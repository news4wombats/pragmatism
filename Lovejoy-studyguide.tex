\documentclass[10pt]{article}
\usepackage{graphicx,fancyhdr,hyperref}

\setlength{\parskip}{.2 em}
\addtolength{\textheight}{.4 in}

% LECTURE NOTES TEMPLATE 13nov2012

% sectioning
\renewcommand*{\thesection}{$\bullet\bullet\bullet$}
\renewcommand*{\thesubsection}{\ $\bullet$}
\renewcommand*{\thesubsubsection}{\ $\Rightarrow$}
\providecommand*{\Lbit}[1]{\paragraph{\S#1}}
\providecommand*{\newbit}{\paragraph{\P}}
\providecommand*{\newex}{\par\medskip\noindent{$\bullet$} }



\renewcommand{\labelitemi}{$\ast$}


\newcommand*{\todothing}[1]{\par\medskip\centerline{$\Rightarrow$ #1 $\Leftarrow$}\medskip\par}
\providecommand*{\aloud}{}
\renewcommand*{\aloud}[1]{\todothing{READ #1}}

\providecommand*{\therefore}{${_\circ}$\hspace{-2pt}${^\circ}$\hspace{-2pt}${_\circ}$}

\renewcommand*{\headrulewidth}{0pt}
\renewcommand*{\footrulewidth}{0pt}


\begin{document}
%preparing styling
\pagestyle{empty}
% --- BEGIN NOTES ---
\section{Thirteen Pragmatisms}

\subsection{Pragmatism as a theory of meaning}
Crudely: The meaning of an idea or judgement is its practical consequences.


\begin{itemize}
\item The meaning of a claim is just the future consequences it predicts, whether we believe it or not. \marginpar{[1]}

\item The meaning of a claim is just the future consequences of believing it. \marginpar{[2]}

\item The meaning of a claim is \emph{in part} in relation to some conscious purpose. \marginpar{[13]}

\end{itemize}


\subsection{Pragmatism as theory of truth}
\begin{itemize}
\item The truth of a claim \emph{consists in} the realization of the experiences to which it is pointed. \marginpar{[3]}
\end{itemize}


\subsection{Pragmatism as an ontology}
\marginpar{[6]}

\begin{itemize}
\item \emph{Temporal becoming} is real. The future is open.
\end{itemize}


\subsection{Pragmatism as theory of knowledge}
\begin{itemize}
\item General propositions are justified if its predictions have held so far. \marginpar{[4]}


\item A claim merits belief if believing it makes life go well. \marginpar{[5]}
\end{itemize}

\subsection{Pragmatism and the satisfaction of knowing}
\begin{itemize}
\item As a psychological fact, believing the truth is satisfying \marginpar{[7]}

\item Claims are justified if they are sufficiently satisfying. \marginpar{[8]}

\item Claims should count as true if they are satisfying in a special, theoretic way. \marginpar{[9]}

\end{itemize}

\subsection{How far can we go beyond empiricism?}

Begin with empiricism: All that we know to exist immediately is the objects of our direct experience.
\begin{itemize}

\item We may and must make postulates. \marginpar{[10]}

\item Logic and math are a priori. \marginpar{[11]}


\item Scientific, ethical, and religious postulates are OK too. \marginpar{[12]}



\end{itemize}
% --- END NOTES ---
\label{theend}
\end{document}