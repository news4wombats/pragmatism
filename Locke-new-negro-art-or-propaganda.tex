\documentclass[12pt]{article}
\usepackage{hyperref,geometry,setspace,graphicx}
\usepackage[tagged]{accessibility}
\usepackage[symbol*,perpage]{footmisc}
\geometry{width=6in,height=9in,centering}
\usepackage[T1]{fontenc}
% changes font to TeX Gyre Schola (Century Schoolbook)
\usepackage{tgschola}
% changes font to TeX Gyre Bonum (Bookman)
%\usepackage{tgbonum}


\begin{document}
%\renewcommand*{\thefootnote}{\fnsymbol{footnote}}
\setcounter{footnote}{0}
\hypersetup{pdfinfo={Title={The New Negro, Art or propaganda?}, Author={Alain Locke}}, pdfborder = {0 0 0 0}}

\setlength{\parskip}{4pt}
\setlength{\parindent}{14pt}


\setstretch{1.1}

\section*{Alain Locke}

\paragraph{About this text:} Prepared by P.D. Magnus (Fall 2025). The text of ``The New Negro'' is based on \href{https://scalar.lehigh.edu/african-american-poetry-a-digital-anthology/the-new-negro-an-interpretation-edited-by-alain-locke-1925}{the on-line version prepared by Amardeep Singh}. The text of ``Art or propaganda?'' is based on a version at \href{https://nationalhumanitiescenter.org/pds/maai3/index.htm}{the National Humanities Center}.

\begin{itemize}
\item Alain Locke, ``The New Negro'' from Alain Locke (ed.), \emph{The New Negro: An Interpretation}, 1925. New York: Albert and Charles Boni. pp. 3--16.

\item Alain Locke, ``Art or propaganda?'' \emph{Harlem}, November 1928, 1(1).
\end{itemize}



\section*{The New Negro}

In the last decade something beyond the watch and guard of statistics has happened in the life of the American Negro and the three norns who have traditionally presided over the Negro problem have a changeling in their laps. The Sociologist, the Philanthropist, the Race-leader are not unaware of the New Negro, but they are at a loss to account for him. He simply cannot be swathed in their formulæ. For the younger generation is vibrant with a new psychology; the new spirit is awake in the masses, and under the very eyes of the professional observers is transforming what has been a perennial problem into the progressive phases of contemporary Negro life.

Could such a metamorphosis have taken place as suddenly as it has appeared to? The answer is no; not because the New Negro is not here, but because the Old Negro had long become more of a myth than a man. The Old Negro, we must remember, was a creature of moral debate and historical controversy. His has been a stock figure perpetuated as an historical fiction partly in innocent sentimentalism, partly in deliberate reactionism. The Negro himself has contributed his share to this through a sort of protective social mimicry forced upon him by the adverse circumstances of dependence. So for generations in the mind of America, the Negro has been more of a formula than a human being--- a something to be argued about, condemned or defended, to be ``kept down,'' or ``in his place,'' or ``helped up,'' to be worried with or worried over, harassed or patronized, a social bogey or a social burden. The thinking Negro even has been induced to share this same general attitude, to focus his attention on controversial issues, to see himself in the distorted perspective of a social problem. His shadow, so to speak, has been more real to him than his personality. Through having had to appeal from the unjust stereotypes of his oppressors and traducers to those of his liberators, friends and benefactors he has had to subscribe to the traditional positions from which his case has been viewed. Little true social or self-understanding has or could come from such a situation.

But while the minds of most of us, black and white, have thus burrowed in the trenches of the Civil War and Reconstruction, the actual march of development has simply flanked these positions, necessitating a sudden reorientation of view. We have not been watching in the right direction; set North and South on a sectional axis, we have not noticed the East till the sun has us blinking.

Recall how suddenly the Negro spirituals revealed themselves; suppressed for generations under the stereotypes of Wesleyan hymn harmony, secretive, half-ashamed, until the courage of being natural brought them out--- and behold, there was folk-music. Similarly the mind of the Negro seems suddenly to have slipped from under the tyranny of social intimidation and to be shaking off the psychology of imitation and implied inferiority. By shedding the old chrysalis of the Negro problem we are achieving something like a spiritual emancipation. Until recently, lacking self-understanding, we have been almost as much of a problem to ourselves as we still are to others. But the decade that found us with a problem has left us with only a task. The multitude perhaps feels as yet only a strange relief and a new vague urge, but the thinking few know that in the reaction the vital inner grip of prejudice has been broken.

With this renewed self-respect and self-dependence, the life of the Negro community is bound to enter a new dynamic phase, the buoyancy from within compensating for whatever pressure there may be of conditions from without. The migrant masses, shifting from countryside to city, hurdle several generations of experience at a leap, but more important, the same thing happens spiritually in the life-attitudes and self-expression of the Young Negro, in his poetry, his art, his education and his new outlook, with the additional advantage, of course, of the poise and greater certainty of knowing what it is all about. From this comes the promise and warrant of a new leadership. As one of them has discerningly put it:

\begin{quote}
   We have tomorrow\\
   Bright before us\\
   Like a flame.\\
   Yesterday, a night-gone thing\\
   A sun-down name.\\
   And dawn today\\
   Broad arch above the road we came.\\
   We march!\footnote{Langston Hughes, ``Youth'' [1924]}
\end{quote}

This is what, even more than any ``most creditable record of fifty years of freedom,'' requires that the Negro of to-day be seen through other than the dusty spectacles of past controversy. The day of ``aunties,'' ``uncles'' and ``mammies'' is equally gone. Uncle Tom and Sambo have passed on, and even the ``Colonel'' and ``George'' play barnstorm rôles from which they escape with relief when the public spotlight is off. The popular melodrama has about played itself out, and it is time to scrap the fictions, garret the bogeys and settle down to a realistic facing of facts.

First we must observe some of the changes which since the traditional lines of opinion were drawn have rendered these quite obsolete. A main change has been, of course, that shifting of the Negro population which has made the Negro problem no longer exclusively or even predominantly Southern. Why should our minds remain sectionalized, when the problem itself no longer is? Then the trend of migration has not only been toward the North and the Central Midwest, but city-ward and to the great centers of industry--- the problems of adjustment are new, practical, local and not peculiarly racial. Rather they are an integral part of the large industrial and social problems of our present-day democracy. And finally, with the Negro rapidly in process of class differentiation, if it ever was warrantable to regard and treat the Negro en masse it is becoming with every day less possible, more unjust and more ridiculous.

In the very process of being transplanted, the Negro is becoming transformed.

The tide of Negro migration, northward and city-ward, is not to be fully explained as a blind flood started by the demands of war industry coupled with the shutting off of foreign migration, or by the pressure of poor crops coupled with increased social terrorism in certain sections of the South and Southwest. Neither labor demand, the boll-weevil nor the Ku Klux Klan is a basic factor, however contributory any or all of them may have been. The wash and rush of this human tide on the beach line of the northern city centers is to be explained primarily in terms of a new vision of opportunity, of social and economic freedom, of a spirit to seize, even in the face of an extortionate and heavy toll, a chance for the improvement of conditions. With each successive wave of it, the movement of the Negro becomes more and more a mass movement toward the larger and the more democratic chance--- in the Negro's case a deliberate flight not only from countryside to city, but from medieval America to modern.

Take Harlem as an instance of this. Here in Manhattan is not merely the largest Negro community in the world, but the first concentration in history of so many diverse elements of Negro life. It has attracted the African, the West Indian, the Negro American; has brought together the Negro of the North and the Negro of the South; the man from the city and the man from the town and village; the peasant, the student, the business man, the professional man, artist, poet, musician, adventurer and worker, preacher and criminal, exploiter and social outcast. Each group has come with its own separate motives and for its own special ends, but their greatest experience has been the finding of one another. Proscription and prejudice have thrown these dissimilar elements into a common area of contact and interaction. Within this area, race sympathy and unity have determined a further fusing of sentiment and experience. So what began in terms of segregation becomes more and more, as its elements mix and react, the laboratory of a great race-welding. Hitherto, it must be admitted that American Negroes have been a race more in name than in fact, or to be exact, more in sentiment than in experience. The chief bond between them has been that of a common condition rather than a common consciousness; a problem in common rather than a life in common. In Harlem, Negro life is seizing upon its first chances for group expression and self-determination. It is--- or promises at least to be--- a race capital. That is why our comparison is taken with those nascent centers of folk-expression and self-determination which are playing a creative part in the world to-day. Without pretense to their political significance, Harlem has the same rôle to play for the New Negro as Dublin has had for the New Ireland or Prague for the New Czechoslovakia.

Harlem, I grant you, isn't typical--- but it is significant, it is prophetic. No sane observer, however sympathetic to the new trend, would contend that the great masses are articulate as yet, but they stir, they move, they are more than physically restless. The challenge of the new intellectuals among them is clear enough--- the ``race radicals'' and realists who have broken with the old epoch of philanthropic guidance, sentimental appeal and protest. But are we after all only reading into the stirrings of a sleeping giant the dreams of an agitator? The answer is in the migrating peasant. It is the ``man farthest down'' who is most active in getting up. One of the most characteristic symptoms of this is the professional man, himself migrating to recapture his constituency after a vain effort to maintain in some Southern corner what for years back seemed an established living and clientele. The clergyman following his errant flock, the physician or lawyer trailing his clients, supply the true clues. In a real sense it is the rank and file who are leading, and the leaders who are following. A transformed and transforming psychology permeates the masses.

When the racial leaders of twenty years ago spoke of developing race-pride and stimulating race-consciousness, and of the desirability of race solidarity, they could not in any accurate degree have anticipated the abrupt feeling that has surged up and now pervades the awakened centers. Some of the recognized Negro leaders and a powerful section of white opinion identified with ``race work'' of the older order have indeed attempted to discount this feeling as a ``passing phase,'' an attack of ``race nerves'' so to speak, an ``aftermath of the war,'' and the like. It has not abated, however, if we are to gauge by the present tone and temper of the Negro press, or by the shift in popular support from the officially recognized and orthodox spokesmen to those of the independent, popular, and often radical type who are unmistakable symptoms of a new order. It is a social disservice to blunt the fact that the Negro of the Northern centers has reached a stage where tutelage, even of the most interested and well-intentioned sort, must give place to new relationships, where positive self-direction must be reckoned with in ever increasing measure. The American mind must reckon with a fundamentally changed Negro.

The Negro too, for his part, has idols of the tribe to smash. If on the one hand the white man has erred in making the Negro appear to be that which would excuse or extenuate his treatment of him, the Negro, in turn, has too often unnecessarily excused himself because of the way he has been treated. The intelligent Negro of to-day is resolved not to make discrimination an extenuation for his shortcomings in performance, individual or collective; he is trying to hold himself at par, neither inflated by sentimental allowances nor depreciated by current social discounts. For this he must know himself and be known for precisely what he is, and for that reason he welcomes the new scientific rather than the old sentimental interest. Sentimental interest in the Negro has ebbed. We used to lament this as the falling off of our friends; now we rejoice and pray to be delivered both from self-pity and condescension. The mind of each racial group has had a bitter weaning, apathy or hatred on one side matching disillusionment or resentment on the other; but they face each other to-day with the possibility at least of entirely new mutual attitudes.

It does not follow that if the Negro were better known, he would be better liked or better treated. But mutual understanding is basic for any subsequent cooperation and adjustment. The effort toward this will at least have the effect of remedying in large part what has been the most unsatisfactory feature of our present stage of race relationships in America, namely the fact that the more intelligent and representative elements of the two race groups have at so many points got quite out of vital touch with one another.

The fiction is that the life of the races is separate, and increasingly so. The fact is that they have touched too closely at the unfavorable and too lightly at the favorable levels.

While inter-racial councils have sprung up in the South, drawing on forward elements of both races, in the Northern cities manual laborers may brush elbows in their everyday work, but the community and business leaders have experienced no such interplay or far too little of it. These segments must achieve contact or the race situation in America becomes desperate. Fortunately this is happening. There is a growing realization that in social effort the co-operative basis must supplant long-distance philanthropy, and that the only safeguard for mass relations in the future must be provided in the carefully maintained contacts of the enlightened minorities of both race groups. In the intellectual realm a renewed and keen curiosity is replacing the recent apathy; the Negro is being carefully studied, not just talked about and discussed. In art and letters, instead of being wholly caricatured, he is being seriously portrayed and painted.

To all of this the New Negro is keenly responsive as an augury of a new democracy in American culture. He is contributing his share to the new social understanding. But the desire to be understood would never in itself have been sufficient to have opened so completely the protectively closed portals of the thinking Negro's mind. There is still too much possibility of being snubbed or patronized for that. It was rather the necessity for fuller, truer self-expression, the realization of the unwisdom of allowing social discrimination to segregate him mentally, and a counter-attitude to cramp and fetter his own living--- and so the ``spite-wall'' that the intellectuals built over the ``color-line'' has happily been taken down. Much of this reopening of intellectual contacts has centered in New York and has been richly fruitful not merely in the enlarging of personal experience, but in the definite enrichment of American art and letters and in the clarifying of our common vision of the social tasks ahead.

The particular significance in the re-establishment of contact between the more advanced and representative classes is that it promises to offset some of the unfavorable reactions of the past, or at least to re-surface race contacts somewhat for the future. Subtly the conditions that are molding a New Negro are molding a new American attitude.

However, this new phase of things is delicate; it will call for less charity but more justice; less help, but infinitely closer understanding. This is indeed a critical stage of race relationships because of the likelihood, if the new temper is not understood, of engendering sharp group antagonism and a second crop of more calculated prejudice. In some quarters, it has already done so. Having weaned the Negro, public opinion cannot continue to paternalize. The Negro to-day is inevitably moving forward under the control largely of his own objectives. What are these objectives? Those of his outer life are happily already well and finally formulated, for they are none other than the ideals of American institutions and democracy. Those of his inner life are yet in process of formation, for the new psychology at present is more of a consensus of feeling than of opinion, of attitude rather than of program. Still some points seem to have crystallized.

Up to the present one may adequately describe the Negro's ``inner objectives'' as an attempt to repair a damaged group psychology and reshape a warped social perspective. Their realization has required a new mentality for the American Negro. And as it matures we begin to see its effects; at first, negative, iconoclastic, and then positive and constructive. In this new group psychology we note the lapse of sentimental appeal, then the development of a more positive self-respect and self-reliance; the repudiation of social dependence, and then the gradual recovery from hyper-sensitiveness and ``touchy'' nerves, the repudiation of the double standard of judgment with its special philanthropic allowances and then the sturdier desire for objective and scientific appraisal; and finally the rise from social disillusionment to race pride, from the sense of social debt to the responsibilities of social contribution, and offsetting the necessary working and commonsense acceptance of restricted conditions, the belief in ultimate esteem and recognition. Therefore the Negro to-day wishes to be known for what he is, even in his faults and shortcomings, and scorns a craven and precarious survival at the price of seeming to be what he is not. He resents being spoken of as a social ward or minor, even by his own, and to being regarded a chronic patient for the sociological clinic, the sick man of American Democracy. For the same reasons, he himself is through with those social nostrums and panaceas, the so-called ``solutions'' of his ``problem,'' with which he and the country have been so liberally dosed in the past. Religion, freedom, education, money--- in turn, he has ardently hoped for and peculiarly trusted these things; he still believes in them, but not in blind trust that they alone will solve his life-problem.

Each generation, however, will have its creed, and that of the present is the belief in the efficacy of collective effort, in race co-operation. This deep feeling of race is at present the mainspring of Negro life. It seems to be the outcome of the reaction to proscription and prejudice; an attempt, fairly successful on the whole, to convert a defensive into an offensive position, a handicap into an incentive. It is radical in tone, but not in purpose and only the most stupid forms of opposition, misunderstanding or persecution could make it otherwise. Of course, the thinking Negro has shifted a little toward the left with the world-trend, and there is an increasing group who affiliate with radical and liberal movements. But fundamentally for the present the Negro is radical on race matters, conservative on others, in other words, a ``forced radical,'' a social protestant rather than a genuine radical. Yet under further pressure and injustice iconoclastic thought and motives will inevitably increase. Harlem's quixotic radicalisms call for their ounce of democracy to-day lest to-morrow they be beyond cure.

The Negro mind reaches out as yet to nothing but American wants, American ideas. But this forced attempt to build his Americanism on race values is a unique social experiment, and its ultimate success is impossible except through the fullest sharing of American culture and institutions. There should be no delusion about this. American nerves in sections unstrung with race hysteria are often fed the opiate that the trend of Negro advance is wholly separatist, and that the effect of its operation will be to encyst the Negro as a benign foreign body in the body politic. This cannot be--- even if it were desirable. The racialism of the Negro is no limitation or reservation with respect to American life; it is only a constructive effort to build the obstructions in the stream of his progress into an efficient dam of social energy and power. Democracy itself is obstructed and stagnated to the extent that any of its channels are closed. Indeed they cannot be selectively closed. So the choice is not between one way for the Negro and another way for the rest, but between American institutions frustrated on the one hand and American ideals progressively fulfilled and realized on the other.

There is, of course, a warrantably comfortable feeling in being on the right side of the country's professed ideals. We realize that we cannot be undone without America's undoing. It is within the gamut of this attitude that the thinking Negro faces America, but with variations of mood that are if anything more significant than the attitude itself. Sometimes we have it taken with the defiant ironic challenge of McKay:

\begin{quote}
   Mine is the future grinding down to-day\\
   Like a great landslip moving to the sea,\\
   Bearing its freight of débris far away\\
   Where the green hungry waters restlessly\\
   Heave mammoth pyramids, and break and roar\\
   Their eerie challenge to the crumbling shore.\footnote{Claude McKay, ``To the Intrenched Classes'' [1922]}
\end{quote}

Sometimes, perhaps more frequently as yet, it is taken in the fervent and almost filial appeal and counsel of Weldon Johnson's:

\begin{quote}
   O Southland, dear Southland!\\
   Then why do you still cling\\
   To an idle age and a musty page,\\
   To a dead and useless thing?\footnote{James Weldon Johnson, ``O Southland'' [1917]}
\end{quote}

But between defiance and appeal, midway almost between cynicism and hope, the prevailing mind stands in the mood of the same author's \emph{To America}, an attitude of sober query and stoical challenge:

\begin{quote}
   How would you have us, as we are?\\
   Or sinking 'neath the load we bear,\\
   Our eyes fixed forward on a star,\\
   Or gazing empty at despair?\\
   Rising or falling? Men or things?\\
   With dragging pace or footsteps fleet?\\
   Strong, willing sinews in your wings,\\
   Or tightening chains about your feet?\footnote{James Weldon Johnson, ``To America''}
\end{quote}

More and more, however, an intelligent realization of the great discrepancy between the American social creed and the American social practice forces upon the Negro the taking of the moral advantage that is his. Only the steadying and sobering effect of a truly characteristic gentleness of spirit prevents the rapid rise of a definite cynicism and counter-hate and a defiant superiority feeling. Human as this reaction would be, the majority still deprecate its advent, and would gladly see it forestalled by the speedy amelioration of its causes. We wish our race pride to be a healthier, more positive achievement than a feeling based upon a realization of the shortcomings of others. But all paths toward the attainment of a sound social attitude have been difficult; only a relatively few enlightened minds have been able as the phrase puts it ``to rise above'' prejudice. The ordinary man has had until recently only a hard choice between the alternatives of supine and humiliating submission and stimulating but hurtful counter-prejudice. Fortunately from some inner, desperate resourcefulness has recently sprung up the simple expedient of fighting prejudice by mental passive resistance, in other words by trying to ignore it. For the few, this manna may perhaps be effective, but the masses cannot thrive upon it.

Fortunately there are constructive channels opening out into which the balked social feelings of the American Negro can flow freely.

Without them there would be much more pressure and danger than there is. These compensating interests are racial but in a new and enlarged way. One is the consciousness of acting as the advance-guard of the African peoples in their contact with Twentieth Century civilization; the other, the sense of a mission of rehabilitating the race in world esteem from that loss of prestige for which the fate and conditions of slavery have so largely been responsible. Harlem, as we shall see, is the center of both these movements; she is the home of the Negro's ``Zionism.'' The pulse of the Negro world has begun to beat in Harlem. A Negro newspaper carrying news material in English, French and Spanish, gathered from all quarters of America, the West Indies and Africa has maintained itself in Harlem for over five years. Two important magazines, both edited from New York, maintain their news and circulation consistently on a cosmopolitan scale. Under American auspices and backing, three pan-African congresses have been held abroad for the discussion of common interests, colonial questions and the future co-operative development of Africa. In terms of the race question as a world problem, the Negro mind has leapt, so to speak, upon the parapets of prejudice and extended its cramped horizons. In so doing it has linked up with the growing group consciousness of the dark-peoples and is gradually learning their common interests. As one of our writers has recently put it: ``It is imperative that we understand the white world in its relations to the non-white world.'' As with the Jew, persecution is making the Negro international.

As a world phenomenon this wider race consciousness is a different thing from the much asserted rising tide of color. Its inevitable causes are not of our making. The consequences are not necessarily damaging to the best interests of civilization. Whether it actually brings into being new Armadas of conflict or argosies of cultural exchange and enlightenment can only be decided by the attitude of the dominant races in an era of critical change. With the American Negro, his new internationalism is primarily an effort to recapture contact with the scattered peoples of African derivation. Garveyism may be a transient, if spectacular, phenomenon, but the possible rôle of the American Negro in the future development of Africa is one of the most constructive and universally helpful missions that any modern people can lay claim to.

Constructive participation in such causes cannot help giving the Negro valuable group incentives, as well as increased prestigé at home and abroad. Our greatest rehabilitation may possibly come through such channels, but for the present, more immediate hope rests in the revaluation by white and black alike of the Negro in terms of his artistic endowments and cultural contributions, past and prospective. It must be increasingly recognized that the Negro has already made very substantial contributions, not only in his folk-art, music especially, which has always found appreciation, but in larger, though humbler and less acknowledged ways. For generations the Negro has been the peasant matrix of that section of America which has most undervalued him, and here he has contributed not only materially in labor and in social patience, but spiritually as well. The South has unconsciously absorbed the gift of his folk-temperament. In less than half a generation it will be easier to recognize this, but the fact remains that a leaven of humor, sentiment, imagination and tropic nonchalance has gone into the making of the South from a humble, unacknowledged source. A second crop of the Negro's gifts promises still more largely. He now becomes a conscious contributor and lays aside the status of a beneficiary and ward for that of a collaborator and participant in American civilization. The great social gain in this is the releasing of our talented group from the arid fields of controversy and debate to the productive fields of creative expression. The especially cultural recognition they win should in turn prove the key to that revaluation of the Negro which must precede or accompany any considerable further betterment of race relationships. But whatever the general effect, the present generation will have added the motives of self-expression and spiritual development to the old and still unfinished task of making material headway and progress. No one who understandingly faces the situation with its substantial accomplishment or views the new scene with its still more abundant promise can be entirely without hope. And certainly, if in our lifetime the Negro should not be able to celebrate his full initiation into American democracy, he can at least, on the warrant of these things, celebrate the attainment of a significant and satisfying new phase of group development, and with it a spiritual Coming of Age.

\section*{Art or propaganda?}

Artistically it is the one fundamental question for us today,--- Art or Propaganda. Which? Is this more the generation of the prophet or that of the poet; shall our intellectual and cultural leadership preach and exhort or sing? I believe we are at that interesting moment when the prophet becomes the poet and when prophecy becomes the expressive song, the chant of fulfillment. We have had too many Jeremiahs, major and minor;--- and too much of the drab wilderness. My chief objection to propaganda, apart from its besetting sin of monotony and disproportion, is that it perpetuates the position of group inferiority even in crying out against it. For it leaves and speaks under the shadow of a dominant majority whom it harangues, cajoles, threatens or supplicates. It is too extroverted for balance or poise or inner dignity and self-respect. Art in the best sense is rooted in self-expression and whether naive or sophisticated is self-contained. In our spiritual growth genius and talent must more and more choose the role of group expression, or even at times the role of free individualistic expression,--- in a word must choose art and put aside propaganda.

The literature and art of the younger generation already reflects this shift of psychology, this regeneration of spirit. David should be its patron saint: it should confront the Phillistines with its five smooth pebbles fearlessly. There is more strength in a confident camp than in a threatened enemy. The sense of inferiority must be innerly compensated, self-conviction must supplant self-justification and in the dignity of this attitude a convinced minority must confront a condescending majority. Art cannot completely accomplish this, but I believe it can lead the way.

Our espousal of art thus becomes no mere idle acceptance of “art for art's sake,” or cultivation of the last decadences of the over-civilized, but rather a deep realization of the fundamental purpose of art and of its function as a tap root of vigorous, flourishing living. Not all of our younger writers are deep enough in the sub-soil of their native materials--- too many are pot-plants seeking a forced growth according to the exotic tastes of a pampered and decadent public. It is the art of the people that needs to be cultivated, not the art of the coteries. Propaganda itself is preferable to shallow, truckling imitation. Negro things may reasonably be a fad for others; for us they must be a religion. Beauty, however, is its best priest and psalms will be more effective than sermons.

To date we have had little sustained art unsubsidized by propaganda; we must admit this debt to these foster agencies. The three journals which have been vehicles of most of our artistic expressions have been the avowed organs of social movements and organized social programs. All our purely artistic publications have been sporadic. There is all the greater need then for a sustained vehicle of free and purely artistic expression. If HARLEM should happily fill this need, it will perform an honorable and constructive service.\footnote{Locke is referring to the the literary magazine \emph{Harlem} where this was published.} I hope it may, but should it not, the need remains and the path toward it will at least be advanced a little.

We need, I suppose in addition to art some substitute for propaganda. What shall that be? Surely we must take some cognizance of the fact that we live at the centre of a social problem. Propaganda at least nurtured some form of serious social discussion, and social discussion was necessary, is still necessary. On this side; the difficulty and shortcoming of propaganda is its partisanship. It is one-sided and often pre-judging. Should we not then have a journal of free discussion, open to all sides of the problem and to all camps of belief? Difficult, that,--- but intriguing. Even if it has to begin on the note of dissent and criticism and assume Menckenian scepticism to escape the common-places of conformity. Yet, I hope we shall not remain at this negative pole. Can we not cultivate truly free and tolerant discussion, almost Socratically minded for the sake of truth? After Beauty, let Truth come into the Renaissance picture,--- a later cue, but a welcome one. This may be premature, but one hopes not,--- for eventually it must come and if we can accomplish that, instead of having to hang our prophets, we can silence them or change their lamentations to song with a Great Fulfillment.

\end{document}