\documentclass[12pt]{article}
\usepackage{hyperref,geometry,setspace,graphicx}
\usepackage[]{accessibility}
\usepackage[symbol*,perpage]{footmisc}
\geometry{width=6in,height=9in,centering}
\usepackage[T1]{fontenc}
% changes font to TeX Gyre Schola (Century Schoolbook)
\usepackage{tgschola}
% changes font to TeX Gyre Bonum (Bookman)
%\usepackage{tgbonum}


\begin{document}
%\renewcommand*{\thefootnote}{\fnsymbol{footnote}}
\setcounter{footnote}{0}
\hypersetup{pdfinfo={Title={Scientific Method and Individual Thinker}, Author={George H. Mead}}, pdfborder = {0 0 0 0}}

\setlength{\parskip}{4pt}
\setlength{\parindent}{0pt}




George H. Mead, `Scientific Method and Individual Thinker' in John Dewey et al., \emph{Creative Intelligence: Essays in the Pragmatic Attitude}, New York: Henry Holt and Company, 1917.

\newcommand*{\marginpage}[1]{\marginpar{\tiny{#1}}}
\paragraph{About this text:} The text is based on the \href{http://www.gutenberg.org/}{Project Gutenberg} edition produced by Adrian Mastronardi, Turgut Dincer, and the Online Distributed Proofreading Team (EBook \#33727). Prepared by P.D. Magnus, Fall 2025. Marginal page numbers reflect the original.

\begin{center}
\centerline{\includegraphics[height=1in]{creative-intelligence-cover-owl.png}}

\textsc{\Large Scientific Method and Individual Thinker}

George H. Mead
\end{center}

\setstretch{1.2}

\setlength{\parindent}{0pt}
\marginpage{176}


The scientist in the ancient world found his test of
reality in the evidence of the presence of the essence
of the object. This evidence came by way of observation,
even to the Platonist. Plato could treat this
evidence as the awaking of memories of the ideal essence
of the object seen in a world beyond the heavens during
a former stage of the existence of the soul. In the
language of Theatetus it was the agreement of fluctuating
sensual content with the thought-content imprinted
in or viewed by the soul. In Aristotle it is again
the agreement of the organized sensuous experience
with the vision which the mind gets of the essence of
the object through the perceptual experience of a number
of instances. That which gives the stamp of reality
is the coincidence of the percept with a rational content
which must in some sense be in the mind to insure
knowledge, as it must be in the cosmos to insure existence,
of the object. The relation of this test of reality
to an analytical method is evident. Our perceptual
world is always more crowded and confused than the
ideal contents by which the reality of its meaning is
to be tested. The aim of the analysis varies with the
character of the science. In the case of Aristotle's
theoretical sciences, such as mathematics and metaphysics,\marginpage{177}
where one proceeds by demonstration from the
given existences, analysis isolates such elements as
numbers, points, lines, surfaces, and solids, essences
and essential accidents. Aristotle approaches nature,
however, as he approaches the works of human art.
Indeed, he speaks of nature as the artificer par excellence.
In the study of nature, then, as in the study
of the practical and productive arts, it is of the first
importance that the observer should have the idea—the
final cause—as the means of deciphering the nature of
living forms. Here analysis proceeds to isolate characters
which are already present in forms whose functions
are assumed to be known. By analogy such identities
as that of fish fins with limbs of other vertebrates are
assumed, and some very striking anticipations of modern
biological conceptions and discoveries are reached.
Aristotle recognizes that the theory of the nature of
the form or essence must be supported by observation
of the actual individual. What is lacking is any body
of observation which has value apart from some theory.
He tests his theory by the observed individual which
is already an embodied theory, rather than by what
we are wont to call the facts. He refers to other observers
to disagree with them. He does not present
their observations apart from their theories as material
which has existential value, independent for the
time being of any hypothesis. And it is consistent
with this attitude that he never presents the observations
of others in support of his own doctrine. His
analysis within this field of biological observation does
not bring him back to what, in modern science, are the\marginpage{178}
data, but to general characters which make up the
definition of the form. His induction involves a gathering
of individuals rather than of data. Thus analysis
in the theoretical, the natural, the practical,
and the productive sciences, leads back to universals.
This is quite consistent with Aristotle's metaphysical
position that since the matter of natural objects has
reality through its realization in the form, whatever
appears without such meaning can be accounted for
only as the expression of the resistance which matter
offers to this realization. This is the field of a blind
necessity, not that of a constructive science.


Continuous advance in science has been possible
only when analysis of the object of knowledge has supplied
not elements of meanings as the objects have
been conceived but elements abstracted from those
meanings. That is, scientific advance implies a willingness
to remain on terms of tolerant acceptance of
the reality of what cannot be stated in the accepted
doctrine of the time, but what must be stated in the
form of contradiction with these accepted doctrines.
The domain of what is usually connoted by the term
facts or data belongs to the field lying between the old
doctrine and the new. This field is not inhabited by
the Aristotelian individual, for the individual is but
the realization of the form or universal essence. When
the new theory has displaced the old, the new individual
appears in the place of its predecessor, but
during the period within which the old theory is being
dislodged and the new is arising, a consciously growing
science finds itself occupied with what is on the one\marginpage{179}
hand the débris of the old and on the other the building
material of the new. Obviously, this must find its immediate
\emph{raison d'\^{e}tre} in something other than the
meaning that is gone or the meaning that is not yet
here. It is true that the barest facts do not lack
meaning, though a meaning which has been theirs in
the past is lost. The meaning, however, that is still
theirs is confessedly inadequate, otherwise there would
be no scientific problem to be solved. Thus, when
older theories of the spread of infectious diseases lost
their validity because of instances where these explanations
could not be applied, the diagnoses and accounts
which could still be given of the cases of the
sickness themselves were no explanation of the spread
of the infection. The facts of the spread of the infection
could be brought neither under a doctrine of
contagion which was shattered by actual events nor
under a doctrine of the germ theory of disease, which
was as yet unborn. The logical import of the dependence
of these facts upon observation, and hence
upon the individual experience of the scientist, I shall
have occasion to discuss later; what I am referring to
here is that the conscious growth of science is accompanied
by the appearance of this sort of material.


There were two fields of ancient science, those of
mathematics and of astronomy, within which very considerable
advance was achieved, a fact which would
seem therefore to offer exception to the statement just
made. The theory of the growth of mathematics is a
disputed territory, but whether mathematical discovery
and invention take place by steps which can be identified\marginpage{180}
with those which mark the advance in the experimental
sciences or not, the individual processes in which
the discoveries and inventions have arisen are almost
uniformly lost to view in the demonstration which presents
the results. It would be improper to state that
no new data have arisen in the development of mathematics,
in the face of such innovations as the minus
quantity, the irrational, the imaginary, the infinitesimal,
or the transfinite number, and yet the innovations
appear as the recasting of the mathematical theories
rather than as new facts. It is of course true that
these advances have depended upon problems such as
those which in the researches of Kepler and Galileo
led to the early concepts of the infinitesimal procedure,
and upon such undertakings as bringing the combined
theories of geometry and algebra to bear upon the
experiences of continuous change. For a century after
the formulation of the infinitesimal method men were
occupied in carrying the new tool of analysis into
every field where its use promised advance. The
conceptions of the method were uncritical. Its applications
were the center of attention. The next
century undertook to bring order into the concepts,
consistency into the doctrine, and rigor into the
reasoning. The dominating trend of this movement
was logical rather than methodological. The development
was in the interest of the foundations of
mathematics rather than in the use of mathematics
as a method for solving scientific problems. Of course
this has in no way interfered with the freedom of
application of mathematical technique to the problems\marginpage{181}
of physical science. On the contrary, it was on account
of the richness and variety of the contents which the
use of mathematical methods in the physical sciences
imported into the doctrine that this logical housecleaning
became necessary in mathematics. The movement
has been not only logical as distinguished from methodological
but logical as distinguished from metaphysical
as well. It has abandoned a Euclidean space with its
axioms as a metaphysical presupposition, and it has
abandoned an Aristotelian subsumptive logic for which
definition is a necessary presupposition. It recognizes
that everything cannot be proved, but it does not undertake
to state what the axiomata shall be; and it also
recognizes that not everything can be defined, and does
not undertake to determine what shall be defined implicitly
and what explicitly. Its constants are logical
constants, as the proposition, the class and the relation.
With these and their like and with relatively
few primitive ideas, which are represented by symbols,
and used according to certain given postulates, it becomes
possible to bring the whole body of mathematics
within a single treatment. The development of this
pure mathematics, which comes to be a logic of the
mathematical sciences, has been made possible by such
a generalization of number theory and theories of the
elements of space and time that the rigor of mathematical
reasoning is secured, while the physical
scientist is left the widest freedom in the choice
and construction of concepts and imagery for his
hypotheses. The only compulsion is a logical compulsion.
The metaphysical compulsion has disappeared\marginpage{182}
from mathematics and the sciences whose techniques it
provides.


It was just this compulsion which confined ancient
science. Euclidian geometry defined the limits of
mathematics. Even mechanics was cultivated largely
as a geometrical field. The metaphysical doctrine
according to which physical objects had their own places
and their own motions determined the limits within
which astronomical speculations could be carried on.
Within these limits Greek mathematical genius achieved
marvelous results. The achievements of any period
will be limited by two variables: the type of problem
against which science formulates its methods, and the
materials which analysis puts at the scientist's disposal
in attacking the problems. The technical problems
of the trisection of an angle and the duplication
of a cube are illustrations of the problems which
characterize a geometrical doctrine that was finding
its technique. There appears also the method of analysis
of the problem into simpler problems, the
assumption of the truth of the conclusion to be proved
and the process of arguing from this to a known truth.
The more fundamental problem which appears first
as the squaring of the circle, which becomes that of
the determination of the relation of the circle to its
diameter and development of the method of exhaustion,
leads up to the sphere, the regular polyhedra, to conic
sections and the beginnings of trigonometry. Number
was not freed from the relations of geometrical magnitudes,
though Archimedes could conceive of a number
greater or smaller than any assignable magnitude.\marginpage{183}
With the method of exhaustion, with the conceptions
of number found in writings of Archimedes and others,
with the beginnings of spherical geometry and trigonometry,
and with the slow growth of algebra finding
its highest expression in that last flaring up of Greek
mathematical creation, the work of Diophantes; there
were present all the conceptions which were necessary
for attack upon the problems of velocities and changing
velocities, and the development of the method of
analysis which has been the revolutionary tool of Europe
since the Renaissance. But the problems of a
relation between the time and space of a motion that
should change just as a motion, without reference to
the essence of the object in motion, were problems
which did not, perhaps could not, arise to confront
the Greek mind. In any case its mathematics was firmly
embedded in a Euclidian space. Though there are
indications of some distrust, even in Greek times, of
the parallel axiom, the suggestion that mathematical
reasoning could be made rigorous and comprehensive
independently of the specific content of axiom and
definition was an impossible one for the Greek, because
such a suggestion could be made only on the
presupposition of a number theory and an algebra
capable of stating a continuum in terms which are
independent of the sensuous intuition of space and time
and of the motion that takes place within space and
time. In the same fashion mechanics came back to
fundamental generalizations of experience with reference
to motions which served as axioms of mechanics,
both celestial and terrestrial: the assumptions of the\marginpage{184}
natural motion of earthly substances to their own
places in straight lines, and of celestial bodies in circles
and uniform velocities, of an equilibrium where equal
weights operate at equal distances from the fulcrum.


The incommensurable of Pythagoras and the paradoxes
of Zeno present the ``no thoroughfares'' of ancient
mathematical thought. Neither the continuum
of space nor of motion could be broken up into ultimate
units, when incommensurable ratios existed which
could not be expressed, and when motion refused to be
divided into positions of space or time since these are
functions of motion. It was not until an algebraic
theory of number led mathematicians to the use of
expressions for the irrational, the minus, and the imaginary
numbers through the logical development of
generalized expressions, that problems could be formulated
in which these irrational ratios and quantities
were involved, though it is also true that the effort
to deal with problems of this character was in no small
degree responsible for the development of the algebra.
Fixed metaphysical assumptions in regard to number,
space, time, motion, and the nature of physical objects
determined the limits within which scientific investigation
could take place. Thus though the hypothesis
of Copernicus and in all probability of Tycho Brahe
were formulated by Greek astronomers, their physical
doctrine was unable to use them because they were in
flagrant contradiction with the definitions the ancient
world gave to earthly and celestial bodies and their
natural motions. The atomic doctrine with Democritus'
thoroughgoing undertaking to substitute a quantitative\marginpage{185}
for a qualitative conception of matter with the
location of the qualitative aspects of the world in the
experience of the soul appealed only to the Epicurean
who used the theory as an exorcism to drive out of the
universe the spirits which disturbed the calm of the
philosopher.


There was only one field in which ancient science
seemed to break away from the fixed assumptions
of its metaphysics and from the definitions of natural
objects which were the bases for their scientific inferences,
this was the field of astronomy in the period
after Eudoxus. Up to and including the theories of
Eudoxus, physical and mathematical astronomy went
hand in hand. Eudoxus' nests of spheres within spheres
hung on different axes revolving in different uniform
periods was the last attempt of the mathematician
philosopher to state the anomalies of the heavens, and
to account for the stations, the retrogressions, and
varying velocities of planetary bodies by a theory resolving
all phenomena of these bodies into motions of
uniform velocities in perfect circles, and also placing
these phenomena within a physical theory consistent
with the prevailing conceptions of the science and philosophy
of the time. As a physicist Aristotle felt the
necessity of introducing further spheres between the
nests of spheres assigned by Eudoxus to the planetary
bodies, spheres whose peculiar motions should correct
the tendency of the different groups of spheres to pass
their motions on to each other. Since the form of the
orbits of heavenly bodies and their velocities could not
be considered to be the results of their masses and of\marginpage{186}
their relative positions with reference to one another;
since it was not possible to calculate the velocities and
orbits from the physical characters of the bodies, since
in a word these physical characters did not enter into
the problem of calculating the positions of the bodies
nor offer explanations for the anomalies which the
mathematical astronomer had to explain, it was not
strange that he disinterested himself from the metaphysical
celestial mechanics of his time and concentrated
his attention upon the geometrical hypotheses
by means of which he could hope to resolve into uniform
revolutions in circular orbits the anomalous motions
of the planetary bodies. The introduction of the
epicycle with the deferent and the eccentric as working
hypotheses to solve the anomalies of the heavens is
to be comprehended largely in view of the isolation of
the mathematical as distinguished from the physical
problem of astronomy. In no sense were these conceptions
working hypotheses of a celestial mechanics.
They were the only means of an age whose mathematics
was almost entirely geometrical for accomplishing
what a later generation could accomplish by an algebraic
theory of functions. As has been pointed out,
the undertaking of the ancient mathematical astronomer
to resolve the motions of planetary bodies into
circular, uniform, continuous, symmetrical movements
is comparable to the theorem of Fourier which allows
the mathematician to replace any one periodic function
by a sum of circular functions. In other words,
the astronomy of the Alexandrian period is a somewhat
cumbrous development of the mathematical technique\marginpage{187}
of the time to enable the astronomer to bring the anomalies
of the planetary bodies, as they increased under
observation, within the axioms of a metaphysical
physics. The genius exhibited in the development of
the mathematical technique places the names of Apollonius
of Perga, Hipparchus of Nicaea, and Ptolemy
among the great mathematicians of the world, but they
never felt themselves free to attack by their hypotheses
the fundamental assumptions of the ancient metaphysical
doctrine of the universe. Thus it was said of
Hipparchus by Adrastus, a philosopher of the first
century A. D., in explaining his preference for the
epicycle to the eccentric as a means of analyzing the
motions of the planetary bodies: ``He preferred and
adopted the principle of the epicycle as more probable
to his mind, because it ordered the system of the
heavens with more symmetry and with a more intimate
dependence with reference to the center of the universe.
Although he guarded himself from assuming the rôle
of the physicist in devoting himself to the investigations
of the real movements of the stars, and in undertaking
to distinguish between the motions which nature has
adopted from those which the appearances present to
our eyes, he assumed that every planet revolved along
an epicycle, the center of which describes a circumference
concentric with the earth.'' Even mathematical
astronomy does not offer an exception to the scientific
method of the ancient world, that of bringing to
consciousness the concepts involved in their world
of experience, organizing these concepts with reference
to each, analyzing and restating them within\marginpage{188}
the limits of their essential accidents, and assimilating
the concrete objects of experience to these typical
forms as more or less complete realizations.


At the beginning of the process of Greek self-conscious
reflection and analysis, the mind ran riot among
the concepts and their characters until the contradictions
which arose from these unsystematized speculations
brought the Greek mind up to the problems of
criticism and scientific method. Criticism led to the
separation of the many from the one, the imperfect copy
from the perfect type, the sensuous and passionate from
the rational and the intrinsically good, the impermanent
particular from the incorruptible universal. The line
of demarcation ran between the lasting reality that
answered to critical objective thought and the realm
of perishing imperfect instances, of partially realized
forms full of unmeaning differences due to distortion
and imperfection, the realm answering to a sensuous
passionate unreflective experience. It would be a quite
inexcusable mistake to put all that falls on the wrong
side of the line into a subjective experience, for these
characters belonged not alone to the experience, but also
to the passing show, to the world of imperfectly developed
matter which belonged to the perceptual passionate
experience. While it may not then be classed
as subjective, the Greeks of the Sophistic period felt
that this phase of existence was an experience which
belongs to the man in his individual life, that life in
which he revolts from the conventions of society, in
which he questions accepted doctrine, in which he differentiates
himself from his fellows. Protagoras seems\marginpage{189}
even to have undertaken to make this experience of the
individual, the stuff of the known world. It is difficult
adequately to assess Protagoras' undertaking. He
seems to be insisting both that the man's experience as
his own must be the measure of reality as known and
on the other hand that these experiences present norms
which offer a choice in conduct. If this is true Protagoras
conceived of the individual's experience in its
atypical and revolutionary form as not only real but
the possible source of fuller realities than the world
of convention. The undertaking failed both in philosophic
doctrine and in practical politics. It failed
in both fields because the subjectivist, both in theory
and practice, did not succeed in finding a place
for the universal character of the object, its meaning,
in the mind of the individual and thus in finding
in this experience the hypothesis for the reconstruction
of the real world. In the ancient world the
atypical individual, the revolutionist, the non-conformist
was a self-seeking adventurer or an anarchist, not
an innovator or reformer, and subjectivism in ancient
philosophy remained a skeptical attitude which could
destroy but could not build up.


Hippocrates and his school came nearer consciously
using the experience of the individual as the actual
material of the object of knowledge. In the skeptical
period in which they flourished they rejected on the
one hand the magic of traditional medicine and on the
other the empty theorizing that had been called out
among the physicians by the philosophers. Their practical
tasks held them to immediate experience. Their\marginpage{190}
functions in the gymnasia gave their medicine an interest
in health as well as in disease, and directed their
attention largely toward diet, exercise, and climate
in the treatment even of disease. In its study they have
left the most admirable sets of observations, including
even accounts of acknowledged errors and the results
of different treatments of cases, which ancient science
can present. It was the misfortune of their science
that it dealt with a complicated subject-matter dependent
for its successful treatment upon the whole
body of physical, chemical, and biological disciplines
as well as the discovery and invention of complicated
techniques. They were forced after all to adopt a
hopelessly inadequate physiological theory—that of
the four humors—with the corresponding doctrine of
health and disease as the proper and improper mixture
of these fluids. Their marvelously fine observation
of symptoms led only to the definition of types and a
medical practice which was capable of no consistent
progress outside of certain fields of surgery. Thus
even Greek medicine was unable to develop a different
type of scientific method except in so far as it kept alive
an empiricism which played a not unimportant part
in post-Aristotelian philosophy. Within the field of
astronomy in explaining the anomalies of the heavens
involved in their metaphysical assumptions, they built
up a marvelously perfect Euclidian geometry, for here
refined and exhaustive definition of all the elements was
possible. The problems involved in propositions to be
proved appeared in the individual experience of the
geometrician, but this experience in space was uniform\marginpage{191}
with that of every one else and took on a universal not
an individual form. The test of the solution was given
in a demonstration which holds for every one living
in the same Euclidian space. When the mathematician
found himself carried by his mathematical technique
beyond the assumptions of a metaphysical physics he
abandoned the field of physical astronomy and confined
himself to the development of his mathematical expressions.


In other fields Greek science analyzed with varying
success and critical skill only the conceptions found in
the experience of their time and world. Nor did Greek
thought succeed in formulating any adequate method
by which the ultimate concepts in any field of science
were to be determined. It is in Aristotle's statement of
induction and the process of definition that we appreciate
most clearly the inadequacy of their method. This
inadequacy lies fundamentally in Aristotle's conception
of observation which, as I have already noted, implies
the recognition of an individual, that is, an object
which is an embodied form or idea. The function of
knowledge is to bring out this essence. The mind sees
through the individuals the universal nature. The
value of the observation lies, then, not in the controlled
perception of certain data as observed facts,
but in the insight with which he recognizes the nature
of the object. When this nature has been seen it is
to be analyzed into essential characters and thus
formulated into the definition. In Aristotle's methodology
there is no procedure by which the mind can
deliberately question the experience of the community\marginpage{192}
and by a controlled method reconstruct its received
world. Thus the natural sciences were as really fixed
by the conceptions of the community as were the exact
sciences by the conceptions of a Euclidian geometry
and the mathematics which the Greeks formulated
within it. The individual within whose peculiar experience
arises a contradiction to the prevailing conceptions
of the community and in whose creative
intelligence appears the new hypothesis which makes
possible a new heaven and a new earth could utilize
his individual experience only in destructive skepticism.
Subjectivism served in ancient thought to invalidate
knowledge not to enlarge it.


Zeller has sketched a parallelism between the ideal
state of Plato and the social structure of the medieval
world. The philosopher-king is represented by the Pope,
below him answering to the warrior class in the Platonic
state stands the warrior class of the Holy Roman
Empire, who in theory enforce the dictates of the
Roman curia, while at the bottom in both communities
stand the mass of the people bound to obedience to the
powers above. There is, however, one profound difference
between the two, and that is to be found in the
relative positions of the ideal worlds that dominate
each. Plato's ideal world beyond the heavens gives
what reality it has to this through the participation
by the world of becoming in the ideas. Opinion dimly
sensed the ideas in the evanescent objects about it, and
though Plato's memory theory of knowledge assumed
that the ideas had been seen in former existence and
men could thus recognize the copies here, the ideal\marginpage{193}
world was not within the mind but without. In a real
sense the Kingdom of Heaven was within men in the
medieval world, as was the Holy Roman Empire. They
were ideal communities that ought to exist on earth,
and it was due to the depravity of men that they did
not exist. From time to time men undertook in various
upheavals to realize in some part these spiritual and
political ideals which they carried within them. And
men not only carried within them the ideas of a New
Jerusalem in which the interest of one was the interest
of all and of an earthly state ordered by a divine
decree to fulfil this Christian ideal, but the determining
causes of the present condition and the future
realization depended also upon the inner attitudes and
experiences of the individuals themselves.


Without carrying the analogy here too far, this
relation between the experience of the individual and
the world which may arise through the realization of
his ideas is the basis of the most profound distinction
between the ancient world and the modern. Before
the logic of this attitude could appear in science a long
period of intellectual and social growth was necessary.
The most essential part of this growth was the slow
but steady development of psychological doctrine which
placed the objective world in the experience of the
individual. It is not of interest here to bring out the
modern epistemological problem that grew out of this,
or to present this in the world of Leibnitzian monads
that had no windows or in the Berkeleyan subjective
idealism. What is of interest is to point out that this
attitude established a functional relationship between\marginpage{194}
even the subjective experience of the individual and the
object of knowledge. A skepticism based upon subjectivism
might thereafter question the justification of
the reference of experience beyond itself; it could not
question knowledge and its immediate object.


Kant formalizes the relation of what was subjective
and what was objective by identifying the former with
the sensuous content of experience and the latter with
the application of the forms of sensibility and understanding
to this content. The relationship was formal
and dead. Kant recognized no functional relationship
between the nature of the \emph{Mannigfaltigkeit} of sensuous
experience and the forms into which it was poured. The
forms remained external to the content, but the relationship
was one which existed within experience, not
without it, and within this experience could be found
the necessity and universality which had been located
in the world independent of experience. The melting
of these fixed Kantian categories came with the spring
floods of the romantic idealism that followed Kant.


The starting-point of this idealism was Kantian.
Within experience lay the object of knowledge. The
Idealist's principal undertaking was to overcome the
skepticism that attached to the object of knowledge
because of its reference to what lies outside itself. If,
as Kant had undertaken to prove, the reality which
knowledge implies must reach beyond experience, then,
on the Kantian doctrine that knowledge lies within
experience, knowledge itself is infected with skepticism.
Kant's practical bridge from the world of experience
to the world of things-in-themselves, which he walked\marginpage{195}
by faith and not by sight, was found in the postulates
of the conduct of the self as a moral being, as a personality.
The romantic idealists advance by the same
road, though as romanticists not critical philosophers,
they fashioned the world of reality, that transcends
experience, out of experience itself, by centering the
self in the absolute self and conceiving the whole infinite
universe as the experience of the absolute self.
The interesting phase of this development is that the
form which experience takes in becoming objective is
found in the nature and thought of the individual, and
that this process of epistemological experience becomes
thus a process of nature, if the objective is the natural.
In Kant's terms our minds give laws to nature. But
this nature constantly exhibits its dependence upon
underlying noumena that must therefore transcend the
laws given by the understanding. The Romanticist insists
that this other reality must be the same stuff as
that of experience, that in experience arise forms which
transcend those which bound the experience in its earlier
phase. If in experience the forms of the objective
world are themselves involved, the process of knowledge
sets no limits to itself, which it may not, does not, by
implication transcend. As further indication of the
shift by which thought had passed into possession of
the world of things in themselves stands the antinomy
which in Kantian experience marks the limit of our
knowledge while in post-Kantian idealism it becomes
the antithesis that leads to the synthesis upon the
higher plane. Contradiction marks the phase at which
the spirit becomes creative, not simply giving an empty\marginpage{196}
formal law to nature, but creating the concrete universe
in which content and form merge in true actuality.
The relation of the sensuous content to the
conceptual form is not dead, as in Kant's doctrine. It
is fused as perception into concept and carries its
immediacy and concreteness of detail into the concrete
universal as the complete organization of stimulation
and response pass into the flexible habit. And yet
in the Hegelian logic, the movement is always away
from the perceptual experience toward the higher realm
of the \emph{Idee}. Thought is creative in the movement, but
in its ultimate reality it transcends spatial and temporal
experience, the experience with which the natural
and mathematical sciences deal. Thought is not a
means of solving the problems of this world as they
arise, but a great process of realization in which this
world is forever transcended. Its abstract particularities
of sensuous detail belong only to the finite experience
of the partial self. This world is, therefore,
always incomplete in its reality and, in so far, always
untrue. Truth and full reality belong not to the field
of scientific investigation.


In its metaphysics Romantic Idealism, though it
finds a place for scientific discovery and reconstruction,
leaves these disdainfully behind, as incomplete phases
of the ultimate process of reality, as infected with untruth
and deceptive unwarranted claims. The world is
still too much with us. We recognize here three striking
results of the development of reflective consciousness
in the modern world:—first, it is assumed that the
objective world of knowledge can be placed within\marginpage{197}
the experience of the individual without losing thereby
its nature as an object, that all characters of that
object can be presented as belonging to that experience,
whether adequately or not is another question;
and second, it is assumed that the contradictions in
its nature which are associated with its inclusion in
individual experience, its references beyond itself when
so included, may themselves be the starting-point of
a reconstruction which at least carries that object
beyond the experience within which these contradictions
arose; and third, it is assumed that this growth
takes place in a world of reality within which the
incomplete experience of the individual is an essential
part of the process, in which it is not a mere fiction,
destroying reality by its representation, but is a growing-point
in that reality itself.


These characters of philosophic interpretation, the
inclusion of the object of knowledge in the individual
experience and the turning of the conflicts in that experience
into the occasion for the creation of new
objects transcending these contradictions, are the
characters in the conscious method, of modern science,
which most profoundly distinguish it from the method
of ancient science. This, of course, is tantamount to
saying that they are those which mark the experimental
method in science.


That phase of the method upon which I have touched
already has been its occupation with the so-called data
or facts as distinguished from Aristotelian individuals.


Whenever we reduce the objects of scientific investigation
to facts and undertake to record them as such,\marginpage{198}
they become events, happenings, whose hard factual
character lies in the circumstance that they have taken
place, and this quite independently of any explanation
of their taking place. When they are explained they
have ceased to be facts and have become instances of
a law, that is, Aristotelian individuals, embodied
theories, and their actuality as events is lost in the
necessity of their occurrence as expressions of the law;
with this change their particularity as events or happenings
disappears. They are but the specific values
of the equation when constants are substituted for
variables. Before the equation is known or the law
discovered they have no such ground of existence. Up
to this point they find their ground for existence in
their mere occurrence, to which the law which is to
explain them must accommodate itself.


There are here suggested two points of view from
which these facts may be regarded. Considered with
reference to a uniformity or law by which they will
be ordered and explained they are the phenomena with
which the positivist deals; as existencies to be identified
and localized before they are placed within such a uniformity
they fall within the domain of the psychological
philosopher who can at least place them in their relation
to the other events in the experience of the individual
who observes them. Considered as having
a residual meaning apart from the law to which they
have become exceptions, they can become the subject-matter
of the rationalist. It is important that we
recognize that neither the positivist nor the rationalist
is able to identify the nature of the fact or datum\marginpage{199}
to which they refer. I refer to such stubborn facts as
those of the sporadic appearance of infectious diseases
before the germ theory of the disease was discovered.
Here was a fact which contradicted the doctrine of the
spread of the infection by contact. It appeared not as
an instance of a law, but as an exception to a law.
As such, its nature is found in its having happened
at a given place and time. If the case had appeared
in the midst of an epidemic, its nature as a case of the
infectious disease would have been cared for in the
accepted doctrine, and for its acceptance as an object
of knowledge its location in space and time as an event
would not have been required. Its geographical and
historical traits would have followed from the theory
of the infection, as we identify by our calculations the
happy fulfilment of Thales' prophecy. The happening
of an instance of a law is accounted for by the law.
Its happening may and in most instances does escape
observation, while as an exception to an accepted law
it captures attention. Its nature as an event is, then,
found in its appearance in the experience of some individual,
whose observation is controlled and recorded
as his experience. Without its reference to this individual's
experience it could not appear as a fact
for further scientific consideration.


Now the attitude of the positivist toward this fact
is that induced by its relation to the law which is \emph{subsequently}
discovered. It has then fallen into place in
a series, and his doctrine is that all laws are but uniformities
of such events. He treats the fact when it
is an exception to law as an instance of the new law\marginpage{200}
and assumes that the exception to the old law and
the instance of the new are identical. And this
is a great mistake,—the mistake made also by the neo-realist
when he assumes that the object of knowledge
is the same within and without the mind, that nothing
happens to what is to be known when it by chance
strays into the realm of conscious cognition. Any as
yet unexplained exception to an old theory can happen
only in the experience of an individual, and that which
has its existence as an event in some one's biography
is a different thing from the future instance which is
not beholden to any one for its existence. Yet there
are, as I indicated earlier, meanings in this exceptional
event which, at least for the time, are unaffected by
the exceptional character of the occurrence. For example,
certain clinical symptoms by which an infectious
disease is identified have remained unchanged in diagnosis
since the days of Hippocrates. These characters
remain as characters of the instance of the law of germ-origin
when this law has been discovered. This may
lead us to say that the exception which appears for
the time being as a unique incident in a biography
is identical with the instance of a germ-induced disease.
Indeed, we are likely to go further and, in the assurance
of the new doctrine, state that former exceptions can
(or with adequate acquaintance with the facts could)
be proved to be necessarily an instance of a disease
carried by a germ. The positivist is therefore confident
that the field of scientific knowledge is made up
of events which are instances of uniform series, although
under conditions of inadequate information\marginpage{201}
some of them appear as exceptions to the statements
of uniformities, in truth the latter being no uniformities
at all.


That this is not a true statement of the nature of
the exception and of the instance, it is not difficult to
show if we are willing to accept the accounts which
the scientists themselves give of their own observation,
the changing forms which the hypothesis assumes during
the effort to reach a solution and the ultimate
reconstruction which attends the final tested solution.
Wherever we are fortunate enough, as in the biographies
of men such as Darwin and Pasteur, to follow
a number of the steps by which they recognized problems
and worked out tenable hypotheses for their
solution, we find that the direction which is given to
attention in the early stage of scientific investigation
is toward conflicts between current theories and observed
phenomena, and that since the form which these
observations take is determined by the opposition, it is
determined by a statement which itself is later abandoned.
We find that the scope and character of the
observations change at once when the investigator sets
about gathering as much of the material as he can
secure, and changes constantly as he formulates tentative
hypotheses for the solution of the problem, which,
moreover, generally changes its form during the investigation.
I am aware that this change in the form
of the data will be brushed aside by many as belonging
only to the attitude of mind of the investigator, while
it is assumed that the ``facts'' themselves, however
selected and organized in his observation and thought,\marginpage{202}
remain identical in their nature throughout. Indeed,
the scientist himself carries with him in the whole procedure
the confidence that the fact-structure of reality
is unchanged, however varied are the forms of the observations
which refer to the same entities.\footnote{An analysis which has been many times carried out has made it clear that scientific data never do more than approximate the laws and entities upon which our science rests. It is equally evident that the forms of these laws and entities themselves shift in the reconstructions of incessant research, or where they seem most secure could consistently be changed, or at least could be fundamentally different were our psychological structure or even our conventions of thought different. I need only refer to the \emph{Science et Hypoth\`{e}se} of Poincar\'{e} and the \emph{Problems of Science} of Enriques. The positivist who undertakes to carry the structure of the world back to the data of observation, and the uniformities appearing in the accepted hypotheses of growing sciences cannot maintain that we ever succeed in isolating data which must remain the same in the kaleidoscope of our research science; nor are we better served if we retreat to the ultimate elements of points and instants which our pure mathematics assumes and implicitly defines, and in connection with which it has worked out the modern theory of the number and continuous series, its statements of continuity and infinity.}


The analysis of the fact-structure of reality shows
in the first place that the scientist undertakes to form
such an hypothesis that all the data of observation
will find their place in the objective world, and in the
second place to bring them into such a structure that
future experience will lead to anticipated results. He
does not undertake to preserve facts in the form in
which they existed in experience before the problem
arose nor to construct a world independent of experience
or that will not be subject itself to future reconstructions
in experience. He merely insists that future
\marginpage{203}
reconstructions will take into account the old in re-adjusting
it to the new. In such a process it is evident
that the change of the form in the data is not due to
a subjective attitude of the investigator which can be
abstracted from the facts. When Darwin, for instance,
found that the marl dressings which farmers
spread over their soil did not sink through the soil
by the force of gravity as was supposed, but that the
earthworm castings were thrown up above these dressings
at nearly the same rate at which they disappeared,
he did not correct a subjective attitude of mind. He
created in experience a humus which took the place
of a former soil, and justified itself by fitting it into
the whole process of disintegration of the earth's surface.
It would be impossible to separate in the earlier
experiences certain facts and certain attitudes of mind
entertained by men with reference to these facts. Certain
objects have replaced other objects. It is only after
the process of analysis, which arose out of the conflicting
observations, has broken up the old object that what
was a part of the object, heavier-things-pushing-their
way-through-soil-of-lighter-texture, can become a mere
idea. Earlier it was an object. Until it could be
tested the earthworm as the cause of the disappearance
of the dressings was also Darwin's idea. It became fact.
For science at least it is quite impossible to distinguish
between what in an object must be fact and what may
be idea. The distinction when it is made is dependent
upon the form of the problem and is functional to its
solution, not metaphysical. So little can a consistent
line of cleavage between facts and ideas be indicated,\marginpage{204}
that we can never tell where in our world of observation
the problem of science will arise, or what will be regarded
as structure of reality or what erroneous
idea.


There is a strong temptation to lodge these supposititious
fact-structures in a world of conceptual objects,
molecules, atoms, electrons, and the like. For these at
least lie beyond the range of perception by their very
definition. They seem to be in a realm of things-in-themselves.
Yet they also are found now in the field
of fact and now in that of ideas. Furthermore, a study
of their structure as they exist in the world of constructive
science shows that their infra-sensible character
is due simply to the nature of our sense-processes,
not to a different metaphysical nature. They occupy
space, have measurable dimensions, mass, and are subject
to the same laws of motion as are sensible objects.
We even bring them indirectly into the field of vision
and photograph their paths of motion.


The ultimate elements referred to above provide a
consistent symbolism for the finding and formulating of
applied mathematical sciences, within which lies the whole
field of physics, including Euclidian geometry as well.
However, they have succeeded in providing nothing
more than a language and logic pruned of the obstinate
contradictions, inaccuracies, and unanalyzed sensuous
stuff of earlier mathematical science. Such a rationalistic
doctrine can never present in an unchanged form
the objects with which natural science deals in any of
the stages of its investigation. It can deal only with
ultimate elements and forms of propositions. It is\marginpage{205}
compelled to fall back on a theory of analysis
which reaches ultimate elements and an assumption of
inference as an indefinable. Such an analysis is actually
impossible either in the field of the conceptual objects
into which physical science reduces physical objects, or
in the field of sensuous experience. Atoms can be reduced
into positive and negative electrical elements and
these may, perhaps do, imply a structure of ether that
again invites further analysis and so on ad infinitum.
None of the hypothetical constructs carry with themselves
the character of being ultimate elements unless
they are purely metaphysical. If they are fashioned
to meet the actual problems of scientific research they
will admit of possible further analysis, because they
must be located and defined in the continuity of space
and time. They cannot \emph{be} the points and instants of
modern mathematical theory. Nor can we reach ultimate
elements in sensuous experience, for this lies
also within a continuum. Furthermore, our scientific
analyses are dependent upon the form that our objects
assume. There is no general analysis which research
in science has ever used. The assumption that psychology
provides us with an analysis of experience
which can be carried to ultimate elements or facts, and
which thereby provides the elements out of which the
objects of our physical world must be constructed,
denies to psychology its rights as a natural science of
which it is so jealous, turning it into a Berkeleyan
metaphysics.


This most modern form of rationalism being unable
to find ultimate elements in the field of actual science\marginpage{206}
is compelled to take what it can find there. Now the
results of the analysis of the classical English psychological
school give the impression of being what Mr.
Russell calls ``hard facts,'' i.e., facts which cannot be
broken up into others. They seem to be the data of
experience. Moreover, the term hard is not so uncompromising
as is the term element. A fact can be more
or less hard, while an ultimate element cannot be more
or less ultimate. Furthermore, the entirely formal
character of the logic enables it to deal with equal
facility with any content. One can operate with the
more or less hard sense-data, putting them in to satisfy
the seeming variables of the propositions, and reach
conclusions which are formally correct. There is no
necessity for scrutinizing the data under these circumstances,
if one can only assume that the data are those
which science is actually using. The difficulty is that
no scientist ever analyzed his objects into such sense-data.
They exist only in philosophical text-books.
Even the psychologists recognize that these sensations
are abstractions which are not the elements out of which
objects of sense are constructed. They are abstractions
made from those objects whose ground for isolation
is found in the peculiar problems of experimental
psychology, such as those of color or tone perception.
It would be impossible to make anything in terms of
Berkeleyan sense-data and of symbolic logic out of any
scientific discovery. Research defines its problem by
isolating certain facts which appear for the time being
not as the sense-data of a solipsistic mind, but as experiences
of an individual in a highly organized society,\marginpage{207}
facts which, because they are in conflict with accepted
doctrines, must be described so that they can be
experienced by others under like conditions. The
ground for the analysis which leads to such facts is
found in the conflict between the accepted theory and
the experience of the individual scientist. The analysis
is strictly \emph{ad hoc}. As far as possible the exception is
stated in terms of accepted meanings. Only where the
meaning is in contradiction with the experience does
the fact appear as the happening to an individual and
become a paragraph out of his biography. But as such
an event, whose existence for science depends upon the
acceptance of the description of him to whom it has
happened, it must have all the setting of circumstantial
evidence. Part of this circumstantial evidence is found
in so-called scientific control, that is, the evidence that
conditions were such that similar experiences could
happen to others and could be described as they are
described in the account given. Other parts of this
evidence which we call corroborative are found in the
statements of others which bear out details of this
peculiar event, though it is important to note that
these details have to be wrenched from their settings
to give this corroborative value. To be most conclusive
they must have no intentional connection with
the experience of the scientist. In other words,
those individuals who corroborate the facts are made,
in spite of themselves, experiencers of the same facts.
The perfection of this evidence is attained when the
fact can happen to others and the observer simply
details the conditions under which he made the observation,\marginpage{208}
which can be then so perfectly reproduced that
others may repeat the exceptional experience.


This process is not an analysis of a known world
into ultimate elements and their relations. Such an
analysis never isolates this particular exception which
constitutes the scientific problems as an individual experience.
The extent to which the analysis is carried
depends upon the exigencies of the problem. It is the
indefinite variety of the problems which accounts for
the indefinite variety of the facts. What constitutes
them facts in the sense in which we are using the
term is their \emph{exceptional} nature; formally they appear
as particular judgments, being denials of universal
judgments, whether positive or negative. This exceptional
nature robs the events of a reality which would
have belonged to them as instances of a universal law.
It leaves them, however, with the rest of their meaning.
But the value which they have lost is just that which
was essential to give them their place in the world as
it has existed for thought. Banished from that universally
valid structure, their ground for existence is
found in the experience of the puzzled observer. Such
an observation was that of the moons of Jupiter made
possible by the primitive telescope of Galileo. For
those who lived in a Ptolemaic cosmos, these could have
existence only as observations of individuals. As moons
they had distinct meaning, circling Jupiter as our
moon circles the earth, but being in contradiction with
the Ptolemaic order they could depend for their existence
only on the evidence of the senses, until a Copernican
order could give them a local habitation and a\marginpage{209}
name. Then they were observed not as the experiences
of individuals but as instances of planetary order in
a heliocentric system. It would be palpably absurd to
refer to them as mere sense-data, mere sensations. They
are for the time being inexplicable experiences of certain
individuals. They are inexplicable because they
have a meaning which is at variance with the structure
of the whole world to which they belong. They are the
phenomena termed accidental by Aristotle and rejected
as full realities by him, but which have become, in the
habitat of individual experience, the headstone of the
structure of modern research of science.


A rationalism which relegates implication to the
indefinables cannot present the process of modern
science. Implication is exactly that process by which
these events pass from their individual existence into
that of universal reality, and the scientist is at pains
to define it as the experimental method. It is true that
a proposition implies implication. But the proposition
is the statement of the result of the process by which
an object has arisen for knowledge and merely indicates
the structure of the object. In discovery, invention,
and research the escape from the exceptional, from the
data of early stages of observation, is by way of an
hypothesis; and every hypothesis so far as it is tenable
and workable in its form is universal. No one would
waste his time with a hypothesis which confessedly was
not applicable to all instances of the problem. An
hypothesis may be again and again abandoned, it may
prove to be faulty and contradictory, but in so far as
it is an instrument of research it is assumed to be\marginpage{210}
universal and to perfect a system which has broken
down at the point indicated by the problem. Implication
and more elaborated instances flow from the
structure of this hypothesis. The classical illustration
which stands at the door of modern experimental
science is the hypothesis which Galileo formed of the
rate of the velocity of a falling body. He conceived
that this was in proportion to the time elapsed during
the fall and then elaborated the consequences of this
hypothesis by working it into the accepted mathematical
doctrines of the physical world, until it led to an
anticipated result which would be actually secured and
which would be so characteristic an instance of a falling
body that it would answer to every other instance
as he had defined them. In this fashion he defined his
inference as the anticipation of a result because this
result was a part of the world as he presented it
amended by his hypothesis. It is true that back of the
specific implication of this result lay a mass of other
implications, many not even presented specifically in
thought and many others presented by symbols which
generalized innumerable instances. These implications
are for the scientist more or less implicit meanings, but
they are meanings each of which may be brought into
question and tested in the same fashion if it should become
an actual problem. Many of them which would
not have occurred to Galileo as possible problems have
been questioned since his day. What has remained after
this period of determined questioning of the foundations
of mathematics and the structure of the world
of physical science is a method of agreement with one\marginpage{211}self
and others, in (a) the identification of the object
of thought, in (b) the accepted values of assent and
denial called truth and falsehood, and in (c) referring
to meaning, in its relation to what is meant. In any
case the achievement of symbolic logic, with its indefinables
and axioms has been to reduce this logic to
a statement of the most generalized form of possible
consistent thought intercourse, with entire abstraction
from the content of the object to which it refers.
If, however, we abstract from its value in giving
a consistent theory of number, continuity, and infinity,
this complete abstraction from the content has
carried the conditions of thinking in agreement with
self and others so far away from the actual problem
of science that symbolic logic has never been used as
a research method. It has indeed emphasized the fact
that thinking deals with problems which have reference
to uses to which it can be put, not to a metaphysical
world lying beyond experience. Symbolic logic has to
do with the world of discourse, not with the world of
things.


What Russell pushes to one side as a happy guess
is the actual process of implication by which, for example,
the minute form in the diseased human system is
identified with unicellular life and the history of the
disease with the life history of this form. This identification
implies reclassification of these forms and a
treatment of the disease that answers to their life history.
Having made this identification we anticipate the
result of this treatment, calling it an inference.


Implication belongs to the reconstruction of the object.\marginpage{212}
As long as no question has arisen, the object
is what it means or means what it is. It does not
imply any feature of itself. When through conflict
with the experience of the individual some feature of
the object is divorced from some meaning the relationship
between these becomes a false implication. When
a hypothetically reconstructed object finds us anticipating
a result which accords with the nature of such
objects we assert an implication of this meaning. To
carry this relation of implication back into objects
which are subject to no criticism or question would of
course resolve the world into elements connected by
external relations, with the added consequence that
these elements can have no content, since every content
in the face of such an analysis must be subject
to further analysis. We reach inevitably symbols such
as X, Y, and Z, which can symbolize nothing. Theoretically
we can assume an implication between any
elements of an object, but in this abstract assumption
the symbolic logician overlooks the fact that he is also
assuming some content which is not analyzed and which
is the ground of the implication. In other words this
logician confuses the scientific attitude of being ready
to question anything with an attitude of being willing
to question everything at once. It is only in an unquestioned
objective world that the exceptional instance
appears and it is only in such a world that an
experimental science tests the implications of the hypothetically
reconstructed object.


The guess is happy because it carries with it the consequences
which follow from its fitting into the world,\marginpage{213}
and the guess, in other words the hypothesis, takes on
this happy form solely because of the material reconstruction
which by its nature removes the unhappy
contradiction and promises the successful carrying out
of the conflicting attitudes in the new objective world.
There is no such thing as formal implication.


Where no reconstruction of the world is involved in
our identification of objects that belong to it and
where, therefore, no readjustment of conduct is demanded,
such a logic symbolizes what takes place in
our direct recognition of objects and our response
to them. Then ``X is a man implies X is mortal for
all values of X'' exactly symbolizes the attitude toward
a man subject to a disease supposedly mortal. But
it fails to symbolize the biological research which starting
with inexplicable sporadic cases of an infectious
disease carries over from the study of the life history
of infusoria a hypothetical reconstruction of the history
of disease and then acts upon the result of this
assumption. Research-science presents a world whose
form is always universal, but this universal form is
neither a metaphysical assumption nor a fixed form of
the understanding. While the scientist may as a metaphysician
assume the existence of realities which lie
beyond a possible experience, or be a Kantian or Neo-Kantian,
neither of these attitudes is necessary for his
research. He may be a positivist—a disciple of Hume
or of John Stuart Mill. He may be a pluralist who
conceives, with William James, that the order which
we detect in parts of the universe is possibly one that
is rising out of the chaos and which may never be as\marginpage{214}
universal as our hypothesis demands. None of these
attitudes has any bearing upon his scientific method.
This simplifies his thinking, enables him to identify the
object in which he is interested wherever he finds it, and
to abstract in the world as he conceives it those features
which carry with them the occurrence he is endeavoring
to place. Especially it enables him to make his thought
a part of the socially accepted and socially organized
science to which his thought belongs. He is far too
modest to demand that the world be as his inference
demands.


He asks that his view of the world be cogent and
convincing to all those whose thinking has made his
own possible, and be an acceptable premise for the
conduct of that society to which he belongs. The
hypothesis has no universal and necessary characters
except those that belong to the thought which preserves
the same meanings to the same objects, the same relations
between the same relata, the same attributes of
assent and dissent under the same conditions, the same
results of the same combinations of the same things.
For scientific research the meanings, the relations with
the relata, the assent and dissent, the combinations
and the things combined are all in the world of experience.
Thinking in its abstractions and identifications
and reconstructions undertakes to preserve the values
that it finds, and the necessity of its thinking lies in its
ability to so identify, preserve, and combine what it
has isolated that the thought structure will have an
identical import under like conditions for the thinker
with all other thinkers to whom these instruments of\marginpage{215}
research conduct are addressed. Whatever conclusions
the scientist draws as necessary and universal results
from his hypothesis for a world independent of his
thought are due, not to the cogency of his logic, but
to other considerations. For he knows if he reflects
that another problem may arise which will in its solution
change the face of the world built upon the present
hypothesis. He will defend the inexorableness of his
reasoning, but the premises may change. Even the
contents of tridimensional space and sensuous time are
not essential to the cogency of that reasoning nor can
the unbroken web of the argument assure the content
of the world as invariable. His universals, when applied
to nature, are all hypothetical universals; hence
the import of experiment as the test of an hypothesis.
Experience does not rule out the possible cropping up of
a new problem which may shift the values attained.
Experience simply reveals that the new hypothesis fits
into the meanings of the world which are not shaken;
it shows that, with the reconstruction which the hypothesis
offers, it is possible for scientific conduct to
proceed.


But if the universal character of the hypothesis and
the tested theory belong to the instrumental character
of thought in so reconstructing a world that has proved
to be imperfect, and inadequate to conduct, the stuff
of the world and of the new hypothesis are the same.
At least this is true for the scientist who has no interest
in an epistemological problem that does not
affect his scientific undertakings in one way nor another.
I have already pointed out that from the standpoint\marginpage{216}
of logical and psychological analysis the things
with which science deals can be neither ultimate elements
nor sense-data; but that they must be phases and
characters and parts of things in some whole, parts
which can only be isolated because of the conflict between
an accepted meaning and some experience. I have
pointed out that an analysis is guided by the practical
demands of a solution of this conflict; that even that
which is individual in its most unique sense in the conflict
and in attempts at its solution does not enter into
the field of psychology—which has its own problems
peculiar to its science. Certain psychological problems
belong to the problems of other sciences, as, for example,
that of the personal equation belongs to astronomy
or that of color vision to the theory of light. But they
bulk small in these sciences. It cannot be successfully
maintained that a scientific observation of the most
unique sort, one which is accepted for the time being
simply as a happening in this or that scientist's experience,
is as such a psychological datum, for the data in
psychological text-books have reference to \emph{psychological}
problems. Psychology deals with the consciousness
of the individual in its dependence upon the physiological
organism and upon those contents which detach
themselves from the objects outside the individual and
which are identified with his inner experience. It deals
with the laws and processes and structures of this consciousness
in all its experiences, not with \emph{exceptional}
experiences. It is necessary to emphasize again that
for science these particular experiences arise within a
world which is in its logical structure organized and\marginpage{217}
universal. They arise only through the conflict of the
individual's experience with such an accepted structure.
For science individual experience \emph{presupposes} the organized
structure; hence it cannot provide the material
out of which the structure is built up. This is the error
of both the positivist and of the psychological philosopher,
if scientific procedure gives us in any sense a
picture of the situation.


A sharp contrast appears between the accepted hypothesis
with its universal form and the experiences
which invalidate the earlier theory. The reality of
these experiences lies in their happening. They were
unpredictable. They are not instances of a law. The
later theory, the one which explains these occurrences,
changes their character and status, making them necessary
results of the world as that is conceived under
this new doctrine. This new standpoint carries with it
a backward view, which explains the erroneous doctrine,
and accounts for the observations which invalidated it.
Every new theory must take up into itself earlier
doctrines and rationalize the earlier exceptions.
A generalization of this attitude places the scientist
in the position of anticipating later reconstructions.
He then must conceive of his world as subject to
continuous reconstructions. A familiar interpretation
of his attitude is that the hypothesis is thus
approaching nearer and nearer toward a reality which
would never change if it could be attained, or, from the
standpoint of the Hegelian toward a goal at infinity.
The Hegelian also undertakes to make this continuous
process of reconstruction an organic phase in reality\marginpage{218}
and to identify with nature the process of finding exceptions
and of correcting them. The fundamental
difference between this position and that of the scientist
who looks before and after is that the Hegelian undertakes
to make the exception in its exceptional character
a part of the reality which transcends it, while the
scientist usually relegates the exception to the experience
of individuals who were simply caught in an error
which later investigation removes.


The error remains as an historical incident explicable
perhaps as a result of the conditions under which it
occurred, but in so far as it was an error, not a part
of reality. It is customary to speak of it as subjective,
though this implies that we are putting the man who
was unwittingly in error into the position of the one
who has corrected it. To entertain that error in the
face of its correction would be subjective. A result
of this interpretation is that the theories are abstracted
from the world and regarded as something outside it.
It is assumed that the theories are mental or subjective
and change while the facts remain unchanged. Even
when it is assumed that theories and facts agree, men
speak of a correspondence or parallelism between idea
and the reality to which it refers. While this attitude
seems to be that of science toward the disproved theories
which lie behind it, it is not its attitude to the
theories which it accepts. These are not regarded as
merely parallel to realities, as abstracted from the
structure of things. These meanings go into the makeup
of the world. It is true that the scientist who looks
before and after realizes that any specific meaning\marginpage{219}
which is now accepted may be questioned and discarded.
If he carries his refection far enough he sees that a complete
elimination of all the meanings which might conceivably
be so discredited would leave nothing but logical
constants, a world with no facts in any sense. In this
position he may of course take an agnostic attitude
and be satisfied with the attitude of Hume or Mill or
Russell. But if he does so, he will pass into the camp
of the psychological philosophers and will have left
the position of the scientist. The scientist always deals
with an \emph{actual} problem, and even when he looks before
and after he does so in so far as he is facing in inquiry
some actual problem. No actual problem could
conceivably take on the form of a conflict involving
the whole world of meaning. The conflict always arises
between an individual experience and certain laws, certain
meanings while others are unaffected. These others
form the necessary field without which no conflict can
arise. They give the man of research his [place to stand]\footnote{Mead has the Greek ``(${\pi}o{\upsilon}~{\sigma}{\tau}{\omega}$)'', which is a bit of the quote from Archimedes: \emph{Give me a place to stand, and I can move the world}.}
%(που στω)
upon which he can formulate his problem and undertake
its solution. The possible calling in question of
any content, whatever it may be, means always that
there is left a field of unquestioned reality. The attitude
of the scientist never contemplates or could contemplate
the possibility of a world in which there
would be no reality by which to test his hypothetical
solution of the problem that arises. Nor does this
attitude when applied to past discarded theories necessarily
carry with it the implication that these older
theories were subjective ideas in men's minds, while
the reality lay beside and beyond them unmingled with\marginpage{220}
ideas. It always finds a standpoint from which these
ideas in the earlier situation are still recognized as
reliable, for there are no scientific data without meanings.
There could be no history of science on any other
basis. No history of science goes back to ultimate elements
or sense-data, or to any combination of bare data
on one hand and logical elements on the other. The
world of the scientist is always there as one in which
reconstruction is taking place with continual shifting
of problems, but as a real world within which the problems
arise. The errors of the past and present appear
as untenable hypotheses which could not bear the test
of experiment if the experience were sufficiently enlarged
and interpreted. But they are not mere errors
to be thrown into the scrap heap. They become a part
of a different phase of reality which a fuller history
of the past records or a fuller account of the present
interprets, giving them thereby their proper place in
a real world.\footnote{In other words, science assumes that every error is \emph{ex post facto} explicable as a function of the real conditions under which it really arose. Hence, ``consciousness,'' set over against Reality, was not its condition.}


The completion of this program, however, awaits the
solution of the scientific problem of the relation of the
psychical and the physical with the attendant problem
of the meaning of the so-called origin of consciousness
in the history of the world. My own feeling is that
these problems must be attacked from the standpoint
of the social nature of so-called consciousness. The
clear indications of this I find in the reference of our
logical constants to the structure of thought as a
\marginpage{221}
means of communication, in the explanation of errors
in the history of science by their social determination,
and in the interpretation of the inner field of experience
as the importation of social intercourse into the
conscious conduct of the individual. But whatever
may be the solution of these problems, it must
carry with it such a treatment of the experience of the
individual that the latter will never be regarded merely
as a subjective state, however inadequate it may have
proved itself as a scientific hypothesis. This seems
to me to be involved in the conception of psychology
as a natural science and in any legitimate carrying
out of the Hegelian program of giving reality and
creative import to individual experience. The experience
of the individual in its exceptional character is
the growing-point of science, first of all in the recognition
of data upon which the older theories break, and
second in the hypothesis which arises in the individual
and is tested by the experiment which reconstructs the
world. A scientific history and a scientific psychology
from which epistemology has been banished must place
these observations and hypotheses together with erroneous
conceptions and mistaken observations \emph{within} the
real world in such a fashion that their reference to the
experience of the individual and to the world to which
he belongs will be comprehensible. As I have indicated,
the scientific theory of the physical and conscious individual
in the world implied in this problem has still
to be adequately developed. But there is implied in
the conception of such a theory such a location of the
process of thought in the process of reality as will\marginpage{222}
give it an import both in the meaning of things and
in the individual's thinking. We have the beginning of
such a doctrine in the conception of a functional value
of consciousness in the conduct of living forms, and the
development of reflective thought out of such a consciousness
which puts it within the act and gives it the
function of preparation where adjustment is necessary.
Such a process creates the situation with reference to
which the form acts. In all adjustment or adaptation
the result is that the form which is adjusted finds that
by its adjustment it has created an environment. The
ancients by their formulation of the Ptolemaic theory
committed themselves to the world in which the fixed
values of the heavenly over against the earthly obtained.
Such a world was the interpretation of the experience
involved in their physical and social attitudes. They
could not accept the hypothesis of Aristarchus because
it conflicted with the world which they had created, with
the values which were determining values for them. The
same was true of the hypothesis of Democritus. They
could not, as they conceived the physical world, accept
its purely quantitative character. The conception of a
disinterested truth which we have cherished since the
Middle Ages is itself a value that has a social basis as
really as had the dogma of the church. The earliest
statement of it was perhaps that of Francis Bacon.
Freeing investigation from the church dogma and its
attendant logic meant to him the freedom to find in
nature what men needed and could use for the amelioration
of their social and physical condition. The full
implication of the doctrine has been recognized as that\marginpage{223}
of freedom, freedom to effect not only values already
recognized, but freedom to attain as well such complete
acquaintance with nature that new and unrecognized
uses would be at our disposal; that is, that progress
should be one toward any possible use to which increased
knowledge might lead. The cult of increasing
knowledge, of continually reconstructing the world, took
the place both of the ancient conception of adequately
organizing the world as presented in thought,
and of the medieval conception of a systematic formulation
on the basis of the statement in church dogma of social
values. This modern conception proceeds from the
standpoint not of formulating values, but giving society
at the moment the largest possible number of alternatives
of conduct, i.e., undertaking to fix from moment
to moment the widest possible field of conduct. The purposes
of conduct are to be determined in the presence
of a field of alternative possibilities of action. The ends
of conduct are not to be determined in advance, but in
view of the interests that fuller knowledge of conditions
awaken. So there appears a conception of determining
the field that shall be quite independent of given
values. A real world which consists not of an unchanged
universe, but of a universe which may be continually
readjusted according to the problems arising
in the consciousness of the individuals within society.
The seemingly fixed character of such a world is found
in the generally fixed conditions which underlie the
type of problems which we find. We determine the
important conditions incident to the working out of
the great problems which face us. Our conception of a\marginpage{224}
given universe is formed in the effort to mobilize all
the material about us in relation to these problems—the
structure of the self, the structure of matter, the
physical process of life, the laws of change and the
interrelation of changes. With reference to these
problems certain conditions appear fixed and become
the statement of the world by which we must determine
by experimental test the viability of our hypotheses.
There arises then the conception of a world
which is unquestioned over against any particular
problem. While our science continually changes that
world, at least it must be always realized as there.
On the other hand, these conceptions are after all
relative to the ends of social conduct which may
be formulated in the presence of any freedom of
action.


We postulate freedom of action as the condition of
formulating the ends toward which our conduct shall
be directed. Ancient thought assured itself of its ends
of conduct and allowed these to determine the world
which tested its hypothesis. We insist such ends may
not be formulated until we know the field of possible
action. The formulation of the ends is essentially a
social undertaking and seems to follow the statement
of the field of possible conduct, while in fact the statement
of the possible field of conduct is actually dependent
on the push toward action. A moving end
which is continually reconstructing itself follows upon
the continually enlarging field of opportunities of
conduct.


The conception of a world of existence, then, is the\marginpage{225}
result of the determination at the moment of the conditions
of the solution of the given problems. These
problems constitute the conditions of conduct, and the
ends of conduct can only be determined as we realize
the possibilities which changing conditions carry with
them. Our world of reality thus becomes independent
of any special ends or purposes and we reach an entirely
disinterested knowledge. And yet the value and
import of this knowledge is found in our conduct and
in our continually changing conditions. Knowledge
for its own sake is the slogan of freedom, for it alone
makes possible the continual reconstruction and enlargement
of the ends of conduct.


The individual in his experiences is continually creating
a world which becomes real through his discovery.
In so far as new conduct arises under the
conditions made possible by his experience and his hypothesis
the world, which may be made the test of
reality, has been modified and enlarged.


I have endeavored to present the world which is an
implication of the scientific method of discovery with
entire abstraction from any epistemological or metaphysical
presuppositions or complications. Scientific
method is indifferent to a world of things-in-themselves,
or to the previous condition of philosophic servitude of
those to whom its teachings are addressed. It is a
method not of knowing the unchangeable but of determining
the form of the world within which we live as
it changes from moment to moment. It undertakes to
tell us what we may expect to happen when we act in
such or such a fashion. It has become a matter of\marginpage{226}
serious consideration for a philosophy which is interested
in a world of things-in-themselves, and the epistemological
problem. For the cherished structures of
the metaphysical world, having ceased to house the
values of mankind, provide good working materials in
the hypothetical structures of science, on condition of
surrendering their metaphysical reality; and the epistemological
problem, having seemingly died of inanition,
has been found to be at bottom a problem of method
or logic. My attempt has been to present what seems
to me to be two capital instances of these transformations.
Science always has a world of reality by which
to test its hypotheses, but this world is not a world
independent of scientific experience, but the immediate
world surrounding us within which we must act. Our
next action may find these conditions seriously changed,
and then science will formulate this world so that in
view of this problem we may logically construct our
next plan of action. The plan of action should be
made self-consistent and universal in its form, not that
we may thus approach nearer to a self-consistent and
universal reality which is independent of our conduct,
but because our plan of action needs to be intelligent
and generally applicable. Again science advances by
the experiences of individuals, experiences which are
different from the world in which they have arisen
and which refer to a world which is not yet in existence,
so far as scientific experience is concerned. But this
relation to the old and new is not that of a subjective
world to an objective universe, but is a process of
logical reconstruction by which out of exceptions the\marginpage{227}
new law arises to replace a structure that has become
inadequate.


In both of these processes, that of determining the
structure of experience which will test by experiment
the legitimacy of the new hypothesis, and that of formulating
the problem and the hypothesis for its solution,
the individual functions in his full particularity,
and yet in organic relationship with the society that
is responsible for him. It is the import for scientific
method of this relationship that promises most for the
interpretation of the philosophic problems involved.
\end{document}