\documentclass{article}
% changes font to TeX Gyre Schola (Century Schoolbook)
\usepackage{tgschola}\usepackage[T1]{fontenc}

\newcommand{\spacer}{\medskip\centerline{* * *}\medskip}


\begin{document}

\paragraph{`Pragmatism'} in \emph{Dictionary of Philosophy and Psychology}, v. II 1902. James Mark Baldwin, ed. New York: the Macmillan Co. pp. 321--323

\spacer

\paragraph{About this text:} This is transcribed from scans of the original by P.D. Magnus, who also wrote this introductory bit.

The multivolume dictionary says in the front matter that it was ``written by many hands and edited by James Mark Baldwin.'' Baldwin seems most often to have taken passages as they were written and edited them by adding context rather than by rewriting them.

This entry includes contributions both from Charles Sanders Peirce and from William James, juxtaposed in a way that highlights their disagreements. Baldwin writes the final part of the entry himself, trying to pull it all together somehow.

The sections from Peirce conclude `C.S.P.', from James `W.J.', and from Baldwin `J.M.B.'

The first part of the entry, included for the sake of completeness, is by James Seth of Edinburgh (`J.S.')


\spacer






\marginpar{321}

\textbf{Pragmatic} (1) and (2) \textbf{Pragmatism} [Gr. $\pi\rho\alpha\gamma\mu\alpha\gamma\iota\kappa$\'{o}$\varsigma$, versed in affairs]: Ger. \textit{pragmatisch, Pragmatismus}; Fr. \textit{pragmatique, pragmatisme}; Ital. \textit{prammatico, prammatismo}. (1) This term is applied by Kant to the species of hypothetical imperative which he otherwise denominates `counsel of prudence,' and characterizes as `assertorial,' those, namely which prescribe the means necessary to the attainment of happiness, an end which we may postulate for all sentient beings (\textit{Grundlegung z. Met. d. Sitten}, ed. Rosenkranz, 42; Eng. trans., Abbot, 34). \hfill (J.S.)

\textit{Pragmatic anthropology}, according to Kant, is a practical ethics.

\textit{Pragmatic horizon} is the adaptation of our general knowledge to influencing our morals.


(2) The opinion that metaphysics is to be largely cleared up by the application of the following maxim for attaining clearness of apprehension: `Consider what effects, that might conceivably have practical bearings, we conceive the object of our conception to have. Then, our conception of these effects is the whole of our conception of the object.' \hfill (C.S.P.)

The doctrine that the whole `meaning' of a conception expresses itself in the practical consequences, consequences either in the shape of conduct to be recommended, or in that of experiences to be expected, if the conception be true; which consequences would be different if it were untrue, and must be different from the consequences by which the meaning of other conceptions is in turn expressed. If a second conception should not appear to have other consequences, then it must really be only the first conception under a different name. In methodology it is certain that to trace and compare their respective consequences is an admirable way of establishing the different meanings of different conceptions. \hfill (W.J.)

\marginpar{322}

This maxim was first proposed by C.S. Peirce in the \textit{Popular Science Monthly} for January, 1878 (xii. 287); and he explained how it was to be applied to the doctrine of reality. The writer was led to the maxim by reflection upon Kant's \textit{Critic of Pure Reason}. Substantially the same way dealing with ontology seems to have been practised by the Stoics. The writer subsequently saw that the principle might easily be misapplied, so as to sweep away the whole doctrine of incommensurables, and, in fact, the whole Weierstrassian way of regarding the calculus. In 1896 William James published his \textit{Will to Believe}, and later his \textit{Philos. Conceptions and Pract. Results}, which pushed this method to such extremes as must tend to give us pause. The doctrine appears to assume that the end of man is action---a stoical axiom which, to the present writer at the age of sixty, does not recommend itself so forcibly as it did at thirty. If it be admitted, on the contrary, that action wants an end, and that that end must be something of a general description, then the spirit of the maxim itself, which is that we must look to the upshot of our concepts in order rightly to apprehend them, would direct us towards something different from practical facts, namely, to general ideas, as the true interpreters of our thought. Nevertheless, the maxim has approved itself to the writer, after many years of trial, as of great utility in leading to a relatively high grade of clearness of thought. He would venture to suggest that it should always be put into practice with conscientious thoroughness, but that, when that has been done, and not before, a still higher grade of clearness of thought can be attained by remembering that the only ultimate good which the practical facts to which it directs attention can subserve is to further the development of concrete reasonableness; so that the meaning of the concept does not lie in any individual reactions at all, but in the manner in which those reactions contribute to that development. Indeed in the article of 1878, above referred to, the writer practiced better than he preached; for he applied the stoical maxim most unstoically, in such a sense as to insist upon the reality of the objects of general ideas in their generality.

A widely current opinion during the last quarter of a century has been that reasonableness is not a good in itself, but only for the sake of something else. Whether it be so or not seems to be a synthetical question, not to be settled by an appeal to the principle of contradiction---as if a reason for reasonableness were absurd. Almost everybody will now agree that the ultimate good lies in the evolutionary process in some way. If so, it is not in the individual reactions in their segregation, but in something general or continuous. Synechism is founded on the notion that the coalescence, the becoming continuous, the becoming governed by laws, the becoming instinct with general ideas, are but phases of one and the same process of the growth of reasonableness. This is first shown to be true with mathematical exactitude in the field of logic, and thence inferred to hold good metaphysically. It is not opposed to pragmatism in the manner in which C.S. Peirce applied it, but includes that procedure as a step. \hfill (C.S.P.)

It is of course legitimate to demand a reason for reasonableness; to do so is only to ask why we think --- a question to which a genetic answer would seem to be afforded by certain forms of pragmatism. We may say (cf. \textsc{Selection}, in psychology) that reasonableness, or truth, is due to practical adjustments, and that the system of truths is developed by the selection of concrete relationships which `work.' But it is quite another thing to make this genetic account of the origin and selection of `truth' a philosophy of reality. For just the general or universal meaning of the system as a whole, the purpose or function which the concrete items selected as workable subserves, and the environment or real world in which the entire movement takes place --- all these are by definition outside the sphere of operation of pragmatism. Pragmatism is really an attempt to construe all reality `retrospectively' --- as adequately given in the system of concrete practically derived truths --- i.e. as empiricial `science'; and while nominalism may invoke it, it still remains to prove nominalism. Cf. what is said under \textsc{Origin} \emph{versus} \textsc{Nature}. In the words of Peirce (comment on this article): `Nominalism, up to that of Hegel, looks at reality retrospectively. What all modern philosophy does is deny that there is any \emph{esse in futuro}.' Urban (\textit{Psychol. Rev.}, July, 1897) holds that while the concrete details of empirical knowledge may be due to `utility selection'---as practical `workables'---yet the structural principles of thought cannot be so accounted for. They have no application \textit{as generals}, and so would
\marginpar{323}
have to the pragmatist no adequate `reason for being.'

The definition by W.J. above, however, seems, by including `experiences to be expected,' to broaden the application of the principle. 

\textit{Literature:} besides the works of \textsc{Peirce} and \textsc{James}, as cited, see \textsc{Caldwell}, Pragmatism, in Mind, Oct., 1900; \textsc{Miller}, Philos. Rev., viii. (1899) 166; cf. \textsc{Clifford}, Lect. and Essays (1886), 85 ff.; also the literature of \textsc{Selective Thinking}. \hfill(J.M.B.)

\end{document}