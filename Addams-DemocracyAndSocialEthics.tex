\documentclass[]{article}
\usepackage{hyperref,setspace,graphicx,fancyhdr,multicol,needspace}
\usepackage[width=6in,height=9.5in,top=.5in,centering]{geometry}
% changes font to TeX Gyre Schola (Century Schoolbook)
\usepackage{tgschola}\usepackage[T1]{fontenc}

\setlength{\columnsep}{.3 in}


\hypersetup{pdfinfo={Title={Democracy and Social Ethics}, Author={Jane Addams}}, pdfborder = {0 0 0 0}}  

\newcounter{authornote}[page]
\newcommand*{\authornote}[1]{\renewcommand{\thefootnote}{\fnsymbol{footnote}}\stepcounter{authornote}\footnote[\value{authornote}]{#1}\renewcommand{\thefootnote}{\arabic{footnote}}}

\newcommand*{\hbreak}{\par\noindent\begin{tabular*}{\linewidth}{c}\hline\hline\end{tabular*}\par}

\newcommand*{\authortitle}[1]{\medskip\centerline{\Huge\sc #1}\bigskip}
\newcommand*{\itemtitle}[1]{\setstretch{1.8}\pagebreak[2]\begin{center}{\LARGE\sc #1}\end{center}\setstretch{1.2}}
\newcommand*{\itemsection}[2]{\needspace{36pt}\begin{center}\addcontentsline{toc}{section}{#1. #2}\rfoot{\sc Democracy and Social Ethics --- #1 #2}\textbf{\Large #1. #2}\nopagebreak\stepcounter{section}\end{center}}

\renewcommand{\thesection}{}

\renewcommand\headrulewidth{0pt}


\begin{document}


\newcounter{addamsparagraph}[section]
\newcommand*{\addamsparagraph}{\stepcounter{addamsparagraph}\paragraph{\small\P\arabic{addamsparagraph}}}
%\renewcommand*{\addamsparagraph}{}


\newenvironment{sectionbody}{\begin{multicols}{2}%
%\par%
%\everypar{\noindent \stepcounter{addamsparagraph}{\small\P\arabic{addamsparagraph}} \hspace{1em}}%
}{\end{multicols}}


\itemtitle{Democracy and Social Ethics [selections]}

\authortitle{Jane Addams}


\paragraph{About this edition:} The text is based on the Project Gutenberg edition prepared by by Alicia Williams, Joel Schlosberg, and the Online Distributed Proofreading Team. Williams et al. followed the 1902 edition published by Macmillan\&Company.

EBook \#15487, \url{http://www.gutenberg.org/}

Prepared by P.D. Magnus, Spring 2018. The included chapters are unabridged. In order to allow reference to other editions, the paragraphs within each chapter are numbered consecutively.

\setstretch{1}


\setcounter{tocdepth}{1}
\tableofcontents

\pagestyle{fancy}
\lfoot{\thepage}
\cfoot{}
\rfoot{\sc Democracy and Social Ethics}

\setstretch{1.2}

\bigskip

%: Chapter I
\itemsection{I}{Introduction}

\begin{sectionbody}

\addamsparagraph It is well to remind ourselves, from time to time, that ``Ethics'' is but
another word for ``righteousness,'' that for which many men and women of
every generation have hungered and thirsted, and without which life
becomes meaningless.

\addamsparagraph Certain forms of personal righteousness have become to a majority of the
community almost automatic. It is as easy for most of us to keep from
stealing our dinners as it is to digest them, and there is quite as much
voluntary morality involved in one process as in the other. To steal
would be for us to fall sadly below the standard of habit and
expectation which makes virtue easy. In the same way we have been
carefully reared to a sense of family obligation, to be kindly and
considerate to the members of our own households, and to feel
responsible for their well-being. As the rules of conduct have become
established in regard to our self-development and our families, so they
have been in regard to limited circles of friends. If the fulfilment of
these claims were all that a righteous life required, the hunger and
thirst would be stilled for many good men and women, and the clew of
right living would lie easily in their hands.

\addamsparagraph But we all know that each generation has its own test, the
contemporaneous and current standard by which alone it can adequately
judge of its own moral achievements, and that it may not legitimately
use a previous and less vigorous test. The advanced test must indeed
include that which has already been attained; but if it includes no
more, we shall fail to go forward, thinking complacently that we have
``arrived'' when in reality we have not yet started.

\addamsparagraph To attain individual morality in an age demanding social morality, to
pride one's self on the results of personal effort when the time demands
social adjustment, is utterly to fail to apprehend the situation.

\addamsparagraph It is perhaps significant that a German critic has of late reminded us
that the one test which the most authoritative and dramatic portrayal of
the Day of Judgment offers, is the social test. The stern questions are
not in regard to personal and family relations, but did ye visit the
poor, the criminal, the sick, and did ye feed the hungry?

\addamsparagraph All about us are men and women who have become unhappy in regard to
their attitude toward the social order itself; toward the dreary round
of uninteresting work, the pleasures narrowed down to those of appetite,
the declining consciousness of brain power, and the lack of mental food
which characterizes the lot of the large proportion of their
fellow-citizens. These men and women have caught a moral challenge
raised by the exigencies of contemporaneous life; some are bewildered,
others who are denied the relief which sturdy action brings are even
seeking an escape, but all are increasingly anxious concerning their
actual relations to the basic organization of society.

\addamsparagraph The test which they would apply to their conduct is a social test. They
fail to be content with the fulfilment of their family and personal
obligations, and find themselves striving to respond to a new demand
involving a social obligation; they have become conscious of another
requirement, and the contribution they would make is toward a code of
social ethics. The conception of life which they hold has not yet
expressed itself in social changes or legal enactment, but rather in a
mental attitude of maladjustment, and in a sense of divergence between
their consciences and their conduct. They desire both a clearer
definition of the code of morality adapted to present day demands and a
part in its fulfilment, both a creed and a practice of social morality.
In the perplexity of this intricate situation at least one thing is
becoming clear: if the latter day moral ideal is in reality that of a
social morality, it is inevitable that those who desire it must be
brought in contact with the moral experiences of the many in order to
procure an adequate social motive.

\addamsparagraph These men and women have realized this and have disclosed the fact in
their eagerness for a wider acquaintance with and participation in the
life about them. They believe that experience gives the easy and
trustworthy impulse toward right action in the broad as well as in the
narrow relations. We may indeed imagine many of them saying: ``Cast our
experiences in a larger mould if our lives are to be animated by the
larger social aims. We have met the obligations of our family life, not
because we had made resolutions to that end, but spontaneously, because
of a common fund of memories and affections, from which the obligation
naturally develops, and we see no other way in which to prepare
ourselves for the larger social duties.'' Such a demand is reasonable,
for by our daily experience we have discovered that we cannot
mechanically hold up a moral standard, then jump at it in rare moments
of exhilaration when we have the strength for it, but that even as the
ideal itself must be a rational development of life, so the strength to
attain it must be secured from interest in life itself. We slowly learn
that life consists of processes as well as results, and that failure may
come quite as easily from ignoring the adequacy of one's method as from
selfish or ignoble aims. We are thus brought to a conception of
Democracy not merely as a sentiment which desires the well-being of all
men, nor yet as a creed which believes in the essential dignity and
equality of all men, but as that which affords a rule of living as well
as a test of faith.

\addamsparagraph We are learning that a standard of social ethics is not attained by
travelling a sequestered byway, but by mixing on the thronged and common
road where all must turn out for one another, and at least see the size
of one another's burdens. To follow the path of social morality results
perforce in the temper if not the practice of the democratic spirit, for
it implies that diversified human experience and resultant sympathy
which are the foundation and guarantee of Democracy.

\addamsparagraph There are many indications that this conception of Democracy is growing
among us. We have come to have an enormous interest in human life as
such, accompanied by confidence in its essential soundness. We do not
believe that genuine experience can lead us astray any more than
scientific data can.

\addamsparagraph We realize, too, that social perspective and sanity of judgment come
only from contact with social experience; that such contact is the
surest corrective of opinions concerning the social order, and
concerning efforts, however humble, for its improvement. Indeed, it is a
consciousness of the illuminating and dynamic value of this wider and
more thorough human experience which explains in no small degree that
new curiosity regarding human life which has more of a moral basis than
an intellectual one.

\addamsparagraph The newspapers, in a frank reflection of popular demand, exhibit an
omniverous curiosity equally insistent upon the trivial and the
important. They are perhaps the most obvious manifestations of that
desire to know, that ``What is this?'' and ``Why do you do that?'' of the
child. The first dawn of the social consciousness takes this form, as
the dawning intelligence of the child takes the form of constant
question and insatiate curiosity.

\addamsparagraph Literature, too, portrays an equally absorbing though better adjusted
desire to know all kinds of life. The popular books are the novels,
dealing with life under all possible conditions, and they are widely
read not only because they are entertaining, but also because they in a
measure satisfy an unformulated belief that to see farther, to know all
sorts of men, in an indefinite way, is a preparation for better social
adjustment---for the remedying of social ills.

\addamsparagraph Doubtless one under the conviction of sin in regard to social ills finds
a vague consolation in reading about the lives of the poor, and derives
a sense of complicity in doing good. He likes to feel that he knows
about social wrongs even if he does not remedy them, and in a very
genuine sense there is a foundation for this belief.

\addamsparagraph Partly through this wide reading of human life, we find in ourselves a
new affinity for all men, which probably never existed in the world
before. Evil itself does not shock us as it once did, and we count only
that man merciful in whom we recognize an understanding of the criminal.
We have learned as common knowledge that much of the insensibility and
hardness of the world is due to the lack of imagination which prevents a
realization of the experiences of other people. Already there is a
conviction that we are under a moral obligation in choosing our
experiences, since the result of those experiences must ultimately
determine our understanding of life. We know instinctively that if we
grow contemptuous of our fellows, and consciously limit our intercourse
to certain kinds of people whom we have previously decided to respect,
we not only tremendously circumscribe our range of life, but limit the
scope of our ethics.

\addamsparagraph We can recall among the selfish people of our acquaintance at least one
common characteristic,---the conviction that they are different from
other men and women, that they need peculiar consideration because they
are more sensitive or more refined. Such people ``refuse to be bound by
any relation save the personally luxurious ones of love and admiration,
or the identity of political opinion, or religious creed.'' We have
learned to recognize them as selfish, although we blame them not for the
will which chooses to be selfish, but for a narrowness of interest which
deliberately selects its experience within a limited sphere, and we say
that they illustrate the danger of concentrating the mind on narrow and
unprogressive issues.

\addamsparagraph We know, at last, that we can only discover truth by a rational and
democratic interest in life, and to give truth complete social
expression is the endeavor upon which we are entering. Thus the
identification with the common lot which is the essential idea of
Democracy becomes the source and expression of social ethics. It is as
though we thirsted to drink at the great wells of human experience,
because we knew that a daintier or less potent draught would not carry
us to the end of the journey, going forward as we must in the heat and
jostle of the crowd.

\addamsparagraph The six following chapters are studies of various types and groups who
are being impelled by the newer conception of Democracy to an acceptance
of social obligations involving in each instance a new line of conduct.
No attempt is made to reach a conclusion, nor to offer advice beyond the
assumption that the cure for the ills of Democracy is more Democracy,
but the quite unlooked-for result of the studies would seem to indicate
that while the strain and perplexity of the situation is felt most
keenly by the educated and self-conscious members of the community, the
tentative and actual attempts at adjustment are largely coming through
those who are simpler and less analytical.

\end{sectionbody}

%: Chapter II
\itemsection{II}{Charitable Effort}

\begin{sectionbody}

\addamsparagraph All those hints and glimpses of a larger and more satisfying democracy,
which literature and our own hopes supply, have a tendency to slip away
from us and to leave us sadly unguided and perplexed when we attempt to
act upon them.

\addamsparagraph Our conceptions of morality, as all our other ideas, pass through a
course of development; the difficulty comes in adjusting our conduct,
which has become hardened into customs and habits, to these changing
moral conceptions. When this adjustment is not made, we suffer from the
strain and indecision of believing one hypothesis and acting upon
another.

\addamsparagraph Probably there is no relation in life which our democracy is changing
more rapidly than the charitable relation---that relation which obtains
between benefactor and beneficiary; at the same time there is no point
of contact in our modern experience which reveals so clearly the lack of
that equality which democracy implies. We have reached the moment when
democracy has made such inroads upon this relationship, that the
complacency of the old-fashioned charitable man is gone forever; while,
at the same time, the very need and existence of charity, denies us the
consolation and freedom which democracy will at last give.

\addamsparagraph It is quite obvious that the ethics of none of us are clearly defined,
and we are continually obliged to act in circles of habit, based upon
convictions which we no longer hold. Thus our estimate of the effect of
environment and social conditions has doubtless shifted faster than our
methods of administrating charity have changed. Formerly when it was
believed that poverty was synonymous with vice and laziness, and that
the prosperous man was the righteous man, charity was administered
harshly with a good conscience; for the charitable agent really blamed
the individual for his poverty, and the very fact of his own superior
prosperity gave him a certain consciousness of superior morality. We
have learned since that time to measure by other standards, and have
ceased to accord to the money-earning capacity exclusive respect; while
it is still rewarded out of all proportion to any other, its possession
is by no means assumed to imply the possession of the highest moral
qualities. We have learned to judge men by their social virtues as well
as by their business capacity, by their devotion to intellectual and
disinterested aims, and by their public spirit, and we naturally resent
being obliged to judge poor people so solely upon the industrial side.
Our democratic instinct instantly takes alarm. It is largely in this
modern tendency to judge all men by one democratic standard, while the
old charitable attitude commonly allowed the use of two standards, that
much of the difficulty adheres. We know that unceasing bodily toil
becomes wearing and brutalizing, and our position is totally untenable
if we judge large numbers of our fellows solely upon their success in
maintaining it.

\addamsparagraph The daintily clad charitable visitor who steps into the little house
made untidy by the vigorous efforts of her hostess, the washerwoman, is
no longer sure of her superiority to the latter; she recognizes that her
hostess after all represents social value and industrial use, as over
against her own parasitic cleanliness and a social standing attained
only through status.

\addamsparagraph The only families who apply for aid to the charitable agencies are those
who have come to grief on the industrial side; it may be through
sickness, through loss of work, or for other guiltless and inevitable
reasons; but the fact remains that they are industrially ailing, and
must be bolstered and helped into industrial health. The charity
visitor, let us assume, is a young college woman, well-bred and
open-minded; when she visits the family assigned to her, she is often
embarrassed to find herself obliged to lay all the stress of her
teaching and advice upon the industrial virtues, and to treat the
members of the family almost exclusively as factors in the industrial
system. She insists that they must work and be self-supporting, that the
most dangerous of all situations is idleness, that seeking one's own
pleasure, while ignoring claims and responsibilities, is the most
ignoble of actions. The members of her assigned family may have other
charms and virtues---they may possibly be kind and considerate of each
other, generous to their friends, but it is her business to stick to the
industrial side. As she daily holds up these standards, it often occurs
to the mind of the sensitive visitor, whose conscience has been made
tender by much talk of brotherhood and equality, that she has no right
to say these things; that her untrained hands are no more fitted to
cope with actual conditions than those of her broken-down family.

\addamsparagraph The grandmother of the charity visitor could have done the industrial
preaching very well, because she did have the industrial virtues and
housewifely training. In a generation our experiences have changed, and
our views with them; but we still keep on in the old methods, which
could be applied when our consciences were in line with them, but which
are daily becoming more difficult as we divide up into people who work
with their hands and those who do not. The charity visitor belonging to
the latter class is perplexed by recognitions and suggestions which the
situation forces upon her. Our democracy has taught us to apply our
moral teaching all around, and the moralist is rapidly becoming so
sensitive that when his life does not exemplify his ethical convictions,
he finds it difficult to preach.

\addamsparagraph Added to this is a consciousness, in the mind of the visitor, of a
genuine misunderstanding of her motives by the recipients of her
charity, and by their neighbors. Let us take a neighborhood of poor
people, and test their ethical standards by those of the charity
visitor, who comes with the best desire in the world to help them out of
their distress. A most striking incongruity, at once apparent, is the
difference between the emotional kindness with which relief is given by
one poor neighbor to another poor neighbor, and the guarded care with
which relief is given by a charity visitor to a charity recipient. The
neighborhood mind is at once confronted not only by the difference of
method, but by an absolute clashing of two ethical standards.

\addamsparagraph A very little familiarity with the poor districts of any city is
sufficient to show how primitive and genuine are the neighborly
relations. There is the greatest willingness to lend or borrow anything,
and all the residents of the given tenement know the most intimate
family affairs of all the others. The fact that the economic condition
of all alike is on a most precarious level makes the ready outflow of
sympathy and material assistance the most natural thing in the world.
There are numberless instances of self-sacrifice quite unknown in the
circles where greater economic advantages make that kind of intimate
knowledge of one's neighbors impossible. An Irish family in which the
man has lost his place, and the woman is struggling to eke out the
scanty savings by day's work, will take in the widow and her five
children who have been turned into the street, without a moment's
reflection upon the physical discomforts involved. The most maligned
landlady who lives in the house with her tenants is usually ready to
lend a scuttle full of coal to one of them who may be out of work, or to
share her supper. A woman for whom the writer had long tried in vain to
find work failed to appear at the appointed time when employment was
secured at last. Upon investigation it transpired that a neighbor
further down the street was taken ill, that the children ran for the
family friend, who went of course, saying simply when reasons for her
non-appearance were demanded, ``It broke me heart to leave the place, but
what could I do?'' A woman whose husband was sent up to the city prison
for the maximum term, just three months, before the birth of her child
found herself penniless at the end of that time, having gradually sold
her supply of household furniture. She took refuge with a friend whom
she supposed to be living in three rooms in another part of town. When
she arrived, however, she discovered that her friend's husband had been
out of work so long that they had been reduced to living in one room.
The friend, however, took her in, and the friend's husband was obliged
to sleep upon a bench in the park every night for a week, which he did
uncomplainingly if not cheerfully. Fortunately it was summer, ``and it
only rained one night.'' The writer could not discover from the young
mother that she had any special claim upon the ``friend'' beyond the fact
that they had formerly worked together in the same factory. The husband
she had never seen until the night of her arrival, when he at once went
forth in search of a midwife who would consent to come upon his promise
of future payment.

\addamsparagraph The evolutionists tell us that the instinct to pity, the impulse to aid
his fellows, served man at a very early period, as a rude rule of right
and wrong. There is no doubt that this rude rule still holds among many
people with whom charitable agencies are brought into contact, and that
their ideas of right and wrong are quite honestly outraged by the
methods of these agencies. When they see the delay and caution with
which relief is given, it does not appear to them a conscientious
scruple, but as the cold and calculating action of a selfish man. It is
not the aid that they are accustomed to receive from their neighbors,
and they do not understand why the impulse which drives people to ``be
good to the poor'' should be so severely supervised. They feel,
remotely, that the charity visitor is moved by motives that are alien
and unreal. They may be superior motives, but they are different, and
they are ``agin nature.'' They cannot comprehend why a person whose
intellectual perceptions are stronger than his natural impulses, should
go into charity work at all. The only man they are accustomed to see
whose intellectual perceptions are stronger than his tenderness of
heart, is the selfish and avaricious man who is frankly ``on the make.''
If the charity visitor is such a person, why does she pretend to like
the poor? Why does she not go into business at once?

\addamsparagraph We may say, of course, that it is a primitive view of life, which thus
confuses intellectuality and business ability; but it is a view quite
honestly held by many poor people who are obliged to receive charity
from time to time. In moments of indignation the poor have been known to
say: ``What do you want, anyway? If you have nothing to give us, why not
let us alone and stop your questionings and investigations?'' ``They
investigated me for three weeks, and in the end gave me nothing but a
black character,'' a little woman has been heard to assert. This
indignation, which is for the most part taciturn, and a certain kindly
contempt for her abilities, often puzzles the charity visitor. The
latter may be explained by the standard of worldly success which the
visited families hold. Success does not ordinarily go, in the minds of
the poor, with charity and kind-heartedness, but rather with the
opposite qualities. The rich landlord is he who collects with sternness,
who accepts no excuse, and will have his own. There are moments of
irritation and of real bitterness against him, but there is still
admiration, because he is rich and successful. The good-natured
landlord, he who pities and spares his poverty-pressed tenants, is
seldom rich. He often lives in the back of his house, which he has owned
for a long time, perhaps has inherited; but he has been able to
accumulate little. He commands the genuine love and devotion of many a
poor soul, but he is treated with a certain lack of respect. In one
sense he is a failure. The charity visitor, just because she is a person
who concerns herself with the poor, receives a certain amount of this
good-natured and kindly contempt, sometimes real affection, but little
genuine respect. The poor are accustomed to help each other and to
respond according to their kindliness; but when it comes to worldly
judgment, they use industrial success as the sole standard. In the case
of the charity visitor who has neither natural kindness nor dazzling
riches, they are deprived of both standards, and they find it of course
utterly impossible to judge of the motive of organized charity.

\addamsparagraph Even those of us who feel most sorely the need of more order in
altruistic effort and see the end to be desired, find something
distasteful in the juxtaposition of the words ``organized'' and ``charity.''
We say in defence that we are striving to turn this emotion into a
motive, that pity is capricious, and not to be depended on; that we mean
to give it the dignity of conscious duty. But at bottom we distrust a
little a scheme which substitutes a theory of social conduct for the
natural promptings of the heart, even although we appreciate the
complexity of the situation. The poor man who has fallen into distress,
when he first asks aid, instinctively expects tenderness, consideration,
and forgiveness. If it is the first time, it has taken him long to make
up his mind to take the step. He comes somewhat bruised and battered,
and instead of being met with warmth of heart and sympathy, he is at
once chilled by an investigation and an intimation that he ought to
work. He does not recognize the disciplinary aspect of the situation.

\addamsparagraph The only really popular charity is that of the visiting nurses, who by
virtue of their professional training render services which may easily
be interpreted into sympathy and kindness, ministering as they do to
obvious needs which do not require investigation.

\addamsparagraph The state of mind which an investigation arouses on both sides is most
unfortunate; but the perplexity and clashing of different standards,
with the consequent misunderstandings, are not so bad as the moral
deterioration which is almost sure to follow.

\addamsparagraph When the agent or visitor appears among the poor, and they discover that
under certain conditions food and rent and medical aid are dispensed
from some unknown source, every man, woman, and child is quick to learn
what the conditions may be, and to follow them. Though in their eyes a
glass of beer is quite right and proper when taken as any
self-respecting man should take it; though they know that cleanliness is
an expensive virtue which can be required of few; though they realize
that saving is well-nigh impossible when but a few cents can be laid by
at a time; though their feeling for the church may be something quite
elusive of definition and quite apart from daily living: to the visitor
they gravely laud temperance and cleanliness and thrift and religious
observance. The deception in the first instances arises from a wondering
inability to understand the ethical ideals which can require such
impossible virtues, and from an innocent desire to please. It is easy to
trace the development of the mental suggestions thus received. When A
discovers that B, who is very little worse off than he, receives good
things from an inexhaustible supply intended for the poor at large, he
feels that he too has a claim for his share, and step by step there is
developed the competitive spirit which so horrifies charity visitors
when it shows itself in a tendency to ``work'' the relief-giving agencies.

\addamsparagraph The most serious effect upon the poor comes when dependence upon the
charitable society is substituted for the natural outgoing of human love
and sympathy, which, happily, we all possess in some degree. The
spontaneous impulse to sit up all night with the neighbor's sick child
is turned into righteous indignation against the district nurse,
because she goes home at six o'clock, and doesn't do it herself. Or the
kindness which would have prompted the quick purchase of much needed
medicine is transformed into a voluble scoring of the dispensary,
because it gives prescriptions and not drugs; and ``who can get well on a
piece of paper?''

\addamsparagraph If a poor woman knows that her neighbor next door has no shoes, she is
quite willing to lend her own, that her neighbor may go decently to
mass, or to work; for she knows the smallest item about the scanty
wardrobe, and cheerfully helps out. When the charity visitor comes in,
all the neighbors are baffled as to what her circumstances may be. They
know she does not need a new pair of shoes, and rather suspect that she
has a dozen pairs at home; which, indeed, she sometimes has. They
imagine untold stores which they may call upon, and her most generous
gift is considered niggardly, compared with what she might do. She ought
to get new shoes for the family all round, ``she sees well enough that
they need them.'' It is no more than the neighbor herself would do, has
practically done, when she lent her own shoes. The charity visitor has
broken through the natural rule of giving, which, in a primitive
society, is bounded only by the need of the recipient and the resources
of the giver; and she gets herself into untold trouble when she is
judged by the ethics of that primitive society.

\addamsparagraph The neighborhood understands the selfish rich people who stay in their
own part of town, where all their associates have shoes and other
things. Such people don't bother themselves about the poor; they are
like the rich landlords of the neighborhood experience. But this lady
visitor, who pretends to be good to the poor, and certainly does talk as
though she were kind-hearted, what does she come for, if she does not
intend to give them things which are so plainly needed?

\addamsparagraph The visitor says, sometimes, that in holding her poor family so hard to
a standard of thrift she is really breaking down a rule of higher living
which they formerly possessed; that saving, which seems quite
commendable in a comfortable part of town, appears almost criminal in a
poorer quarter where the next-door neighbor needs food, even if the
children of the family do not.

\addamsparagraph She feels the sordidness of constantly being obliged to urge the
industrial view of life. The benevolent individual of fifty years ago
honestly believed that industry and self-denial in youth would result in
comfortable possessions for old age. It was, indeed, the method he had
practised in his own youth, and by which he had probably obtained
whatever fortune he possessed. He therefore reproved the poor family for
indulging their children, urged them to work long hours, and was utterly
untouched by many scruples which afflict the contemporary charity
visitor. She says sometimes, ``Why must I talk always of getting work and
saving money, the things I know nothing about? If it were anything else
I had to urge, I could do it; anything like Latin prose, which I had
worried through myself, it would not be so hard.'' But she finds it
difficult to connect the experiences of her youth with the experiences
of the visited family.

\addamsparagraph Because of this diversity in experience, the visitor is continually
surprised to find that the safest platitude may be challenged. She
refers quite naturally to the ``horrors of the saloon,'' and discovers
that the head of her visited family does not connect them with ``horrors''
at all. He remembers all the kindnesses he has received there, the free
lunch and treating which goes on, even when a man is out of work and not
able to pay up; the loan of five dollars he got there when the charity
visitor was miles away and he was threatened with eviction. He may
listen politely to her reference to ``horrors,'' but considers it only
``temperance talk.''

\addamsparagraph The charity visitor may blame the women for lack of gentleness toward
their children, for being hasty and rude to them, until she learns that
the standard of breeding is not that of gentleness toward the children
so much as the observance of certain conventions, such as the
punctilious wearing of mourning garments after the death of a child. The
standard of gentleness each mother has to work out largely by herself,
assisted only by the occasional shame-faced remark of a neighbor, ``That
they do better when you are not too hard on them''; but the wearing of
mourning garments is sustained by the definitely expressed sentiment of
every woman in the street. The mother would have to bear social blame, a
certain social ostracism, if she failed to comply with that requirement.
It is not comfortable to outrage the conventions of those among whom we
live, and, if our social life be a narrow one, it is still more
difficult. The visitor may choke a little when she sees the lessened
supply of food and the scanty clothing provided for the remaining
children in order that one may be conventionally mourned, but she
doesn't talk so strongly against it as she would have done during her
first month of experience with the family since bereaved.

\addamsparagraph The subject of clothes indeed perplexes the visitor constantly, and the
result of her reflections may be summed up somewhat in this wise: The
girl who has a definite social standing, who has been to a fashionable
school or to a college, whose family live in a house seen and known by
all her friends and associates, may afford to be very simple, or even
shabby as to her clothes, if she likes. But the working girl, whose
family lives in a tenement, or moves from one small apartment to
another, who has little social standing and has to make her own place,
knows full well how much habit and style of dress has to do with her
position. Her income goes into her clothing, out of all proportion to
the amount which she spends upon other things. But, if social
advancement is her aim, it is the most sensible thing she can do. She is
judged largely by her clothes. Her house furnishing, with its pitiful
little decorations, her scanty supply of books, are never seen by the
people whose social opinions she most values. Her clothes are her
background, and from them she is largely judged. It is due to this fact
that girls' clubs succeed best in the business part of town, where
``working girls'' and ``young ladies'' meet upon an equal footing, and where
the clothes superficially look very much alike. Bright and ambitious
girls will come to these down-town clubs to eat lunch and rest at noon,
to study all sorts of subjects and listen to lectures, when they might
hesitate a long time before joining a club identified with their own
neighborhood, where they would be judged not solely on their own merits
and the unconscious social standing afforded by good clothes, but by
other surroundings which are not nearly up to these. For the same
reason, girls' clubs are infinitely more difficult to organize in little
towns and villages, where every one knows every one else, just how the
front parlor is furnished, and the amount of mortgage there is upon the
house. These facts get in the way of a clear and unbiassed judgment;
they impede the democratic relationship and add to the
self-consciousness of all concerned. Every one who has had to do with
down-town girls' clubs has had the experience of going into the home of
some bright, well-dressed girl, to discover it uncomfortable and perhaps
wretched, and to find the girl afterward carefully avoiding her,
although the working girl may not have been at home when the call was
made, and the visitor may have carried herself with the utmost courtesy
throughout. In some very successful down-town clubs the home address is
not given at all, and only the ``business address'' is required. Have we
worked out our democracy further in regard to clothes than anything
else?

\addamsparagraph The charity visitor has been rightly brought up to consider it vulgar to
spend much money upon clothes, to care so much for ``appearances.'' She
realizes dimly that the care for personal decoration over that for one's
home or habitat is in some way primitive and undeveloped; but she is
silenced by its obvious need. She also catches a glimpse of the fact
that the disproportionate expenditure of the poor in the matter of
clothes is largely due to the exclusiveness of the rich who hide from
them the interior of their houses, and their more subtle pleasures,
while of necessity exhibiting their street clothes and their street
manners. Every one who goes shopping at the same time may see the
clothes of the richest women in town, but only those invited to her
receptions see the Corot on her walls or the bindings in her library.
The poor naturally try to bridge the difference by reproducing the
street clothes which they have seen. They are striving to conform to a
common standard which their democratic training presupposes belongs to
all of us. The charity visitor may regret that the Italian peasant
woman has laid aside her picturesque kerchief and substituted a cheap
street hat. But it is easy to recognize the first attempt toward
democratic expression.

\addamsparagraph The charity visitor finds herself still more perplexed when she comes to
consider such problems as those of early marriage and child labor; for
she cannot deal with them according to economic theories, or according
to the conventions which have regulated her own life. She finds both of
these fairly upset by her intimate knowledge of the situation, and her
sympathy for those into whose lives she has gained a curious insight.
She discovers how incorrigibly bourgeois her standards have been, and it
takes but a little time to reach the conclusion that she cannot insist
so strenuously upon the conventions of her own class, which fail to fit
the bigger, more emotional, and freer lives of working people. The
charity visitor holds well-grounded views upon the imprudence of early
marriages, quite naturally because she comes from a family and circle
of professional and business people. A professional man is scarcely
equipped and started in his profession before he is thirty. A business
man, if he is on the road to success, is much nearer prosperity at
thirty-five than twenty-five, and it is therefore wise for these men not
to marry in the twenties; but this does not apply to the workingman. In
many trades he is laid upon the shelf at thirty-five, and in nearly all
trades he receives the largest wages in his life between twenty and
thirty. If the young workingman has all his wages to himself, he will
probably establish habits of personal comfort, which he cannot keep up
when he has to divide with a family---habits which he can, perhaps, never
overcome.

\addamsparagraph The sense of prudence, the necessity for saving, can never come to a
primitive, emotional man with the force of a conviction; but the
necessity of providing for his children is a powerful incentive. He
naturally regards his children as his savings-bank; he expects them to
care for him when he gets old, and in some trades old age comes very
early. A Jewish tailor was quite lately sent to the Cook County
poorhouse, paralyzed beyond recovery at the age of thirty-five. Had his
little boy of nine been but a few years older, he might have been spared
this sorrow of public charity. He was, in fact, better able to well
support a family when he was twenty than when he was thirty-five, for
his wages had steadily grown less as the years went on. Another tailor
whom I know, who is also a Socialist, always speaks of saving as a
bourgeois virtue, one quite impossible to the genuine workingman. He
supports a family consisting of himself, a wife and three children, and
his two parents on eight dollars a week. He insists it would be criminal
not to expend every penny of this amount upon food and shelter, and he
expects his children later to care for him.

\addamsparagraph This economic pressure also accounts for the tendency to put children to
work overyoung and thus cripple their chances for individual
development and usefulness, and with the avaricious parent also leads to
exploitation. ``I have fed her for fourteen years, now she can help me
pay my mortgage'' is not an unusual reply when a hardworking father is
expostulated with because he would take his bright daughter out of
school and put her into a factory.

\addamsparagraph It has long been a common error for the charity visitor, who is strongly
urging her ``family'' toward self-support, to suggest, or at least
connive, that the children be put to work early, although she has not
the excuse that the parents have. It is so easy, after one has been
taking the industrial view for a long time, to forget the larger and
more social claim; to urge that the boy go to work and support his
parents, who are receiving charitable aid. She does not realize what a
cruel advantage the person who distributes charity has, when she gives
advice.

\addamsparagraph The manager in a huge mercantile establishment employing many children
was able to show during a child-labor investigation, that the only
children under fourteen years of age in his employ were prot�g�s who had
been urged upon him by philanthropic ladies, not only acquaintances of
his, but valued patrons of the establishment. It is not that the charity
visitor is less wise than other people, but she has fixed her mind so
long upon the industrial lameness of her family that she is eager to
seize any crutch, however weak, which may enable them to get on.

\addamsparagraph She has failed to see that the boy who attempts to prematurely support
his widowed mother may lower wages, add an illiterate member to the
community, and arrest the development of a capable workingman. As she
has failed to see that the rules which obtain in regard to the age of
marriage in her own family may not apply to the workingman, so also she
fails to understand that the present conditions of employment
surrounding a factory child are totally unlike those which obtained
during the energetic youth of her father.

\addamsparagraph The child who is prematurely put to work is constantly oppressed by this
never ending question of the means of subsistence, and even little
children are sometimes almost crushed with the cares of life through
their affectionate sympathy. The writer knows a little Italian lad of
six to whom the problems of food, clothing, and shelter have become so
immediate and pressing that, although an imaginative child, he is unable
to see life from any other standpoint. The goblin or bugaboo, feared by
the more fortunate child, in his mind, has come to be the need of coal
which caused his father hysterical and demonstrative grief when it
carried off his mother's inherited linen, the mosaic of St. Joseph, and,
worst of all, his own rubber boots. He once came to a party at
Hull-House, and was interested in nothing save a gas stove which he saw
in the kitchen. He became excited over the discovery that fire could be
produced without fuel. ``I will tell my father of this stove. You buy no
coal, you need only a match. Anybody will give you a match.'' He was
taken to visit at a country-house and at once inquired how much rent was
paid for it. On being told carelessly by his hostess that they paid no
rent for that house, he came back quite wild with interest that the
problem was solved. ``Me and my father will go to the country. You get a
big house, all warm, without rent.'' Nothing else in the country
interested him but the subject of rent, and he talked of that with an
exclusiveness worthy of a single taxer.

\addamsparagraph The struggle for existence, which is so much harsher among people near
the edge of pauperism, sometimes leaves ugly marks on character, and the
charity visitor finds these indirect results most mystifying. Parents
who work hard and anticipate an old age when they can no longer earn,
take care that their children shall expect to divide their wages with
them from the very first. Such a parent, when successful, impresses the
immature nervous system of the child thus tyrannically establishing
habits of obedience, so that the nerves and will may not depart from
this control when the child is older. The charity visitor, whose family
relation is lifted quite out of this, does not in the least understand
the industrial foundation for this family tyranny.

\addamsparagraph The head of a kindergarten training-class once addressed a club of
working women, and spoke of the despotism which is often established
over little children. She said that the so-called determination to break
a child's will many times arose from a lust of dominion, and she urged
the ideal relationship founded upon love and confidence. But many of the
women were puzzled. One of them remarked to the writer as she came out
of the club room, ``If you did not keep control over them from the time
they were little, you would never get their wages when they are grown
up.'' Another one said, ``Ah, of course she (meaning the speaker) doesn't
have to depend upon her children's wages. She can afford to be lax with
them, because even if they don't give money to her, she can get along
without it.''

\addamsparagraph There are an impressive number of children who uncomplainingly and
constantly hand over their weekly wages to their parents, sometimes
receiving back ten cents or a quarter for spending-money, but quite as
often nothing at all; and the writer knows one girl of twenty-five who
for six years has received two cents a week from the constantly falling
wages which she earns in a large factory. Is it habit or virtue which
holds her steady in this course? If love and tenderness had been
substituted for parental despotism, would the mother have had enough
affection, enough power of expression to hold her daughter's sense of
money obligation through all these years? This girl who spends her
paltry two cents on chewing-gum and goes plainly clad in clothes of her
mother's choosing, while many of her friends spend their entire wages on
those clothes which factory girls love so well, must be held by some
powerful force.

\addamsparagraph The charity visitor finds these subtle and elusive problems most
harrowing. The head of a family she is visiting is a man who has become
black-listed in a strike. He is not a very good workman, and this, added
to his agitator's reputation, keeps him out of work for a long time. The
fatal result of being long out of work follows: he becomes less and less
eager for it, and gets a ``job'' less and less frequently. In order to
keep up his self-respect, and still more to keep his wife's respect for
him, he yields to the little self-deception that this prolonged idleness
follows because he was once blacklisted, and he gradually becomes a
martyr. Deep down in his heart perhaps---but who knows what may be deep
down in his heart? Whatever may be in his wife's, she does not show for
an instant that she thinks he has grown lazy, and accustomed to see her
earn, by sewing and cleaning, most of the scanty income for the family.
The charity visitor, however, does see this, and she also sees that the
other men who were in the strike have gone back to work. She further
knows by inquiry and a little experience that the man is not skilful.
She cannot, however, call him lazy and good-for-nothing, and denounce
him as worthless as her grandmother might have done, because of certain
intellectual conceptions at which she has arrived. She sees other
workmen come to him for shrewd advice; she knows that he spends many
more hours in the public library reading good books than the average
workman has time to do. He has formed no bad habits and has yielded only
to those subtle temptations toward a life of leisure which come to the
intellectual man. He lacks the qualifications which would induce his
union to engage him as a secretary or organizer, but he is a constant
speaker at workingmen's meetings, and takes a high moral attitude on the
questions discussed there. He contributes a certain intellectuality to
his friends, and he has undoubted social value. The neighboring women
confide to the charity visitor their sympathy with his wife, because
she has to work so hard, and because her husband does not ``provide.''
Their remarks are sharpened by a certain resentment toward the
superiority of the husband's education and gentle manners. The charity
visitor is ashamed to take this point of view, for she knows that it is
not altogether fair. She is reminded of a college friend of hers, who
told her that she was not going to allow her literary husband to write
unworthy potboilers for the sake of earning a living. ``I insist that we
shall live within my own income; that he shall not publish until he is
ready, and can give his genuine message.'' The charity visitor recalls
what she has heard of another acquaintance, who urged her husband to
decline a lucrative position as a railroad attorney, because she wished
him to be free to take municipal positions, and handle public questions
without the inevitable suspicion which unaccountably attaches itself in
a corrupt city to a corporation attorney. The action of these two women
seemed noble to her, but in their cases they merely lived on a lesser
income. In the case of the workingman's wife, she faced living on no
income at all, or on the precarious one which she might be able to get
together.

\addamsparagraph She sees that this third woman has made the greatest sacrifice, and she
is utterly unwilling to condemn her while praising the friends of her
own social position. She realizes, of course, that the situation is
changed by the fact that the third family needs charity, while the other
two do not; but, after all, they have not asked for it, and their plight
was only discovered through an accident to one of the children. The
charity visitor has been taught that her mission is to preserve the
finest traits to be found in her visited family, and she shrinks from
the thought of convincing the wife that her husband is worthless and she
suspects that she might turn all this beautiful devotion into
complaining drudgery. To be sure, she could give up visiting the family
altogether, but she has become much interested in the progress of the
crippled child who eagerly anticipates her visits, and she also suspects
that she will never know many finer women than the mother. She is
unwilling, therefore, to give up the friendship, and goes on bearing her
perplexities as best she may.

\addamsparagraph The first impulse of our charity visitor is to be somewhat severe with
her shiftless family for spending money on pleasures and indulging their
children out of all proportion to their means. The poor family which
receives beans and coal from the county, and pays for a bicycle on the
instalment plan, is not unknown to any of us. But as the growth of
juvenile crime becomes gradually understood, and as the danger of giving
no legitimate and organized pleasure to the child becomes clearer, we
remember that primitive man had games long before he cared for a house
or regular meals.

\addamsparagraph There are certain boys in many city neighborhoods who form themselves
into little gangs with a leader who is somewhat more intrepid than the
rest. Their favorite performance is to break into an untenanted house,
to knock off the faucets, and cut the lead pipe, which they sell to the
nearest junk dealer. With the money thus procured they buy beer and
drink it in little free-booter's groups sitting in the alley. From
beginning to end they have the excitement of knowing that they may be
seen and caught by the ``coppers,'' and are at times quite breathless with
suspense. It is not the least unlike, in motive and execution, the
practice of country boys who go forth in squads to set traps for rabbits
or to round up a coon.

\addamsparagraph It is characterized by a pure spirit for adventure, and the vicious
training really begins when they are arrested, or when an older boy
undertakes to guide them into further excitements. From the very
beginning the most enticing and exciting experiences which they have
seen have been connected with crime. The policeman embodies all the
majesty of successful law and established government in his brass
buttons and dazzlingly equipped patrol wagon.

\addamsparagraph The boy who has been arrested comes back more or less a hero with a tale
to tell of the interior recesses of the mysterious police station. The
earliest public excitement the child remembers is divided between the
rattling fire engines, ``the time there was a fire in the next block,''
and all the tense interest of the patrol wagon ``the time the drunkest
lady in our street was arrested.''

\addamsparagraph In the first year of their settlement the Hull-House residents took
fifty kindergarten children to Lincoln Park, only to be grieved by their
apathetic interest in trees and flowers. As they came back with an
omnibus full of tired and sleepy children, they were surprised to find
them galvanized into sudden life because a patrol wagon rattled by.
Their eager little heads popped out of the windows full of questioning:
``Was it a man or a woman?'' ``How many policemen inside?'' and eager little
tongues began to tell experiences of arrests which baby eyes had
witnessed.

\addamsparagraph The excitement of a chase, the chances of competition, and the love of a
fight are all centred in the outward display of crime. The parent who
receives charitable aid and yet provides pleasure for his child, and is
willing to indulge him in his play, is blindly doing one of the wisest
things possible; and no one is more eager for playgrounds and vacation
schools than the conscientious charity visitor.

\addamsparagraph This very imaginative impulse and attempt to live in a pictured world of
their own, which seems the simplest prerogative of childhood, often
leads the boys into difficulty. Three boys aged seven, nine, and ten
were once brought into a neighboring police station under the charge of
pilfering and destroying property. They had dug a cave under a railroad
viaduct in which they had spent many days and nights of the summer
vacation. They had ``swiped'' potatoes and other vegetables from
hucksters' carts, which they had cooked and eaten in true brigand
fashion; they had decorated the interior of the excavation with stolen
junk, representing swords and firearms, to their romantic imaginations.
The father of the ringleader was a janitor living in a building five
miles away in a prosperous portion of the city. The landlord did not
want an active boy in the building, and his mother was dead; the janitor
paid for the boy's board and lodging to a needy woman living near the
viaduct. She conscientiously gave him his breakfast and supper, and left
something in the house for his dinner every morning when she went to
work in a neighboring factory; but was too tired by night to challenge
his statement that he ``would rather sleep outdoors in the summer,'' or to
investigate what he did during the day. In the meantime the three boys
lived in a world of their own, made up from the reading of adventurous
stories and their vivid imaginations, steadily pilfering more and more
as the days went by, and actually imperilling the safety of the traffic
passing over the street on the top of the viaduct. In spite of vigorous
exertions on their behalf, one of the boys was sent to the Reform
School, comforting himself with the conclusive remark, ``Well, we had fun
anyway, and maybe they will let us dig a cave at the School; it is in
the country, where we can't hurt anything.''

\addamsparagraph In addition to books of adventure, or even reading of any sort, the
scenes and ideals of the theatre largely form the manners and morals of
the young people. ``Going to the theatre'' is indeed the most common and
satisfactory form of recreation. Many boys who conscientiously give all
their wages to their mothers have returned each week ten cents to pay
for a seat in the gallery of a theatre on Sunday afternoon. It is their
one satisfactory glimpse of life---the moment when they ``issue forth from
themselves'' and are stirred and thoroughly interested. They quite simply
adopt as their own, and imitate as best they can, all that they see
there. In moments of genuine grief and excitement the words and the
gestures they employ are those copied from the stage, and the tawdry
expression often conflicts hideously with the fine and genuine emotion
of which it is the inadequate and vulgar vehicle.

\addamsparagraph As in the matter of dress, more refined and simpler manners and mode of
expressions are unseen by them, and they must perforce copy what they
know.

\addamsparagraph If we agree with a recent definition of Art, as that which causes the
spectator to lose his sense of isolation, there is no doubt that the
popular theatre, with all its faults, more nearly fulfils the function
of art for the multitude of working people than all the ``free galleries''
and picture exhibits combined.

\addamsparagraph The greatest difficulty is experienced when the two standards come
sharply together, and when both sides make an attempt at understanding
and explanation. The difficulty of making clear one's own ethical
standpoint is at times insurmountable. A woman who had bought and sold
school books stolen from the school fund,---books which are all plainly
marked with a red stamp,---came to Hull House one morning in great
distress because she had been arrested, and begged a resident ``to speak
to the judge.'' She gave as a reason the fact that the House had known
her for six years, and had once been very good to her when her little
girl was buried. The resident more than suspected that her visitor knew
the school books were stolen when buying them, and any attempt to talk
upon that subject was evidently considered very rude. The visitor wished
to get out of her trial, and evidently saw no reason why the House
should not help her. The alderman was out of town, so she could not go
to him. After a long conversation the visitor entirely failed to get
another point of view and went away grieved and disappointed at a
refusal, thinking the resident simply disobliging; wondering, no doubt,
why such a mean woman had once been good to her; leaving the resident,
on the other hand, utterly baffled and in the state of mind she would
have been in, had she brutally insisted that a little child should lift
weights too heavy for its undeveloped muscles.

\addamsparagraph Such a situation brings out the impossibility of substituting a higher
ethical standard for a lower one without similarity of experience, but
it is not as painful as that illustrated by the following example, in
which the highest ethical standard yet attained by the charity recipient
is broken down, and the substituted one not in the least understood:---

\addamsparagraph A certain charity visitor is peculiarly appealed to by the weakness and
pathos of forlorn old age. She is responsible for the well-being of
perhaps a dozen old women to whom she sustains a sincerely affectionate
and almost filial relation. Some of them learn to take her benefactions
quite as if they came from their own relatives, grumbling at all she
does, and scolding her with a family freedom. One of these poor old
women was injured in a fire years ago. She has but the fragment of a
hand left, and is grievously crippled in her feet. Through years of pain
she had become addicted to opium, and when she first came under the
visitor's care, was only held from the poorhouse by the awful thought
that she would there perish without her drug. Five years of tender care
have done wonders for her. She lives in two neat little rooms, where
with her thumb and two fingers she makes innumerable quilts, which she
sells and gives away with the greatest delight. Her opium is regulated
to a set amount taken each day, and she has been drawn away from much
drinking. She is a voracious reader, and has her head full of strange
tales made up from books and her own imagination. At one time it seemed
impossible to do anything for her in Chicago, and she was kept for two
years in a suburb, where the family of the charity visitor lived, and
where she was nursed through several hazardous illnesses. She now lives
a better life than she did, but she is still far from being a model old
woman. The neighbors are constantly shocked by the fact that she is
supported and comforted by a ``charity lady,'' while at the same time she
occasionally ``rushes the growler,'' scolding at the boys lest they jar
her in her tottering walk. The care of her has broken through even that
second standard, which the neighborhood had learned to recognize as the
standard of charitable societies, that only the ``worthy poor'' are to be
helped; that temperance and thrift are the virtues which receive the
plums of benevolence. The old lady herself is conscious of this
criticism. Indeed, irate neighbors tell her to her face that she doesn't
in the least deserve what she gets. In order to disarm them, and at the
same time to explain what would otherwise seem loving-kindness so
colossal as to be abnormal, she tells them that during her sojourn in
the suburb she discovered an awful family secret,---a horrible scandal
connected with the long-suffering charity visitor; that it is in order
to prevent the divulgence of this that she constantly receives her
ministrations. Some of her perplexed neighbors accept this explanation
as simple and offering a solution of this vexed problem. Doubtless many
of them have a glimpse of the real state of affairs, of the love and
patience which ministers to need irrespective of worth. But the
standard is too high for most of them, and it sometimes seems
unfortunate to break down the second standard, which holds that people
who ``rush the growler'' are not worthy of charity, and that there is a
certain justice attained when they go to the poorhouse. It is certainly
dangerous to break down the lower, unless the higher is made clear.

\addamsparagraph Just when our affection becomes large enough to care for the unworthy
among the poor as we would care for the unworthy among our own kin, is
certainly a perplexing question. To say that it should never be so, is a
comment upon our democratic relations to them which few of us would be
willing to make.

\addamsparagraph Of what use is all this striving and perplexity? Has the experience any
value? It is certainly genuine, for it induces an occasional charity
visitor to live in a tenement house as simply as the other tenants do.
It drives others to give up visiting the poor altogether, because, they
claim, it is quite impossible unless the individual becomes a member of
a sisterhood, which requires, as some of the Roman Catholic sisterhoods
do, that the member first take the vows of obedience and poverty, so
that she can have nothing to give save as it is first given to her, and
thus she is not harassed by a constant attempt at adjustment.

\addamsparagraph Both the tenement-house resident and the sister assume to have put
themselves upon the industrial level of their neighbors, although they
have left out the most awful element of poverty, that of imminent fear
of starvation and a neglected old age.

\addamsparagraph The young charity visitor who goes from a family living upon a most
precarious industrial level to her own home in a prosperous part of the
city, if she is sensitive at all, is never free from perplexities which
our growing democracy forces upon her.

\addamsparagraph We sometimes say that our charity is too scientific, but we would
doubtless be much more correct in our estimate if we said that it is not
scientific enough. We dislike the entire arrangement of cards
alphabetically classified according to streets and names of families,
with the unrelated and meaningless details attached to them. Our feeling
of revolt is probably not unlike that which afflicted the students of
botany and geology in the middle of the last century, when flowers were
tabulated in alphabetical order, when geology was taught by colored
charts and thin books. No doubt the students, wearied to death, many
times said that it was all too scientific, and were much perplexed and
worried when they found traces of structure and physiology which their
so-called scientific principles were totally unable to account for. But
all this happened before science had become evolutionary and scientific
at all, before it had a principle of life from within. The very
indications and discoveries which formerly perplexed, later illumined
and made the study absorbing and vital.

\addamsparagraph We are singularly slow to apply this evolutionary principle to human
affairs in general, although it is fast being applied to the education
of children. We are at last learning to follow the development of the
child; to expect certain traits under certain conditions; to adapt
methods and matter to his growing mind. No ``advanced educator'' can allow
himself to be so absorbed in the question of what a child ought to be
as to exclude the discovery of what he is. But in our charitable efforts
we think much more of what a man ought to be than of what he is or of
what he may become; and we ruthlessly force our conventions and
standards upon him, with a sternness which we would consider stupid
indeed did an educator use it in forcing his mature intellectual
convictions upon an undeveloped mind.

\addamsparagraph Let us take the example of a timid child, who cries when he is put to
bed because he is afraid of the dark. The ``soft-hearted'' parent stays
with him, simply because he is sorry for him and wants to comfort him.
The scientifically trained parent stays with him, because he realizes
that the child is in a stage of development in which his imagination has
the best of him, and in which it is impossible to reason him out of a
belief in ghosts. These two parents, wide apart in point of view, after
all act much alike, and both very differently from the pseudo-scientific
parent, who acts from dogmatic conviction and is sure he is right. He
talks of developing his child's self-respect and good sense, and leaves
him to cry himself to sleep, demanding powers of self-control and
development which the child does not possess. There is no doubt that our
development of charity methods has reached this pseudo-scientific and
stilted stage. We have learned to condemn unthinking, ill-regulated
kind-heartedness, and we take great pride in mere repression much as the
stern parent tells the visitor below how admirably he is rearing the
child, who is hysterically crying upstairs and laying the foundation for
future nervous disorders. The pseudo-scientific spirit, or rather, the
undeveloped stage of our philanthropy, is perhaps most clearly revealed
in our tendency to lay constant stress on negative action. ``Don't give;''
``don't break down self-respect,'' we are constantly told. We distrust the
human impulse as well as the teachings of our own experience, and in
their stead substitute dogmatic rules for conduct. We forget that the
accumulation of knowledge and the holding of convictions must finally
result in the application of that knowledge and those convictions to
life itself; that the necessity for activity and a pull upon the
sympathies is so severe, that all the knowledge in the possession of the
visitor is constantly applied, and she has a reasonable chance for an
ultimate intellectual comprehension. Indeed, part of the perplexity in
the administration of charity comes from the fact that the type of
person drawn to it is the one who insists that her convictions shall not
be unrelated to action. Her moral concepts constantly tend to float away
from her, unless they have a basis in the concrete relation of life. She
is confronted with the task of reducing her scruples to action, and of
converging many wills, so as to unite the strength of all of them into
one accomplishment, the value of which no one can foresee.

\addamsparagraph On the other hand, the young woman who has succeeded in expressing her
social compunction through charitable effort finds that the wider
social activity, and the contact with the larger experience, not only
increases her sense of social obligation but at the same time recasts
her social ideals. She is chagrined to discover that in the actual task
of reducing her social scruples to action, her humble beneficiaries are
far in advance of her, not in charity or singleness of purpose, but in
self-sacrificing action. She reaches the old-time virtue of humility by
a social process, not in the old way, as the man who sits by the side of
the road and puts dust upon his head, calling himself a contrite sinner,
but she gets the dust upon her head because she has stumbled and fallen
in the road through her efforts to push forward the mass, to march with
her fellows. She has socialized her virtues not only through a social
aim but by a social process.

\addamsparagraph The Hebrew prophet made three requirements from those who would join the
great forward-moving procession led by Jehovah. ``To love mercy'' and at
the same time ``to do justly'' is the difficult task; to fulfil the first
requirement alone is to fall into the error of indiscriminate giving
with all its disastrous results; to fulfil the second solely is to
obtain the stern policy of withholding, and it results in such a dreary
lack of sympathy and understanding that the establishment of justice is
impossible. It may be that the combination of the two can never be
attained save as we fulfil still the third requirement---``to walk humbly
with God,'' which may mean to walk for many dreary miles beside the
lowliest of His creatures, not even in that peace of mind which the
company of the humble is popularly supposed to afford, but rather with
the pangs and throes to which the poor human understanding is subjected
whenever it attempts to comprehend the meaning of life.

\end{sectionbody}

%: Chapter V
\itemsection{V}{Industrial Amelioration}
\begin{sectionbody}
\addamsparagraph There is no doubt that the great difficulty we experience in reducing to
action our imperfect code of social ethics arises from the fact that we
have not yet learned to act together, and find it far from easy even to
fuse our principles and aims into a satisfactory statement. We have all
been at times entertained by the futile efforts of half a dozen highly
individualized people gathered together as a committee. Their aimless
attempts to find a common method of action have recalled the wavering
motion of a baby's arm before he has learned to co�rdinate his muscles.

\addamsparagraph If, as is many times stated, we are passing from an age of individualism
to one of association, there is no doubt that for decisive and
effective action the individual still has the best of it. He will secure
efficient results while committees are still deliberating upon the best
method of making a beginning. And yet, if the need of the times demand
associated effort, it may easily be true that the action which appears
ineffective, and yet is carried out upon the more highly developed line
of associated effort, may represent a finer social quality and have a
greater social value than the more effective individual action. It is
possible that an individual may be successful, largely because he
conserves all his powers for individual achievement and does not put any
of his energy into the training which will give him the ability to act
with others. The individual acts promptly, and we are dazzled by his
success while only dimly conscious of the inadequacy of his code.
Nowhere is this illustrated more clearly than in industrial relations,
as existing between the owner of a large factory and his employees.

\addamsparagraph A growing conflict may be detected between the democratic ideal, which
urges the workmen to demand representation in the administration of
industry, and the accepted position, that the man who owns the capital
and takes the risks has the exclusive right of management. It is in
reality a clash between individual or aristocratic management, and
corporate or democratic management. A large and highly developed factory
presents a sharp contrast between its socialized form and
individualistic ends.

\addamsparagraph It is possible to illustrate this difference by a series of events which
occurred in Chicago during the summer of 1894. These events epitomized
and exaggerated, but at the same time challenged, the code of ethics
which regulates much of our daily conduct, and clearly showed that
so-called social relations are often resting upon the will of an
individual, and are in reality regulated by a code of individual
ethics.

\addamsparagraph As this situation illustrates a point of great difficulty to which we
have arrived in our development of social ethics, it may be justifiable
to discuss it at some length. Let us recall the facts, not as they have
been investigated and printed, but as they remain in our memories.

\addamsparagraph A large manufacturing company had provided commodious workshops, and, at
the instigation of its president, had built a model town for the use of
its employees. After a series of years it was deemed necessary, during a
financial depression, to reduce the wages of these employees by giving
each workman less than full-time work ``in order to keep the shops open.''
This reduction was not accepted by the men, who had become discontented
with the factory management and the town regulations, and a strike
ensued, followed by a complete shut-down of the works. Although these
shops were non-union shops, the strikers were hastily organized and
appealed for help to the American Railway Union, which at that moment
was holding its biennial meeting in Chicago. After some days' discussion
and some futile attempts at arbitration, a sympathetic strike was
declared, which gradually involved railway men in all parts of the
country, and orderly transportation was brought to a complete
standstill. In the excitement which followed, cars were burned and
tracks torn up. The police of Chicago did not cope with the disorder,
and the railway companies, apparently distrusting the Governor of the
State, and in order to protect the United States mails, called upon the
President of the United States for the federal troops, the federal
courts further enjoined all persons against any form of interference
with the property or operation of the railroads, and the situation
gradually assumed the proportions of internecine warfare. During all of
these events the president of the manufacturing company first involved,
steadfastly refused to have the situation submitted to arbitration, and
this attitude naturally provoked much discussion. The discussion was
broadly divided between those who held that the long kindness of the
president of the company had been most ungratefully received, and those
who maintained that the situation was the inevitable outcome of the
social consciousness developing among working people. The first defended
the president of the company in his persistent refusal to arbitrate,
maintaining that arbitration was impossible after the matter had been
taken up by other than his own employees, and they declared that a man
must be allowed to run his own business. They considered the firm stand
of the president a service to the manufacturing interests of the entire
country. The others claimed that a large manufacturing concern has
ceased to be a private matter; that not only a number of workmen and
stockholders are concerned in its management, but that the interests of
the public are so involved that the officers of the company are in a
real sense administering a public trust.

\addamsparagraph This prolonged strike clearly puts in a concrete form the ethics of an
individual, in this case a benevolent employer, and the ethics of a mass
of men, his employees, claiming what they believed to be their moral
rights.

\addamsparagraph These events illustrate the difficulty of managing an industry which has
become organized into a vast social operation, not with the co�peration
of the workman thus socialized, but solely by the dictation of the
individual owning the capital. There is a sharp divergence between the
social form and the individual aim, which becomes greater as the
employees are more highly socialized and dependent. The president of the
company under discussion went further than the usual employer does. He
socialized not only the factory, but the form in which his workmen were
living. He built, and in a great measure regulated, an entire town,
without calling upon the workmen either for self-expression or
self-government. He honestly believed that he knew better than they what
was for their good, as he certainly knew better than they how to conduct
his business. As his factory developed and increased, making money each
year under his direction, he naturally expected the town to prosper in
the same way.

\addamsparagraph He did not realize that the men submitted to the undemocratic conditions
of the factory organization because the economic pressure in our
industrial affairs is so great that they could not do otherwise. Under
this pressure they could be successfully discouraged from organization,
and systematically treated on the individual basis.

\addamsparagraph Social life, however, in spite of class distinctions, is much freer than
industrial life, and the men resented the extension of industrial
control to domestic and social arrangements. They felt the lack of
democracy in the assumption that they should be taken care of in these
matters, in which even the humblest workman has won his independence.
The basic difficulty lay in the fact that an individual was directing
the social affairs of many men without any consistent effort to find out
their desires, and without any organization through which to give them
social expression. The president of the company was, moreover, so
confident of the righteousness of his aim that he had come to test the
righteousness of the process by his own feelings and not by those of the
men. He doubtless built the town from a sincere desire to give his
employees the best surroundings. As it developed, he gradually took
toward it the artist attitude toward his own creation, which has no
thought for the creation itself but is absorbed in the idea it stands
for, and he ceased to measure the usefulness of the town by the standard
of the men's needs. This process slowly darkened his glints of memory,
which might have connected his experience with that of his men. It is
possible to cultivate the impulses of the benefactor until the power of
attaining a simple human relationship with the beneficiaries, that of
frank equality with them, is gone, and there is left no mutual interest
in a common cause. To perform too many good deeds may be to lose the
power of recognizing good in others; to be too absorbed in carrying out
a personal plan of improvement may be to fail to catch the great moral
lesson which our times offer.

\addamsparagraph The president of this company fostered his employees for many years; he
gave them sanitary houses and beautiful parks; but in their extreme
need, when they were struggling with the most difficult situation which
the times could present to them, he lost his touch and had nothing
wherewith to help them. The employer's conception of goodness for his
men had been cleanliness, decency of living, and, above all, thrift and
temperance. Means had been provided for all this, and opportunities had
also been given for recreation and improvement. But this employer
suddenly found his town in the sweep of a world-wide moral impulse. A
movement had been going on about him and among his working men, of which
he had been unconscious, or concerning which he had heard only by rumor.

\addamsparagraph Outside the ken of philanthropists the proletariat had learned to say in
many languages, that ``the injury of one is the concern of all.'' Their
watchwords were brotherhood, sacrifice, the subordination of individual
and trade interests, to the good of the working classes, and they were
moved by a determination to free that class from the untoward conditions
under which they were laboring.

\addamsparagraph Compared to these watchwords, the old ones which this philanthropic
employer had given his town were negative and inadequate. He had
believed strongly in temperance and steadiness of individual effort, but
had failed to apprehend the greater movement of combined abstinence and
concerted action. With all his fostering, the president had not attained
to a conception of social morality for his men and had imagined that
virtue for them largely meant absence of vice.

\addamsparagraph When the labor movement finally stirred his town, or, to speak more
fairly, when, in their distress and perplexity, his own employees
appealed to an organized manifestation of this movement, they were quite
sure that simply because they were workmen in distress they would not be
deserted by it. This loyalty on the part of a widely ramified and
well-organized union toward the workmen in a ``non-union shop,'' who had
contributed nothing to its cause, was certainly a manifestation of moral
power.

\addamsparagraph In none of his utterances or correspondence did the president for an
instant recognize this touch of nobility, although one would imagine
that he would gladly point out this bit of virtue, in what he must have
considered the moral ruin about him. He stood throughout for the
individual virtues, those which had distinguished the model workmen of
his youth; those which had enabled him and so many of his
contemporaries to rise in life, when ``rising in life'' was urged upon
every promising boy as the goal of his efforts.

\addamsparagraph Of the code of social ethics he had caught absolutely nothing. The
morals he had advocated in selecting and training his men did not fail
them in the hour of confusion. They were self-controlled, and they
themselves destroyed no property. They were sober and exhibited no
drunkenness, even although obliged to hold their meetings in the saloon
hall of a neighboring town. They repaid their employer in kind, but he
had given them no rule for the life of association into which they were
plunged.

\addamsparagraph The president of the company desired that his employees should possess
the individual and family virtues, but did nothing to cherish in them
the social virtues which express themselves in associated effort.

\addamsparagraph Day after day, during that horrible time of suspense, when the wires
constantly reported the same message, ``the President of the Company
holds that there is nothing to arbitrate,'' one was forced to feel that
the ideal of one-man rule was being sustained in its baldest form. A
demand from many parts of the country and from many people was being
made for social adjustment, against which the commercial training and
the individualistic point of view held its own successfully.

\addamsparagraph The majority of the stockholders, not only of this company but of
similar companies, and many other citizens, who had had the same
commercial experience, shared and sustained this position. It was quite
impossible for them to catch the other point of view. They not only felt
themselves right from the commercial standpoint, but had gradually
accustomed themselves also to the philanthropic standpoint, until they
had come to consider their motives beyond reproach. Habit held them
persistent in this view of the case through all changing conditions.

\addamsparagraph A wise man has said that ``the consent of men and your own conscience
are two wings given you whereby you may rise to God.'' It is so easy for
the good and powerful to think that they can rise by following the
dictates of conscience, by pursuing their own ideals, that they are
prone to leave those ideals unconnected with the consent of their
fellow-men. The president of the company thought out within his own mind
a beautiful town. He had power with which to build this town, but he did
not appeal to nor obtain the consent of the men who were living in it.
The most unambitious reform, recognizing the necessity for this consent,
makes for slow but sane and strenuous progress, while the most ambitious
of social plans and experiments, ignoring this, is prone to failure.

\addamsparagraph The man who insists upon consent, who moves with the people, is bound to
consult the ``feasible right'' as well as the absolute right. He is often
obliged to attain only Mr. Lincoln's ``best possible,'' and then has the
sickening sense of compromise with his best convictions. He has to move
along with those whom he leads toward a goal that neither he nor they
see very clearly till they come to it. He has to discover what people
really want, and then ``provide the channels in which the growing moral
force of their lives shall flow.'' What he does attain, however, is not
the result of his individual striving, as a solitary mountain-climber
beyond that of the valley multitude but it is sustained and upheld by
the sentiments and aspirations of many others. Progress has been slower
perpendicularly, but incomparably greater because lateral. He has not
taught his contemporaries to climb mountains, but he has persuaded the
villagers to move up a few feet higher; added to this, he has made
secure his progress. A few months after the death of the promoter of
this model town, a court decision made it obligatory upon the company to
divest itself of the management of the town as involving a function
beyond its corporate powers. The parks, flowers, and fountains of this
far-famed industrial centre were dismantled, with scarcely a protest
from the inhabitants themselves.

\addamsparagraph The man who disassociates his ambition, however disinterested, from the
co�peration of his fellows, always takes this risk of ultimate failure.
He does not take advantage of the great conserver and guarantee of his
own permanent success which associated efforts afford. Genuine
experiments toward higher social conditions must have a more democratic
faith and practice than those which underlie private venture. Public
parks and improvements, intended for the common use, are after all only
safe in the hands of the public itself; and associated effort toward
social progress, although much more awkward and stumbling than that same
effort managed by a capable individual, does yet enlist deeper forces
and evoke higher social capacities.

\addamsparagraph The successful business man who is also the philanthropist is in more
than the usual danger of getting widely separated from his employees.
The men already have the American veneration for wealth and successful
business capacity, and, added to this, they are dazzled by his good
works. The workmen have the same kindly impulses as he, but while they
organize their charity into mutual benefit associations and distribute
their money in small amounts in relief for the widows and insurance for
the injured, the employer may build model towns, erect college
buildings, which are tangible and enduring, and thereby display his
goodness in concentrated form.

\addamsparagraph By the very exigencies of business demands, the employer is too often
cut off from the social ethics developing in regard to our larger social
relationships, and from the great moral life springing from our common
experiences. This is sure to happen when he is good ``to'' people rather
than ``with'' them, when he allows himself to decide what is best for them
instead of consulting them. He thus misses the rectifying influence of
that fellowship which is so big that it leaves no room for
sensitiveness or gratitude. Without this fellowship we may never know
how great the divergence between ourselves and others may become, nor
how cruel the misunderstandings.

\addamsparagraph During a recent strike of the employees of a large factory in Ohio, the
president of the company expressed himself as bitterly disappointed by
the results of his many kindnesses, and evidently considered the
employees utterly unappreciative. His state of mind was the result of
the fallacy of ministering to social needs from an individual impulse
and expecting a socialized return of gratitude and loyalty. If the
lunch-room was necessary, it was a necessity in order that the employees
might have better food, and, when they had received the better food, the
legitimate aim of the lunch-room was met. If baths were desirable, and
the fifteen minutes of calisthenic exercise given the women in the
middle of each half day brought a needed rest and change to their
muscles, then the increased cleanliness and the increased bodily
comfort of so many people should of themselves have justified the
experiment.

\addamsparagraph To demand, as a further result, that there should be no strikes in the
factory, no revolt against the will of the employer because the
employees were filled with loyalty as the result of the kindness, was of
course to take the experiment from an individual basis to a social one.

\addamsparagraph Large mining companies and manufacturing concerns are constantly
appealing to their stockholders for funds, or for permission to take a
percentage of the profits, in order that the money may be used for
educational and social schemes designed for the benefit of the
employees. The promoters of these schemes use as an argument and as an
appeal, that better relations will be thus established, that strikes
will be prevented, and that in the end the money returned to the
stockholders will be increased. However praiseworthy this appeal may be
in motive, it involves a distinct confusion of issues, and in theory
deserves the failure it so often meets with in practice. In the clash
which follows a strike, the employees are accused of an ingratitude,
when there was no legitimate reason to expect gratitude; and useless
bitterness, which has really a factitious basis, may be developed on
both sides.

\addamsparagraph Indeed, unless the relation becomes a democratic one, the chances of
misunderstanding are increased, when to the relation of employer and
employees is added the relation of benefactor to beneficiaries, in so
far as there is still another opportunity for acting upon the individual
code of ethics.

\addamsparagraph There is no doubt that these efforts are to be commended, not only from
the standpoint of their social value but because they have a marked
industrial significance. Failing, as they do, however, to touch the
question of wages and hours, which are almost invariably the points of
trades-union effort, the employers confuse the mind of the public when
they urge the amelioration of conditions and the kindly relation
existing between them and their men as a reason for the discontinuance
of strikes and other trades-union tactics. The men have individually
accepted the kindness of the employers as it was individually offered,
but quite as the latter urges his inability to increase wages unless he
has the co�peration of his competitors, so the men state that they are
bound to the trades-union struggle for an increase in wages because it
can only be undertaken by combinations of labor.

\addamsparagraph Even the much more democratic effort to divide a proportion of the
profits at the end of the year among the employees, upon the basis of
their wages and efficiency, is also exposed to a weakness, from the fact
that the employing side has the power of determining to whom the benefit
shall accrue.

\addamsparagraph Both individual acts of self-defence on the part of the wage earner and
individual acts of benevolence on the part of the employer are most
useful as they establish standards to which the average worker and
employer may in time be legally compelled to conform. Progress must
always come through the individual who varies from the type and has
sufficient energy to express this variation. He first holds a higher
conception than that held by the mass of his fellows of what is
righteous under given conditions, and expresses this conviction in
conduct, in many instances formulating a certain scruple which the
others share, but have not yet defined even to themselves. Progress,
however, is not secure until the mass has conformed to this new
righteousness. This is equally true in regard to any advance made in the
standard of living on the part of the trades-unionists or in the
improved conditions of industry on the part of reforming employers. The
mistake lies, not in overpraising the advance thus inaugurated by
individual initiative, but in regarding the achievement as complete in a
social sense when it is still in the realm of individual action. No sane
manufacturer regards his factory as the centre of the industrial system.
He knows very well that the cost of material, wages, and selling prices
are determined by industrial conditions completely beyond his control.
Yet the same man may quite calmly regard himself and his own private
principles as merely self-regarding, and expect results from casual
philanthropy which can only be accomplished through those common rules
of life and labor established by the community for the common good.

\addamsparagraph Outside of and surrounding these smaller and most significant efforts
are the larger and irresistible movements operating toward combination.
This movement must tend to decide upon social matters from the social
standpoint. Until then it is difficult to keep our minds free from a
confusion of issues. Such a confusion occurs when the gift of a large
sum to the community for a public and philanthropic purpose, throws a
certain glamour over all the earlier acts of a man, and makes it
difficult for the community to see possible wrongs committed against it,
in the accumulation of wealth so beneficently used. It is possible also
that the resolve to be thus generous unconsciously influences the man
himself in his methods of accumulation. He keeps to a certain individual
rectitude, meaning to make an individual restitution by the old paths of
generosity and kindness, whereas if he had in view social restitution on
the newer lines of justice and opportunity, he would throughout his
course doubtless be watchful of his industrial relationships and his
social virtues.

\addamsparagraph The danger of professionally attaining to the power of the righteous
man, of yielding to the ambition ``for doing good'' on a large scale,
compared to which the ambition for politics, learning, or wealth, are
vulgar and commonplace, ramifies through our modern life; and those most
easily beset by this temptation are precisely the men best situated to
experiment on the larger social lines, because they so easily dramatize
their acts and lead public opinion. Very often, too, they have in their
hands the preservation and advancement of large vested interests, and
often see clearly and truly that they are better able to administer the
affairs of the community than the community itself: sometimes they see
that if they do not administer them sharply and quickly, as only an
individual can, certain interests of theirs dependent upon the community
will go to ruin.

\addamsparagraph The model employer first considered, provided a large sum in his will
with which to build and equip a polytechnic school, which will doubtless
be of great public value. This again shows the advantage of individual
management, in the spending as well as in the accumulating of wealth,
but this school will attain its highest good, in so far as it incites
the ambition to provide other schools from public funds. The town of
Zurich possesses a magnificent polytechnic institute, secured by the
vote of the entire people and supported from public taxes. Every man who
voted for it is interested that his child should enjoy its benefits,
and, of course, the voluntary attendance must be larger than in a
school accepted as a gift to the community.

\addamsparagraph In the educational efforts of model employers, as in other attempts
toward social amelioration, one man with the best of intentions is
trying to do what the entire body of employees should have undertaken to
do for themselves. The result of his efforts will only attain its
highest value as it serves as an incentive to procure other results by
the community as well as for the community.

\addamsparagraph There are doubtless many things which the public would never demand
unless they were first supplied by individual initiative, both because
the public lacks the imagination, and also the power of formulating
their wants. Thus philanthropic effort supplies kindergartens, until
they become so established in the popular affections that they are
incorporated in the public school system. Churches and missions
establish reading rooms, until at last the public library system dots
the city with branch reading rooms and libraries. For this willingness
to take risks for the sake of an ideal, for those experiments which must
be undertaken with vigor and boldness in order to secure didactic value
in failure as well as in success, society must depend upon the
individual possessed with money, and also distinguished by earnest and
unselfish purpose. Such experiments enable the nation to use the
Referendum method in its public affairs. Each social experiment is thus
tested by a few people, given wide publicity, that it may be observed
and discussed by the bulk of the citizens before the public prudently
makes up its mind whether or not it is wise to incorporate it into the
functions of government. If the decision is in its favor and it is so
incorporated, it can then be carried on with confidence and enthusiasm.

\addamsparagraph But experience has shown that we can only depend upon successful men for
a certain type of experiment in the line of industrial amelioration and
social advancement. The list of those who found churches, educational
institutions, libraries, and art galleries, is very long, as is again
the list of those contributing to model dwellings, recreation halls, and
athletic fields. At the present moment factory employers are doing much
to promote ``industrial betterment'' in the way of sanitary surroundings,
opportunities for bathing, lunch rooms provided with cheap and wholesome
food, club rooms, and guild halls. But there is a line of social
experiment involving social righteousness in its most advanced form, in
which the number of employers and the ``favored class'' are so few that it
is plain society cannot count upon them for continuous and valuable
help. This lack is in the line of factory legislation and that sort of
social advance implied in shorter hours and the regulation of wages; in
short, all that organization and activity that is involved in such a
maintenance and increase of wages as would prevent the lowering of the
standard of life.

\addamsparagraph A large body of people feel keenly that the present industrial system
is in a state of profound disorder, and that there is no guarantee that
the pursuit of individual ethics will ever right it. They claim that
relief can only come through deliberate corporate effort inspired by
social ideas and guided by the study of economic laws, and that the
present industrial system thwarts our ethical demands, not only for
social righteousness but for social order. Because they believe that
each advance in ethics must be made fast by a corresponding advance in
politics and legal enactment, they insist upon the right of state
regulation and control. While many people representing all classes in a
community would assent to this as to a general proposition, and would
even admit it as a certain moral obligation, legislative enactments
designed to control industrial conditions have largely been secured
through the efforts of a few citizens, mostly those who constantly see
the harsh conditions of labor and who are incited to activity by their
sympathies as well as their convictions.

\addamsparagraph This may be illustrated by the series of legal enactments regulating the
occupations in which children may be allowed to work, also the laws in
regard to the hours of labor permitted in those occupations, and the
minimum age below which children may not be employed. The first child
labor laws were enacted in England through the efforts of those members
of parliament whose hearts were wrung by the condition of the little
parish apprentices bound out to the early textile manufacturers of the
north; and through the long years required to build up the code of child
labor legislation which England now possesses, knowledge of the
conditions has always preceded effective legislation. The efforts of
that small number in every community who believe in legislative control
have always been re�nforced by the efforts of trades-unionists rather
than by the efforts of employers. Partly because the employment of
workingmen in the factories brings them in contact with the children who
tend to lower wages and demoralize their trades, and partly because
workingmen have no money nor time to spend in alleviating philanthropy,
and must perforce seize upon agitation and legal enactment as the only
channel of redress which is open to them.

\addamsparagraph We may illustrate by imagining a row of people seated in a moving
street-car, into which darts a boy of eight, calling out the details of
the last murder, in the hope of selling an evening newspaper. A
comfortable looking man buys a paper from him with no sense of moral
shock; he may even be a trifle complacent that he has helped along the
little fellow, who is making his way in the world. The philanthropic
lady sitting next to him may perhaps reflect that it is a pity that such
a bright boy is not in school. She may make up her mind in a moment of
compunction to redouble her efforts for various newsboys' schools and
homes, that this poor child may have better teaching, and perhaps a
chance at manual training. She probably is convinced that he alone, by
his unaided efforts, is supporting a widowed mother, and her heart is
moved to do all she can for him. Next to her sits a workingman trained
in trades-union methods. He knows that the boy's natural development is
arrested, and that the abnormal activity of his body and mind uses up
the force which should go into growth; moreover, that this premature use
of his powers has but a momentary and specious value. He is forced to
these conclusions because he has seen many a man, entering the factory
at eighteen and twenty, so worn out by premature work that he was ``laid
on the shelf'' within ten or fifteen years. He knows very well that he
can do nothing in the way of ameliorating the lot of this particular
boy; that his only possible chance is to agitate for proper child-labor
laws; to regulate, and if possible prohibit, street-vending by children,
in order that the child of the poorest may have his school time secured
to him, and may have at least his short chance for growth.

\addamsparagraph These three people, sitting in the street car, are all honest and
upright, and recognize a certain duty toward the forlorn children of the
community. The self-made man is encouraging one boy's own efforts; the
philanthropic lady is helping on a few boys; the workingman alone is
obliged to include all the boys of his class. Workingmen, because of
their feebleness in all but numbers, have been forced to appeal to the
state, in order to secure protection for themselves and for their
children. They cannot all rise out of their class, as the occasionally
successful man has done; some of them must be left to do the work in the
factories and mines, and they have no money to spend in philanthropy.

\addamsparagraph Both public agitation and a social appeal to the conscience of the
community is necessary in order to secure help from the state, and,
curiously enough, child-labor laws once enacted and enforced are a
matter of great pride, and even come to be regarded as a register of the
community's humanity and enlightenment. If the method of public
agitation could find quiet and orderly expression in legislative
enactment, and if labor measures could be submitted to the examination
and judgment of the whole without a sense of division or of warfare, we
should have the ideal development of the democratic state.

\addamsparagraph But we judge labor organizations as we do other living institutions, not
by their declaration of principles, which we seldom read, but by their
blundering efforts to apply their principles to actual conditions, and
by the oft-time failure of their representatives, when the individual
finds himself too weak to become the organ of corporate action.

\addamsparagraph The very blunders and lack of organization too often characterizing a
union, in marked contrast to the orderly management of a factory, often
confuse us as to the real issues involved, and we find it hard to trust
uncouth and unruly manifestations of social effort. The situation is
made even more complicated by the fact that those who are formulating a
code of associated action so often break through the established code of
law and order. As society has a right to demand of the reforming
individual that he be sternly held to his personal and domestic claims,
so it has a right to insist that labor organizations shall keep to the
hardly won standards of public law and order; and the community performs
but its plain duty when it registers its protest every time law and
order are subverted, even in the interest of the so-called social
effort. Yet in moments of industrial stress and strain the community is
confronted by a moral perplexity which may arise from the mere fact that
the good of yesterday is opposed to the good of today, and that which
may appear as a choice between virtue and vice is really but a choice
between virtue and virtue. In the disorder and confusion sometimes
incident to growth and progress, the community may be unable to see
anything but the unlovely struggle itself.

\addamsparagraph The writer recalls a conversation between two workingmen who were
leaving a lecture on ``Organic Evolution.'' The first was much puzzled,
and anxiously inquired of the second ``if evolution could mean that one
animal turned into another.'' The challenged workman stopped in the rear
of the hall, put his foot upon a chair, and expounded what he thought
evolution did mean; and this, so nearly as the conversation can be
recalled, is what he said: ``You see a lot of fishes are living in a
stream, which overflows in the spring and strands some of them upon the
bank. The weak ones die up there, but others make a big effort to get
back into the water. They dig their fins into the sand, breathe as much
air as they can with their gills, and have a terrible time. But after a
while their fins turn into legs and their gills into lungs, and they
have become frogs. Of course they are further along than the sleek,
comfortable fishes who sail up and down the stream waving their tails
and despising the poor damaged things thrashing around on the bank.
He---the lecturer---did not say anything about men, but it is easy enough
to think of us poor devils on the dry bank, struggling without enough to
live on, while the comfortable fellows sail along in the water with all
they want and despise us because we thrash about.'' His listener did not
reply, and was evidently dissatisfied both with the explanation and the
application. Doubtless the illustration was bungling in more than its
setting forth, but the story is suggestive.

\addamsparagraph At times of social disturbance the law-abiding citizen is naturally so
anxious for peace and order, his sympathies are so justly and inevitably
on the side making for the restoration of law, that it is difficult for
him to see the situation fairly. He becomes insensible to the unselfish
impulse which may prompt a sympathetic strike in behalf of the workers
in a non-union shop, because he allows his mind to dwell exclusively on
the disorder which has become associated with the strike. He is
completely side-tracked by the ugly phases of a great moral movement. It
is always a temptation to assume that the side which has respectability,
authority, and superior intelligence, has therefore righteousness as
well, especially when the same side presents concrete results of
individual effort as over against the less tangible results of
associated effort.

\addamsparagraph It is as yet most difficult for us to free ourselves from the
individualistic point of view sufficiently to group events in their
social relations and to judge fairly those who are endeavoring to
produce a social result through all the difficulties of associated
action. The philanthropist still finds his path much easier than do
those who are attempting a social morality. In the first place, the
public, anxious to praise what it recognizes as an undoubted moral
effort often attended with real personal sacrifice, joyfully seizes upon
this manifestation and overpraises it, recognizing the philanthropist
as an old friend in the paths of righteousness, whereas the others are
strangers and possibly to be distrusted as aliens. It is easy to confuse
the response to an abnormal number of individual claims with the
response to the social claim. An exaggerated personal morality is often
mistaken for a social morality, and until it attempts to minister to a
social situation its total inadequacy is not discovered. To attempt to
attain a social morality without a basis of democratic experience
results in the loss of the only possible corrective and guide, and ends
in an exaggerated individual morality but not in social morality at all.
We see this from time to time in the care-worn and overworked
philanthropist, who has taxed his individual will beyond the normal
limits and has lost his clew to the situation among a bewildering number
of cases. A man who takes the betterment of humanity for his aim and end
must also take the daily experiences of humanity for the constant
correction of his process. He must not only test and guide his
achievement by human experience, but he must succeed or fail in
proportion as he has incorporated that experience with his own.
Otherwise his own achievements become his stumbling-block, and he comes
to believe in his own goodness as something outside of himself. He makes
an exception of himself, and thinks that he is different from the rank
and file of his fellows. He forgets that it is necessary to know of the
lives of our contemporaries, not only in order to believe in their
integrity, which is after all but the first beginnings of social
morality, but in order to attain to any mental or moral integrity for
ourselves or any such hope for society.

\end{sectionbody}

%: Chapter VII
\itemsection{VII}{Political reform}
\begin{sectionbody}

\addamsparagraph Throughout this volume we have assumed that much of our ethical
maladjustment in social affairs arises from the fact that we are acting
upon a code of ethics adapted to individual relationships, but not to
the larger social relationships to which it is bunglingly applied. In
addition, however, to the consequent strain and difficulty, there is
often an honest lack of perception as to what the situation demands.

\addamsparagraph Nowhere is this more obvious than in our political life as it manifests
itself in certain quarters of every great city. It is most difficult to
hold to our political democracy and to make it in any sense a social
expression and not a mere governmental contrivance, unless we take pains
to keep on common ground in our human experiences. Otherwise there is
in various parts of the community an inevitable difference of ethical
standards which becomes responsible for much misunderstanding.

\addamsparagraph It is difficult both to interpret sympathetically the motives and ideals
of those who have acquired rules of conduct in experience widely
different from our own, and also to take enough care in guarding the
gains already made, and in valuing highly enough the imperfect good so
painfully acquired and, at the best, so mixed with evil. This wide
difference in daily experience exhibits itself in two distinct attitudes
toward politics. The well-to-do men of the community think of politics
as something off by itself; they may conscientiously recognize political
duty as part of good citizenship, but political effort is not the
expression of their moral or social life. As a result of this
detachment, ``reform movements,'' started by business men and the better
element, are almost wholly occupied in the correction of political
machinery and with a concern for the better method of administration,
rather than with the ultimate purpose of securing the welfare of the
people. They fix their attention so exclusively on methods that they
fail to consider the final aims of city government. This accounts for
the growing tendency to put more and more responsibility upon executive
officers and appointed commissions at the expense of curtailing the
power of the direct representatives of the voters. Reform movements tend
to become negative and to lose their educational value for the mass of
the people. The reformers take the r�le of the opposition. They give
themselves largely to criticisms of the present state of affairs, to
writing and talking of what the future must be and of certain results
which should be obtained. In trying to better matters, however, they
have in mind only political achievements which they detach in a curious
way from the rest of life, and they speak and write of the purification
of politics as of a thing set apart from daily life.

\addamsparagraph On the other hand, the real leaders of the people are part of the entire
life of the community which they control, and so far as they are
representative at all, are giving a social expression to democracy. They
are often politically corrupt, but in spite of this they are proceeding
upon a sounder theory. Although they would be totally unable to give it
abstract expression, they are really acting upon a formulation made by a
shrewd English observer; namely, that, ``after the enfranchisement of the
masses, social ideals enter into political programmes, and they enter
not as something which at best can be indirectly promoted by government,
but as something which it is the chief business of government to advance
directly.''

\addamsparagraph Men living near to the masses of voters, and knowing them intimately,
recognize this and act upon it; they minister directly to life and to
social needs. They realize that the people as a whole are clamoring for
social results, and they hold their power because they respond to that
demand. They are corrupt and often do their work badly; but they at
least avoid the mistake of a certain type of business men who are
frightened by democracy, and have lost their faith in the people. The
two standards are similar to those seen at a popular exhibition of
pictures where the cultivated people care most for the technique of a
given painting, the moving mass for a subject that shall be domestic and
human.

\addamsparagraph This difference may be illustrated by the writer's experience in a
certain ward of Chicago, during three campaigns, when efforts were made
to dislodge an alderman who had represented the ward for many years. In
this ward there are gathered together fifty thousand people,
representing a score of nationalities; the newly emigrated Latin,
Teuton, Celt, Greek, and Slav who live there have little in common save
the basic experiences which come to men in all countries and under all
conditions. In order to make fifty thousand people, so heterogeneous in
nationality, religion, and customs, agree upon any demand, it must be
founded upon universal experiences which are perforce individual and not
social.

\addamsparagraph An instinctive recognition of this on the part of the alderman makes it
possible to understand the individualistic basis of his political
success, but it remains extremely difficult to ascertain the reasons for
the extreme leniency of judgment concerning the political corruption of
which he is constantly guilty.

\addamsparagraph This leniency is only to be explained on the ground that his
constituents greatly admire individual virtues, and that they are at the
same time unable to perceive social outrages which the alderman may be
committing. They thus free the alderman from blame because his
corruption is social, and they honestly admire him as a great man and
hero, because his individual acts are on the whole kindly and generous.

\addamsparagraph In certain stages of moral evolution, a man is incapable of action
unless the results will benefit himself or some one of his
acquaintances, and it is a long step in moral progress to set the good
of the many before the interest of the few, and to be concerned for the
welfare of a community without hope of an individual return. How far the
selfish politician befools his constituents into believing that their
interests are identical with his own; how far he presumes upon their
inability to distinguish between the individual and social virtues, an
inability which he himself shares with them; and how far he dazzles them
by the sense of his greatness, and a conviction that they participate
therein, it is difficult to determine.

\addamsparagraph Morality certainly develops far earlier in the form of moral fact than
in the form of moral ideas, and it is obvious that ideas only operate
upon the popular mind through will and character, and must be dramatized
before they reach the mass of men, even as the biography of the saints
have been after all ``the main guide to the stumbling feet of thousands
of Christians to whom the Credo has been but mysterious words.''

\addamsparagraph Ethics as well as political opinions may be discussed and disseminated
among the sophisticated by lectures and printed pages, but to the common
people they can only come through example---through a personality which
seizes the popular imagination. The advantage of an unsophisticated
neighborhood is, that the inhabitants do not keep their ideas as
treasures---they are untouched by the notion of accumulating them, as
they might knowledge or money, and they frankly act upon those they
have. The personal example promptly rouses to emulation. In a
neighborhood where political standards are plastic and undeveloped, and
where there has been little previous experience in self-government, the
office-holder himself sets the standard, and the ideas that cluster
around him exercise a specific and permanent influence upon the
political morality of his constituents.

\addamsparagraph Nothing is more certain than that the quality which a heterogeneous
population, living in one of the less sophisticated wards, most admires
is the quality of simple goodness; that the man who attracts them is the
one whom they believe to be a good man. We all know that children long
``to be good'' with an intensity which they give to no other ambition. We
can all remember that the earliest strivings of our childhood were in
this direction, and that we venerated grown people because they had
attained perfection.

\addamsparagraph Primitive people, such as the South Italian peasants, are still in this
stage. They want to be good, and deep down in their hearts they admire
nothing so much as the good man. Abstract virtues are too difficult for
their untrained minds to apprehend, and many of them are still simple
enough to believe that power and wealth come only to good people.

\addamsparagraph The successful candidate, then, must be a good man according to the
morality of his constituents. He must not attempt to hold up too high a
standard, nor must he attempt to reform or change their standards. His
safety lies in doing on a large scale the good deeds which his
constituents are able to do only on a small scale. If he believes what
they believe and does what they are all cherishing a secret ambition to
do, he will dazzle them by his success and win their confidence. There
is a certain wisdom in this course. There is a common sense in the mass
of men which cannot be neglected with impunity, just as there is sure to
be an eccentricity in the differing and reforming individual which it is
perhaps well to challenge.

\addamsparagraph The constant kindness of the poor to each other was pointed out in a
previous chapter, and that they unfailingly respond to the need and
distresses of their poorer neighbors even when in danger of bankruptcy
themselves. The kindness which a poor man shows his distressed neighbor
is doubtless heightened by the consciousness that he himself may be in
distress next week; he therefore stands by his friend when he gets too
drunk to take care of himself, when he loses his wife or child, when he
is evicted for non-payment of rent, when he is arrested for a petty
crime. It seems to such a man entirely fitting that his alderman should
do the same thing on a larger scale---that he should help a constituent
out of trouble, merely because he is in trouble, irrespective of the
justice involved.

\addamsparagraph The alderman therefore bails out his constituents when they are
arrested, or says a good word to the police justice when they appear
before him for trial, uses his pull with the magistrate when they are
likely to be fined for a civil misdemeanor, or sees what he can do to
``fix up matters'' with the state's attorney when the charge is really a
serious one, and in doing this he follows the ethics held and practised
by his constituents. All this conveys the impression to the
simple-minded that law is not enforced, if the lawbreaker have a
powerful friend. One may instance the alderman's action in standing by
an Italian padrone of the ward when he was indicted for violating the
civil service regulations. The commissioners had sent out notices to
certain Italian day-laborers who were upon the eligible list that they
were to report for work at a given day and hour. One of the padrones
intercepted these notifications and sold them to the men for five
dollars apiece, making also the usual bargain for a share of their
wages. The padrone's entire arrangement followed the custom which had
prevailed for years before the establishment of civil service laws. Ten
of the laborers swore out warrants against the padrone, who was
convicted and fined seventy-five dollars. This sum was promptly paid by
the alderman, and the padrone, assured that he would be protected from
any further trouble, returned uninjured to the colony. The simple
Italians were much bewildered by this show of a power stronger than that
of the civil service, which they had trusted as they did the one in
Italy. The first violation of its authority was made, and various
sinister acts have followed, until no Italian who is digging a sewer or
sweeping a street for the city feels quite secure in holding his job
unless he is backed by the friendship of the alderman. According to the
civil service law, a laborer has no right to a trial; many are
discharged by the foreman, and find that they can be reinstated only
upon the aldermanic recommendation. He thus practically holds his old
power over the laborers working for the city. The popular mind is
convinced that an honest administration of civil service is impossible,
and that it is but one more instrument in the hands of the powerful.

\addamsparagraph It will be difficult to establish genuine civil service among these men,
who learn only by experience, since their experiences have been of such
a nature that their unanimous vote would certainly be that ``civil
service'' is ``no good.''

\addamsparagraph As many of his constituents in this case are impressed with the fact
that the aldermanic power is superior to that of government, so
instances of actual lawbreaking might easily be cited. A young man may
enter a saloon long after midnight, the legal closing hour, and seat
himself at a gambling table, perfectly secure from interruption or
arrest, because the place belongs to an alderman; but in order to secure
this immunity the policeman on the beat must pretend not to see into the
windows each time that he passes, and he knows, and the young man knows
that he knows, that nothing would embarrass ``Headquarters'' more than to
have an arrest made on those premises. A certain contempt for the whole
machinery of law and order is thus easily fostered.

\addamsparagraph Because of simple friendliness the alderman is expected to pay rent for
the hard-pressed tenant when no rent is forthcoming, to find ``jobs'' when
work is hard to get, to procure and divide among his constituents all
the places which he can seize from the city hall. The alderman of the
ward we are considering at one time could make the proud boast that he
had twenty-six hundred people in his ward upon the public pay-roll.
This, of course, included day laborers, but each one felt under
distinct obligations to him for getting a position. When we reflect that
this is one-third of the entire vote of the ward, we realize that it is
very important to vote for the right man, since there is, at the least,
one chance out of three for securing work.

\addamsparagraph If we recollect further that the franchise-seeking companies pay
respectful heed to the applicants backed by the alderman, the question
of voting for the successful man becomes as much an industrial one as a
political one. An Italian laborer wants a ``job'' more than anything else,
and quite simply votes for the man who promises him one. It is not so
different from his relation to the padrone, and, indeed, the two
strengthen each other.

\addamsparagraph The alderman may himself be quite sincere in his acts of kindness, for
an office seeker may begin with the simple desire to alleviate
suffering, and this may gradually change into the desire to put his
constituents under obligations to him; but the action of such an
individual becomes a demoralizing element in the community when kindly
impulse is made a cloak for the satisfaction of personal ambition, and
when the plastic morals of his constituents gradually conform to his own
undeveloped standards.

\addamsparagraph The alderman gives presents at weddings and christenings. He seizes
these days of family festivities for making friends. It is easiest to
reach them in the holiday mood of expansive good-will, but on their side
it seems natural and kindly that he should do it. The alderman procures
passes from the railroads when his constituents wish to visit friends or
attend the funerals of distant relatives; he buys tickets galore for
benefit entertainments given for a widow or a consumptive in peculiar
distress; he contributes to prizes which are awarded to the handsomest
lady or the most popular man. At a church bazaar, for instance, the
alderman finds the stage all set for his dramatic performance. When
others are spending pennies, he is spending dollars. When anxious
relatives are canvassing to secure votes for the two most beautiful
children who are being voted upon, he recklessly buys votes from both
sides, and laughingly declines to say which one he likes best, buying
off the young lady who is persistently determined to find out, with five
dollars for the flower bazaar, the posies, of course, to be sent to the
sick of the parish. The moral atmosphere of a bazaar suits him exactly.
He murmurs many times, ``Never mind, the money all goes to the poor; it
is all straight enough if the church gets it, the poor won't ask too
many questions.'' The oftener he can put such sentiments into the minds
of his constituents, the better he is pleased. Nothing so rapidly
prepares them to take his view of money getting and money spending. We
see again the process disregarded, because the end itself is considered
so praiseworthy.

\addamsparagraph There is something archaic in a community of simple people in their
attitude toward death and burial. There is nothing so easy to collect
money for as a funeral, and one involuntarily remembers that the early
religious tithes were paid to ward off death and ghosts. At times one
encounters almost the Greek feeling in regard to burial. If the alderman
seizes upon times of festivities for expressions of his good-will, much
more does he seize upon periods of sorrow. At a funeral he has the
double advantage of ministering to a genuine craving for comfort and
solace, and at the same time of assisting a bereaved constituent to
express that curious feeling of remorse, which is ever an accompaniment
of quick sorrow, that desire to ``make up'' for past delinquencies, to
show the world how much he loved the person who has just died, which is
as natural as it is universal.

\addamsparagraph In addition to this, there is, among the poor, who have few social
occasions, a great desire for a well-arranged funeral, the grade of
which almost determines their social standing in the neighborhood. The
alderman saves the very poorest of his constituents from that awful
horror of burial by the county; he provides carriages for the poor, who
otherwise could not have them. It may be too much to say that all the
relatives and friends who ride in the carriages provided by the
alderman's bounty vote for him, but they are certainly influenced by his
kindness, and talk of his virtues during the long hours of the ride back
and forth from the suburban cemetery. A man who would ask at such a time
where all the money thus spent comes from would be considered sinister.
The tendency to speak lightly of the faults of the dead and to judge
them gently is transferred to the living, and many a man at such a time
has formulated a lenient judgment of political corruption, and has heard
kindly speeches which he has remembered on election day. ``Ah, well, he
has a big Irish heart. He is good to the widow and the fatherless.'' ``He
knows the poor better than the big guns who are always talking about
civil service and reform.''

\addamsparagraph Indeed, what headway can the notion of civic purity, of honesty of
administration make against this big manifestation of human
friendliness, this stalking survival of village kindness? The notions of
the civic reformer are negative and impotent before it. Such an alderman
will keep a standing account with an undertaker, and telephone every
week, and sometimes more than once, the kind of funeral he wishes
provided for a bereaved constituent, until the sum may roll up into
``hundreds a year.'' He understands what the people want, and ministers
just as truly to a great human need as the musician or the artist. An
attempt to substitute what we might call a later standard was made at
one time when a delicate little child was deserted in the Hull-House
nursery. An investigation showed that it had been born ten days
previously in the Cook County hospital, but no trace could be found of
the unfortunate mother. The little child lived for several weeks, and
then, in spite of every care, died. It was decided to have it buried by
the county authorities, and the wagon was to arrive at eleven o'clock;
about nine o'clock in the morning the rumor of this awful deed reached
the neighbors. A half dozen of them came, in a very excited state of
mind, to protest. They took up a collection out of their poverty with
which to defray a funeral. The residents of Hull-House were then
comparatively new in the neighborhood and did not realize that they were
really shocking a genuine moral sentiment of the community. In their
crudeness they instanced the care and tenderness which had been expended
upon the little creature while it was alive; that it had had every
attention from a skilled physician and a trained nurse, and even
intimated that the excited members of the group had not taken part in
this, and that it now lay with the nursery to decide that it should be
buried as it had been born, at the county's expense. It is doubtful if
Hull-House has ever done anything which injured it so deeply in the
minds of some of its neighbors. It was only forgiven by the most
indulgent on the ground that the residents were spinsters, and could not
know a mother's heart. No one born and reared in the community could
possibly have made a mistake like that. No one who had studied the
ethical standards with any care could have bungled so completely.

\addamsparagraph We are constantly underestimating the amount of sentiment among simple
people. The songs which are most popular among them are those of a
reminiscent old age, in which the ripened soul calmly recounts and
regrets the sins of his youth, songs in which the wayward daughter is
forgiven by her loving parents, in which the lovers are magnanimous and
faithful through all vicissitudes. The tendency is to condone and
forgive, and not hold too rigidly to a standard. In the theatres it is
the magnanimous man, the kindly reckless villain who is always
applauded. So shrewd an observer as Samuel Johnson once remarked that it
was surprising to find how much more kindness than justice society
contained.

\addamsparagraph On the same basis the alderman manages several saloons, one down town
within easy access of the city hall, where he can catch the more
important of his friends. Here again he has seized upon an old tradition
and primitive custom, the good fellowship which has long been best
expressed when men drink together. The saloons offer a common meeting
ground, with stimulus enough to free the wits and tongues of the men who
meet there.

\addamsparagraph He distributes each Christmas many tons of turkeys not only to voters,
but to families who are represented by no vote. By a judicious
management some families get three or four turkeys apiece; but what of
that, the alderman has none of the nagging rules of the charitable
societies, nor does he declare that because a man wants two turkeys for
Christmas, he is a scoundrel who shall never be allowed to eat turkey
again. As he does not distribute his Christmas favors from any hardly
acquired philanthropic motive, there is no disposition to apply the
carefully evolved rules of the charitable societies to his
beneficiaries. Of course, there are those who suspect that the
benevolence rests upon self-seeking motives, and feel themselves quite
freed from any sense of gratitude; others go further and glory in the
fact that they can thus ``soak the alderman.'' An example of this is the
young man who fills his pockets with a handful of cigars, giving a sly
wink at the others. But this freedom from any sense of obligation is
often the first step downward to the position where he is willing to
sell his vote to both parties, and then scratch his ticket as he
pleases. The writer recalls a conversation with a man in which he
complained quite openly, and with no sense of shame, that his vote had
``sold for only two dollars this year,'' and that he was ``awfully
disappointed.'' The writer happened to know that his income during the
nine months previous had been but twenty-eight dollars, and that he was
in debt thirty-two dollars, and she could well imagine the eagerness
with which he had counted upon this source of revenue. After some years
the selling of votes becomes a commonplace, and but little attempt is
made upon the part of the buyer or seller to conceal the fact, if the
transaction runs smoothly.

\addamsparagraph A certain lodging-house keeper at one time sold the votes of his entire
house to a political party and was ``well paid for it too''; but being of
a grasping turn, he also sold the house for the same election to the
rival party. Such an outrage could not be borne. The man was treated to
a modern version of tar and feathers, and as a result of being held
under a street hydrant in November, contracted pneumonia which resulted
in his death. No official investigation took place, since the doctor's
certificate of pneumonia was sufficient for legal burial, and public
sentiment sustained the action. In various conversations which the
writer had concerning the entire transaction, she discovered great
indignation concerning his duplicity and treachery, but none whatever
for his original offence of selling out the votes of his house.

\addamsparagraph A club will be started for the express purpose of gaining a reputation
for political power which may later be sold out. The president and
executive committee of such a club, who will naturally receive the
funds, promise to divide with ``the boys'' who swell the size of the
membership. A reform movement is at first filled with recruits who are
active and loud in their assertions of the number of votes they can
``deliver.'' The reformers are delighted with this display of zeal, and
only gradually find out that many of the recruits are there for the
express purpose of being bought by the other side; that they are most
active in order to seem valuable, and thus raise the price of their
allegiance when they are ready to sell. Reformers seeing them drop away
one by one, talk of desertion from the ranks of reform, and of the power
of money over well-meaning men, who are too weak to withstand
temptation; but in reality the men are not deserters because they have
never actually been enrolled in the ranks. The money they take is
neither a bribe nor the price of their loyalty, it is simply the
consummation of a long-cherished plan and a well-earned reward. They
came into the new movement for the purpose of being bought out of it,
and have successfully accomplished that purpose.

\addamsparagraph Hull-House assisted in carrying on two unsuccessful campaigns against
the same alderman. In the two years following the end of the first one,
nearly every man who had been prominent in it had received an office
from the re�lected alderman. A printer had been appointed to a clerkship
in the city hall; a driver received a large salary for services in the
police barns; the candidate himself, a bricklayer, held a position in
the city construction department. At the beginning of the next
campaign, the greatest difficulty was experienced in finding a
candidate, and each one proposed, demanded time to consider the
proposition. During this period he invariably became the recipient of
the alderman's bounty. The first one, who was foreman of a large
factory, was reported to have been bought off by the promise that the
city institutions would use the product of his firm. The second one, a
keeper of a grocery and family saloon, with large popularity, was
promised the aldermanic nomination on the regular ticket at the
expiration of the term of office held by the alderman's colleague, and
it may be well to state in passing that he was thus nominated and
successfully elected. The third proposed candidate received a place for
his son in the office of the city attorney.

\addamsparagraph Not only are offices in his gift, but all smaller favors as well. Any
requests to the council, or special licenses, must be presented by the
alderman of the ward in which the person desiring the favor resides.
There is thus constant opportunity for the alderman to put his
constituents under obligations to him, to make it difficult for a
constituent to withstand him, or for one with large interests to enter
into political action at all. From the Italian pedler who wants a
license to peddle fruit in the street, to the large manufacturing
company who desires to tunnel an alley for the sake of conveying pipes
from one building to another, everybody is under obligations to his
alderman, and is constantly made to feel it. In short, these very
regulations for presenting requests to the council have been made, by
the aldermen themselves, for the express purpose of increasing the
dependence of their constituents, and thereby augmenting aldermanic
power and prestige.

\addamsparagraph The alderman has also a very singular hold upon the property owners of
his ward. The paving, both of the streets and sidewalks throughout his
district, is disgraceful; and in the election speeches the reform side
holds him responsible for this condition, and promises better paving
under another r�gime. But the paving could not be made better without a
special assessment upon the property owners of the vicinity, and paying
more taxes is exactly what his constituents do not want to do. In
reality, ``getting them off,'' or at the worst postponing the time of the
improvement, is one of the genuine favors which he performs. A movement
to have the paving done from a general fund would doubtless be opposed
by the property owners in other parts of the city who have already paid
for the asphalt bordering their own possessions, but they have no
conception of the struggle and possible bankruptcy which repaving may
mean to the small property owner, nor how his chief concern may be to
elect an alderman who cares more for the feelings and pocket-books of
his constituents than he does for the repute and cleanliness of his
city.

\addamsparagraph The alderman exhibited great wisdom in procuring from certain of his
down-town friends the sum of three thousand dollars with which to
uniform and equip a boys' temperance brigade which had been formed in
one of the ward churches a few months before his campaign. Is it strange
that the good leader, whose heart was filled with innocent pride as he
looked upon these promising young scions of virtue, should decline to
enter into a reform campaign? Of what use to suggest that uniforms and
bayonets for the purpose of promoting temperance, bought with money
contributed by a man who was proprietor of a saloon and a gambling
house, might perhaps confuse the ethics of the young soldiers? Why take
the pains to urge that it was vain to lecture and march abstract virtues
into them, so long as the ``champion boodler'' of the town was the man
whom the boys recognized as a loyal and kindhearted friend, the
public-spirited citizen, whom their fathers enthusiastically voted for,
and their mothers called ``the friend of the poor.'' As long as the actual
and tangible success is thus embodied, marching whether in kindergartens
or brigades, talking whether in clubs or classes, does little to change
the code of ethics.

\addamsparagraph The question of where does the money come from which is spent so
successfully, does of course occur to many minds. The more primitive
people accept the truthful statement of its sources without any shock to
their moral sense. To their simple minds he gets it ``from the rich'' and,
so long as he again gives it out to the poor as a true Robin Hood, with
open hand, they have no objections to offer. Their ethics are quite
honestly those of the merry-making foresters. The next less primitive
people of the vicinage are quite willing to admit that he leads the
``gang'' in the city council, and sells out the city franchises; that he
makes deals with the franchise-seeking companies; that he guarantees to
steer dubious measures through the council, for which he demands liberal
pay; that he is, in short, a successful ``boodler.'' When, however, there
is intellect enough to get this point of view, there is also enough to
make the contention that this is universally done, that all the
aldermen do it more or less successfully, but that the alderman of this
particular ward is unique in being so generous; that such a state of
affairs is to be deplored, of course; but that that is the way business
is run, and we are fortunate when a kind-hearted man who is close to the
people gets a large share of the spoils; that he serves franchised
companies who employ men in the building and construction of their
enterprises, and that they are bound in return to give work to his
constituents. It is again the justification of stealing from the rich to
give to the poor. Even when they are intelligent enough to complete the
circle, and to see that the money comes, not from the pockets of the
companies' agents, but from the street-car fares of people like
themselves, it almost seems as if they would rather pay two cents more
each time they ride than to give up the consciousness that they have a
big, warm-hearted friend at court who will stand by them in an
emergency. The sense of just dealing comes apparently much later than
the desire for protection and indulgence. On the whole, the gifts and
favors are taken quite simply as an evidence of genuine loving-kindness.
The alderman is really elected because he is a good friend and neighbor.
He is corrupt, of course, but he is not elected because he is corrupt,
but rather in spite of it. His standard suits his constituents. He
exemplifies and exaggerates the popular type of a good man. He has
attained what his constituents secretly long for.

\addamsparagraph At one end of the ward there is a street of good houses, familiarly
called ``Con Row.'' The term is perhaps quite unjustly used, but it is
nevertheless universally applied, because many of these houses are
occupied by professional office holders. This row is supposed to form a
happy hunting-ground of the successful politician, where he can live in
prosperity, and still maintain his vote and influence in the ward. It
would be difficult to justly estimate the influence which this group of
successful, prominent men, including the alderman who lives there, have
had upon the ideals of the youth in the vicinity. The path which leads
to riches and success, to civic prominence and honor, is the path of
political corruption. We might compare this to the path laid out by
Benjamin Franklin, who also secured all of these things, but told young
men that they could be obtained only by strenuous effort and frugal
living, by the cultivation of the mind, and the holding fast to
righteousness; or, again, we might compare it to the ideals which were
held up to the American youth fifty years ago, lower, to be sure, than
the revolutionary ideal, but still fine and aspiring toward honorable
dealing and careful living. They were told that the career of the
self-made man was open to every American boy, if he worked hard and
saved his money, improved his mind, and followed a steady ambition. The
writer remembers that when she was ten years old, the village
schoolmaster told his little flock, without any mitigating clauses,
that Jay Gould had laid the foundation of his colossal fortune by always
saving bits of string, and that, as a result, every child in the village
assiduously collected party-colored balls of twine. A bright Chicago boy
might well draw the inference that the path of the corrupt politician
not only leads to civic honors, but to the glories of benevolence and
philanthropy. This lowering of standards, this setting of an ideal, is
perhaps the worst of the situation, for, as we said in the first
chapter, we determine ideals by our daily actions and decisions not only
for ourselves, but largely for each other.

\addamsparagraph We are all involved in this political corruption, and as members of the
community stand indicted. This is the penalty of a democracy,---that we
are bound to move forward or retrograde together. None of us can stand
aside; our feet are mired in the same soil, and our lungs breathe the
same air.

\addamsparagraph That the alderman has much to do with setting the standard of life and
desirable prosperity may be illustrated by the following incident:
During one of the campaigns a clever cartoonist drew a poster
representing the successful alderman in portraiture drinking champagne
at a table loaded with pretentious dishes and surrounded by other
revellers. In contradistinction was his opponent, a bricklayer, who sat
upon a half-finished wall, eating a meagre dinner from a workingman's
dinner-pail, and the passer-by was asked which type of representative he
preferred, the presumption being that at least in a workingman's
district the bricklayer would come out ahead. To the chagrin of the
reformers, however, it was gradually discovered that, in the popular
mind, a man who laid bricks and wore overalls was not nearly so
desirable for an alderman as the man who drank champagne and wore a
diamond in his shirt front. The district wished its representative ``to
stand up with the best of them,'' and certainly some of the constituents
would have been ashamed to have been represented by a bricklayer. It is
part of that general desire to appear well, the optimistic and
thoroughly American belief, that even if a man is working with his hands
to-day, he and his children will quite likely be in a better position in
the swift coming to-morrow, and there is no need of being too closely
associated with common working people. There is an honest absence of
class consciousness, and a na�ve belief that the kind of occupation
quite largely determines social position. This is doubtless exaggerated
in a neighborhood of foreign people by the fact that as each nationality
becomes more adapted to American conditions, the scale of its occupation
rises. Fifty years ago in America ``a Dutchman'' was used as a term of
reproach, meaning a man whose language was not understood, and who
performed menial tasks, digging sewers and building railroad
embankments. Later the Irish did the same work in the community, but as
quickly as possible handed it on to the Italians, to whom the name
``dago'' is said to cling as a result of the digging which the Irishman
resigned to him. The Italian himself is at last waking up to this fact.
In a political speech recently made by an Italian padrone, he bitterly
reproached the alderman for giving the-four-dollars-a-day ``jobs'' of
sitting in an office to Irishmen and the-dollar-and-a-half-a-day ``jobs''
of sweeping the streets to the Italians. This general struggle to rise
in life, to be at least politically represented by one of the best, as
to occupation and social status, has also its negative side. We must
remember that the imitative impulse plays an important part in life, and
that the loss of social estimation, keenly felt by all of us, is perhaps
most dreaded by the humblest, among whom freedom of individual conduct,
the power to give only just weight to the opinion of neighbors, is but
feebly developed. A form of constraint, gentle, but powerful, is
afforded by the simple desire to do what others do, in order to share
with them the approval of the community. Of course, the larger the
number of people among whom an habitual mode of conduct obtains, the
greater the constraint it puts upon the individual will. Thus it is that
the political corruption of the city presses most heavily where it can
be least resisted, and is most likely to be imitated.

\addamsparagraph According to the same law, the positive evils of corrupt government are
bound to fall heaviest upon the poorest and least capable. When the
water of Chicago is foul, the prosperous buy water bottled at distant
springs; the poor have no alternative but the typhoid fever which comes
from using the city's supply. When the garbage contracts are not
enforced, the well-to-do pay for private service; the poor suffer the
discomfort and illness which are inevitable from a foul atmosphere. The
prosperous business man has a certain choice as to whether he will treat
with the ``boss'' politician or preserve his independence on a smaller
income; but to an Italian day laborer it is a choice between obeying the
commands of a political ``boss'' or practical starvation. Again, a more
intelligent man may philosophize a little upon the present state of
corruption, and reflect that it is but a phase of our commercialism,
from which we are bound to emerge; at any rate, he may give himself the
solace of literature and ideals in other directions, but the more
ignorant man who lives only in the narrow present has no such resource;
slowly the conviction enters his mind that politics is a matter of
favors and positions, that self-government means pleasing the ``boss'' and
standing in with the ``gang.'' This slowly acquired knowledge he hands on
to his family. During the month of February his boy may come home from
school with rather incoherent tales about Washington and Lincoln, and
the father may for the moment be fired to tell of Garibaldi, but such
talk is only periodic, and the long year round the fortunes of the
entire family, down to the opportunity to earn food and shelter, depend
upon the ``boss.''

\addamsparagraph In a certain measure also, the opportunities for pleasure and recreation
depend upon him. To use a former illustration, if a man happens to have
a taste for gambling, if the slot machine affords him diversion, he goes
to those houses which are protected by political influence. If he and
his friends like to drop into a saloon after midnight, or even want to
hear a little music while they drink together early in the evening, he
is breaking the law when he indulges in either of them, and can only be
exempt from arrest or fine because the great political machine is
friendly to him and expects his allegiance in return.

\addamsparagraph During the campaign, when it was found hard to secure enough local
speakers of the moral tone which was desired, orators were imported from
other parts of the town, from the so-called ``better element.'' Suddenly
it was rumored on all sides that, while the money and speakers for the
reform candidate were coming from the swells, the money which was
backing the corrupt alderman also came from a swell source; that the
president of a street-car combination, for whom he performed constant
offices in the city council, was ready to back him to the extent of
fifty thousand dollars; that this president, too, was a good man, and
sat in high places; that he had recently given a large sum of money to
an educational institution and was therefore as philanthropic, not to
say good and upright, as any man in town; that the corrupt alderman had
the sanction of the highest authorities, and that the lecturers who were
talking against corruption, and the selling and buying of franchises,
were only the cranks, and not the solid business men who had developed
and built up Chicago.

\addamsparagraph All parts of the community are bound together in ethical development. If
the so-called more enlightened members accept corporate gifts from the
man who buys up the council, and the so-called less enlightened members
accept individual gifts from the man who sells out the council, we
surely must take our punishment together. There is the difference, of
course, that in the first case we act collectively, and in the second
case individually; but is the punishment which follows the first any
lighter or less far-reaching in its consequences than the more obvious
one which follows the second?

\addamsparagraph Have our morals been so captured by commercialism, to use Mr. Chapman's
generalization, that we do not see a moral dereliction when business or
educational interests are served thereby, although we are still shocked
when the saloon interest is thus served?

\addamsparagraph The street-car company which declares that it is impossible to do
business without managing the city council, is on exactly the same moral
level with the man who cannot retain political power unless he has a
saloon, a large acquaintance with the semi-criminal class, and
questionable money with which to debauch his constituents. Both sets of
men assume that the only appeal possible is along the line of
self-interest. They frankly acknowledge money getting as their own
motive power, and they believe in the cupidity of all the men whom they
encounter. No attempt in either case is made to put forward the claims
of the public, or to find a moral basis for action. As the corrupt
politician assumes that public morality is impossible, so many business
men become convinced that to pay tribute to the corrupt aldermen is on
the whole cheaper than to have taxes too high; that it is better to pay
exorbitant rates for franchises, than to be made unwilling partners in
transportation experiments. Such men come to regard political reformers
as a sort of monomaniac, who are not reasonable enough to see the
necessity of the present arrangement which has slowly been evolved and
developed, and upon which business is safely conducted. A reformer who
really knew the people and their great human needs, who believed that
it was the business of government to serve them, and who further
recognized the educative power of a sense of responsibility, would
possess a clew by which he might analyze the situation. He would find
out what needs, which the alderman supplies, are legitimate ones which
the city itself could undertake, in counter-distinction to those which
pander to the lower instincts of the constituency. A mother who eats her
Christmas turkey in a reverent spirit of thankfulness to the alderman
who gave it to her, might be gradually brought to a genuine sense of
appreciation and gratitude to the city which supplies her little
children with a Kindergarten, or, to the Board of Health which properly
placarded a case of scarlet-fever next door and spared her sleepless
nights and wearing anxiety, as well as the money paid with such
difficulty to the doctor and the druggist. The man who in his emotional
gratitude almost kneels before his political friend who gets his boy out
of jail, might be made to see the kindness and good sense of the city
authorities who provided the boy with a playground and reading room,
where he might spend his hours of idleness and restlessness, and through
which his temptations to petty crime might be averted. A man who is
grateful to the alderman who sees that his gambling and racing are not
interfered with, might learn to feel loyal and responsible to the city
which supplied him with a gymnasium and swimming tank where manly and
well-conducted sports are possible. The voter who is eager to serve the
alderman at all times, because the tenure of his job is dependent upon
aldermanic favor, might find great relief and pleasure in working for
the city in which his place was secured by a well-administered civil
service law.

\addamsparagraph After all, what the corrupt alderman demands from his followers and
largely depends upon is a sense of loyalty, a standing-by the man who is
good to you, who understands you, and who gets you out of trouble. All
the social life of the voter from the time he was a little boy and
played ``craps'' with his ``own push,'' and not with some other ``push,'' has
been founded on this sense of loyalty and of standing in with his
friends. Now that he is a man, he likes the sense of being inside a
political organization, of being trusted with political gossip, of
belonging to a set of fellows who understand things, and whose interests
are being cared for by a strong friend in the city council itself. All
this is perfectly legitimate, and all in the line of the development of
a strong civic loyalty, if it were merely socialized and enlarged. Such
a voter has already proceeded in the forward direction in so far as he
has lost the sense of isolation, and has abandoned the conviction that
city government does not touch his individual affairs. Even Mill claims
that the social feelings of man, his desire to be at unity with his
fellow-creatures, are the natural basis for morality, and he defines a
man of high moral culture as one who thinks of himself, not as an
isolated individual, but as a part in a social organism.

\addamsparagraph Upon this foundation it ought not to be difficult to build a structure
of civic virtue. It is only necessary to make it clear to the voter that
his individual needs are common needs, that is, public needs, and that
they can only be legitimately supplied for him when they are supplied
for all. If we believe that the individual struggle for life may widen
into a struggle for the lives of all, surely the demand of an individual
for decency and comfort, for a chance to work and obtain the fulness of
life may be widened until it gradually embraces all the members of the
community, and rises into a sense of the common weal.

\addamsparagraph In order, however, to give him a sense of conviction that his individual
needs must be merged into the needs of the many, and are only important
as they are thus merged, the appeal cannot be made along the line of
self-interest. The demand should be universalized; in this process it
would also become clarified, and the basis of our political organization
become perforce social and ethical.

\addamsparagraph Would it be dangerous to conclude that the corrupt politician himself,
because he is democratic in method, is on a more ethical line of social
development than the reformer, who believes that the people must be made
over by ``good citizens'' and governed by ``experts''? The former at least
are engaged in that great moral effort of getting the mass to express
itself, and of adding this mass energy and wisdom to the community as a
whole.

\addamsparagraph The wide divergence of experience makes it difficult for the good
citizen to understand this point of view, and many things conspire to
make it hard for him to act upon it. He is more or less a victim to that
curious feeling so often possessed by the good man, that the righteous
do not need to be agreeable, that their goodness alone is sufficient,
and that they can leave the arts and wiles of securing popular favor to
the self-seeking. This results in a certain repellent manner, commonly
regarded as the apparel of righteousness, and is further responsible for
the fatal mistake of making the surroundings of ``good influences''
singularly unattractive; a mistake which really deserves a reprimand
quite as severe as the equally reprehensible deed of making the
surroundings of ``evil influences'' so beguiling. Both are akin to that
state of mind which narrows the entrance into a wider morality to the
eye of a needle, and accounts for the fact that new moral movements have
ever and again been inaugurated by those who have found themselves in
revolt against the conventionalized good.

\addamsparagraph The success of the reforming politician who insists upon mere purity of
administration and upon the control and suppression of the unruly
elements in the community, may be the easy result of a narrowing and
selfish process. For the painful condition of endeavoring to minister
to genuine social needs, through the political machinery, and at the
same time to remodel that machinery so that it shall be adequate to its
new task, is to encounter the inevitable discomfort of a transition into
a new type of democratic relation. The perplexing experiences of the
actual administration, however, have a genuine value of their own. The
economist who treats the individual cases as mere data, and the social
reformer who labors to make such cases impossible, solely because of the
appeal to his reason, may have to share these perplexities before they
feel themselves within the grasp of a principle of growth, working
outward from within; before they can gain the exhilaration and uplift
which comes when the individual sympathy and intelligence is caught into
the forward intuitive movement of the mass. This general movement is not
without its intellectual aspects, but it has to be transferred from the
region of perception to that of emotion before it is really
apprehended. The mass of men seldom move together without an emotional
incentive. The man who chooses to stand aside, avoids much of the
perplexity, but at the same time he loses contact with a great source of
vitality.

\addamsparagraph Perhaps the last and greatest difficulty in the paths of those who are
attempting to define and attain a social morality, is that which arises
from the fact that they cannot adequately test the value of their
efforts, cannot indeed be sure of their motives until their efforts are
reduced to action and are presented in some workable form of social
conduct or control. For action is indeed the sole medium of expression
for ethics. We continually forget that the sphere of morals is the
sphere of action, that speculation in regard to morality is but
observation and must remain in the sphere of intellectual comment, that
a situation does not really become moral until we are confronted with
the question of what shall be done in a concrete case, and are obliged
to act upon our theory. A stirring appeal has lately been made by a
recognized ethical lecturer who has declared that ``It is insanity to
expect to receive the data of wisdom by looking on. We arrive at moral
knowledge only by tentative and observant practice. We learn how to
apply the new insight by having attempted to apply the old and having
found it to fail.''

\addamsparagraph This necessity of reducing the experiment to action throws out of the
undertaking all timid and irresolute persons, more than that, all those
who shrink before the need of striving forward shoulder to shoulder with
the cruder men, whose sole virtue may be social effort, and even that
not untainted by self-seeking, who are indeed pushing forward social
morality, but who are doing it irrationally and emotionally, and often
at the expense of the well-settled standards of morality.

\addamsparagraph The power to distinguish between the genuine effort and the adventitious
mistakes is perhaps the most difficult test which comes to our fallible
intelligence. In the range of individual morals, we have learned to
distrust him who would reach spirituality by simply renouncing the
world, or by merely speculating upon its evils. The result, as well as
the process of virtues attained by repression, has become distasteful to
us. When the entire moral energy of an individual goes into the
cultivation of personal integrity, we all know how unlovely the result
may become; the character is upright, of course, but too coated over
with the result of its own endeavor to be attractive. In this effort
toward a higher morality in our social relations, we must demand that
the individual shall be willing to lose the sense of personal
achievement, and shall be content to realize his activity only in
connection with the activity of the many.

\addamsparagraph The cry of ``Back to the people'' is always heard at the same time, when
we have the prophet's demand for repentance or the religious cry of
``Back to Christ,'' as though we would seek refuge with our fellows and
believe in our common experiences as a preparation for a new moral
struggle.

\addamsparagraph As the acceptance of democracy brings a certain life-giving power, so it
has its own sanctions and comforts. Perhaps the most obvious one is the
curious sense which comes to us from time to time, that we belong to the
whole, that a certain basic well being can never be taken away from us
whatever the turn of fortune. Tolstoy has portrayed the experience in
``Master and Man.'' The former saves his servant from freezing, by
protecting him with the heat of his body, and his dying hours are filled
with an ineffable sense of healing and well-being. Such experiences, of
which we have all had glimpses, anticipate in our relation to the living
that peace of mind which envelopes us when we meditate upon the great
multitude of the dead. It is akin to the assurance that the dead
understand, because they have entered into the Great Experience, and
therefore must comprehend all lesser ones; that all the
misunderstandings we have in life are due to partial experience, and
all life's fretting comes of our limited intelligence; when the last and
Great Experience comes, it is, perforce, attended by mercy and
forgiveness. Consciously to accept Democracy and its manifold
experiences is to anticipate that peace and freedom.
\end{sectionbody}

\end{document}