\documentclass{article}
\usepackage{hyperref,fancyhdr}
\usepackage[]{accessibility}
% changes font to TeX Gyre Schola (Century Schoolbook)
\usepackage{tgschola}\usepackage[T1]{fontenc}
%\usepackage{baskervald}
\title{Self-Reliance}
\author{Ralph Waldo Emerson}
\date{1841}
\fancyhead{}
\fancyfoot{}
\fancyhead[L]{\textsc{Self-Reliance}}
\fancyhead[R]{\arabic{page} of \pageref{theend}}
\begin{document}
\pagestyle{fancy}
\maketitle

{\it ``Ne te quaesiveris extra.''}\footnote{Latin for, ``Do not seek outside yourself.''}

\begin{verse}
``Man is his own star; and the soul that can\\
Render an honest and a perfect man,\\
Commands all light, all influence, all fate;\\
Nothing to him falls early or too late.\\
Our acts our angels are, or good or ill,\\
Our fatal shadows that walk by us still.''\\
\end{verse}

{\it Epilogue to Beaumont and Fletcher's Honest Man's Fortune}


\begin{verse}
Cast the bantling on the rocks,\\
Suckle him with the she-wolf's teat;\\
Wintered with the hawk and fox,\\
Power and speed be hands and feet.
\end{verse}

I read the other day some verses written by an eminent painter which were
original and not conventional. The soul always hears an admonition in such
lines, let the subject be what it may. The sentiment they instill is of more
value than any thought they may contain. To believe your own thought, to
believe that what is true for you in your private heart is true for all
men, --- that is genius. Speak your latent conviction, and it shall be the
universal sense; for the inmost in due time becomes the outmost, --- and our
first thought is rendered back to us by the trumpets of the Last Judgment.
Familiar as the voice of the mind is to each, the highest merit we ascribe
to Moses, Plato, and Milton is, that they set at naught books and
traditions, and spoke not what men but what they thought. A man should learn
to detect and watch that gleam of light which flashes across his mind from
within, more than the lustre of the firmament of bards and sages. Yet he
dismisses without notice his thought, because it is his. In every work of
genius we recognize our own rejected thoughts: they come back to us with a
certain alienated majesty. Great works of art have no more affecting lesson
for us than this. They teach us to abide by our spontaneous impression with
good-humored inflexibility then most when the whole cry of voices is on the
other side. Else, to-morrow a stranger will say with masterly good sense
precisely what we have thought and felt all the time, and we shall be forced
to take with shame our own opinion from another.

There is a time in every man's education when he arrives at the conviction
that envy is ignorance; that imitation is suicide; that he must take himself
for better, for worse, as his portion; that though the wide universe is full
of good, no kernel of nourishing corn can come to him but through his toil
bestowed on that plot of ground which is given to him to till. The power
which resides in him is new in nature, and none but he knows what that is
which he can do, nor does he know until he has tried. Not for nothing one
face, one character, one fact, makes much impression on him, and another
none. This sculpture in the memory is not without preestablished harmony.
The eye was placed where one ray should fall, that it might testify of that
particular ray. We but half express ourselves, and are ashamed of that
divine idea which each of us represents. It may be safely trusted as
proportionate and of good issues, so it be faithfully imparted, but God will
not have his work made manifest by cowards. A man is relieved and gay when
he has put his heart into his work and done his best; but what he has said
or done otherwise, shall give him no peace. It is a deliverance which does
not deliver. In the attempt his genius deserts him; no muse befriends; no
invention, no hope.

Trust thyself: every heart vibrates to that iron string. Accept the place
the divine providence has found for you, the society of your contemporaries,
the connection of events. Great men have always done so, and confided
themselves childlike to the genius of their age, betraying their perception
that the absolutely trustworthy was seated at their heart, working through
their hands, predominating in all their being. And we are now men, and must
accept in the highest mind the same transcendent destiny; and not minors and
invalids in a protected corner, not cowards fleeing before a revolution, but
guides, redeemers, and benefactors, obeying the Almighty effort, and
advancing on Chaos and the Dark.

What pretty oracles nature yields us on this text, in the face and behaviour
of children, babes, and even brutes! That divided and rebel mind, that
distrust of a sentiment because our arithmetic has computed the strength and
means opposed to our purpose, these have not. Their mind being whole, their
eye is as yet unconquered, and when we look in their faces, we are
disconcerted. Infancy conforms to nobody: all conform to it, so that one
babe commonly makes four or five out of the adults who prattle and play to
it. So God has armed youth and puberty and manhood no less with its own
piquancy and charm, and made it enviable and gracious and its claims not to
be put by, if it will stand by itself. Do not think the youth has no force,
because he cannot speak to you and me. Hark! in the next room his voice is
sufficiently clear and emphatic. It seems he knows how to speak to his
contemporaries. Bashful or bold, then, he will know how to make us seniors
very unnecessary.

The nonchalance of boys who are sure of a dinner, and would disdain as much
as a lord to do or say aught to conciliate one, is the healthy attitude of
human nature. A boy is in the parlour what the pit is in the playhouse;
independent, irresponsible, looking out from his corner on such people and
facts as pass by, he tries and sentences them on their merits, in the swift,
summary way of boys, as good, bad, interesting, silly, eloquent,
troublesome. He cumbers himself never about consequences, about interests:
he gives an independent, genuine verdict. You must court him: he does not
court you. But the man is, as it were, clapped into jail by his
consciousness. As soon as he has once acted or spoken with eclat, he is a
committed person, watched by the sympathy or the hatred of hundreds, whose
affections must now enter into his account. There is no Lethe for this. Ah,
that he could pass again into his neutrality! Who can thus avoid all
pledges, and having observed, observe again from the same unaffected,
unbiased, unbribable, unaffrighted innocence, must always be formidable. He
would utter opinions on all passing affairs, which being seen to be not
private, but necessary, would sink like darts into the ear of men, and put
them in fear.

These are the voices which we hear in solitude, but they grow faint and
inaudible as we enter into the world. Society everywhere is in conspiracy
against the manhood of every one of its members. Society is a joint-stock
company, in which the members agree, for the better securing of his bread to
each shareholder, to surrender the liberty and culture of the eater. The
virtue in most request is conformity. Self-reliance is its aversion. It
loves not realities and creators, but names and customs.

Whoso would be a man must be a nonconformist. He who would gather immortal
palms must not be hindered by the name of goodness, but must explore if it
be goodness. Nothing is at last sacred but the integrity of your own mind.
Absolve you to yourself, and you shall have the suffrage of the world. I
remember an answer which when quite young I was prompted to make to a valued
adviser, who was wont to importune me with the dear old doctrines of the
church. On my saying, What have I to do with the sacredness of traditions,
if I live wholly from within? my friend suggested, --- ``But these impulses
may be from below, not from above.'' I replied, ``They do not seem to me to be
such; but if I am the Devil's child, I will live then from the Devil.'' No
law can be sacred to me but that of my nature. Good and bad are but names
very readily transferable to that or this; the only right is what is after
my constitution, the only wrong what is against it. A man is to carry
himself in the presence of all opposition, as if every thing were titular
and ephemeral but he. I am ashamed to think how easily we capitulate to
badges and names, to large societies and dead institutions. Every decent and
well-spoken individual affects and sways me more than is right. I ought to
go upright and vital, and speak the rude truth in all ways. If malice and
vanity wear the coat of philanthropy, shall that pass? If an angry bigot
assumes this bountiful cause of Abolition, and comes to me with his last
news from Barbadoes, why should I not say to him, `Go love thy infant; love
thy wood-chopper: be good-natured and modest: have that grace; and never
varnish your hard, uncharitable ambition with this incredible tenderness for
black folk a thousand miles off. Thy love afar is spite at home.' Rough and
graceless would be such greeting, but truth is handsomer than the
affectation of love. Your goodness must have some edge to it, --- else it is
none. The doctrine of hatred must be preached as the counteraction of the
doctrine of love when that pules and whines. I shun father and mother and
wife and brother, when my genius calls me. I would write on the lintels of
the door-post, {\it Whim}. I hope it is somewhat better than whim at last, but
we cannot spend the day in explanation. Expect me not to show cause why I
seek or why I exclude company. Then, again, do not tell me, as a good man
did to-day, of my obligation to put all poor men in good situations. Are
they {\it my} poor? I tell thee, thou foolish philanthropist, that I grudge the
dollar, the dime, the cent, I give to such men as do not belong to me and to
whom I do not belong. There is a class of persons to whom by all spiritual
affinity I am bought and sold; for them I will go to prison, if need be; but
your miscellaneous popular charities; the education at college of fools; the
building of meeting-houses to the vain end to which many now stand; alms to
sots; and the thousandfold Relief Societies; --- though I confess with shame
I sometimes succumb and give the dollar, it is a wicked dollar which by and
by I shall have the manhood to withhold.

Virtues are, in the popular estimate, rather the exception than the rule.
There is the man {\it and} his virtues. Men do what is called a good action, as
some piece of courage or charity, much as they would pay a fine in expiation
of daily non-appearance on parade. Their works are done as an apology or
extenuation of their living in the world, --- as invalids and the insane pay
a high board. Their virtues are penances. I do not wish to expiate, but to
live. My life is for itself and not for a spectacle. I much prefer that it
should be of a lower strain, so it be genuine and equal, than that it should
be glittering and unsteady. I wish it to be sound and sweet, and not to need
diet and bleeding. I ask primary evidence that you are a man, and refuse
this appeal from the man to his actions. I know that for myself it makes no
difference whether I do or forbear those actions which are reckoned
excellent. I cannot consent to pay for a privilege where I have intrinsic
right. Few and mean as my gifts may be, I actually am, and do not need for
my own assurance or the assurance of my fellows any secondary testimony.

What I must do is all that concerns me, not what the people think. This
rule, equally arduous in actual and in intellectual life, may serve for the
whole distinction between greatness and meanness. It is the harder, because
you will always find those who think they know what is your duty better than
you know it. It is easy in the world to live after the world's opinion; it
is easy in solitude to live after our own; but the great man is he who in
the midst of the crowd keeps with perfect sweetness the independence of
solitude.

The objection to conforming to usages that have become dead to you is, that
it scatters your force. It loses your time and blurs the impression of your
character. If you maintain a dead church, contribute to a dead
Bible-society, vote with a great party either for the government or against
it, spread your table like base housekeepers, --- under all these screens I
have difficulty to detect the precise man you are. And, of course, so much
force is withdrawn from your proper life. But do your work, and I shall know
you. Do your work, and you shall reinforce yourself. A man must consider
what a blindman's-buff is this game of conformity. If I know your sect, I
anticipate your argument. I hear a preacher announce for his text and topic
the expediency of one of the institutions of his church. Do I not know
beforehand that not possibly can he say a new and spontaneous word? Do I not
know that, with all this ostentation of examining the grounds of the
institution, he will do no such thing? Do I not know that he is pledged to
himself not to look but at one side, --- the permitted side, not as a man,
but as a parish minister? He is a retained attorney, and these airs of the
bench are the emptiest affectation. Well, most men have bound their eyes
with one or another handkerchief, and attached themselves to some one of
these communities of opinion. This conformity makes them not false in a few
particulars, authors of a few lies, but false in all particulars. Their
every truth is not quite true. Their two is not the real two, their four not
the real four; so that every word they say chagrins us, and we know not
where to begin to set them right. Meantime nature is not slow to equip us in
the prison-uniform of the party to which we adhere. We come to wear one cut
of face and figure, and acquire by degrees the gentlest asinine expression.
There is a mortifying experience in particular, which does not fail to wreak
itself also in the general history; I mean ``the foolish face of praise,'' the
forced smile which we put on in company where we do not feel at ease in
answer to conversation which does not interest us. The muscles, not
spontaneously moved, but moved by a low usurping wilfulness, grow tight
about the outline of the face with the most disagreeable sensation.

For nonconformity the world whips you with its displeasure. And therefore a
man must know how to estimate a sour face. The by-standers look askance on
him in the public street or in the friend's parlour. If this aversation had
its origin in contempt and resistance like his own, he might well go home
with a sad countenance; but the sour faces of the multitude, like their
sweet faces, have no deep cause, but are put on and off as the wind blows
and a newspaper directs. Yet is the discontent of the multitude more
formidable than that of the senate and the college. It is easy enough for a
firm man who knows the world to brook the rage of the cultivated classes.
Their rage is decorous and prudent, for they are timid as being very
vulnerable themselves. But when to their feminine rage the indignation of
the people is added, when the ignorant and the poor are aroused, when the
unintelligent brute force that lies at the bottom of society is made to
growl and mow, it needs the habit of magnanimity and religion to treat it
godlike as a trifle of no concernment.

The other terror that scares us from self-trust is our consistency; a
reverence for our past act or word, because the eyes of others have no other
data for computing our orbit than our past acts, and we are loath to
disappoint them.

But why should you keep your head over your shoulder? Why drag about this
corpse of your memory, lest you contradict somewhat you have stated in this
or that public place? Suppose you should contradict yourself; what then? It
seems to be a rule of wisdom never to rely on your memory alone, scarcely
even in acts of pure memory, but to bring the past for judgment into the
thousand-eyed present, and live ever in a new day. In your metaphysics you
have denied personality to the Deity: yet when the devout motions of the
soul come, yield to them heart and life, though they should clothe God with
shape and color. Leave your theory, as Joseph his coat in the hand of the
harlot, and flee.

A foolish consistency is the hobgoblin of little minds, adored by little
statesmen and philosophers and divines. With consistency a great soul has
simply nothing to do. He may as well concern himself with his shadow on the
wall. Speak what you think now in hard words, and to-morrow speak what
to-morrow thinks in hard words again, though it contradict every thing you
said to-day. --- `Ah, so you shall be sure to be misunderstood.' --- Is it so
bad, then, to be misunderstood? Pythagoras was misunderstood, and Socrates,
and Jesus, and Luther, and Copernicus, and Galileo, and Newton, and every
pure and wise spirit that ever took flesh. To be great is to be
misunderstood.

I suppose no man can violate his nature. All the sallies of his will are
rounded in by the law of his being, as the inequalities of Andes and
Himmaleh are insignificant in the curve of the sphere. Nor does it matter
how you gauge and try him. A character is like an acrostic or Alexandrian
stanza; --- read it forward, backward, or across, it still spells the same
thing. In this pleasing, contrite wood-life which God allows me, let me
record day by day my honest thought without prospect or retrospect, and, I
cannot doubt, it will be found symmetrical, though I mean it not, and see it
not. My book should smell of pines and resound with the hum of insects. The
swallow over my window should interweave that thread or straw he carries in
his bill into my web also. We pass for what we are. Character teaches above
our wills. Men imagine that they communicate their virtue or vice only by
overt actions, and do not see that virtue or vice emit a breath every
moment.

There will be an agreement in whatever variety of actions, so they be each
honest and natural in their hour. For of one will, the actions will be
harmonious, however unlike they seem. These varieties are lost sight of at a
little distance, at a little height of thought. One tendency unites them
all. The voyage of the best ship is a zigzag line of a hundred tacks. See
the line from a sufficient distance, and it straightens itself to the
average tendency. Your genuine action will explain itself, and will explain
your other genuine actions. Your conformity explains nothing. Act singly,
and what you have already done singly will justify you now. Greatness
appeals to the future. If I can be firm enough to-day to do right, and scorn
eyes, I must have done so much right before as to defend me now. Be it how
it will, do right now. Always scorn appearances, and you always may. The
force of character is cumulative. All the foregone days of virtue work their
health into this. What makes the majesty of the heroes of the senate and the
field, which so fills the imagination? The consciousness of a train of great
days and victories behind. They shed an united light on the advancing actor.
He is attended as by a visible escort of angels. That is it which throws
thunder into Chatham's voice, and dignity into Washington's port, and
America into Adams's eye. Honor is venerable to us because it is no
ephemeris. It is always ancient virtue. We worship it to-day because it is
not of to-day. We love it and pay it homage, because it is not a trap for
our love and homage, but is self-dependent, self-derived, and therefore of
an old immaculate pedigree, even if shown in a young person.

I hope in these days we have heard the last of conformity and consistency.
Let the words be gazetted and ridiculous henceforward. Instead of the gong
for dinner, let us hear a whistle from the Spartan fife. Let us never bow
and apologize more. A great man is coming to eat at my house. I do not wish
to please him; I wish that he should wish to please me. I will stand here
for humanity, and though I would make it kind, I would make it true. Let us
affront and reprimand the smooth mediocrity and squalid contentment of the
times, and hurl in the face of custom, and trade, and office, the fact which
is the upshot of all history, that there is a great responsible Thinker and
Actor working wherever a man works; that a true man belongs to no other time
or place, but is the centre of things. Where he is, there is nature. He
measures you, and all men, and all events. Ordinarily, every body in society
reminds us of somewhat else, or of some other person. Character, reality,
reminds you of nothing else; it takes place of the whole creation. The man
must be so much, that he must make all circumstances indifferent. Every true
man is a cause, a country, and an age; requires infinite spaces and numbers
and time fully to accomplish his design; --- and posterity seem to follow his
steps as a train of clients. A man Caesar is born, and for ages after we
have a Roman Empire. Christ is born, and millions of minds so grow and
cleave to his genius, that he is confounded with virtue and the possible of
man. An institution is the lengthened shadow of one man; as, Monachism, of
the Hermit Antony; the Reformation, of Luther; Quakerism, of Fox; Methodism,
of Wesley; Abolition, of Clarkson. Scipio, Milton called ``the height of
Rome''; and all history resolves itself very easily into the biography of a
few stout and earnest persons.

Let a man then know his worth, and keep things under his feet. Let him not
peep or steal, or skulk up and down with the air of a charity-boy, a
bastard, or an interloper, in the world which exists for him. But the man in
the street, finding no worth in himself which corresponds to the force which
built a tower or sculptured a marble god, feels poor when he looks on these.
To him a palace, a statue, or a costly book have an alien and forbidding
air, much like a gay equipage, and seem to say like that, `Who are you,
Sir?' Yet they all are his, suitors for his notice, petitioners to his
faculties that they will come out and take possession. The picture waits for
my verdict: it is not to command me, but I am to settle its claims to
praise. That popular fable of the sot who was picked up dead drunk in the
street, carried to the duke's house, washed and dressed and laid in the
duke's bed, and, on his waking, treated with all obsequious ceremony like
the duke, and assured that he had been insane, owes its popularity to the
fact, that it symbolizes so well the state of man, who is in the world a
sort of sot, but now and then wakes up, exercises his reason, and finds
himself a true prince.

Our reading is mendicant and sycophantic. In history, our imagination plays
us false. Kingdom and lordship, power and estate, are a gaudier vocabulary
than private John and Edward in a small house and common day's work; but the
things of life are the same to both; the sum total of both is the same. Why
all this deference to Alfred, and Scanderbeg, and Gustavus? Suppose they
were virtuous; did they wear out virtue? As great a stake depends on your
private act to-day, as followed their public and renowned steps. When
private men shall act with original views, the lustre will be transferred
from the actions of kings to those of gentlemen.

The world has been instructed by its kings, who have so magnetized the eyes
of nations. It has been taught by this colossal symbol the mutual reverence
that is due from man to man. The joyful loyalty with which men have
everywhere suffered the king, the noble, or the great proprietor to walk
among them by a law of his own, make his own scale of men and things, and
reverse theirs, pay for benefits not with money but with honor, and
represent the law in his person, was the hieroglyphic by which they
obscurely signified their consciousness of their own right and comeliness,
the right of every man.

The magnetism which all original action exerts is explained when we inquire
the reason of self-trust. Who is the Trustee? What is the aboriginal Self,
on which a universal reliance may be grounded? What is the nature and power
of that science-baffling star, without parallax, without calculable
elements, which shoots a ray of beauty even into trivial and impure actions,
if the least mark of independence appear? The inquiry leads us to that
source, at once the essence of genius, of virtue, and of life, which we call
Spontaneity or Instinct. We denote this primary wisdom as Intuition, whilst
all later teachings are tuitions. In that deep force, the last fact behind
which analysis cannot go, all things find their common origin. For, the
sense of being which in calm hours rises, we know not how, in the soul, is
not diverse from things, from space, from light, from time, from man, but
one with them, and proceeds obviously from the same source whence their life
and being also proceed. We first share the life by which things exist, and
afterwards see them as appearances in nature, and forget that we have shared
their cause. Here is the fountain of action and of thought. Here are the
lungs of that inspiration which giveth man wisdom, and which cannot be
denied without impiety and atheism. We lie in the lap of immense
intelligence, which makes us receivers of its truth and organs of its
activity. When we discern justice, when we discern truth, we do nothing of
ourselves, but allow a passage to its beams. If we ask whence this comes, if
we seek to pry into the soul that causes, all philosophy is at fault. Its
presence or its absence is all we can affirm. Every man discriminates
between the voluntary acts of his mind, and his involuntary perceptions, and
knows that to his involuntary perceptions a perfect faith is due. He may err
in the expression of them, but he knows that these things are so, like day
and night, not to be disputed. My wilful actions and acquisitions are but
roving; --- the idlest reverie, the faintest native emotion, command my
curiosity and respect. Thoughtless people contradict as readily the
statement of perceptions as of opinions, or rather much more readily; for,
they do not distinguish between perception and notion. They fancy that I
choose to see this or that thing. But perception is not whimsical, but
fatal. If I see a trait, my children will see it after me, and in course of
time, all mankind, --- although it may chance that no one has seen it before
me. For my perception of it is as much a fact as the sun.

The relations of the soul to the divine spirit are so pure, that it is
profane to seek to interpose helps. It must be that when God speaketh he
should communicate, not one thing, but all things; should fill the world
with his voice; should scatter forth light, nature, time, souls, from the
centre of the present thought; and new date and new create the whole.
Whenever a mind is simple, and receives a divine wisdom, old things pass
away, --- means, teachers, texts, temples fall; it lives now, and absorbs
past and future into the present hour. All things are made sacred by
relation to it, --- one as much as another. All things are dissolved to their
centre by their cause, and, in the universal miracle, petty and particular
miracles disappear. If, therefore, a man claims to know and speak of God,
and carries you backward to the phraseology of some old mouldered nation in
another country, in another world, believe him not. Is the acorn better than
the oak which is its fulness and completion? Is the parent better than the
child into whom he has cast his ripened being? Whence, then, this worship of
the past? The centuries are conspirators against the sanity and authority of
the soul. Time and space are but physiological colors which the eye makes,
but the soul is light; where it is, is day; where it was, is night; and
history is an impertinence and an injury, if it be any thing more than a
cheerful apologue or parable of my being and becoming.

Man is timid and apologetic; he is no longer upright; he dares not say `I
think,' `I am,' but quotes some saint or sage. He is ashamed before the
blade of grass or the blowing rose. These roses under my window make no
reference to former roses or to better ones; they are for what they are;
they exist with God to-day. There is no time to them. There is simply the
rose; it is perfect in every moment of its existence. Before a leaf-bud has
burst, its whole life acts; in the full-blown flower there is no more; in
the leafless root there is no less. Its nature is satisfied, and it
satisfies nature, in all moments alike. But man postpones or remembers; he
does not live in the present, but with reverted eye laments the past, or,
heedless of the riches that surround him, stands on tiptoe to foresee the
future. He cannot be happy and strong until he too lives with nature in the
present, above time.

This should be plain enough. Yet see what strong intellects dare not yet
hear God himself, unless he speak the phraseology of I know not what David,
or Jeremiah, or Paul. We shall not always set so great a price on a few
texts, on a few lives. We are like children who repeat by rote the sentences
of grandames and tutors, and, as they grow older, of the men of talents and
character they chance to see, --- painfully recollecting the exact words they
spoke; afterwards, when they come into the point of view which those had who
uttered these sayings, they understand them, and are willing to let the
words go; for, at any time, they can use words as good when occasion comes.
If we live truly, we shall see truly. It is as easy for the strong man to be
strong, as it is for the weak to be weak. When we have new perception, we
shall gladly disburden the memory of its hoarded treasures as old rubbish.
When a man lives with God, his voice shall be as sweet as the murmur of the
brook and the rustle of the corn.

And now at last the highest truth on this subject remains unsaid; probably
cannot be said; for all that we say is the far-off remembering of the
intuition. That thought, by what I can now nearest approach to say it, is
this. When good is near you, when you have life in yourself, it is not by
any known or accustomed way; you shall not discern the foot-prints of any
other; you shall not see the face of man; you shall not hear any name; ---
the way, the thought, the good, shall be wholly strange and new. It shall
exclude example and experience. You take the way from man, not to man. All
persons that ever existed are its forgotten ministers. Fear and hope are
alike beneath it. There is somewhat low even in hope. In the hour of vision,
there is nothing that can be called gratitude, nor properly joy. The soul
raised over passion beholds identity and eternal causation, perceives the
self-existence of Truth and Right, and calms itself with knowing that all
things go well. Vast spaces of nature, the Atlantic Ocean, the South Sea, ---
long intervals of time, years, centuries, --- are of no account. This which I
think and feel underlay every former state of life and circumstances, as it
does underlie my present, and what is called life, and what is called death.

Life only avails, not the having lived. Power ceases in the instant of
repose; it resides in the moment of transition from a past to a new state,
in the shooting of the gulf, in the darting to an aim. This one fact the
world hates, that the soul {\it becomes}; for that for ever degrades the past,
turns all riches to poverty, all reputation to a shame, confounds the saint
with the rogue, shoves Jesus and Judas equally aside. Why, then, do we prate
of self-reliance? Inasmuch as the soul is present, there will be power not
confident but agent. To talk of reliance is a poor external way of speaking.
Speak rather of that which relies, because it works and is. Who has more
obedience than I masters me, though he should not raise his finger. Round
him I must revolve by the gravitation of spirits. We fancy it rhetoric, when
we speak of eminent virtue. We do not yet see that virtue is Height, and
that a man or a company of men, plastic and permeable to principles, by the
law of nature must overpower and ride all cities, nations, kings, rich men,
poets, who are not.

This is the ultimate fact which we so quickly reach on this, as on every
topic, the resolution of all into the ever-blessed ONE. Self-existence is
the attribute of the Supreme Cause, and it constitutes the measure of good
by the degree in which it enters into all lower forms. All things real are
so by so much virtue as they contain. Commerce, husbandry, hunting, whaling,
war, eloquence, personal weight, are somewhat, and engage my respect as
examples of its presence and impure action. I see the same law working in
nature for conservation and growth. Power is in nature the essential measure
of right. Nature suffers nothing to remain in her kingdoms which cannot help
itself. The genesis and maturation of a planet, its poise and orbit, the
bended tree recovering itself from the strong wind, the vital resources of
every animal and vegetable, are demonstrations of the self-sufficing, and
therefore self-relying soul.

Thus all concentrates: let us not rove; let us sit at home with the cause.
Let us stun and astonish the intruding rabble of men and books and
institutions, by a simple declaration of the divine fact. Bid the invaders
take the shoes from off their feet, for God is here within. Let our
simplicity judge them, and our docility to our own law demonstrate the
poverty of nature and fortune beside our native riches.

But now we are a mob. Man does not stand in awe of man, nor is his genius
admonished to stay at home, to put itself in communication with the internal
ocean, but it goes abroad to beg a cup of water of the urns of other men. We
must go alone. I like the silent church before the service begins, better
than any preaching. How far off, how cool, how chaste the persons look,
begirt each one with a precinct or sanctuary! So let us always sit. Why
should we assume the faults of our friend, or wife, or father, or child,
because they sit around our hearth, or are said to have the same blood? All
men have my blood, and I have all men's. Not for that will I adopt their
petulance or folly, even to the extent of being ashamed of it. But your
isolation must not be mechanical, but spiritual, that is, must be elevation.
At times the whole world seems to be in conspiracy to importune you with
emphatic trifles. Friend, client, child, sickness, fear, want, charity, all
knock at once at thy closet door, and say, --- `Come out unto us.' But keep
thy state; come not into their confusion. The power men possess to annoy me,
I give them by a weak curiosity. No man can come near me but through my act.
``What we love that we have, but by desire we bereave ourselves of the love.''

If we cannot at once rise to the sanctities of obedience and faith, let us
at least resist our temptations; let us enter into the state of war, and
wake Thor and Woden, courage and constancy, in our Saxon breasts. This is to
be done in our smooth times by speaking the truth. Check this lying
hospitality and lying affection. Live no longer to the expectation of these
deceived and deceiving people with whom we converse. Say to them, O father,
O mother, O wife, O brother, O friend, I have lived with you after
appearances hitherto. Henceforward I am the truth's. Be it known unto you
that henceforward I obey no law less than the eternal law. I will have no
covenants but proximities. I shall endeavour to nourish my parents, to
support my family, to be the chaste husband of one wife, --- but these
relations I must fill after a new and unprecedented way. I appeal from your
customs. I must be myself. I cannot break myself any longer for you, or you.
If you can love me for what I am, we shall be the happier. If you cannot, I
will still seek to deserve that you should. I will not hide my tastes or
aversions. I will so trust that what is deep is holy, that I will do
strongly before the sun and moon whatever inly rejoices me, and the heart
appoints. If you are noble, I will love you; if you are not, I will not hurt
you and myself by hypocritical attentions. If you are true, but not in the
same truth with me, cleave to your companions; I will seek my own. I do this
not selfishly, but humbly and truly. It is alike your interest, and mine,
and all men's, however long we have dwelt in lies, to live in truth. Does
this sound harsh to-day? You will soon love what is dictated by your nature
as well as mine, and, if we follow the truth, it will bring us out safe at
last. --- But so you may give these friends pain. Yes, but I cannot sell my
liberty and my power, to save their sensibility. Besides, all persons have
their moments of reason, when they look out into the region of absolute
truth; then will they justify me, and do the same thing.

The populace think that your rejection of popular standards is a rejection
of all standard, and mere antinomianism; and the bold sensualist will use
the name of philosophy to gild his crimes. But the law of consciousness
abides. There are two confessionals, in one or the other of which we must be
shriven. You may fulfil your round of duties by clearing yourself in the
{\it direct}, or in the {\it reflex} way. Consider whether you
have satisfied your
relations to father, mother, cousin, neighbour, town, cat, and dog; whether
any of these can upbraid you. But I may also neglect this reflex standard,
and absolve me to myself. I have my own stern claims and perfect circle. It
denies the name of duty to many offices that are called duties. But if I can
discharge its debts, it enables me to dispense with the popular code. If any
one imagines that this law is lax, let him keep its commandment one day.

And truly it demands something godlike in him who has cast off the common
motives of humanity, and has ventured to trust himself for a taskmaster.
High be his heart, faithful his will, clear his sight, that he may in good
earnest be doctrine, society, law, to himself, that a simple purpose may be
to him as strong as iron necessity is to others!

If any man consider the present aspects of what is called by distinction
{\it society}, he will see the need of these ethics. The sinew and heart of man
seem to be drawn out, and we are become timorous, desponding whimperers. We
are afraid of truth, afraid of fortune, afraid of death, and afraid of each
other. Our age yields no great and perfect persons. We want men and women
who shall renovate life and our social state, but we see that most natures
are insolvent, cannot satisfy their own wants, have an ambition out of all
proportion to their practical force, and do lean and beg day and night
continually. Our housekeeping is mendicant, our arts, our occupations, our
marriages, our religion, we have not chosen, but society has chosen for us.
We are parlour soldiers. We shun the rugged battle of fate, where strength
is born.

If our young men miscarry in their first enterprises, they lose all heart.
If the young merchant fails, men say he is {\it ruined}. If the finest genius
studies at one of our colleges, and is not installed in an office within one
year afterwards in the cities or suburbs of Boston or New York, it seems to
his friends and to himself that he is right in being disheartened, and in
complaining the rest of his life. A sturdy lad from New Hampshire or
Vermont, who in turn tries all the professions, who {\it teams it},
{\it farms it}, {\it peddles}, keeps a school, preaches,
edits a newspaper, goes to Congress,
buys a township, and so forth, in successive years, and always, like a cat,
falls on his feet, is worth a hundred of these city dolls. He walks abreast
with his days, and feels no shame in not `studying a profession,' for he
does not postpone his life, but lives already. He has not one chance, but a
hundred chances. Let a Stoic open the resources of man, and tell men they
are not leaning willows, but can and must detach themselves; that with the
exercise of self-trust, new powers shall appear; that a man is the word made
flesh, born to shed healing to the nations, that he should be ashamed of our
compassion, and that the moment he acts from himself, tossing the laws, the
books, idolatries, and customs out of the window, we pity him no more, but
thank and revere him, --- and that teacher shall restore the life of man to
splendor, and make his name dear to all history.

It is easy to see that a greater self-reliance must work a revolution in all
the offices and relations of men; in their religion; in their education; in
their pursuits; their modes of living; their association; in their property;
in their speculative views.

1. In what prayers do men allow themselves! That which they call a holy
office is not so much as brave and manly. Prayer looks abroad and asks for
some foreign addition to come through some foreign virtue, and loses itself
in endless mazes of natural and supernatural, and mediatorial and
miraculous. Prayer that craves a particular commodity, --- any thing less
than all good, --- is vicious. Prayer is the contemplation of the facts of
life from the highest point of view. It is the soliloquy of a beholding and
jubilant soul. It is the spirit of God pronouncing his works good. But
prayer as a means to effect a private end is meanness and theft. It supposes
dualism and not unity in nature and consciousness. As soon as the man is at
one with God, he will not beg. He will then see prayer in all action. The
prayer of the farmer kneeling in his field to weed it, the prayer of the
rower kneeling with the stroke of his oar, are true prayers heard throughout
nature, though for cheap ends. Caratach, in Fletcher's Bonduca, when
admonished to inquire the mind of the god Audate, replies, ---

\begin{verse}
``His hidden meaning lies in our endeavours;\\
Our valors are our best gods.''
\end{verse}

Another sort of false prayers are our regrets. Discontent is the want of
self-reliance: it is infirmity of will. Regret calamities, if you can
thereby help the sufferer; if not, attend your own work, and already the
evil begins to be repaired. Our sympathy is just as base. We come to them
who weep foolishly, and sit down and cry for company, instead of imparting
to them truth and health in rough electric shocks, putting them once more in
communication with their own reason. The secret of fortune is joy in our
hands. Welcome evermore to gods and men is the self-helping man. For him all
doors are flung wide: him all tongues greet, all honors crown, all eyes
follow with desire. Our love goes out to him and embraces him, because he
did not need it. We solicitously and apologetically caress and celebrate
him, because he held on his way and scorned our disapprobation. The gods
love him because men hated him. ``To the persevering mortal,'' said Zoroaster,
``the blessed Immortals are swift.''

As men's prayers are a disease of the will, so are their creeds a disease of
the intellect. They say with those foolish Israelites, `Let not God speak to
us, lest we die. Speak thou, speak any man with us, and we will obey.'
Everywhere I am hindered of meeting God in my brother, because he has shut
his own temple doors, and recites fables merely of his brother's, or his
brother's brother's God. Every new mind is a new classification. If it prove
a mind of uncommon activity and power, a Locke, a Lavoisier, a Hutton, a
Bentham, a Fourier, it imposes its classification on other men, and lo! a
new system. In proportion to the depth of the thought, and so to the number
of the objects it touches and brings within reach of the pupil, is his
complacency. But chiefly is this apparent in creeds and churches, which are
also classifications of some powerful mind acting on the elemental thought
of duty, and man's relation to the Highest. Such is Calvinism, Quakerism,
Swedenborgism. The pupil takes the same delight in subordinating every thing
to the new terminology, as a girl who has just learned botany in seeing a
new earth and new seasons thereby. It will happen for a time, that the pupil
will find his intellectual power has grown by the study of his master's
mind. But in all unbalanced minds, the classification is idolized, passes
for the end, and not for a speedily exhaustible means, so that the walls of
the system blend to their eye in the remote horizon with the walls of the
universe; the luminaries of heaven seem to them hung on the arch their
master built. They cannot imagine how you aliens have any right to see, ---
how you can see; `It must be somehow that you stole the light from us.' They
do not yet perceive, that light, unsystematic, indomitable, will break into
any cabin, even into theirs. Let them chirp awhile and call it their own. If
they are honest and do well, presently their neat new pinfold will be too
strait and low, will crack, will lean, will rot and vanish, and the immortal
light, all young and joyful, million-orbed, million-colored, will beam over
the universe as on the first morning.

2. It is for want of self-culture that the superstition of Travelling, whose
idols are Italy, England, Egypt, retains its fascination for all educated
Americans. They who made England, Italy, or Greece venerable in the
imagination did so by sticking fast where they were, like an axis of the
earth. In manly hours, we feel that duty is our place. The soul is no
traveller; the wise man stays at home, and when his necessities, his duties,
on any occasion call him from his house, or into foreign lands, he is at
home still, and shall make men sensible by the expression of his
countenance, that he goes the missionary of wisdom and virtue, and visits
cities and men like a sovereign, and not like an interloper or a valet.

I have no churlish objection to the circumnavigation of the globe, for the
purposes of art, of study, and benevolence, so that the man is first
domesticated, or does not go abroad with the hope of finding somewhat
greater than he knows. He who travels to be amused, or to get somewhat which
he does not carry, travels away from himself, and grows old even in youth
among old things. In Thebes, in Palmyra, his will and mind have become old
and dilapidated as they. He carries ruins to ruins.

Travelling is a fool's paradise. Our first journeys discover to us the
indifference of places. At home I dream that at Naples, at Rome, I can be
intoxicated with beauty, and lose my sadness. I pack my trunk, embrace my
friends, embark on the sea, and at last wake up in Naples, and there beside
me is the stern fact, the sad self, unrelenting, identical, that I fled
from. I seek the Vatican, and the palaces. I affect to be intoxicated with
sights and suggestions, but I am not intoxicated. My giant goes with me
wherever I go.

3. But the rage of travelling is a symptom of a deeper unsoundness affecting
the whole intellectual action. The intellect is vagabond, and our system of
education fosters restlessness. Our minds travel when our bodies are forced
to stay at home. We imitate; and what is imitation but the travelling of the
mind? Our houses are built with foreign taste; our shelves are garnished
with foreign ornaments; our opinions, our tastes, our faculties, lean, and
follow the Past and the Distant. The soul created the arts wherever they
have flourished. It was in his own mind that the artist sought his model. It
was an application of his own thought to the thing to be done and the
conditions to be observed. And why need we copy the Doric or the Gothic
model? Beauty, convenience, grandeur of thought, and quaint expression are
as near to us as to any, and if the American artist will study with hope and
love the precise thing to be done by him, considering the climate, the soil,
the length of the day, the wants of the people, the habit and form of the
government, he will create a house in which all these will find themselves
fitted, and taste and sentiment will be satisfied also.

Insist on yourself; never imitate. Your own gift you can present every
moment with the cumulative force of a whole life's cultivation; but of the
adopted talent of another, you have only an extemporaneous, half possession.
That which each can do best, none but his Maker can teach him. No man yet
knows what it is, nor can, till that person has exhibited it. Where is the
master who could have taught Shakspeare? Where is the master who could have
instructed Franklin, or Washington, or Bacon, or Newton? Every great man is
a unique. The Scipionism of Scipio is precisely that part he could not
borrow. Shakspeare will never be made by the study of Shakspeare. Do that
which is assigned you, and you cannot hope too much or dare too much. There
is at this moment for you an utterance brave and grand as that of the
colossal chisel of Phidias, or trowel of the Egyptians, or the pen of Moses,
or Dante, but different from all these. Not possibly will the soul all rich,
all eloquent, with thousand-cloven tongue, deign to repeat itself; but if
you can hear what these patriarchs say, surely you can reply to them in the
same pitch of voice; for the ear and the tongue are two organs of one
nature. Abide in the simple and noble regions of thy life, obey thy heart,
and thou shalt reproduce the Foreworld again.

4. As our Religion, our Education, our Art look abroad, so does our spirit
of society. All men plume themselves on the improvement of society, and no
man improves.

Society never advances. It recedes as fast on one side as it gains on the
other. It undergoes continual changes; it is barbarous, it is civilized, it
is christianized, it is rich, it is scientific; but this change is not
amelioration. For every thing that is given, something is taken. Society
acquires new arts, and loses old instincts. What a contrast between the
well-clad, reading, writing, thinking American, with a watch, a pencil, and
a bill of exchange in his pocket, and the naked New Zealander, whose
property is a club, a spear, a mat, and an undivided twentieth of a shed to
sleep under! But compare the health of the two men, and you shall see that
the white man has lost his aboriginal strength. If the traveller tell us
truly, strike the savage with a broad axe, and in a day or two the flesh
shall unite and heal as if you struck the blow into soft pitch, and the same
blow shall send the white to his grave.

The civilized man has built a coach, but has lost the use of his feet. He is
supported on crutches, but lacks so much support of muscle. He has a fine
Geneva watch, but he fails of the skill to tell the hour by the sun. A
Greenwich nautical almanac he has, and so being sure of the information when
he wants it, the man in the street does not know a star in the sky. The
solstice he does not observe; the equinox he knows as little; and the whole
bright calendar of the year is without a dial in his mind. His note-books
impair his memory; his libraries overload his wit; the insurance-office
increases the number of accidents; and it may be a question whether
machinery does not encumber; whether we have not lost by refinement some
energy, by a Christianity entrenched in establishments and forms, some vigor
of wild virtue. For every Stoic was a Stoic; but in Christendom where is the
Christian?

There is no more deviation in the moral standard than in the standard of
height or bulk. No greater men are now than ever were. A singular equality
may be observed between the great men of the first and of the last ages; nor
can all the science, art, religion, and philosophy of the nineteenth century
avail to educate greater men than Plutarch's heroes, three or four and
twenty centuries ago. Not in time is the race progressive. Phocion,
Socrates, Anaxagoras, Diogenes, are great men, but they leave no class. He
who is really of their class will not be called by their name, but will be
his own man, and, in his turn, the founder of a sect. The arts and
inventions of each period are only its costume, and do not invigorate men.
The harm of the improved machinery may compensate its good. Hudson and
Behring accomplished so much in their fishing-boats, as to astonish Parry
and Franklin, whose equipment exhausted the resources of science and art.
Galileo, with an opera-glass, discovered a more splendid series of celestial
phenomena than any one since. Columbus found the New World in an undecked
boat. It is curious to see the periodical disuse and perishing of means and
machinery, which were introduced with loud laudation a few years or
centuries before. The great genius returns to essential man. We reckoned the
improvements of the art of war among the triumphs of science, and yet
Napoleon conquered Europe by the bivouac, which consisted of falling back on
naked valor, and disencumbering it of all aids. The Emperor held it
impossible to make a perfect army, says Las Casas, ``without abolishing our
arms, magazines, commissaries, and carriages, until, in imitation of the
Roman custom, the soldier should receive his supply of corn, grind it in his
hand-mill, and bake his bread himself.''

Society is a wave. The wave moves onward, but the water of which it is
composed does not. The same particle does not rise from the valley to the
ridge. Its unity is only phenomenal. The persons who make up a nation
to-day, next year die, and their experience with them.

And so the reliance on Property, including the reliance on governments which
protect it, is the want of self-reliance. Men have looked away from
themselves and at things so long, that they have come to esteem the
religious, learned, and civil institutions as guards of property, and they
deprecate assaults on these, because they feel them to be assaults on
property. They measure their esteem of each other by what each has, and not
by what each is. But a cultivated man becomes ashamed of his property, out
of new respect for his nature. Especially he hates what he has, if he see
that it is accidental, --- came to him by inheritance, or gift, or crime;
then he feels that it is not having; it does not belong to him, has no root
in him, and merely lies there, because no revolution or no robber takes it
away. But that which a man is, does always by necessity acquire, and what the
man acquires is living property, which does not wait the beck of rulers, or
mobs, or revolutions, or fire, or storm, or bankruptcies, but perpetually
renews itself wherever the man breathes. ``Thy lot or portion of life,'' said
the Caliph Ali, ``is seeking after thee; therefore be at rest from seeking
after it.'' Our dependence on these foreign goods leads us to our slavish
respect for numbers. The political parties meet in numerous conventions; the
greater the concourse, and with each new uproar of announcement, The
delegation from Essex! The Democrats from New Hampshire! The Whigs of Maine!
the young patriot feels himself stronger than before by a new thousand of
eyes and arms. In like manner the reformers summon conventions, and vote and
resolve in multitude. Not so, O friends! will the God deign to enter and
inhabit you, but by a method precisely the reverse. It is only as a man puts
off all foreign support, and stands alone, that I see him to be strong and
to prevail. He is weaker by every recruit to his banner. Is not a man better
than a town? Ask nothing of men, and in the endless mutation, thou only firm
column must presently appear the upholder of all that surrounds thee. He who
knows that power is inborn, that he is weak because he has looked for good
out of him and elsewhere, and so perceiving, throws himself unhesitatingly
on his thought, instantly rights himself, stands in the erect position,
commands his limbs, works miracles; just as a man who stands on his feet is
stronger than a man who stands on his head.

So use all that is called Fortune. Most men gamble with her, and gain all,
and lose all, as her wheel rolls. But do thou leave as unlawful these
winnings, and deal with Cause and Effect, the chancellors of God. In the
Will work and acquire, and thou hast chained the wheel of Chance, and shalt
sit hereafter out of fear from her rotations. A political victory, a rise of
rents, the recovery of your sick, or the return of your absent friend, or
some other favorable event, raises your spirits, and you think good days are
preparing for you. Do not believe it. Nothing can bring you peace but
yourself. Nothing can bring you peace but the triumph of principles.
\label{theend}
\end{document}
