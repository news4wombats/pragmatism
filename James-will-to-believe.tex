\documentclass[12pt]{article}
\usepackage{hyperref,geometry,setspace}
\usepackage[symbol*,perpage]{footmisc}
\geometry{width=6in,height=9in,centering}
% changes font to TeX Gyre Schola (Century Schoolbook)
\usepackage{tgschola}\usepackage[T1]{fontenc}

\begin{document}
%\renewcommand*{\thefootnote}{\fnsymbol{footnote}}
\setcounter{footnote}{0}
\hypersetup{pdfinfo={Title={The Will to Believe}, Author={William James}}, pdfborder = {0 0 0 0}}

\paragraph{A note about the text:}This text was prepared by P.D. Magnus from the Project Gutenberg edition, which in turn was produced by Al Haines. The marginal numbers indicate the pagination of the edition Haines worked from:

\smallskip
\noindent\emph{The Will to Believe and Other Essays in Popular Philosophy}. New York: Longmans, Green, And Co. 1912.

\section*{\textsc{The Will To Believe}\footnote{An Address to the Philosophical Clubs of Yale and Brown Universities. Published in the New World, June, 1896.}}

{\textsc{William James}}

\medskip

\marginpar{1}
In the recently published Life by Leslie Stephen of his brother, Fitz-James, there is an account of a school to which the latter went when he was a boy.  The teacher, a certain Mr. Guest, used to converse with his pupils in this wise: ``Gurney, what is the difference between justification and sanctification?--- Stephen, prove the omnipotence of God!'' etc.  In the midst of our Harvard freethinking and indifference we are prone to imagine that here at your good old orthodox College conversation continues to be somewhat upon this order; and to show you that we at Harvard have not lost all interest in these vital subjects, I have brought with me to-night something like a sermon on justification by faith to read to you,--- I mean an essay in justification \emph{of} faith, a defence of our right to adopt a believing attitude in religious matters, in spite of the fact that our merely logical \marginpar{2} intellect may not have been coerced.  `The Will to Believe,' accordingly, is the title of my paper.

I have long defended to my own students the lawfulness of voluntarily adopted faith; but as soon as they have got well imbued with the logical spirit, they have as a rule refused to admit my contention to be lawful philosophically, even though in point of fact they were personally all the time chock-full of some faith or other themselves. I am all the while, however, so profoundly convinced that my own position is correct, that your invitation has seemed to me a good occasion to make my statements more clear.  Perhaps your minds will be more open than those with which I have hitherto had to deal.  I will be as little technical as I can, though I must begin by setting up some technical distinctions that will help us in the end.

\section*{I.}

Let us give the name of \emph{hypothesis} to anything that may be proposed to our belief; and just as the electricians speak of live and dead wires, let us speak of any hypothesis as either \emph{live} or \emph{dead}.  A live hypothesis is one which appeals as a real possibility to him to whom it is proposed.  If I ask you to believe in the Mahdi, the notion makes no electric connection with your nature,--- it refuses to scintillate with any credibility at all.  As an hypothesis it is completely dead.  To an Arab, however (even if he be not one of the Mahdi's followers), the hypothesis is among the mind's possibilities: it is alive.  This shows that deadness and liveness in an hypothesis are not intrinsic properties, but relations to the \marginpar{3} individual thinker.  They are measured by his willingness to act.  The maximum of liveness in an hypothesis means willingness to act irrevocably. Practically, that means belief; but there is some believing tendency wherever there is willingness to act at all.

Next, let us call the decision between two hypotheses an \emph{option}. Options may be of several kinds.  They may be--- 1, \emph{living} or \emph{dead}; 2, \emph{forced} or \emph{avoidable}; 3, \emph{momentous} or \emph{trivial}; and for our purposes we may call an option a \emph{genuine} option when it is of the forced, living, and momentous kind.

1.  A living option is one in which both hypotheses are live ones.  If I say to you: ``Be a theosophist or be a Mohammedan,'' it is probably a dead option, because for you neither hypothesis is likely to be alive. But if I say: ``Be an agnostic or be a Christian,'' it is otherwise: trained as you are, each hypothesis makes some appeal, however small, to your belief.

2.  Next, if I say to you: ``Choose between going out with your umbrella or without it,'' I do not offer you a genuine option, for it is not forced.  You can easily avoid it by not going out at all.  Similarly, if I say, ``Either love me or hate me,'' ``Either call my theory true or call it false,'' your option is avoidable.  You may remain indifferent to me, neither loving nor hating, and you may decline to offer any judgment as to my theory.  But if I say, ``Either accept this truth or go without it,'' I put on you a forced option, for there is no standing place outside of the alternative.  Every dilemma based on a complete logical disjunction, with no possibility of not choosing, is an option of this forced kind.

\marginpar{4}

3.  Finally, if I were Dr. Nansen and proposed to you to join my North Pole expedition, your option would be momentous; for this would probably be your only similar opportunity, and your choice now would either exclude you from the North Pole sort of immortality altogether or put at least the chance of it into your hands.  He who refuses to embrace a unique opportunity loses the prize as surely as if he tried and failed.  \emph{Per contra}, the option is trivial when the opportunity is not unique, when the stake is insignificant, or when the decision is reversible if it later prove unwise.  Such trivial options abound in the scientific life.  A chemist finds an hypothesis live enough to spend a year in its verification: he believes in it to that extent. But if his experiments prove inconclusive either way, he is quit for his loss of time, no vital harm being done.

It will facilitate our discussion if we keep all these distinctions well in mind.

\section*{II.}

The next matter to consider is the actual psychology of human opinion. When we look at certain facts, it seems as if our passional and volitional nature lay at the root of all our convictions.  When we look at others, it seems as if they could do nothing when the intellect had once said its say.  Let us take the latter facts up first.

Does it not seem preposterous on the very face of it to talk of our opinions being modifiable at will?  Can our will either help or hinder our intellect in its perceptions of truth?  Can we, by just willing it, believe that Abraham Lincoln's existence is a myth, \marginpar{5} and that the portraits of him in McClure's Magazine are all of some one else?  Can we, by any effort of our will, or by any strength of wish that it were true, believe ourselves well and about when we are roaring with rheumatism in bed, or feel certain that the sum of the two one-dollar bills in our pocket must be a hundred dollars?  We can say any of these things, but we are absolutely impotent to believe them; and of just such things is the whole fabric of the truths that we do believe in made up,--- matters of fact, immediate or remote, as Hume said, and relations between ideas, which are either there or not there for us if we see them so, and which if not there cannot be put there by any action of our own.

In Pascal's Thoughts there is a celebrated passage known in literature as Pascal's wager.  In it he tries to force us into Christianity by reasoning as if our concern with truth resembled our concern with the stakes in a game of chance.  Translated freely his words are these: You must either believe or not believe that God is--- which will you do? Your human reason cannot say.  A game is going on between you and the nature of things which at the day of judgment will bring out either heads or tails.  Weigh what your gains and your losses would be if you should stake all you have on heads, or God's existence: if you win in such case, you gain eternal beatitude; if you lose, you lose nothing at all.  If there were an infinity of chances, and only one for God in this wager, still you ought to stake your all on God; for though you surely risk a finite loss by this procedure, any finite loss is reasonable, even a certain one is reasonable, if there is but the possibility of \marginpar{6} infinite gain.  Go, then, and take holy water, and have masses said; belief will come and stupefy your scruples,--- \emph{Cela vous fera croire et vous ab�tira}.  Why should you not?  At bottom, what have you to lose?

You probably feel that when religious faith expresses itself thus, in the language of the gaming-table, it is put to its last trumps.  Surely Pascal's own personal belief in masses and holy water had far other springs; and this celebrated page of his is but an argument for others, a last desperate snatch at a weapon against the hardness of the unbelieving heart.  We feel that a faith in masses and holy water adopted wilfully after such a mechanical calculation would lack the inner soul of faith's reality; and if we were ourselves in the place of the Deity, we should probably take particular pleasure in cutting off believers of this pattern from their infinite reward.  It is evident that unless there be some pre-existing tendency to believe in masses and holy water, the option offered to the will by Pascal is not a living option.  Certainly no Turk ever took to masses and holy water on its account; and even to us Protestants these means of salvation seem such foregone impossibilities that Pascal's logic, invoked for them specifically, leaves us unmoved.  As well might the Mahdi write to us, saying, ``I am the Expected One whom God has created in his effulgence. You shall be infinitely happy if you confess me; otherwise you shall be cut off from the light of the sun.  Weigh, then, your infinite gain if I am genuine against your finite sacrifice if I am not!''  His logic would be that of Pascal; but he would vainly use it on us, for the hypothesis he offers us is dead.  No tendency to act on it exists in us to any degree.

\marginpar{7}

The talk of believing by our volition seems, then, from one point of view, simply silly.  From another point of view it is worse than silly, it is vile.  When one turns to the magnificent edifice of the physical sciences, and sees how it was reared; what thousands of disinterested moral lives of men lie buried in its mere foundations; what patience and postponement, what choking down of preference, what submission to the icy laws of outer fact are wrought into its very stones and mortar; how absolutely impersonal it stands in its vast augustness,--- then how besotted and contemptible seems every little sentimentalist who comes blowing his voluntary smoke-wreaths, and pretending to decide things from out of his private dream!  Can we wonder if those bred in the rugged and manly school of science should feel like spewing such subjectivism out of their mouths?  The whole system of loyalties which grow up in the schools of science go dead against its toleration; so that it is only natural that those who have caught the scientific fever should pass over to the opposite extreme, and write sometimes as if the incorruptibly truthful intellect ought positively to prefer bitterness and unacceptableness to the heart in its cup.

  It fortifies my soul to know   That, though I perish, Truth is so--- 

sings Clough, while Huxley exclaims: ``My only consolation lies in the reflection that, however bad our posterity may become, so far as they hold by the plain rule of not pretending to believe what they have no reason to believe, because it may be to their advantage so to pretend [the word `pretend' is surely here redundant], they will not have reached the \marginpar{8} lowest depth of immorality.''  And that delicious \emph{enfant terrible} Clifford writes; ``Belief is desecrated when given to unproved and unquestioned statements for the solace and private pleasure of the believer,\ldots  Whoso would deserve well of his fellows in this matter will guard the purity of his belief with a very fanaticism of jealous care, lest at any time it should rest on an unworthy object, and catch a stain which can never be wiped away\ldots. If [a] belief has been accepted on insufficient evidence [even though the belief be true, as Clifford on the same page explains] the pleasure is a stolen one\ldots.  It is sinful because it is stolen in defiance of our duty to mankind.  That duty is to guard ourselves from such beliefs as from a pestilence which may shortly master our own body and then spread to the rest of the town\ldots.  It is wrong always, everywhere, and for every one, to believe anything upon insufficient evidence.''

\section*{III.}

All this strikes one as healthy, even when expressed, as by Clifford, with somewhat too much of robustious pathos in the voice.  Free-will and simple wishing do seem, in the matter of our credences, to be only fifth wheels to the coach.  Yet if any one should thereupon assume that intellectual insight is what remains after wish and will and sentimental preference have taken wing, or that pure reason is what then settles our opinions, he would fly quite as directly in the teeth of the facts.

It is only our already dead hypotheses that our willing nature is unable to bring to life again  But what has made them dead for us is for the most part \marginpar{9} a previous action of our willing nature of an antagonistic kind.  When I say `willing nature,' I do not mean only such deliberate volitions as may have set up habits of belief that we cannot now escape from,--- I mean all such factors of belief as fear and hope, prejudice and passion, imitation and partisanship, the circumpressure of our caste and set.  As a matter of fact we find ourselves believing, we hardly know how or why.  Mr. Balfour gives the name of `authority' to all those influences, born of the intellectual climate, that make hypotheses possible or impossible for us, alive or dead.  Here in this room, we all of us believe in molecules and the conservation of energy, in democracy and necessary progress, in Protestant Christianity and the duty of fighting for `the doctrine of the immortal Monroe,' all for no reasons worthy of the name.  We see into these matters with no more inner clearness, and probably with much less, than any disbeliever in them might possess.  His unconventionality would probably have some grounds to show for its conclusions; but for us, not insight, but the \emph{prestige} of the opinions, is what makes the spark shoot from them and light up our sleeping magazines of faith.  Our reason is quite satisfied, in nine hundred and ninety-nine cases out of every thousand of us, if it can find a few arguments that will do to recite in case our credulity is criticised by some one else.  Our faith is faith in some one else's faith, and in the greatest matters this is most the case.  Our belief in truth itself, for instance, that there is a truth, and that our minds and it are made for each other,--- what is it but a passionate affirmation of desire, in which our social system backs us up?  We want to have a truth; we want to believe that our \marginpar{10} experiments and studies and discussions must put us in a continually better and better position towards it; and on this line we agree to fight out our thinking lives.  But if a pyrrhonistic sceptic asks us \emph{how we know} all this, can our logic find a reply?  No! certainly it cannot.  It is just one volition against another,--- we willing to go in for life upon a trust or assumption which he, for his part, does not care to make.\footnote{Compare the admirable page 310 in S. H. Hodgson's ``Time and Space,'' London, 1865.}

As a rule we disbelieve all facts and theories for which we have no use.  Clifford's cosmic emotions find no use for Christian feelings. Huxley belabors the bishops because there is no use for sacerdotalism in his scheme of life.  Newman, on the contrary, goes over to Romanism, and finds all sorts of reasons good for staying there, because a priestly system is for him an organic need and delight.  Why do so few `scientists' even look at the evidence for telepathy, so called? Because they think, as a leading biologist, now dead, once said to me, that even if such a thing were true, scientists ought to band together to keep it suppressed and concealed.  It would undo the uniformity of Nature and all sorts of other things without which scientists cannot carry on their pursuits.  But if this very man had been shown something which as a scientist he might \emph{do} with telepathy, he might not only have examined the evidence, but even have found it good enough.  This very law which the logicians would impose upon us--- if I may give the name of logicians to those who would rule out our willing nature here--- is based on nothing but their own natural wish to exclude all elements for \marginpar{11} which they, in their professional quality of logicians, can find no use.

Evidently, then, our non-intellectual nature does influence our convictions.  There are passional tendencies and volitions which run before and others which come after belief, and it is only the latter that are too late for the fair; and they are not too late when the previous passional work has been already in their own direction. Pascal's argument, instead of being powerless, then seems a regular clincher, and is the last stroke needed to make our faith in masses and holy water complete.  The state of things is evidently far from simple; and pure insight and logic, whatever they might do ideally, are not the only things that really do produce our creeds.

\section*{IV.}

Our next duty, having recognized this mixed-up state of affairs, is to ask whether it be simply reprehensible and pathological, or whether, on the contrary, we must treat it as a normal element in making up our minds.  The thesis I defend is, briefly stated, this:  \emph{Our passional nature not only lawfully may, but must, decide an option between propositions, whenever it is a genuine option that cannot by its nature be decided on intellectual grounds; for to say, under such circumstances, ``Do not decide, but leave the question open,'' is itself a passional decision,--- just like deciding yes or no,--- and is attended with the same risk of losing the truth}.  The thesis thus abstractly expressed will, I trust, soon become quite clear.  But I must first indulge in a bit more of preliminary work.

 \marginpar{12}

\section*{V.}

It will be observed that for the purposes of this discussion we are on `dogmatic' ground,--- ground, I mean, which leaves systematic philosophical scepticism altogether out of account.  The postulate that there is truth, and that it is the destiny of our minds to attain it, we are deliberately resolving to make, though the sceptic will not make it.  We part company with him, therefore, absolutely, at this point. But the faith that truth exists, and that our minds can find it, may be held in two ways.  We may talk of the \emph{empiricist} way and of the \emph{absolutist} way of believing in truth.  The absolutists in this matter say that we not only can attain to knowing truth, but we can \emph{know when} we have attained to knowing it; while the empiricists think that although we may attain it, we cannot infallibly know when.  To \emph{know} is one thing, and to know for certain \emph{that} we know is another.  One may hold to the first being possible without the second; hence the empiricists and the absolutists, although neither of them is a sceptic in the usual philosophic sense of the term, show very different degrees of dogmatism in their lives.

If we look at the history of opinions, we see that the empiricist tendency has largely prevailed in science, while in philosophy the absolutist tendency has had everything its own way.  The characteristic sort of happiness, indeed, which philosophies yield has mainly consisted in the conviction felt by each successive school or system that by it bottom-certitude had been attained.  ``Other philosophies are collections of opinions, mostly false; \emph{my} philosophy \marginpar{13} gives standing-ground forever,''--- who does not recognize in this the key-note of every system worthy of the name?  A system, to be a system at all, must come as a \emph{closed} system, reversible in this or that detail, perchance, but in its essential features never!

Scholastic orthodoxy, to which one must always go when one wishes to find perfectly clear statement, has beautifully elaborated this absolutist conviction in a doctrine which it calls that of `objective evidence.'  If, for example, I am unable to doubt that I now exist before you, that two is less than three, or that if all men are mortal then I am mortal too, it is because these things illumine my intellect irresistibly.  The final ground of this objective evidence possessed by certain propositions is the \emph{adaequatio intellect�s nostri cum r�}. The certitude it brings involves an \emph{aptitudinem ad extorquendum certum assensum} on the part of the truth envisaged, and on the side of the subject a \emph{quietem in cognitione}, when once the object is mentally received, that leaves no possibility of doubt behind; and in the whole transaction nothing operates but the \emph{entitas ipsa} of the object and the \emph{entitas ipsa} of the mind.  We slouchy modern thinkers dislike to talk in Latin,--- indeed, we dislike to talk in set terms at all; but at bottom our own state of mind is very much like this whenever we uncritically abandon ourselves: You believe in objective evidence, and I do.  Of some things we feel that we are certain: we know, and we know that we do know.  There is something that gives a click inside of us, a bell that strikes twelve, when the hands of our mental clock have swept the dial and meet over the meridian hour.  The greatest empiricists among us are only empiricists on reflection: when \marginpar{14} left to their instincts, they dogmatize like infallible popes.  When the Cliffords tell us how sinful it is to be Christians on such `insufficient evidence,' insufficiency is really the last thing they have in mind. For them the evidence is absolutely sufficient, only it makes the other way.  They believe so completely in an anti-christian order of the universe that there is no living option: Christianity is a dead hypothesis from the start.

\section*{VI.}

But now, since we are all such absolutists by instinct, what in our quality of students of philosophy ought we to do about the fact?  Shall we espouse and indorse it?  Or shall we treat it as a weakness of our nature from which we must free ourselves, if we can?

I sincerely believe that the latter course is the only one we can follow as reflective men.  Objective evidence and certitude are doubtless very fine ideals to play with, but where on this moonlit and dream-visited planet are they found?  I am, therefore, myself a complete empiricist so far as my theory of human knowledge goes.  I live, to be sure, by the practical faith that we must go on experiencing and thinking over our experience, for only thus can our opinions grow more true; but to hold any one of them--- I absolutely do not care which--- as if it never could be reinterpretable or corrigible, I believe to be a tremendously mistaken attitude, and I think that the whole history of philosophy will bear me out.  There is but one indefectibly certain truth, and that is the truth that pyrrhonistic scepticism itself leaves \marginpar{15} standing,--- the truth that the present phenomenon of consciousness exists.  That, however, is the bare starting-point of knowledge, the mere admission of a stuff to be philosophized about.  The various philosophies are but so many attempts at expressing what this stuff really is.  And if we repair to our libraries what disagreement do we discover!  Where is a certainly true answer found?  Apart from abstract propositions of comparison (such as two and two are the same as four), propositions which tell us nothing by themselves about concrete reality, we find no proposition ever regarded by any one as evidently certain that has not either been called a falsehood, or at least had its truth sincerely questioned by some one else.  The transcending of the axioms of geometry, not in play but in earnest, by certain of our contemporaries (as Z�llner and Charles H. Hinton), and the rejection of the whole Aristotelian logic by the Hegelians, are striking instances in point.

No concrete test of what is really true has ever been agreed upon. Some make the criterion external to the moment of perception, putting it either in revelation, the \emph{consensus gentium}, the instincts of the heart, or the systematized experience of the race.  Others make the perceptive moment its own test,--- Descartes, for instance, with his clear and distinct ideas guaranteed by the veracity of God; Reid with his `common-sense;' and Kant with his forms of synthetic judgment \emph{a priori}.  The inconceivability of the opposite; the capacity to be verified by sense; the possession of complete organic unity or self-relation, realized when a thing is its own other,--- are standards which, in turn, have been used.  The much \marginpar{16} lauded objective evidence is never triumphantly there, it is a mere aspiration or \emph{Grenzbegriff}, marking the infinitely remote ideal of our thinking life.  To claim that certain truths now possess it, is simply to say that when you think them true and they \emph{are} true, then their evidence is objective, otherwise it is not.  But practically one's conviction that the evidence one goes by is of the real objective brand, is only one more subjective opinion added to the lot.  For what a contradictory array of opinions have objective evidence and absolute certitude been claimed!  The world is rational through and through,--- its existence is an ultimate brute fact; there is a personal God,--- a personal God is inconceivable; there is an extra-mental physical world immediately known,--- the mind can only know its own ideas; a moral imperative exists,--- obligation is only the resultant of desires; a permanent spiritual principle is in every one,--- there are only shifting states of mind; there is an endless chain of causes,--- there is an absolute first cause; an eternal necessity,--- a freedom; a purpose,--- no purpose; a primal One,--- a primal Many; a universal continuity,--- an essential discontinuity in things; an infinity,--- no infinity.  There is this,--- there is that; there is indeed nothing which some one has not thought absolutely true, while his neighbor deemed it absolutely false; and not an absolutist among them seems ever to have considered that the trouble may all the time be essential, and that the intellect, even with truth directly in its grasp, may have no infallible signal for knowing whether it be truth or no.  When, indeed, one remembers that the most striking practical application to life of the doctrine of objective certitude has been \marginpar{17} the conscientious labors of the Holy Office of the Inquisition, one feels less tempted than ever to lend the doctrine a respectful ear.

But please observe, now, that when as empiricists we give up the doctrine of objective certitude, we do not thereby give up the quest or hope of truth itself.  We still pin our faith on its existence, and still believe that we gain an ever better position towards it by systematically continuing to roll up experiences and think.  Our great difference from the scholastic lies in the way we face.  The strength of his system lies in the principles, the origin, the \emph{terminus a quo} of his thought; for us the strength is in the outcome, the upshot, the \emph{terminus ad quem}.  Not where it comes from but what it leads to is to decide.  It matters not to an empiricist from what quarter an hypothesis may come to him: he may have acquired it by fair means or by foul; passion may have whispered or accident suggested it; but if the total drift of thinking continues to confirm it, that is what he means by its being true.

\section*{VII.}

One more point, small but important, and our preliminaries are done. There are two ways of looking at our duty in the matter of opinion,--- ways entirely different, and yet ways about whose difference the theory of knowledge seems hitherto to have shown very little concern.  \emph{We must know the truth}; and \emph{we must avoid error},--- these are our first and great commandments as would-be knowers; but they are not two ways of stating an identical commandment, they are two separable laws.  Although it may indeed happen that when we believe the truth \emph{A}, we escape \marginpar{18} as an incidental consequence from believing the falsehood \emph{B}, it hardly ever happens that by merely disbelieving \emph{B} we necessarily believe \emph{A}.  We may in escaping \emph{B} fall into believing other falsehoods, \emph{C} or \emph{D}, just as bad as \emph{B}; or we may escape \emph{B} by not believing anything at all, not even \emph{A}.

Believe truth!  Shun error!--- these, we see, are two materially different laws; and by choosing between them we may end by coloring differently our whole intellectual life.  We may regard the chase for truth as paramount, and the avoidance of error as secondary; or we may, on the other hand, treat the avoidance of error as more imperative, and let truth take its chance.  Clifford, in the instructive passage which I have quoted, exhorts us to the latter course.  Believe nothing, he tells us, keep your mind in suspense forever, rather than by closing it on insufficient evidence incur the awful risk of believing lies.  You, on the other hand, may think that the risk of being in error is a very small matter when compared with the blessings of real knowledge, and be ready to be duped many times in your investigation rather than postpone indefinitely the chance of guessing true.  I myself find it impossible to go with Clifford.  We must remember that these feelings of our duty about either truth or error are in any case only expressions of our passional life.  Biologically considered, our minds are as ready to grind out falsehood as veracity, and he who says, ``Better go without belief forever than believe a lie!'' merely shows his own preponderant private horror of becoming a dupe.  He may be critical of many of his desires and fears, but this fear he slavishly obeys.  He cannot imagine any one questioning its binding force.  For my own part, I \marginpar{19} have also a horror of being duped; but I can believe that worse things than being duped may happen to a man in this world: so Clifford's exhortation has to my ears a thoroughly fantastic sound.  It is like a general informing his soldiers that it is better to keep out of battle forever than to risk a single wound.  Not so are victories either over enemies or over nature gained.  Our errors are surely not such awfully solemn things.  In a world where we are so certain to incur them in spite of all our caution, a certain lightness of heart seems healthier than this excessive nervousness on their behalf.  At any rate, it seems the fittest thing for the empiricist philosopher.

\section*{VIII.}

And now, after all this introduction, let us go straight at our question.  I have said, and now repeat it, that not only as a matter of fact do we find our passional nature influencing us in our opinions, but that there are some options between opinions in which this influence must be regarded both as an inevitable and as a lawful determinant of our choice.

I fear here that some of you my hearers will begin to scent danger, and lend an inhospitable ear.  Two first steps of passion you have indeed had to admit as necessary,--- we must think so as to avoid dupery, and we must think so as to gain truth; but the surest path to those ideal consummations, you will probably consider, is from now onwards to take no further passional step.

Well, of course, I agree as far as the facts will allow.  Wherever the option between losing truth and gaining it is not momentous, we can throw the \marginpar{20} chance of \emph{gaining truth} away, and at any rate save ourselves from any chance of \emph{believing falsehood}, by not making up our minds at all till objective evidence has come.  In scientific questions, this is almost always the case; and even in human affairs in general, the need of acting is seldom so urgent that a false belief to act on is better than no belief at all.  Law courts, indeed, have to decide on the best evidence attainable for the moment, because a judge's duty is to make law as well as to ascertain it, and (as a learned judge once said to me) few cases are worth spending much time over: the great thing is to have them decided on \emph{any} acceptable principle, and got out of the way.  But in our dealings with objective nature we obviously are recorders, not makers, of the truth; and decisions for the mere sake of deciding promptly and getting on to the next business would be wholly out of place.  Throughout the breadth of physical nature facts are what they are quite independently of us, and seldom is there any such hurry about them that the risks of being duped by believing a premature theory need be faced.  The questions here are always trivial options, the hypotheses are hardly living (at any rate not living for us spectators), the choice between believing truth or falsehood is seldom forced.  The attitude of sceptical balance is therefore the absolutely wise one if we would escape mistakes.  What difference, indeed, does it make to most of us whether we have or have not a theory of the R�ntgen rays, whether we believe or not in mind-stuff, or have a conviction about the causality of conscious states?  It makes no difference.  Such options are not forced on us. On every account it is better not to make them, but still keep weighing reasons \emph{pro et contra} with an indifferent hand.

\marginpar{21}

I speak, of course, here of the purely judging mind.  For purposes of discovery such indifference is to be less highly recommended, and science would be far less advanced than she is if the passionate desires of individuals to get their own faiths confirmed had been kept out of the game.  See for example the sagacity which Spencer and Weismann now display.  On the other hand, if you want an absolute duffer in an investigation, you must, after all, take the man who has no interest whatever in its results: he is the warranted incapable, the positive fool.  The most useful investigator, because the most sensitive observer, is always he whose eager interest in one side of the question is balanced by an equally keen nervousness lest he become deceived.\footnote{Compare Wilfrid Ward's Essay, ``The Wish to Believe,'' in his \emph{Witnesses to the Unseen}, Macmillan \& Co., 1893.}  Science has organized this nervousness into a regular \emph{technique}, her so-called method of verification; and she has fallen so deeply in love with the method that one may even say she has ceased to care for truth by itself at all.  It is only truth as technically verified that interests her.  The truth of truths might come in merely affirmative form, and she would decline to touch it.  Such truth as that, she might repeat with Clifford, would be stolen in defiance of her duty to mankind.  Human passions, however, are stronger than technical rules.  ``Le coeur a ses raisons,'' as Pascal says, ``que la raison ne conna\^{\i}t pas;'' and however indifferent to all but the bare rules of the game the umpire, the abstract intellect, may be, the concrete players who furnish him the materials to judge of are usually, each one of them, in love with some pet `live hypothesis' of his own. Let us agree, however, that wherever there is no forced option, the \marginpar{22} dispassionately judicial intellect with no pet hypothesis, saving us, as it does, from dupery at any rate, ought to be our ideal.

The question next arises: Are there not somewhere forced options in our speculative questions, and can we (as men who may be interested at least as much in positively gaining truth as in merely escaping dupery) always wait with impunity till the coercive evidence shall have arrived?  It seems \emph{a priori} improbable that the truth should be so nicely adjusted to our needs and powers as that.  In the great boarding-house of nature, the cakes and the butter and the syrup seldom come out so even and leave the plates so clean.  Indeed, we should view them with scientific suspicion if they did.

\section*{IX.}

\emph{Moral questions} immediately present themselves as questions whose solution cannot wait for sensible proof.  A moral question is a question not of what sensibly exists, but of what is good, or would be good if it did exist.  Science can tell us what exists; but to compare the \emph{worths}, both of what exists and of what does not exist, we must consult not science, but what Pascal calls our heart.  Science herself consults her heart when she lays it down that the infinite ascertainment of fact and correction of false belief are the supreme goods for man.  Challenge the statement, and science can only repeat it oracularly, or else prove it by showing that such ascertainment and correction bring man all sorts of other goods which man's heart in turn declares.  The question of having moral beliefs at all or not having them is decided by \marginpar{23} our will.  Are our moral preferences true or false, or are they only odd biological phenomena, making things good or bad for \emph{us}, but in themselves indifferent?  How can your pure intellect decide?  If your heart does not \emph{want} a world of moral reality, your head will assuredly never make you believe in one. Mephistophelian scepticism, indeed, will satisfy the head's play-instincts much better than any rigorous idealism can.  Some men (even at the student age) are so naturally cool-hearted that the moralistic hypothesis never has for them any pungent life, and in their supercilious presence the hot young moralist always feels strangely ill at ease.  The appearance of knowingness is on their side, of na�vet� and gullibility on his.  Yet, in the inarticulate heart of him, he clings to it that he is not a dupe, and that there is a realm in which (as Emerson says) all their wit and intellectual superiority is no better than the cunning of a fox.  Moral scepticism can no more be refuted or proved by logic than intellectual scepticism can.  When we stick to it that there \emph{is} truth (be it of either kind), we do so with our whole nature, and resolve to stand or fall by the results.  The sceptic with his whole nature adopts the doubting attitude; but which of us is the wiser, Omniscience only knows.

Turn now from these wide questions of good to a certain class of questions of fact, questions concerning personal relations, states of mind between one man and another.  \emph{Do you like me or not?}--- for example.  Whether you do or not depends, in countless instances, on whether I meet you half-way, am willing to assume that you must like me, and show you trust and expectation.  The previous faith on my part in your liking's existence is in such cases what makes \marginpar{24} your liking come.  But if I stand aloof, and refuse to budge an inch until I have objective evidence, until you shall have done something apt, as the absolutists say, \emph{ad extorquendum assensum meum}, ten to one your liking never comes.  How many women's hearts are vanquished by the mere sanguine insistence of some man that they \emph{must} love him! he will not consent to the hypothesis that they cannot.  The desire for a certain kind of truth here brings about that special truth's existence; and so it is in innumerable cases of other sorts.  Who gains promotions, boons, appointments, but the man in whose life they are seen to play the part of live hypotheses, who discounts them, sacrifices other things for their sake before they have come, and takes risks for them in advance?  His faith acts on the powers above him as a claim, and creates its own verification.

A social organism of any sort whatever, large or small, is what it is because each member proceeds to his own duty with a trust that the other members will simultaneously do theirs.  Wherever a desired result is achieved by the co-operation of many independent persons, its existence as a fact is a pure consequence of the precursive faith in one another of those immediately concerned.  A government, an army, a commercial system, a ship, a college, an athletic team, all exist on this condition, without which not only is nothing achieved, but nothing is even attempted.  A whole train of passengers (individually brave enough) will be looted by a few highwaymen, simply because the latter can count on one another, while each passenger fears that if he makes a movement of resistance, he will be shot before any one else backs him up.  If we believed that the whole car-full would rise \marginpar{25} at once with us, we should each severally rise, and train-robbing would never even be attempted.  There are, then, cases where a fact cannot come at all unless a preliminary faith exists in its coming.  \emph{And where faith in a fact can help create the fact}, that would be an insane logic which should say that faith running ahead of scientific evidence is the `lowest kind of immorality' into which a thinking being can fall.  Yet such is the logic by which our scientific absolutists pretend to regulate our lives!

\section*{X.}

In truths dependent on our personal action, then, faith based on desire is certainly a lawful and possibly an indispensable thing.

But now, it will be said, these are all childish human cases, and have nothing to do with great cosmical matters, like the question of religious faith.  Let us then pass on to that.  Religions differ so much in their accidents that in discussing the religious question we must make it very generic and broad.  What then do we now mean by the religious hypothesis?  Science says things are; morality says some things are better than other things; and religion says essentially two things.

First, she says that the best things are the more eternal things, the overlapping things, the things in the universe that throw the last stone, so to speak, and say the final word.  ``Perfection is eternal,''--- this phrase of Charles Secr�tan seems a good way of putting this first affirmation of religion, an affirmation which obviously cannot yet be verified scientifically at all.

\marginpar{26}

The second affirmation of religion is that we are better off even now if we believe her first affirmation to be true.

Now, let us consider what the logical elements of this situation are \emph{in case the religious hypothesis in both its branches be really true}. (Of course, we must admit that possibility at the outset.  If we are to discuss the question at all, it must involve a living option.  If for any of you religion be a hypothesis that cannot, by any living possibility be true, then you need go no farther.  I speak to the `saving remnant' alone.)  So proceeding, we see, first, that religion offers itself as a \emph{momentous} option.  We are supposed to gain, even now, by our belief, and to lose by our non-belief, a certain vital good.  Secondly, religion is a \emph{forced} option, so far as that good goes.  We cannot escape the issue by remaining sceptical and waiting for more light, because, although we do avoid error in that way \emph{if religion be untrue}, we lose the good, \emph{if it be true}, just as certainly as if we positively chose to disbelieve.  It is as if a man should hesitate indefinitely to ask a certain woman to marry him because he was not perfectly sure that she would prove an angel after he brought her home.  Would he not cut himself off from that particular angel-possibility as decisively as if he went and married some one else?  Scepticism, then, is not avoidance of option; it is option of a certain particular kind of risk.  \emph{Better risk loss of truth than chance of error},--- that is your faith-vetoer's exact position.  He is actively playing his stake as much as the believer is; he is backing the field against the religious hypothesis, just as the believer is backing the religious hypothesis against the field.  To preach scepticism to us as a duty until \marginpar{27} `sufficient evidence' for religion be found, is tantamount therefore to telling us, when in presence of the religious hypothesis, that to yield to our fear of its being error is wiser and better than to yield to our hope that it may be true.  It is not intellect against all passions, then; it is only intellect with one passion laying down its law.  And by what, forsooth, is the supreme wisdom of this passion warranted?  Dupery for dupery, what proof is there that dupery through hope is so much worse than dupery through fear?  I, for one, can see no proof; and I simply refuse obedience to the scientist's command to imitate his kind of option, in a case where my own stake is important enough to give me the right to choose my own form of risk.  If religion be true and the evidence for it be still insufficient, I do not wish, by putting your extinguisher upon my nature (which feels to me as if it had after all some business in this matter), to forfeit my sole chance in life of getting upon the winning side,--- that chance depending, of course, on my willingness to run the risk of acting as if my passional need of taking the world religiously might be prophetic and right.

All this is on the supposition that it really may be prophetic and right, and that, even to us who are discussing the matter, religion is a live hypothesis which may be true.  Now, to most of us religion comes in a still further way that makes a veto on our active faith even more illogical.  The more perfect and more eternal aspect of the universe is represented in our religions as having personal form.  The universe is no longer a mere \emph{It} to us, but a \emph{Thou}, if we are religious; and any relation that may be possible from person to person might be possible \marginpar{28} here.  For instance, although in one sense we are passive portions of the universe, in another we show a curious autonomy, as if we were small active centres on our own account.  We feel, too, as if the appeal of religion to us were made to our own active good-will, as if evidence might be forever withheld from us unless we met the hypothesis half-way.  To take a trivial illustration: just as a man who in a company of gentlemen made no advances, asked a warrant for every concession, and believed no one's word without proof, would cut himself off by such churlishness from all the social rewards that a more trusting spirit would earn,--- so here, one who should shut himself up in snarling logicality and try to make the gods extort his recognition willy-nilly, or not get it at all, might cut himself off forever from his only opportunity of making the gods' acquaintance.  This feeling, forced on us we know not whence, that by obstinately believing that there are gods (although not to do so would be so easy both for our logic and our life) we are doing the universe the deepest service we can, seems part of the living essence of the religious hypothesis.  If the hypothesis \emph{were} true in all its parts, including this one, then pure intellectualism, with its veto on our making willing advances, would be an absurdity; and some participation of our sympathetic nature would be logically required.  I, therefore, for one cannot see my way to accepting the agnostic rules for truth-seeking, or wilfully agree to keep my willing nature out of the game.  I cannot do so for this plain reason, that \emph{a rule of thinking which would absolutely prevent me from acknowledging certain kinds of truth if those kinds of truth were really there, would be an irrational rule}.  That for me \marginpar{29} is the long and short of the formal logic of the situation, no matter what the kinds of truth might materially be.

 I confess I do not see how this logic can be escaped.  But sad experience makes me fear that some of you may still shrink from radically saying with me, \emph{in abstracto}, that we have the right to believe at our own risk any hypothesis that is live enough to tempt our will.  I suspect, however, that if this is so, it is because you have got away from the abstract logical point of view altogether, and are thinking (perhaps without realizing it) of some particular religious hypothesis which for you is dead.  The freedom to `believe what we will' you apply to the case of some patent superstition; and the faith you think of is the faith defined by the schoolboy when he said, ``Faith is when you believe something that you know ain't true.''  I can only repeat that this is misapprehension.  \emph{In concreto}, the freedom to believe can only cover living options which the intellect of the individual cannot by itself resolve; and living options never seem absurdities to him who has them to consider.  When I look at the religious question as it really puts itself to concrete men, and when I think of all the possibilities which both practically and theoretically it involves, then this command that we shall put a stopper on our heart, instincts, and courage, and wait--- acting of course meanwhile more or less as if religion were \emph{not} true\footnote{Since belief is measured by action, he who forbids us to believe religion to be true, necessarily also forbids us to act as we should if we did believe it to be true.  The whole defence of religious faith hinges upon action.  If the action required or inspired by the religious hypothesis is in no way different from that dictated by the naturalistic hypothesis, then religious faith is a pure superfluity, better pruned away, and controversy about its legitimacy is a piece of idle trifling, unworthy of serious minds.  I myself believe, of course, that the religious hypothesis gives to the world an expression which specifically determines our reactions, and makes them in a large part unlike what they might be on a purely naturalistic scheme of belief.}--- till \marginpar{30} doomsday, or till such time as our intellect and senses working together may have raked in evidence enough,--- this command, I say, seems to me the queerest idol ever manufactured in the philosophic cave.  Were we scholastic absolutists, there might be more excuse.  If we had an infallible intellect with its objective certitudes, we might feel ourselves disloyal to such a perfect organ of knowledge in not trusting to it exclusively, in not waiting for its releasing word.  But if we are empiricists, if we believe that no bell in us tolls to let us know for certain when truth is in our grasp, then it seems a piece of idle fantasticality to preach so solemnly our duty of waiting for the bell. Indeed we \emph{may} wait if we will,--- I hope you do not think that I am denying that,--- but if we do so, we do so at our peril as much as if we believed.  In either case we \emph{act}, taking our life in our hands.  No one of us ought to issue vetoes to the other, nor should we bandy words of abuse.  We ought, on the contrary, delicately and profoundly to respect one another's mental freedom: then only shall we bring about the intellectual republic; then only shall we have that spirit of inner tolerance without which all our outer tolerance is soulless, and which is empiricism's glory; then only shall we live and let live, in speculative as well as in practical things.

I began by a reference to Fitz James Stephen; let me end by a quotation from him.  ``What do you think \marginpar{31} of yourself?  What do you think of the world?\ldots  These are questions with which all must deal as it seems good to them.  They are riddles of the Sphinx, and in some way or other we must deal with them\ldots.  In all important transactions of life we have to take a leap in the dark\ldots.  If we decide to leave the riddles unanswered, that is a choice; if we waver in our answer, that, too, is a choice: but whatever choice we make, we make it at our peril.  If a man chooses to turn his back altogether on God and the future, no one can prevent him; no one can show beyond reasonable doubt that he is mistaken.  If a man thinks otherwise and acts as he thinks, I do not see that any one can prove that \emph{he} is mistaken.  Each must act as he thinks best; and if he is wrong, so much the worse for him.  We stand on a mountain pass in the midst of whirling snow and blinding mist, through which we get glimpses now and then of paths which may be deceptive.  If we stand still we shall be frozen to death.  If we take the wrong road we shall be dashed to pieces.  We do not certainly know whether there is any right one.  What must we do?  `Be strong and of a good courage.'  Act for the best, hope for the best, and take what comes\ldots.  If death ends all, we cannot meet death better.''\footnote{Liberty, Equality, Fraternity, p. 353, 2d edition.  London, 1874.}

 



\end{document}