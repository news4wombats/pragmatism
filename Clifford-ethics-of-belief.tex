\documentclass[12pt]{article}
\usepackage{hyperref,geometry,setspace,fancyhdr}
\usepackage{accessibility}
\usepackage[symbol*,perpage]{footmisc}
% changes font to TeX Gyre Schola (Century Schoolbook)
\usepackage{tgschola}\usepackage[T1]{fontenc}

\fancyhf{} % sets both header and footer to nothing
\renewcommand{\headrulewidth}{0pt}
\fancyfoot[R]{\textsc{The Ethics of Belief} \arabic{page}}

\begin{document}
\pagestyle{fancy}
\setcounter{footnote}{0}
\hypersetup{pdfinfo={Title={The Ethics of Belief}, Author={William Kingdon Clifford}}, pdfborder = {0 0 0 0}}


\paragraph{A note about the text:} This text was prepared by P.D. Magnus from a scan of Clifford's \emph{Lectures and Essays} (second edition) edited by Leslie Stephen and Frederick Pollock and published by Macmillan and Co. (London) in 1886.

The marginal numbers indicate page numbers in the original.

\begin{center}

\section*{\textsc{The Ethics of Belief}\footnote{\emph{Contemporary Review}, January 1877.}}

\textsc{William Kingdon Clifford}

\end{center}

\marginpar{339}
\subsection*{I.---\textsc{The Duty of Inquiry}}

A shipowner was about to send to sea an emigrant-ship. He knew that she was old, and not over-well built at the first; that she had seen many seas and climes, and often had needed repairs. Doubts had been suggested to him that possibly she was not seaworthy. These doubts preyed upon his mind and made him unhappy; he thought that perhaps he ought to have her thoroughly overhauled and refitted, even though this should put him to great expense. Before the ship sailed, however, he succeeded in overcoming these melancholy reflections. He said to himself that she had gone safely through so many voyages and weathered so many storms that it was idle to suppose she would not come safely home from this trip also. He would put his trust in Providence, which could hardly fail to protect all these unhappy families that were leaving their fatherland to seek for better times elsewhere. He would dismiss from his mind all ungenerous suspicions about the honesty of builders and contractors. In such ways he acquired a sincere and comfortable conviction that his vessel was thoroughly safe and seaworthy; he watched her departure with a light heart, and benevolent wishes for the success of tho exiles in their strange new home that was to be; and he got his insurance-money when she went down in mid-ocean and told no tales.

What shall we say of him? Surely this, that he was verily guilty of the death of those men. It is admitted that he did sincerely believe in the soundness of his ship; but the\marginpar{340} sincerity of his conviction can in no wise help him, because \emph{he had no right to believe on such evidence as was before him}. He had acquired his belief not by honestly earning it in patient investigation, but by stifling his doubts. And although in the end he may have felt so sure about it that he could not think otherwise, yet inasmuch as he had knowingly and willingly worked himself into that frame of mind, he must be held responsible for it.

Let us alter the case a little, and suppose that the ship was not unsound after all; that she made her voyage safely, and many others after it. Will that diminish the guilt of her owner? Not one jot. When an action is once done, it is right or wrong for ever; no accidental failure of its good or evil fruits can possibly alter that. The man would not have been innocent, he would only have been not found out. The question of right or wrong has to do with the origin of his belief, not the matter of it; not what it was, but how he got it; not whether it turned out to be true or false, but whether he had a right to believe on such evidence as was before him.

There was once an island in which some of the inhabitants professed a religion teaching neither the doctrine of original sin nor that of eternal punishment. A suspicion got abroad that the professors of this religion had made use of unfair means to get their doctrines taught to children. They were accused of wresting the laws of their country in such a way as to remove children from the care of their natural and legal guardians; and even of stealing them away and keeping them concealed from their friends and relations. A certain number of men formed themselves into a society for the purpose of agitating the public about this matter. They published grave accusations against individual citizens of the highest position and character, and did all in their power to injure these citizens in the exercise of their professions. So great was the noise they made, that a Commission was appointed to investigate the facts; but after the Commission had carefully inquired into all the evidence that could be got, it appeared that the accused were innocent. Not only had they been accused on insufficient evidence, but the evidence of their innocence was such as the agitators might easily have obtained,\marginpar{341} if they had attempted a fair inquiry. After these disclosures the inhabitants of that country looked upon the members of the agitating society, not only as persons whose judgment was to be distrusted, but also as no longer to be counted honourable men. For although they had sincerely and conscientiously believed in the charges they had made, \emph{yet they had no right to believe on such evidence as was before them}. Their sincere convictions, instead of being honestly earned by patient inquiring, were stolen by listening to the voice of prejudice and passion.

Let us vary this case also, and suppose, other things remaining as before, that a still more accurate investigation proved the accused to have been really guilty. Would this make any difference in the guilt of the accusers? Clearly not; the question is not whether their belief was true or false, but whether they entertained it on wrong grounds. They would no doubt say, ``Now you see that we were right after all; next time perhaps you will believe us.'' And they might be believed, but they would not thereby become honourable men. They would not be innocent, they would only be not found out. Every one of them, if he chose to examine himself \emph{in foro conscienti\oe}, would know that he had acquired and nourished a belief, when he had no right to believe on such evidence as was before him; and therein he would know that he had done a wrong thing.

It may be said, however, that in both of these supposed cases it is not the belief which is judged to be wrong, but the action following upon it. The shipowner might say, ``I am perfectly certain that my ship is sound, but still I feel it my duty to have her examined, before trusting the lives of so many people to her.'' And it might be said to the agitator, ``However convinced you were of the justice of your cause and the truth of your convictions, you ought not to have made a public attack upon any man's character until you had examined the evidence on both sides with the utmost patience and care.''

In the first place, let us admit that, so far as it goes, this view of the case is right and necessary; right, because even when a man's belief is so fixed that he cannot think otherwise, he still has a choice in regard to the action suggested by it,\marginpar{342} and so cannot escape the duty of investigating on the ground of the strength of his convictions; and necessary, because those who are not yet capable of controlling their feelings and thoughts must have a plain rule dealing with overt acts.

But this being premised as necessary, it becomes clear that it is not sufficient, and that our previous judgment is required to supplement it. For it is not possible so to sever the belief from the action it suggests as to condemn the one without condemning the other. No man holding a strong belief on one side of a question, or even wishing to hold a belief on one side, can investigate it with such fairness and completeness as if he were really in doubt and unbiassed; so that the existence of a belief not founded on fair inquiry unfits a man for the performance of this necessary duty.

Nor is that truly a belief at all which has not some influence upon the actions of him who holds it. He who truly believes that which prompts him to an action has looked upon the action to lust after it, he has committed it already in his heart. If a belief is not realised immediately in open deeds, it is stored up for the guidance of the future. It goes to make a part of that aggregate of beliefs which is the link between sensation and action at every moment of all our lives, and which is so organised and compacted together that no part of it can be isolated from the rest, but every new addition modifies the structure of the whole. No real belief, however trifling and fragmentary it may seem, is ever tmly insignificant; it prepares us to receive more of its like, confirms those which resembled it before, and weakens others; and so gradually it lays a stealthy train in our inmost thoughts, which may some day explode into overt action, and leave its stamp upon our character for ever.

And no one man's belief is in any case a private matter which concerns himself alone. Our lives are guided by that general conception of the course of things which has been created by society for social purposes. Our words, our phrases, our forms and processes and modes of thought, are common property, fashioned and perfected from age to age; an heirloom which every succeeding generation inherits as a precious deposit and a sacred trust to be handed on to the next one, not unchanged but enlarged and purified, with some clear marks\marginpar{343} of its proper handiwork. Into this, for good or ill, is woven every belief of every man who has speech of his fellows. An awful privilege, and an awful responsibility, that we should help to create the world in which posterity will live.

In the two supposed cases which have been considered, it has been judged wrong to believe on insufficient evidence, or to nourish belief by suppressing doubts and avoiding investigation. The reason of this judgment is not far to seek: it is that in both these cases the belief held by one man was of great importance to other men. But forasmuch as no belief held by one man, however seemingly trivial the belief, and however obscure the believer, is ever actually insignificant or without its effect on the fate of mankind, we have no choice but to extend our judgment to all cases of belief whatever. Belief, that sacred faculty which prompts the decisions of our will, and knits into harmonious working all the compacted energies of our being, is ours not for ourselves, but for humanity. It is rightly used on truths which have been established by long experience and waiting to~ and which have stood in the fierce light of free and fearless questioning. Then it helps to bind men together, and to strengthen and direct their common action. It is desecrated when given to unproved and unquestioned statements, for the solace and private pleasure of the believer; to add a tinsel splendour to the plain straight road of our life and display a bright mirage beyond it; or even to drown the common sorrows of our kind by a self-deception which allows them not only to cast down, but also to degrade us. Whoso would deserve well of his fellows in this matter will guard the purity of his belief with a very fanaticism of jealous care, lest at any time it should rest on an unworthy object, and catch a stain which can never be wiped away.

It is not only the leader of men, statesman, philosopher, or poet, that owes this bounden duty to mankind. Every rustic who delivers in the village alehouse his slow, infrequent sentences, may help to kill or keep alive the fatal superstitions which clog his race. Every hard-worked wife of an artisan may transmit to her children beliefs which shall knit society together, or rend it in pieces. No simplicity of mind, no\marginpar{344} obscurity of station, can escape the universal duty of questioning all that we believe.

It is true that this duty is a hard one, and the doubt which comes out of it is often a very bitter thing. It leaves us bare and powerless where we thought that we were safe and strong. To know all about anything is to know how to deal with it under all circumstances. We feel much happier and more secure when we think we know precisely what to do, no matter what happens, than when we have lost our way and do not know where to turn. And if we have supposed ourselves to know all about anything, and to be capable of doing what is fit in regard to it, we naturally do not like to find that we are really ignorant and powerless, that we have to begin again at the beginning, and try to learn what the thing is and how it is to be dealt with--- if indeed anything can be learnt about it. It is the sense of power attached to a sense of knowledge that makes men desirous of believing, and afraid of doubting.

This sense of power is the highest and best of pleasures when the belief on which it is founded is a true belief, and has been fairly earned by investigation. For then we may justly feel that it is common property, and holds good for others as well as for ourselves. Then we may be glad, not that I have learned secrets by which I am safer and stronger, hut that we men have got mastery over more of the world; and we shall be strong, not for ourselves, but in the name of Man and in his strength. But if the belief has been accepted on insufficient evidence, the pleasure is a stolen one. Not only does it deceive ourselves by giving us a sense of power which we do not really possess, but it is sinful, because it is stolen in defiance of our duty to mankind. That duty is to guard ourselves from such beliefs as from a pestilence, which may shortly master our own body and then spread to the rest of the town. What would be thought of one who, for the sake of a sweet fruit, should deliberately rnn the risk of bringing a plague upon his family and his neighbours ~

And, as in other such cases, it is not the risk only which has to be considered; for a bad action is always bad at the time when it is done, no matter what happens afterwards. Every time we let ourselves believe for unworthy reasons, we weaken\marginpar{345} our powers of self-control, of doubting, of judicially and fairly weighing evidence. We all suffer severely enough from the maintenance and support of false beliefs and the fatally wrong actions which they lead to, and the evil born when one such belief is entertained is great and wide. But a greater and wider evil arises when the credulous character is maintained and supported, when a habit of believing for unworthy reasons is fostered and made permanent. If I steal money from any person, there may be no harm done by the mere transfer of possession; he may not feel the loss, or it may prevent him from using the money badly. But I cannot help doing this great wrong towards Man, that I make myself dishonest. What hurts society is not that it should lose its property, but that it should become a den of thieves; for then it must cease to be society. This is why we ought not to do evil that good may come; for at any rate this great evil has come, that we have done evil and are made wicked thereby. In like manner, if I let myself believe anything on insufficient evidence, there may be no great harm done by the mere belief; it may be true after all, or I may never have occasion to exhibit it in outward acts. But I cannot help doing this great wrong towards Man, that I make myself credulous. The danger to society is not merely that it should believe wrong things, though that is great enough; but that it should become credulous, and lose the habit of testing things and inquiring into them; for then it must sink back into savagery.

The harm which is done by credulity in a man is not confined to the fostering of a credulous character in others, and consequent support of false beliefs. Habitual want of care about what I believe leads to habitual want of care in others about the truth of what is told to me. Men speak the truth to one another when each reveres the truth in his own mind and in the other's mind; but how shall my friend revere the truth in my mind when I myself am careless about it, when I believe things because I want to believe them, and because they are comforting and pleasant? Will he not learn to cry, ``Peace,'' to me, when there is no peace? By such a course I shall surround myself with a thick atmosphere of falsehood and fraud, and in that I must live. It may matter little to\marginpar{346} me, in my cloud-castle of sweet illusions and darling lies; but it matters much to Man that I have made my neighbours ready to deceive. The credulous man is father to the liar and the cheat; he lives in the bosom of this his family, and it is no marvel if he should become even as they are. So closely are our duties knit together, that whoso shall keep the whole law, and yet offend in one point, he is guilty of all.

To sum up: it is wrong always, everywhere, and for any one, to believe anything upon insufficient evidence.

If a man, holding a belief which he was taught in childhood or persuaded of afterwards, keeps down and pushes away any doubts which arise about it in his mind, purposely avoids the reading of books and the company of men that call in question or discuss it, and regards as impious those questions which cannot easily be asked without disturbing it--- the life of that man is one long sin against mankind. If this judgment seems harsh when applied to those simple souls who have never known better, who have been brought up from the cradle with a horror of doubt, and taught that their eternal welfare depends on \emph{what} they believe, then it leads to the very serious question, \emph{Who hath made Israel to sin}? It may be permitted me to fortify this judgment with the sentence of Milton\footnote{\emph{Areopagitica}.}---

``A man may be a heretic in the truth; and if he believe things only because his pastor says so, or the assembly so determine, without knowing other reason, though his belief be true, yet the very truth he holds becomes his heresy.''

And with this famous aphorism of Coleridge\footnote{\emph{Aids to Reflection}.}---

``He who begins by loving Christianity better than Truth, will proceed by loving his own sect or Church better than Christianity, and end in loving himself better than all.''

Inquiry into the evidence of a doctrine is not to be made once for all, and then taken as finally settled. It is never lawful to stifle a doubt; for either it can be honestly answered by means of the inquiry already made, or else it proves that the inquiry was not complete.

``But,'' says one, ``I am a busy man; I have no time for the long course of study which would be necessary to make me in any degree a competent judge of certain questions, or\marginpar{347} even able to understand the nature of the arguments.'' Then he should have no time to believe.

\subsection*{II.---\textsc{The Weight of Authority}}

Are we then to become universal sceptics, doubting everything, afraid always to put one foot before the other until we have personally tested the firmness of the road? Are we to deprive ourselves of the help and guidance of that vast body of knowledge which is daily growing upon the world, because neither we nor any other one person can possibly test a hundredth part of it by immediate experiment or observation, and because it would not be completely proved if we did? Shall we steal and tell lies because we have had no personal experience wide enough to justify the belief that it is wrong to do so?

There is no practical danger that such consequences will ever follow from scrupulous care and self-control in the matter of belief. Those men who have most nearly done their duty in this respect have found that certain great principles, and these most fitted for the guidance of life, have stood out more and more clearly in proportion to the care and honesty with which they were tested, and have acquired in this way a practical certainty. The beliefs about right and wrong which guide our actions in dealing with men in society, and the beliefs about physical nature which guide our actions in dealing with animate and inanimate bodies, these never suffer from investigation; they can take care of themselves, without being propped up by ``acts of faith,'' the clamour of paid advocates, or the suppression of contrary evidence. Moreover there are many cases in which it is our duty to act upon probabilities, although the evidence is not such as to justify present belief; because it is precisely by such action, and by observation of its fruits, that evidence is got which may justify future belief. So that we have no reason to fear lest a habit of conscientious inquiry should paralyse the actions of our daily life.

But because it is not enough to say, ``It is wrong to believe on unworthy evidence,'' without saying also what evidence is\marginpar{348} worthy, we shall now go on to inquire under what circumstances it is lawful to believe on the testimony of others; and then, further, we shall inquire more genemlly when and why we may believe that which goes beyond our own experience, or even beyond the experience of mankind.

In what cases, then, let us ask in the first place, is the testimony of a man unworthy of belief? He may say that which is untrue either knowingly or unknowingly. In the first case he is lying, and his moral character is to blame; in the second case he is ignorant or mistaken, and it is only his knowledge or his judgment which is in fault. In order that we may have the right to accept his testimony as ground for believing what he says, we must have reasonable grounds for trusting his \emph{veracity}, that he is really trying to speak the truth so far as he knows it; his \emph{knowledge}, that he has had opportunities of knowing the truth about this matter; and his \emph{judgment}, that he has made proper use of those opportunities in coming to the conclusion which he affirms.

However plain and obvious these reasons may be, so that no man of ordinary intelligence, reflecting upon the matter, could fail to arrive at them, it is nevertheless true that a. great many persons do habitually disregard them in weighing testimony. Of the two questions, equally important to the trustworthiness of a witness, ``Is he dishonest ?'' and ``May he be mistaken ?'' the majority of mankind are perfectly satisfied if \emph{one} can, with some show of probability, be answered in the negative. The excellent moml chamcter of a man is alleged as ground for accepting his statements about things which he cannot possibly have known. A Mohammedan, for example, will tell us that the character of his Prophet was so noble and majestic that it commands the reverence even of those who do not believe in his mission. So admirable was his moral teaching, so wisely put together the great social machine which he created, that his precepts have not only been accepted by a great portion of mankind, but have actually been obeyed. His institutions have on the one hand rescued the negro from savagery, and on the other hand have taught civilisation to the advancing West; and although the races which held the highest forms of his faith, and most fully embodied his mind and thought, have all been conquered and swept away by\marginpar{349} barbaric tribes, yet the history of their marvellous attainments remains as an imperishable glory to Islam. Are we to doubt the word of a man so great and so good? Can we suppose that this magnificent genius, this splendid moral hero, has lied to us about the most solemn and sacred matters? The testimony of Mohammed is clear, that there is but one God, and that he, Mohammed, is his Prophet; that if we believe in him we shall enjoy everlasting felicity, but that if we do not we shall be damned. This testimony rests on the most awful of foundations, the revelation of heaven itself; for was he not visited by the angel Gabriel, as he fasted and prayed in his desert cave; and allowed to enter into the blessed fields of Paradise? Surely God is God and Mohammed is the Prophet of God.

What should we answer to this Mussulman? First, no doubt, we should be tempted to take exception against his view of the character of the Prophet and the uniformly beneficial influence of Islam: before we could go with him altogether in these matters it might seem that we should have to forget many terrible things of which we have heard or read. But if we chose to grant him all these assumptions, for the sake of argument, and because it is difficult both for the faithful and for infidels to discuss them fairly and without passion, still we should have something to say which takes away the ground of his belief, and therefore shows that it is wrong to entertain it. Namely this: the character of Mohammed is excellent evidence that ho was honest and spoke the truth so far as he knew it; but it is no evidence at all that he knew what the truth was. What means could he have of knowing that the form which appeared to him to be the angel Gabriel was not a hallucination, and that his apparent visit to Paradise was not a dream? Grant that he himself was fully persuaded and honestly believed that he had the guidance of heaven, and was the vehicle of a supernatural revelation, how could he know that this strong conviction was not a mistake? Let us put ourselves in his place; we shall find that the more completely we endeavour to realise what passed through his mind, the more clearly we shall perceive that the Prophet could have had no adequate ground for the belief in his own inspiration. It is most probable that he\marginpar{350} himself never doubted of the matter, or thought of asking the question; but we are in the position of those to whom the question has been asked, and who are bound to answer it. It is known to medical observers that solitude and want of food are powerful means of producing delusion and of fostering a tendency to mental disease. Let us suppose, then, that I, like Mohammed, go into desert places to fast and pray; what things can happen to me which will give me the right to believe that I am divinely inspired? Suppose that I get information, apparently from a celestial visitor, which upon being tested is found to be correct. I cannot be sure, in the first place, that the celestial visitor is not a figment of my own mind, and that the information did not come to me, unknown at the time to my consciousness, through some subtle channel of sense. But if my visitor were a real visitor, and for a long time gave me information which was found to be trustworthy, this would indeed be good ground for trusting him in the future as to such matters as fall within human powers of verification; but it would not be ground for trusting his testimony as to any other matters. For although his tested character would justify me in believing that he spoke the truth so far as he knew, yet the same question would present itself--- what ground is there for supposing that he knows?

Even if my supposed visitor had given me such information, subsequently verified by me, as proved him to have means of knowledge about verifiable matters far exceeding my own; this would not justify me in believing what he said about matters that are not at present capable of verification by man. It would be ground for interesting conjecture, and for the hope that, as the fruit of our patient inquiry, we might by and by attain to such a means of verification as should rightly turn conjecture into belief. For belief belongs to man, and to the guidance of human affairs: no belief is real unless it guide our actions, and those very actions supply a test of its truth.

But, it may be replied, the acceptance of Islam as a system is just that action which is prompted by belief in tho mission of the Prophet, and which will serve for a test of its truth. Is it possible to believe that a system which has succeeded so well\marginpar{351} is really founded upon a delusion? Not only have individual saints found joy and peace in believing, and verified those spiritual experiences which are promised to the faithful, but nations also have been raised from savagery or barbarism to a higher social state. Surely we are at liberty to say that the belief has been acted upon, and that it has been verified.

It requires, however, but little consideration to show that what has really been verified is not at all the supernal character of the Prophet's mission, or the trustworthiness of his authority in matters which we ourselves cannot test, but only his practical wisdom in certain very mundane things. The fact that believers have found joy and peace in believing gives us the right to say that the doctrine is a comfortable doctrine, and pleasant to the soul; but it does not give us the right to say that it is true. And the question which our conscience is always asking about that which we are tempted to believe is not, ``Is it comfortable and pleasant?'' but, ``Is it true?'' That the Prophet preached certain doctrines, and predicted that spiritual comfort would be found in them, proves only his sympathy with human nature and his knowledge of it; but it does not prove his superhuman knowledge of theology.

And if we admit for the sake of argument (for it seems that we cannot do more) that the progress made by Moslem nations in certain cases was really due to the system formed and sent forth into the world by Mohammed, we are not at liberty to conclude from this that he was inspired to declare the truth about things which we cannot verify. We are only at liberty to infer the excellence of his moral precepts, or of the means which he devised for so working upon men as so get them obeyed, or of the social and political machinery which he set up. And it would require a great amount of careful examination into the history of those nations to determine which of these things had the greater share in the result. So that here again it is the Prophet's knowledge of human nature, and his sympathy with it, that are verified; not his divine inspiration or his knowledge of theology.

If there were only one Prophet, indeed, it might well seem a difficult and even an ungracious task to decide upon what\marginpar{352} points we would trust him, and on what we would doubt his authority; seeing what help and furtherance all men have gained in all ages from those who saw more clearly, who felt more strongly, and who sought the truth with more single heart than their weaker brethren. But there is not only one Prophet; and while the consent of many upon that which, as men, they had real means of knowing and did know, has endured to the end, and been honourably built into the great fabric of human knowledge, the diverse witness of some about that which they did not and could not know remains as a warning to us that to exaggerate the prophetic authority is to misuse it, and to dishonour those who have sought only to help and further us after their power. It is hardly in human nature that a man should quite accurately gauge the limits of his own insight; but it is the duty of those who profit by his work to consider carefully where he may have been carried beyond it. If we must needs embalm his possible errors along with his solid achievements, and use his authority as an excuse for believing what he cannot have known, we make of his goodness an occasion to sin.

To consider only one other such witness: the followers of the Buddha have at least as much right to appeal to individual and social experience in support of the authority of the Eastern saviour. The special mark of his religion, it is said, that in which it has never been surpassed, is the comfort and consolation which it gives to the sick and sorrowful, the tender sympathy with which it soothes and assuages all the natural griefs of men. And surely no triumph of social morality can be greater or nobler than that which has kept nearly half the human race from persecuting in the name of religion. If we are to trust the accounts of his early followers, he believed himself to have come upon earth with a divine and cosmic mission to set rolling the wheel of the law. Being a prince, he divested himself of his kingdom, and of his free will became acquainted with misery, that be might learn how to meet and subdue it. Could such a man speak falsely about solemn things? And as for his knowledge, was he not a man miraculous with powers more than man's? He was born of woman without the help of man; be rose into the air and was transfigured before his kinsmen; at last he went up bodily into\marginpar{353} heaven from the top of Adam's Peak. Is not his word to be believed in when he testifies of heavenly things?

If there were only he, and no other, with such claims! But there is Mohammed with his testimony; we cannot choose but listen to them both. The Prophet tells us that there is one God, and that we shall live for ever in joy or misery, according as we believe in the Prophet or not. The Buddha says that there is no God, and that we shall be annihilated by and by if we are good enough. Both cannot be infallibly inspired; one or other must have been the victim of a delusion, and thought he knew that which he really did not know. Who shall dare to say which? and how can we justify ourselves in believing that the other was not also deluded?

We are led, then, to these judgments following. The goodness and greatness of a man do not justify us in accepting a belief upon the warrant of his authority, unless there are reasonable grounds for supposing that he knew the truth of what he was saying. And there can be no grounds for supposing that a man knows that which we, without ceasing to be men, could not be supposed to verify.

If a chemist tells me, who am no chemist, that a certain substance can be made by putting together other substances in certain proportions and subjecting them to a known process, I am quite justified in believing this upon his authority, unless I know anything against his character or his judgment. For his professional training is one which tends to encourage veracity and the honest pursuit of truth, and to produce a dislike of hasty conclusions and slovenly investigation. And I have. reasonable ground for supposing that he knows the truth of what he is saying, for although I am no chemist, I can be made to understand so much of the methods and processes of the science as makes it conceivable to me that, without ceasing to be man, I might verify the statement. I may never actually verify it, or even see any experiment which goes towards verifying it; but still I have quite reason enough to justify me in believing that the verification is within the reach of human appliances and powers, and in particular that it has been actually performed by my informant. His result, the belief to which he has been led by his inquiries, is valid\marginpar{354} not only for himself but for others; it is watched and tested by those who are working in the same ground, and who know that no greater service can be rendered to science than the purification of accepted results from the errors which may have crept into them. It is in this way that the result becomes common property, a right object of belief, which is a social affair and matter of public business. Thus it is to be observed that his authority is valid because there are those who question it and verify it; that it is precisely this process of examining and purifying that keeps alive among investigators the love of that which shall stand all possible tests, the sense of public responsibility as of those whose work, if well done, shall remain as the enduring heritage of mankind.

But if my chemist tells me that an atom of oxygen has existed unaltered in weight and rate of vibration throughout all time I have no right to believe this on his authority, for it is a thing which he cannot know without ceasing to be man. He may quite honestly believe that this statement is a fair inference from his experiments, but in that case his judgment is at fault. A very simple consideration of the character of experiments would show him that they never can lead to results of such a kind; that being themselves only approximate and limited, they cannot give us knowledge which is exact and universal. No eminence of character and genius can give a man authority enough to justify us in hclieving him when he makes statements implying exact or universal knowledge.

Again, an Arctic explorer may tell us that in a given latitude and longitude he has experienced such and such a degree of cold, that the sea was of such a depth, and the ice of such a character. We should be quite right to believe him, in the absence of any stain upon his veracity. It is conceivable that we might, without ceasing to be men, go there and verify his statement; it can be tested by the witness of his companions, and there is adequate ground for supposing that he knows the truth of what he is saying. But if an old whaler tells us that the ice is 300 feet thick all the way up to the Pole, we shall not be justified in believing him. For although the statement may he capable of verification by man, it is certainly not capable of verification by \emph{him}, \marginpar{355} with any means and appliances which he has possessed; and he must have persuaded himself of the truth of it by some means which does not attach any credit to his testimony. Even if, therefore, the matter affirmed is within the reach of human knowledge, we have no right to accept it upon authority unless it is within the reach of our informant's knowledge.

What shall we say of that authority, more venerable and august than any individual witness, the time-honoured tradition of the human race? An atmosphere of beliefs and conceptions has been formed by the labours and struggles of our forefathers, which enables us to breathe amid the various and complex circumstances of our life. It is around and about us and within us; we cannot think except in the forms and processes of thought which it supplies. Is it possible to doubt and to test it? and if possible, is it right?

We shall find reason to answer that it is not only possible and right, but our bounden duty; that the main purpose of the tradition itself is to supply us with the means of asking questions, of testing and inquiring into things; that if we misuse it, and take it as a collection of cut-and-dried statements to be accepted without further inquiry, we are not only injuring ourselves here, but, by refusing to do our part towards the building up of the fabric which shall be inherited by our children, we are tending to cut off ourselves and our race from the human line.

Let us first take care to distinguish a kind of tradition which especially requires to be examined and called in question, because it especially shrinks from inquiry. Suppose that a medicine-man in Central Africa tells his tribe that a certain powerful medicine in his tent will be propitiated if they kill their cattle, and that the tribe believe him. Whether the medicine was propitiated or not there are no means of verifying, but the cattle are gone. Still the belief may be kept up in the tribe that propitiation has been effected in this way; and in a later generation it will be all the easier for another medicine-man to persuade them to a similar act. Here the only reason for belief is that everybody has believed the thing for so long that it must be true. And yet the belief was founded on fraud, and has been propagated by credulity. That man will undoubtedly do right, and be a\marginpar{356} friend of men, who shall call it in question and see that there is no evidence for it, help his neighbours to see as he does, and even, if need be, go into the holy tent and break the medicine.

The rule which should guide us in such cases is simple and obvious enough: that the aggregate testimony of our neighbours is subject to the same conditions as the testimony of any one of them. Namely, we have no right to believe a thing true because everybody says so unless there are good grounds for believing that some one person at least has the means of knowing what is true, and is speaking the truth so far as he knows it. However many nations and generations of men are brought into the witness-box they cannot testify to anything which they do not know. Every man who has accepted the statement from somebody else, without himself testing and verifying it, is out of court; his word is worth nothing at all. And when we get back at last to the true birth and beginning of the statement, two serious questions must be disposed of in regard to him who first made it: was he mistaken in thinking that he \emph{knew} about this matter, or was he lying?

This last question is unfortunately a very actual and practical one even to us at this day and in this country. We have no occasion to go to La Salette, or to Central Africa, or to Lourdes, for examples of immoral and debasing superstition. It is only too possible for a child to grow up in London surrounded by an atmosphere of beliefs fit only for the savage, which have in our own time been founded in fraud and propagated by credulity.

Laying aside, then, such tradition as is handed on without testing by successive generations, let us consider that which is truly built up out of the common experience of mankind. This great fabric is for the guidance of our thoughts, and through them of our actions, both in the moral and in the material world. In the moral world, for example, it gives us the conceptions of right in general, of justice, of truth, of beneficence, and the like. These are given as conceptions, not as statements or propositions; they answer to certain definite instincts which are certainly within us, however they came there. That it is right to be beneficent is matter of\marginpar{357} immediate personal experience; for when a man retires within himself and there finds something, wider and more lasting than his solitary personality, which says, ``I want to do right,'' as well as, ``I want to do good to man,'' he can verify by direct observation that one instinct is founded upon and agrees fully with the other. And it is his duty so to verify this and all similar statements.

The tradition says also, at a definite place and time, that such and such actions are just, or true, or beneficent. For all such rules a further inquiry is necessary, since they are sometimes established by an authority other than that of the moral sense founded on experience. Until recently, the moral tradition of our own country--- and indeed of all Europe--- taught that it was beneficent to give money indiscriminately to beggars. But the questioning of this rule, and investigation into it, led men to see that true beneficence is that which helps a man to do the work which he is most fitted for, not that which keeps and encourages him in idleness; and that to neglect this distinction in the present is to prepare pauperism and misery for the future. By this testing and discussion not only has practice been purified and made more beneficent, but the very conception of beneficence has been made wider and wiser. Now here the great social heirloom consists of two parts: the instinct of beneficence, which makes a certain side of our nature, when predominant, wish to do good to men; and the intellectual conception of benefi cence, which we can compare with any proposed course of conduct and ask, ``Is this beneficent or not?'' By the continual asking and answering of such questions the conception grows in breadth and distinctness, and the instinct becomes strengthened and purified It appears, then, that the great use of the conception, the intellectual part of the heirloom, is to enable us to ask questions; that it grows and is kept straight by means of these questions; and if we do not use it for that purpose we shall gradually lose it altogether, and be left with a mere code of regulations which cannot rightly be called morality at all

Such considerations apply even more obviously and clearly, if possible, to the store of beliefs and conceptions which our fathers have amassed for us in respect of the material world.\marginpar{358} We are ready to laugh at the rule of thumb of the Australian who continues to tie his hatchet to the side of the handle, although the Birmingham fitter has made a hole on purpose for him to put the handle in. His people have tied up hatchets so for ages: who is he that he should set himself up against their wisdom? He has sunk so low that he cannot do what some of them must have done in the far distant past--- call in question an established usage, and invent or learn something better. Yet here, in the dim beginning of knowledge, where science and art are one, we find only the same simple rule which applies to the highest and deepest growths of that cosmic Tree; to its loftiest flower-tipped branches as well as to the profoundest of its hidden roots; the rule, namely, that what is stored up and handed down to us is rightly used by those who act as the makers acted, when they stored it up; those who use it to ask further questions, to examine, to investigate; who try honestly and solemnly to find out what is the right way of looking at things and of dealing with them.

A question rightly asked is already half answered, said Jacobi; we may add that the method of solution is the other half of the answer, and that the actual result counts for nothing by the side of these two. For an example let us go to the telegraph, where theory and practice, grown each to years of discretion, are marvellously wedded for the fruitful service of men. Ohm found that the strength of an electric current is directly proportional to the strength of the battery which produces it, and inversely as the length of the wire along which it has to travel. This is called Ohm's law; but the result, regarded as a statement to be believed, is not the valuable part of it. The first half is the question: what relation holds good between these quantities? So put, the question involves already the conception of strength of current, and of strength of battery, as quantities to be measured and compared; it hints clearly that these are the things to be attended to in the study of electric currents. The second half is the method of investigation; how to measure these quantities, what instruments are required for the experiment, and how are they to be used? The student who begins to learn about electricity is not asked to believe in Ohm's law: \marginpar{359} he is made to understand the question, he is placed before the apparatus, and he is taught to verify it. He learns to do things, not to think he knows things; to use instruments and to ask questions, not to accept a traditional statement. The question which required a genius to ask it rightly is answered by a tiro. If Ohm's law were suddenly lost and forgotten by all men, while the question and the method of solution remained, the result could be rediscovered in an hour. But the result by itself, if known to a people who could not comprehend the value of the question or the means of solving it, would be like a watch in the hands of a savage who could not wind it up, or an iron steamship worked by Spanish engineers. In regard, then, to the sacred tradition of humanity, we learn that it consists, not in propositions or statements which are to be accepted and believed on the authority of the tradition, but in questions rightly asked, in conceptions which enable us to ask further questions, and in methods of answering questions. The value of all these things depends on their being tested day by day. The very sacredness of the precious deposit imposes upon us the duty and the responsibility of testing it, of purifying and enlarging it to the utmost of our power. He who makes use of its results to stifle his own doubts, or to hamper the inquiry of others, is guilty of a sacrilege which centuries shall never be able to blot out. When the labours and questionings of honest and brave men shall have built up the fabric of known truth to a glory which we in this generation can neither hope for nor imagine, in that pure and holy temple he shall have no part nor lot, but his name and his works shall be cast out into the darkness of oblivion for ever.

\section*{III.---\textsc{The Limits of Inference}}

The question in what cases we may believe that which goes beyond our experience, is a very large and delicate one, extending to the whole range of scientific method, and requiring a considerable increase in the application of it before it can be answered with anything approaching to completeness. \marginpar{360} But one rule, lying on the threshold of the subject, of extreme simplicity and vast practical importance, may here be touched upon and shortly laid down.

A little reflection will show us that every belief, even the simplest and most fundamental, goes beyond experience when regarded as a guide to our actions. A burnt child dreads the fire, because it believes that the fire will burn it to-day just as it did yesterday; but this belief goes beyond experience, and assumes that the unknown fire of to-day is like the known fire of yesterday. Even the belief that the child was burnt yesterday goes beyond \emph{present} experience, which contains only the memory of a burning, and not the burning itself; it assumes, therefore, that this memory is trustworthy, although we know that a memory may often be mistaken. But if it is to be used as a guide to action, as a hint of what the future is to be, it must assume something about that future, namely, that it will be consistent with the supposition that the burning really took place yesterday; which is going beyond experience. Even the fundamental ``I am,'' which cannot be doubted, is no guide to action until it takes to itself ``I shall be,'' which goes beyond experience. The question is not, therefore, ``May we believe what goes beyond experience?'' for this is involved in the very nature of belief; but ``How far and in what manner may we add to our experience in forming our beliefs?''

And an answer, of utter simplicity and universality, is suggested by the example we have taken: a burnt child dreads the fire. We may go beyond experience by assuming that what we do not know is like what we do know; or, in other words, we may add to our experience on the assumption of a uniformity in nature. What this uniformity precisely is, how we grow in the knowledge of it from generation to generation, these are questions which for the present we lay aside, being content to examine two instances which may serve to make plainer the nature of the rule.

From certain observations made with the spectroscope, we infer the existence of hydrogen in the sun. By looking into the spectroscope when the sun is shining on its slit, we see certain definite bright lines: and experiments made upon bodies on the earth have taught us that when these bright\marginpar{361} lines are seen hydrogen is the source of them. We assume, then, that the unknown bright lines in the sun are like the known bright lines of the laboratory, and that hydrogen in the sun behaves as hydrogen under similar circumstances would behave on the earth.

But are we not trusting our spectroscope too much? Surely, having found it to be trustworthy for terrestrial substances, where its statements can be verified by man, we are justified in accepting its testimony in other like cases; but not when it gives us information about things in the sun, where its testimony cannot be directly verified by man?

Certainly, we want to know a little more before this inference can be justified; and fortunately we do know this. The spectroscope testifies to exactly the same thing in the two cases; namely, that light-vibrations of a certain rate are being sent through it. Its construction is such that if it were wrong about this in one case, it would be wrong in the other. When we come to look into the matter, we find that we have really assumed the matter of the sun to be like the matter of the earth, made up of a certain number of distinct substances; and that each of these, when very hot, has a distinct rate of vibration, by which it may be recognised and singled out from the rest. But this is the kind of assumption which we arc justified in using when we add to our experience. It is an assumption of uniformity in nature, and can only be checked by comparison with many similar assumptions which we have to make in other such cases.

But is this a true belief, of the existence of hydrogen in the sun? Can it help in the right guidance of human action?

Certainly not, if it is accepted on unworthy grounds, and without some understanding of the process by which it is got at. But when this process is taken in as the ground of the belief, it becomes a very serious and practical matter. For if there is no hydrogen in the sun, the spectroscope--- that is to say, the measurement of rates of vibration--- must be an uncertain guide in recognising different substances; and consequently it ought not to be used in chemical analysis--- in assaying, for example--- to the great saving of time, trouble, and money. Whereas the acceptance of the spectroscopic method as trustworthy has enriched us not only: with new \marginpar{362} metals, which is a great thing, but with new processes of investigation, which is vastly greater.

For another example, let us consider the way in which we infer the truth of an historical event--- say the siege of Syracuse in the Peloponnesian war. Our experience is that manuscripts exist which are said to be and which call themselves manuscripts of the history of Thucydides; that in other manuscripts, stated to be by later historians, he is described as living during the time of the war; and that books, supposed to date from the revival of learning, tell us how these manuscripts had been preserved and were then acquired. We find also that men do not, as a rule, forge books and histories without a special motive; we assume that in this respect men in the past were like men in the present; and we observe that in this case no special motive was present. That is, we add to our experience on the assumption of a uniformity in the characters of men. Because our knowledge of this uniformity is far less complete and exact than our knowledge of that which obtains in physics, inferences of the historical kind are more precarious and less exact than inferences in many other sciences.

But if there is any special reason to suspect the character of the persons who wrote or transmitted certain books, the case becomes altered If a group of documents give internal evidence that they were produced among people who forged books in the names of others, and who, in describing events, suppressed those things which did not suit them, while they amplified such as did suit them; who not only committed these crimes, but gloried in them as proofs of humility and zeal; then we must say that upon such documents no true historical inference can be founded, but only unsatisfactory conjecture.

We may, then, add to our experience on the assumption of a uniformity in nature; we may fill in our picture of what is and has been, as experience gives it us, in such a way as to make the whole consistent with this uniformity. And practically demonstrative inference--- that which gives us a right to believe in the result of it--- is a clear showing that in no other way than by the truth of this result can the uniformity of nature be saved.\marginpar{363}

No evidence, therefore, can justify us in believing the truth of a statement which is contrary to, or outside of, the uniformity of nature. If our experience is such that it cannot be filled up consistently with uniformity, all we have a right to conclude is that there is something wrong somewhere; but the possibility of inference is taken away; we must rest in our experience, and not go beyond it at all. If an event really happened which was not a part of the uniformity of nature, it would have two properties: no evidence could give the right to believe it to any except those whose actual experience it was; and no inference worthy of belief could be founded upon it at all.

Are we then bound to believe that nature is absolutely and universally uniform? Certainly not; we have no right to believe anything of this kind. The rule only tells us that in forming beliefs which go beyond our experience, we may make the assumption that nature is practically uniform so far as we are concerned. Within the range of human action and verification, we may form, by help of this assumption, actual beliefs; beyond it, only those hypotheses which serve for the more accurate asking of questions.

To sum up:---

We may believe what goes beyond our experience, only when it is inferred from that experience by the assumption that what we do not know is like what we know. We may believe the statement of another person, when there is reasonable ground for supposing that he knows the matter of which he speaks, and that he is speaking the truth so far as he knows it.

It is wrong in all cases to believe on insufficient evidence; and where it is presumption to doubt and to investigate, there it is worse than presumption to believe.


\label{theend}
\end{document}
