\documentclass[12pt]{article}
\usepackage{hyperref,setspace,graphicx}
%\usepackage[symbol*,perpage]{footmisc}
\usepackage[tagged]{accessibility}
\usepackage[width=6in,height=9in,top=.5in,centering]{geometry}
\usepackage[T1]{fontenc}
% changes font to TeX Gyre Schola (Century Schoolbook)
\usepackage{tgschola}
% changes font to TeX Gyre Bonum (Bookman)
%\usepackage{tgbonum}


\begin{document}
%\renewcommand*{\thefootnote}{\fnsymbol{footnote}}
\setcounter{footnote}{0}
\hypersetup{pdfinfo={Title={selections from Human Nature and Conduct}, Author={John Dewey}}, pdfborder = {0 0 0 0}}

\setlength{\parskip}{4pt}
\setlength{\parindent}{0pt}



\centerline{
\textsc{\Large Human Nature and Conduct}
}


\bigskip



Excerpted from John Dewey, \emph{Human Nature and Conduct: An Introduction to Social Psychology}, New York: Henry Holt and Company. First published in 1922, the book is based on a series of invited lectures that Dewey delivered at Stanford University in spring of 1918.

Dewey concedes in the preface that ``the sub-title requires a word of explanation. The book does not purport to be a treatment of social psychology.'' Perhaps the problem, as Murray G. Murphey suggests, is ``that here, as often before, Dewey is seeking to say something new and that he does not have an adequate vocabulary for his purpose.''

\paragraph{About this text:}
Based on the Project Gutenberg edition, eBook 41386, and prepared by P.D. Magnus in Fall 2025. %Marginal page numbers, following the Project Gutenberg edition, correspond to the 1922 original.
The excerpts are numbered as Chapters 1, 2, 14, 15, and 23 in the Collected Works of John Dewey, Middle Works, volume 14.



\centerline{\includegraphics[height=1in]{dewey-hnc-cover-owl.jpg}}



%\setstretch{1.2}

\providecommand*{\hncpage}[1]{}


\section*{Part One: The Place of Habit in Conduct}
\hncpage{14}

\subsection*{I}

Habits may be profitably compared to physiological
functions, like breathing, digesting. The latter are, to
be sure, involuntary, while habits are acquired. But
important as is this difference for many purposes it
should not conceal the fact that habits are like functions
in many respects, and especially in requiring the
cooperation of organism and environment. Breathing
is an affair of the air as truly as of the lungs; digesting
an affair of food as truly as of tissues of stomach.
Seeing involves light just as certainly as it does the
eye and optic nerve. Walking implicates the ground
as well as the legs; speech demands physical air and
human companionship and audience as well as vocal
organs. We may shift from the biological to the mathematical
use of the word function, and say that natural
operations like breathing and digesting, acquired ones
like speech and honesty, are functions of the surroundings
as truly as of a person. They are things done \emph{by}
the environment by means of organic structures or
acquired dispositions. The same air that under certain
conditions ruffles the pool or wrecks buildings,
\hncpage{15}
under other conditions purifies the blood and conveys
thought. The outcome depends upon what air acts
upon. The social environment acts through native impulses
and speech and moral habitudes manifest themselves.
There are specific good reasons for the usual
attribution of acts to the person from whom they immediately
proceed. But to convert this special reference
into a belief of exclusive ownership is as misleading
as to suppose that breathing and digesting are
complete within the human body. To get a rational
basis for moral discussion we must begin with recognizing
that functions and habits are ways of using and
incorporating the environment in which the latter has
its say as surely as the former.

We may borrow words from a context less technical
than that of biology, and convey the same idea by saying
that habits are arts. They involve skill of sensory
and motor organs, cunning or craft, and objective
materials. They assimilate objective energies, and
eventuate in command of environment. They require
order, discipline, and manifest technique. They have
a beginning, middle and end. Each stage marks progress
in dealing with materials and tools, advance in converting
material to active use. We should laugh at any
one who said that he was master of stone working, but
that the art was cooped up within himself and in no wise
dependent upon support from objects and assistance
from tools.

In morals we are however quite accustomed to such
a fatuity. Moral dispositions are thought of as belonging
\hncpage{16}
exclusively to a self. The self is thereby isolated
from natural and social surroundings. A whole school
of morals flourishes upon capital drawn from restricting
morals to character and then separating character
from conduct, motives from actual deeds. Recognition
of the analogy of moral action with functions and arts
uproots the causes which have made morals subjective
and ``individualistic.'' It brings morals to earth, and
if they still aspire to heaven it is to the heavens of the
earth, and not to another world. Honesty, chastity,
malice, peevishness, courage, triviality, industry, irresponsibility
are not private possessions of a person.
They are working adaptations of personal capacities
with environing forces. All virtues and vices are habits
which incorporate objective forces. They are interactions
of elements contributed by the make-up of an
individual with elements supplied by the out-door world.
They can be studied as objectively as physiological
functions, and they can be modified by change of either
personal or social elements.

If an individual were alone in the world, he would
form his habits (assuming the impossible, namely, that
he would be able to form them) in a moral vacuum.
They would belong to him alone, or to him only in reference
to physical forces. Responsibility and virtue
would be his alone. But since habits involve the support
of environing conditions, a society or some specific
group of fellow-men, is always accessory before and
after the fact. Some activity proceeds from a man;
then it sets up reactions in the surroundings. Others
\hncpage{17}
approve, disapprove, protest, encourage, share and resist.
Even letting a man alone is a definite response.
Envy, admiration and imitation are complicities. Neutrality
is non-existent. Conduct is always shared; this
is the difference between it and a physiological process.
It is not an ethical ``ought'' that conduct \emph{should} be
social. It \emph{is} social, whether bad or good.

Washing one's hands of the guilt of others is a way
of sharing guilt so far as it encourages in others a
vicious way of action. Non-resistance to evil which
takes the form of paying no attention to it is a way
of promoting it. The desire of an individual to keep
his own conscience stainless by standing aloof from
badness may be a sure means of causing evil and thus
of creating personal responsibility for it. Yet there are
circumstances in which passive resistance may be the
most effective form of nullification of wrong action,
or in which heaping coals of fire on the evil-doer may
be the most effective way of transforming conduct. To
sentimentalize over a criminal---to ``forgive'' because
of a glow of feeling---is to incur liability for production
of criminals. But to suppose that infliction of retributive
suffering suffices, without reference to concrete
consequences, is to leave untouched old causes of criminality
and to create new ones by fostering revenge and
brutality. The abstract theory of justice which demands
the ``vindication'' of law irrespective of instruction
and reform of the wrong-doer is as much a
refusal to recognize responsibility as is the sentimental
gush which makes a suffering victim out of a criminal.

\hncpage{18}
Courses of action which put the blame exclusively
on a person as if his evil will were the sole cause of
wrong-doing and those which condone offense on account
of the share of social conditions in producing
bad disposition, are equally ways of making an unreal
separation of man from his surroundings, mind from
the world. Causes for an act always exist, but causes
are not excuses. Questions of causation are physical,
not moral except when they concern future consequences.
It is as causes of future actions that excuses
and accusations alike must be considered. At present
we give way to resentful passion, and then ``rationalize''
our surrender by calling it a vindication of justice.
Our entire tradition regarding punitive justice tends
to prevent recognition of social partnership in producing
crime; it falls in with a belief in metaphysical
free-will. By killing an evil-doer or shutting him up
behind stone walls, we are enabled to forget both him
and our part in creating him. Society excuses itself
by laying the blame on the criminal; he retorts by putting
the blame on bad early surroundings, the temptations
of others, lack of opportunities, and the persecutions
of officers of the law. Both are right, except in
the wholesale character of their recriminations. But
the effect on both sides is to throw the whole matter
back into antecedent causation, a method which refuses
to bring the matter to truly moral judgment. For
morals has to do with acts still within our control, acts
still to be performed. No amount of guilt on the part
\hncpage{19}
of the evil-doer absolves us from responsibility for the
consequences upon him and others of our way of treating
him, or from our continuing responsibility for the
conditions under which persons develop perverse habits.

We need to discriminate between the physical and the
moral question. The former concerns what \emph{has} happened,
and how it happened. To consider this question
is indispensable to morals. Without an answer to it we
cannot tell what forces are at work nor how to direct
our actions so as to improve conditions. Until we
know the conditions which have helped form the characters
we approve and disapprove, our efforts to create
the one and do away with the other will be blind and
halting. But the moral issue concerns the future. It is
prospective. To content ourselves with pronouncing
judgments of merit and demerit without reference to
the fact that our judgments are themselves facts which
have consequences and that their value depends upon
\emph{their} consequences, is complacently to dodge the moral
issue, perhaps even to indulge ourselves in pleasurable
passion just as the person we condemn once indulged
himself. The moral problem is that of modifying the
factors which now influence future results. To change
the working character or will of another we have to
alter objective conditions which enter into his habits.
Our own schemes of judgment, of assigning blame and
praise, of awarding punishment and honor, are part
of these conditions.

In practical life, there are many recognitions of the
\hncpage{20}
part played by social factors in generating personal
traits. One of them is our habit of making social
classifications. We attribute distinctive characteristics
to rich and poor, slum-dweller and captain of industry,
rustic and suburbanite, officials, politicians, professors,
to members of races, sets and parties. These judgments
are usually too coarse to be of much use. But
they show our practical awareness that personal traits
are functions of social situations. When we generalize
this perception and act upon it intelligently we are
committed by it to recognize that we change character
from worse to better only by changing conditions---among
which, once more, are our own ways of dealing
with the one we judge. We cannot change habit directly:
that notion is magic. But we can change it
indirectly by modifying conditions, by an intelligent
selecting and weighting of the objects which engage
attention and which influence the fulfilment of desires.

A savage can travel after a fashion in a jungle.
Civilized activity is too complex to be carried on without
smoothed roads. It requires signals and junction
points; traffic authorities and means of easy and rapid
transportation. It demands a congenial, antecedently
prepared environment. Without it, civilization would
relapse into barbarism in spite of the best of subjective
intention and internal good disposition. The eternal
dignity of labor and art lies in their effecting that permanent
reshaping of environment which is the substantial
foundation of future security and progress. Individuals
\hncpage{21}
flourish and wither away like the grass of the
fields. But the fruits of their work endure and make
possible the development of further activities having
fuller significance. It is of grace not of ourselves that
we lead civilized lives. There is sound sense in the old
pagan notion that gratitude is the root of all virtue.
Loyalty to whatever in the established environment
makes a life of excellence possible is the beginning of
all progress. The best we can accomplish for posterity
is to transmit unimpaired and with some increment of
meaning the environment that makes it possible to
maintain the habits of decent and refined life. Our
individual habits are links in forming the endless chain
of humanity. Their significance depends upon the environment
inherited from our forerunners, and it is
enhanced as we foresee the fruits of our labors in the
world in which our successors live.

For however much has been done, there always remains
more to do. We can retain and transmit our own
heritage only by constant remaking of our own environment.
Piety to the past is not for its own sake nor for
the sake of the past, but for the sake of a present so
secure and enriched that it will create a yet better
future. Individuals with their exhortations, their
preachings and scoldings, their inner aspirations and
sentiments have disappeared, but their habits endure,
because these habits incorporate objective conditions in
themselves. So will it be with \emph{our} activities. We may
desire abolition of war, industrial justice, greater
\hncpage{22}
equality of opportunity for all. But no amount of
preaching good will or the golden rule or cultivation
of sentiments of love and equity will accomplish the
results. There must be change in objective arrangements
and institutions. We must work on the environment
not merely on the hearts of men. To think otherwise
is to suppose that flowers can be raised in a desert
or motor cars run in a jungle. Both things can happen
and without a miracle. But only by first changing the
jungle and desert.

Yet the distinctively personal or subjective factors in
habit count. Taste for flowers may be the initial step
in building reservoirs and irrigation canals. The stimulation
of desire and effort is one preliminary in the
change of surroundings. While personal exhortation,
advice and instruction is a feeble stimulus compared
with that which steadily proceeds from the impersonal
forces and depersonalized habitudes of the environment,
yet they may start the latter going. Taste, appreciation
and effort always spring from some accomplished
objective situation. They have objective
support; they represent the liberation of something
formerly accomplished so that it is useful in further
operation. A genuine appreciation of the beauty of
flowers is not generated within a self-enclosed consciousness.
It reflects a world in which beautiful flowers have
already grown and been enjoyed. Taste and desire
represent a prior objective fact recurring in action to
secure perpetuation and extension. Desire for flowers
comes after actual enjoyment of flowers. But it comes
\hncpage{23}
before the work that makes the desert blossom, it comes
before \emph{cultivation} of plants. Every ideal is preceded by
an actuality; but the ideal is more than a repetition
in inner image of the actual. It projects in securer and
wider and fuller form some good which has been previously
experienced in a precarious, accidental, fleeting
way.


\subsection*{II}
\hncpage{24}

It is a significant fact that in order to appreciate
the peculiar place of habit in activity we have to betake
ourselves to bad habits, foolish idling, gambling,
addiction to liquor and drugs. When we think of such
habits, the union of habit with desire and with propulsive
power is forced upon us. When we think of
habits in terms of walking, playing a musical instrument,
typewriting, we are much given to thinking of
habits as technical abilities existing apart from our
likings and as lacking in urgent impulsion. We think
of them as passive tools waiting to be called into action
from without. A bad habit suggests an inherent tendency
to action and also a hold, command over us. It
makes us do things we are ashamed of, things which we
tell ourselves we prefer not to do. It overrides our
formal resolutions, our conscious decisions. When we
are honest with ourselves we acknowledge that a habit
has this power because it is so intimately a part of ourselves.
It has a hold upon us because we are the habit.

Our self-love, our refusal to face facts, combined
perhaps with a sense of a possible better although
unrealized self, leads us to eject the habit from the
thought of ourselves and conceive it as an evil power
which has somehow overcome us. We feed our conceit
by recalling that the habit was not deliberately formed;
we never intended to become idlers or gamblers or rouès.
\hncpage{25}
And how can anything be deeply ourselves which developed
accidentally, without set intention? These
traits of a bad habit are precisely the things which are
most instructive about all habits and about ourselves.
They teach us that all habits are affections, that all
have projectile power, and that a predisposition
formed by a number of specific acts is an immensely
more intimate and fundamental part of ourselves than
are vague, general, conscious choices. All habits are
demands for certain kinds of activity; and they constitute
the self. In any intelligible sense of the word
will, they \emph{are} will. They form our effective desires and
they furnish us with our working capacities. They
rule our thoughts, determining which shall appear and
be strong and which shall pass from light into
obscurity.

We may think of habits as means, waiting, like tools
in a box, to be used by conscious resolve. But they
are something more than that. They are active means,
means that project themselves, energetic and dominating
ways of acting. We need to distinguish between
materials, tools and means proper. Nails and boards
are not strictly speaking means of a box. They are
only materials for making it. Even the saw and hammer
are means only when they are employed in some
actual making. Otherwise they are tools, or potential
means. They are actual means only when brought in
conjunction with eye, arm and hand in some specific
operation. And eye, arm and hand are, correspondingly,
means proper only when they are in active operation.
\hncpage{26}
And whenever they are in action they are cooperating
with external materials and energies. Without
support from beyond themselves the eye stares blankly
and the hand moves fumblingly. They are means only
when they enter into organization with things which
independently accomplish definite results. These organizations
are habits.

This fact cuts two ways. Except in a contingent
sense, with an ``if,'' neither external materials nor bodily
and mental organs are in themselves means. They
have to be employed in coordinated conjunction with
one another to be actual means, or habits. This statement
may seem like the formulation in technical language
of a common-place. But belief in magic has
played a large part in human history. And the essence
of all hocus-pocus is the supposition that results
can be accomplished without the joint adaptation to
each other of human powers and physical conditions.
A desire for rain may induce men to wave willow
branches and to sprinkle water. The reaction is natural
and innocent. But men then go on to believe that
their act has immediate power to bring rain without
the cooperation of intermediate conditions of nature.
This is magic; while it may be natural or spontaneous,
it is not innocent. It obstructs intelligent study of
operative conditions and wastes human desire and effort
in futilities.

Belief in magic did not cease when the coarser forms
of superstitious practice ceased. The principle of
magic is found whenever it is hoped to get results
\hncpage{27}
without intelligent control of means; and also when it
is supposed that means can exist and yet remain inert
and inoperative. In morals and politics such expectations
still prevail, and in so far the most important
phases of human action are still affected by magic. We
think that by feeling strongly enough about something,
by wishing hard enough, we can get a desirable result,
such as virtuous execution of a good resolve, or peace
among nations, or good will in industry. We slur over
the necessity of the cooperative action of objective
conditions, and the fact that this cooperation is assured
only by persistent and close study. Or, on the
other hand, we fancy we can get these results by
external machinery, by tools or potential means, without
a corresponding functioning of human desires and
capacities. Often times these two false and contradictory
beliefs are combined in the same person. The man
who feels that \emph{his} virtues are his own personal accomplishments
is likely to be also the one who thinks that
by passing laws he can throw the fear of God into
others and make them virtuous by edict and prohibitory
mandate.

Recently a friend remarked to me that there was one
superstition current among even cultivated persons.
They suppose that if one is told what to do, if the
right \emph{end} is pointed to them, all that is required in
order to bring about the right act is will or wish on
the part of the one who is to act. He used as an illustration
the matter of physical posture; the assumption
is that if a man is told to stand up straight, all that
\hncpage{28}
is further needed is wish and effort on his part, and
the deed is done. He pointed out that this belief is on
a par with primitive magic in its neglect of attention
to the means which are involved in reaching an end.
And he went on to say that the prevalence of this belief,
starting with false notions about the control of
the body and extending to control of mind and character,
is the greatest bar to intelligent social progress.
It bars the way because it makes us neglect intelligent
inquiry to discover the means which will produce a
desired result, and intelligent invention to procure the
means. In short, it leaves out the importance of intelligently
controlled habit.

We may cite his illustration of the real nature of a
physical aim or order and its execution in its contrast
with the current false notion.\footnote{I refer to Alexander, \emph{Man's Supreme Inheritance}.} A man who has a bad
habitual posture tells himself, or is told, to stand up
straight. If he is interested and responds, he braces
himself, goes through certain movements, and it is assumed
that the desired result is substantially attained;
and that the position is retained at least as long as
the man keeps the idea or order in his mind. Consider
the assumptions which are here made. It is implied
that the means or effective conditions of the realization
of a purpose exist independently of established
habit and even that they may be set in motion in opposition
to habit. It is assumed that means are there,
so that the failure to stand erect is wholly a matter of
failure of purpose and desire. It needs paralysis or
\hncpage{29}
a broken leg or some other equally gross phenomenon
to make us appreciate the importance of objective
conditions.

Now in fact a man who \emph{can} stand properly does so,
and only a man who can, does. In the former case,
fiats of will are unnecessary, and in the latter useless.
A man who does not stand properly forms a habit of
standing improperly, a positive, forceful habit. The
common implication that his mistake is merely negative,
that he is simply failing to do the right thing, and
that the failure can be made good by an order of will
is absurd. One might as well suppose that the man
who is a slave of whiskey-drinking is merely one who
fails to drink water. Conditions have been formed for
producing a bad result, and the bad result will occur
as long as those conditions exist. They can no more
be dismissed by a direct effort of will than the conditions
which create drought can be dispelled by whistling
for wind. It is as reasonable to expect a fire to go out
when it is ordered to stop burning as to suppose that
a man can stand straight in consequence of a direct
action of thought and desire. The fire can be put out
only by changing objective conditions; it is the same
with rectification of bad posture.

Of course something happens when a man acts upon
his idea of standing straight. For a little while, he
stands differently, but only a different kind of badly.
He then takes the unaccustomed feeling which accompanies
his unusual stand as evidence that he is now
standing right. But there are many ways of standing
\hncpage{30}
badly, and he has simply shifted his usual way to a
compensatory bad way at some opposite extreme.
When we realize this fact, we are likely to suppose that
it exists because control of the \emph{body} is physical and
hence is external to mind and will. Transfer the command
inside character and mind, and it is fancied that
an idea of an end and the desire to realize it will take
immediate effect. After we get to the point of recognizing
that habits must intervene between wish and
execution in the case of bodily acts, we still cherish
the illusions that they can be dispensed with in the case
of mental and moral acts. Thus the net result is to
make us sharpen the distinction between non-moral and
moral activities, and to lead us to confine the latter
strictly within a private, immaterial realm. But in
fact, formation of ideas as well as their execution depends
upon habit. \emph{If} we could form a correct idea
without a correct habit, then possibly we could carry
it out irrespective of habit. But a wish gets definite
form only in connection with an idea, and an idea gets
shape and consistency only when it has a habit back of
it. Only when a man can already perform an act of
standing straight does he know what it is like to have
a right posture and only then can he summon the
idea required for proper execution. The act must come
before the thought, and a habit before an ability to
evoke the thought at will. Ordinary psychology reverses
the actual state of affairs.

Ideas, thoughts of ends, are not spontaneously generated.
There is no immaculate conception of meanings
\hncpage{31}
or purposes. Reason pure of all influence from
prior habit is a fiction. But pure sensations out of
which ideas can be framed apart from habit are equally
fictitious. The sensations and ideas which are the
``stuff'' of thought and purpose are alike affected by
habits manifested in the acts which give rise to sensations
and meanings. The dependence of thought, or
the more intellectual factor in our conceptions, upon
prior experience is usually admitted. But those who
attack the notion of thought pure from the influence
of experience, usually identify experience with sensations
impressed upon an empty mind. They therefore
replace the theory of unmixed thoughts with that of
pure unmixed sensations as the stuff of all conceptions,
purposes and beliefs. But distinct and independent
sensory qualities, far from being original elements, are
the products of a highly skilled analysis which disposes
of immense technical scientific resources. To be able to
single out a definitive sensory element in any field is
evidence of a high degree of previous training, that is,
of well-formed habits. A moderate amount of observation
of a child will suffice to reveal that even such gross
discriminations as black, white, red, green, are the result
of some years of active dealings with things in the
course of which habits have been set up. It is not such
a simple matter to have a clear-cut sensation. The
latter is a sign of training, skill, habit.

Admission that the idea of, say, standing erect is
dependent upon sensory materials is, therefore equivalent
to recognition that it is dependent upon the
\hncpage{32}
habitual attitudes which govern concrete sensory materials.
The medium of habit filters all the material
that reaches our perception and thought. The filter is
not, however, chemically pure. It is a reagent which
adds new qualities and rearranges what is received.
Our ideas truly depend upon experience, but so do our
sensations. And the experience upon which they both
depend is the operation of habits---originally of instincts.
Thus our purposes and commands regarding
action (whether physical or moral) come to us through
the refracting medium of bodily and moral habits. Inability
to think aright is sufficiently striking to have
caught the attention of moralists. But a false psychology
has led them to interpret it as due to a necessary
conflict of flesh and spirit, not as an indication
that our ideas are as dependent, to say the least, upon
our habits as are our acts upon our conscious thoughts
and purposes.

Only the man who can maintain a correct posture
has the stuff out of which to form that idea of standing
erect which can be the starting point of a right act.
Only the man whose habits are already good can know
what the good is. Immediate, seemingly instinctive,
feeling of the direction and end of various lines of behavior
is in reality the feeling of habits working below
direct consciousness. The psychology of illusions of
perception is full of illustrations of the distortion introduced
by habit into observation of objects. The
same fact accounts for the intuitive element in judgments
of action, an element which is valuable or the
\hncpage{33}
reverse in accord with the quality of dominant habits.
For, as Aristotle remarked, the untutored moral perceptions
of a good man are usually trustworthy, those
of a bad character, not. (But he should have added
that the influence of social custom as well as personal
habit has to be taken into account in estimating who
is the good man and the good judge.)

What is true of the dependence of execution of an
idea upon habit is true, then, of the formation and
quality of the idea. Suppose that by a happy chance
a right concrete idea or purpose---concrete, not simply
correct in words---has been hit upon: What happens
when one with an incorrect habit tries to act in accord
with it? Clearly the idea can be carried into execution
only with a mechanism already there. If this is defective
or perverted, the best intention in the world will
yield bad results. In the case of no other engine does
one suppose that a defective machine will turn out good
goods simply because it is invited to. Everywhere else
we recognize that the design and structure of the agency
employed tell directly upon the work done. Given a
bad habit and the ``will'' or mental direction to get a
good result, and the actual happening is a reverse or
looking-glass manifestation of the usual fault---a compensatory
twist in the opposite direction. Refusal
to recognize this fact only leads to a separation of mind
from body, and to supposing that mental or ``psychical''
mechanisms are different in kind from those of
bodily operations and independent of them. So deep
seated is this notion that even so ``scientific'' a theory
\hncpage{34}
as modern psycho-analysis thinks that mental habits
can be straightened out by some kind of purely psychical
manipulation without reference to the distortions
of sensation and perception which are due to bad bodily
sets. The other side of the error is found in the notion
of ``scientific'' nerve physiologists that it is only necessary
to locate a particular diseased cell or local lesion,
independent of the whole complex of organic habits, in
order to rectify conduct.

Means are means; they are intermediates, middle
terms. To grasp this fact is to have done with the
ordinary dualism of means and ends. The ``end'' is
merely a series of acts viewed at a remote stage; and
a means is merely the series viewed at an earlier one.
The distinction of means and end arises in surveying
the \emph{course} of a proposed \emph{line} of action, a connected
series in time. The ``end'' is the last act thought of;
the means are the acts to be performed prior to it in
time. To \emph{reach} an end we must take our mind off from
it and attend to the act which is next to be performed.
We must make that the end. The only exception to
this statement is in cases where customary habit determines
the course of the series. Then all that is
wanted is a cue to set it off. But when the proposed
end involves any deviation from usual action, or any
rectification of it---as in the case of standing straight---then
the main thing is to find some act which is different
from the usual one. The discovery and performance
of this unaccustomed act is the ``end'' to
which we must devote all attention. Otherwise we shall
\hncpage{35}
simply do the old thing over again, no matter what is
our conscious command. The only way of accomplishing
this discovery is through a flank movement. We
must stop even thinking of standing up straight. To
think of it is fatal, for it commits us to the operation of
an established habit of standing wrong. We must find
an act within our power which is disconnected from any
thought about standing. We must start to do another
thing which on one side inhibits our falling into the
customary bad position and on the other side is the
beginning of a series of acts which may lead into the
correct posture.\footnote{The technique of this process is stated in the book of Mr. Alexander already referred to, and the theoretical statement given is borrowed from Mr. Alexander's analysis.} The hard-drinker who keeps thinking
of not drinking is doing what he can to initiate the
acts which lead to drinking. He is starting with the
stimulus to his habit. To succeed he must find some
positive interest or line of action which will inhibit the
drinking series and which by instituting another course
of action will bring him to his desired end. In short,
the man's true aim is to discover some course of action,
having nothing to do with the habit of drink or standing
erect, which will take him where he wants to go.
The discovery of this other series is at once his means
and his end. Until one takes intermediate acts seriously
enough to treat them as ends, one wastes one's
time in any effort at change of habits. Of the intermediate
acts, the most important is the \emph{next} one. The
first or earliest means is the most important \emph{end} to
discover.

\hncpage{36}
Means and ends are two names for the same reality.
The terms denote not a division in reality but a distinction
in judgment. Without understanding this fact
we cannot understand the nature of habits nor can we
pass beyond the usual separation of the moral and
non-moral in conduct. ``End'' is a name for a series
of acts taken collectively---like the term army.
``Means'' is a name for the same series taken distributively---like
this soldier, that officer. To think of the
end signifies to extend and enlarge our view of the act
to be performed. It means to look at the next act in
perspective, not permitting it to occupy the entire field
of vision. To bear the end in mind signifies that we
should not stop thinking about our \emph{next} act until we
form some reasonably clear idea of the \emph{course} of action
to which it commits us. To attain a remote end means
on the other hand to treat the end as a series of means.
To say that an end is remote or distant, to say in fact
that it is an end at all, is equivalent to saying that
obstacles intervene between us and it. If, however, it
remains a distant end, it becomes a \emph{mere} end, that is a
dream. As soon as we have projected it, we must begin
to work backward in thought. We must change \emph{what}
is to be done into a \emph{how}, the means whereby. The
end thus re-appears as a series of ``what nexts,'' and the
what next of chief importance is the one nearest the
present state of the one acting. Only as the end is
converted into means is it definitely conceived, or intellectually
defined, to say nothing of being executable.
Just as end, it is vague, cloudy, impressionistic. We
\hncpage{37}
do not \emph{know} what we are really after until a \emph{course} of
action is mentally worked out. Aladdin with his lamp
could dispense with translating ends into means, but no
one else can do so.

Now the thing which is closest to us, the means
within our power, is a habit. Some habit impeded by
circumstances is the source of the projection of the end.
It is also the primary means in its realization. The
habit is propulsive and moves anyway toward some end,
or result, whether it is projected as an end-in-view or
not. The man who can walk does walk; the man who
can talk does converse---if only with himself. How is
this statement to be reconciled with the fact that we
are not always walking and talking; that our habits
seem so often to be latent, inoperative? Such inactivity
holds only of \emph{overt}, visibly obvious operation. In
actuality each habit operates all the time of waking
life; though like a member of a crew taking his turn
at the wheel, its operation becomes the dominantly
characteristic trait of an act only occasionally or
rarely.

The habit of walking is expressed in what a man
sees when he keeps still, even in dreams. The recognition
of distances and directions of things from his
place at rest is the obvious proof of this statement.
The habit of locomotion is latent in the sense that it is
covered up, counteracted, by a habit of seeing which is
definitely at the fore. But counteraction is not suppression.
Locomotion is a potential energy, not in
any metaphysical sense, but in the physical sense in
\hncpage{38}
which potential energy as well as kinetic has to be taken
account of in any scientific description. Everything
that a man who has the habit of locomotion does and
thinks he does and thinks differently on that account.
This fact is recognized in current psychology, but is
falsified into an association of sensations. Were it not
for the continued operation of all habits in every act,
no such thing as character could exist. There would
be simply a bundle, an untied bundle at that, of isolated
acts. Character is the interpenetration of habits. If
each habit existed in an insulated compartment and
operated without affecting or being affected by others,
character would not exist. That is, conduct would lack
unity being only a juxtaposition of disconnected reactions
to separated situations. But since environments
overlap, since situations are continuous and those remote
from one another contain like elements, a continuous
modification of habits by one another is constantly
going on. A man may give himself away in a look or
a gesture. Character can be read through the medium
of individual acts.

Of course interpenetration is never total. It is most
marked in what we call strong characters. Integration
is an achievement rather than a datum. A weak, unstable,
vacillating character is one in which different
habits alternate with one another rather than embody
one another. The strength, solidity of a habit is not
its own possession but is due to reinforcement by the
force of other habits which it absorbs into itself.
Routine specialization always works against interpenetration.
\hncpage{39}
Men with ``pigeon-hole'' minds are not infrequent.
Their diverse standards and methods of
judgment for scientific, religious, political matters testify
to isolated compartmental habits of action. Character
that is unable to undergo successfully the strain
of thought and effort required to bring competing
tendencies into a unity, builds up barriers between
different systems of likes and dislikes. The emotional
stress incident to conflict is avoided not by readjustment
but by effort at confinement. Yet the exception
proves the rule. Such persons are successful in keeping
different ways of reacting apart from one another in
consciousness rather than in action. Their character
is marked by stigmata resulting from this division.

The mutual modification of habits by one another
enables us to define the nature of the moral situation.
It is not necessary nor advisable to be always considering
the interaction of habits with one another, that
is to say the effect of a particular habit upon character---which
is a name for the total interaction. Such
consideration distracts attention from the problem of
building up an effective habit. A man who is learning
French, or chess-playing or engineering has his hands
full with his particular occupation. He would be confused
and hampered by constant inquiry into its effect
upon character. He would resemble the centipede who
by trying to think of the movement of each leg in relation
to all the others was rendered unable to travel.
At any given time, certain habits must be taken for
granted as a matter of course. Their operation is not
\hncpage{40}
a matter of moral judgment. They are treated as
technical, recreational, professional, hygienic or economic
or esthetic rather than moral. To lug in morals,
or ulterior effect on character at every point, is to
cultivate moral valetudinarianism or priggish posing.
Nevertheless any act, even that one which passes ordinarily
as trivial, may entail such consequences for habit
and character as upon occasion to require judgment
from the standpoint of the whole body of conduct. It
then comes under moral scrutiny. To know when to
leave acts without distinctive moral judgment and
when to subject them to it is itself a large factor in
morality. The serious matter is that this relative
pragmatic, or intellectual, distinction between the moral
and non-moral, has been solidified into a fixed and absolute
distinction, so that some acts are popularly regarded
as forever within and others forever without the
moral domain. From this fatal error recognition of the
relations of one habit to others preserves us. For it
makes us see that character is the name given to the
working interaction of habits, and that the cumulative
effect of insensible modifications worked by a particular
habit in the body of preferences may at any moment
require attention.

The word habit may seem twisted somewhat from
its customary use when employed as we have been using
it. But we need a word to express that kind of human
activity which is influenced by prior activity and in
that sense acquired; which contains within itself a certain
ordering or systematization of minor elements of
\hncpage{41}
action; which is projective, dynamic in quality, ready
for overt manifestation; and which is operative in some
subdued subordinate form even when not obviously
dominating activity. Habit even in its ordinary usage
comes nearer to denoting these facts than any other
word. If the facts are recognized we may also use the
words attitude and disposition. But unless we have
first made clear to ourselves the facts which have been
set forth under the name of habit, these words are more
likely to be misleading than is the word habit. For the
latter conveys explicitly the sense of operativeness,
actuality. Attitude and, as ordinarily used, disposition
suggest something latent, potential, something which
requires a positive stimulus outside themselves to become
active. If we perceive that they denote positive
forms of action which are released merely through
removal of some counteracting ``inhibitory'' tendency,
and then become overt, we may employ them instead of
the word habit to denote subdued, non-patent forms of
the latter.

In this case, we must bear in mind that the word
disposition means predisposition, readiness to act
overtly in a specific fashion whenever opportunity is
presented, this opportunity consisting in removal of
the pressure due to the dominance of some overt habit;
and that attitude means some special case of a predisposition,
the disposition waiting as it were to spring
through an opened door. While it is admitted that the
word habit has been used in a somewhat broader sense
than is usual, we must protest against the tendency in
\hncpage{42}
psychological literature to limit its meaning to repetition.
This usage is much less in accord with popular
usage than is the wider way in which we have used the
word. It assumes from the start the identity of habit
with routine. Repetition is in no sense the essence of
habit. Tendency to repeat acts is an incident of many
habits but not of all. A man with the habit of giving
way to anger may show his habit by a murderous attack
upon some one who has offended. His act is nonetheless
due to habit because it occurs only once in his life.
The essence of habit is an acquired predisposition to
\emph{ways} or modes of response, not to particular acts except
as, under special conditions, these express a way
of behaving. Habit means special sensitiveness or accessibility
to certain classes of stimuli, standing predilections
and aversions, rather than bare recurrence of
specific acts. It means will.

\bigskip
\centerline{\ldots}

\section*{Part Three: The Place of Intelligence in Conduct}
\hncpage{172}

\subsection*{I}

In discussing habit and impulse we have repeatedly
met topics where reference to the work of thought was
imperative. Explicit consideration of the place and
office of intelligence in conduct can hardly begin otherwise
than by gathering together these incidental references
and reaffirming their significance. The stimulation
of reflective imagination by impulse, its dependence
upon established habits, and its effect in transforming
habit and regulating impulse forms, accordingly,
our first theme.

Habits are conditions of intellectual efficiency. They
operate in two ways upon intellect. Obviously, they
restrict its reach, they fix its boundaries. They are
blinders that confine the eyes of mind to the road ahead.
They prevent thought from straying away from its imminent
occupation to a landscape more varied and
picturesque but irrelevant to practice. Outside the
scope of habits, thought works gropingly, fumbling in
confused uncertainty; and yet habit made complete in
routine shuts in thought so effectually that it is no
longer needed or possible. The routineer's road is a
\hncpage{173}
ditch out of which he cannot get, whose sides enclose
him, directing his course so thoroughly that he no
longer thinks of his path or his destination. All habit-forming
involves the beginning of an intellectual specialization
which if unchecked ends in thoughtless
action.

Significantly enough this fullblown result is called
absentmindedness. Stimulus and response are mechanically
linked together in an unbroken chain. Each successive
act facilely evoked by its predecessor pushes us
automatically into the next act of a predetermined series.
Only a signal flag of distress recalls consciousness
to the task of carrying on. Fortunately nature which
beckons us to this path of least resistance also puts
obstacles in the way of our complete acceptance of its
invitation. Success in achieving a ruthless and dull
efficiency of action is thwarted by untoward circumstance.
The most skilful aptitude bumps at times into
the unexpected, and so gets into trouble from which
only observation and invention extricate it. Efficiency
in following a beaten path has then to be converted
into breaking a new road through strange lands.

Nevertheless what in effect is love of ease has masqueraded
morally as love of perfection. A goal of finished
accomplishment has been set up which if it were
attained would mean only mindless action. It has been
called complete and free activity when in truth it is
only a treadmill activity or marching in one place. The
practical impossibility of reaching, in an all around
way and all at once such a ``perfection'' has been recognized.
\hncpage{174}
But such a goal has nevertheless been conceived
as the ideal, and progress has been defined as
approximation to it. Under diverse intellectual skies
the ideal has assumed diverse forms and colors. But
all of them have involved the conception of a completed
activity, a static perfection. Desire and need have been
treated as signs of deficiency, and endeavor as proof
not of power but of incompletion.

In Aristotle this conception of an end which exhausts
all realization and excludes all potentiality appears
as a definition of the highest excellence. It of
necessity excludes all want and struggle and all dependencies.
It is neither practical nor social. Nothing
is left but a self-revolving, self-sufficing thought
engaged in contemplating its own sufficiency. Some
forms of Oriental morals have united this logic with a
profounder psychology, and have seen that the final
terminus on this road is Nirvana, an obliteration of
all thought and desire. In medieval science, the ideal
reappeared as a definition of heavenly bliss accessible
only to a redeemed immortal soul. Herbert Spencer
is far enough away from Aristotle, medieval Christianity
and Buddhism; but the idea re-emerges in his conception
of a goal of evolution in which adaptation of
organism to environment is complete and final. In
popular thought, the conception lives in the vague
thought of a remote state of attainment in which we
shall be beyond ``temptation,'' and in which virtue
by its own inertia will persist as a triumphant consummation.
Even Kant who begins with a complete scorn
\hncpage{175}
for happiness ends with an ``ideal'' of the eternal and
undisturbed union of virtue and joy, though in his
case nothing but a symbolic approximation is admitted
to be feasible.

The fallacy in these versions of the same idea is
perhaps the most pervasive of all fallacies in philosophy.
So common is it that one questions whether it
might not be called \emph{the} philosophical fallacy. It consists
in the supposition that whatever is found true
under certain conditions may forthwith be asserted universally
or without limits and conditions. Because a
thirsty man gets satisfaction in drinking water, bliss
consists in being drowned. Because the success of any
particular struggle is measured by reaching a point of
frictionless action, therefore there is such a thing as an
all-inclusive end of effortless smooth activity endlessly
maintained. It is forgotten that success is success \emph{of}
a specific effort, and satisfaction the fulfilment \emph{of} a
specific demand, so that success and satisfaction become
meaningless when severed from the wants and
struggles whose consummations they are, or when
taken universally. The philosophy of Nirvana comes
the closest to admission of this fact, but even it holds
Nirvana to be desirable.

Habit is however more than a restriction of thought.
Habits become negative limits because they are first
positive agencies. The more numerous our habits the
wider the field of possible observation and foretelling.
The more flexible they are, the more refined is perception
in its discrimination and the more delicate the presentation
\hncpage{176}
evoked by imagination. The sailor is intellectually
at home on the sea, the hunter in the forest,
the painter in his studio, the man of science in his laboratory.
These commonplaces are universally recognized
in the concrete; but their significance is obscured
and their truth denied in the current general theory
of mind. For they mean nothing more or less than
that habits formed in process of exercising biological
aptitudes are the sole agents of observation, recollection,
foresight and judgment: a mind or consciousness
or soul in general which performs these operations is
a myth.

The doctrine of a single, simple and indissoluble soul
was the cause and the effect of failure to recognize that
concrete habits are the means of knowledge and
thought. Many who think themselves scientifically
emancipated and who freely advertise the soul for a
superstition, perpetuate a false notion of what knows,
that is, of a separate knower. Nowadays they usually
fix upon consciousness in general, as a stream or process
or entity; or else, more specifically upon sensations and
images as the tools of intellect. Or sometimes they
think they have scaled the last heights of realism by
adverting grandiosely to a formal knower in general
who serves as one term in the knowing relation;
by dismissing psychology as irrelevant to knowledge
and logic, they think to conceal the psychological monster
they have conjured up.

Now it is dogmatically stated that no such conceptions
of the seat, agent or vehicle will go psychologically
\hncpage{177}
at the present time. Concrete habits do all the
perceiving, recognizing, imagining, recalling, judging,
conceiving and reasoning that is done. ``Consciousness,''
whether as a stream or as special sensations and
images, expresses functions of habits, phenomena of
their formation, operation, their interruption and reorganization.

Yet habit does not, of itself, know, for it does not
of itself stop to think, observe or remember. Neither
does impulse of itself engage in reflection or contemplation.
It just lets go. Habits by themselves are too
organized, too insistent and determinate to need to
indulge in inquiry or imagination. And impulses are
too chaotic, tumultuous and confused to be able to
know even if they wanted to. Habit as such is too
definitely adapted to an environment to survey or analyze
it, and impulse is too indeterminately related to
the environment to be capable of reporting anything
about it. Habit incorporates, enacts or overrides objects,
but it doesn't know them. Impulse scatters and
obliterates them with its restless stir. A certain delicate
combination of habit and impulse is requisite for
observation, memory and judgment. Knowledge which
is not projected against the black unknown lives in the
muscles, not in consciousness.

We may, indeed, be said to \emph{know how} by means of our
habits. And a sensible intimation of the practical function
of knowledge has led men to identify all acquired
practical skill, or even the instinct of animals, with
knowledge. We walk and read aloud, we get off and
\hncpage{178}
on street cars, we dress and undress, and do a thousand
useful acts without thinking of them. We know something,
namely, how to do them. Bergson's philosophy
of intuition is hardly more than an elaborately documented
commentary on the popular conception that by
instinct a bird knows how to build a nest and a spider
to weave a web. But after all, this practical work
done by habit and instinct in securing prompt and exact
adjustment to the environment is not knowledge, except
by courtesy. Or, if we choose to call it knowledge---and
no one has the right to issue an ukase to the contrary---then
other things also called knowledge, knowledge
\emph{of} and \emph{about} things, knowledge \emph{that} things are
thus and so, knowledge that involves reflection and conscious
appreciation, remains of a different sort, unaccounted
for and undescribed.

For it is a commonplace that the more suavely efficient
a habit the more unconsciously it operates. Only
a hitch in its workings occasions emotion and provokes
thought. Carlyle and Rousseau, hostile in temperament
and outlook, yet agree in looking at consciousness
as a kind of disease, since we have no consciousness
of bodily or mental organs as long as they work at ease
in perfect health. The idea of disease is, however, aside
from the point, unless we are pessimistic enough to
regard every slip in total adjustment of a person to its
surroundings as something abnormal---a point of view
which once more would identify well-being with perfect
automatism. The truth is that in every waking moment,
the complete balance of the organism and its
\hncpage{179}
environment is constantly interfered with and as constantly
restored. Hence the ``stream of consciousness''
in general, and in particular that phase of it celebrated
by William James as alternation of flights and
perchings. Life is interruptions and recoveries. Continuous
interruption is not possible in the activities
of an individual. Absence of perfect equilibrium is not
equivalent to a complete crushing of organized activity.
When the disturbance amounts to such a pitch
as that, the self goes to pieces. It is like shell-shock.
Normally, the environment remains sufficiently in harmony
with the body of organized activities to sustain
most of them in active function. But a novel factor
in the surroundings releases some impulse which tends
to initiate a different and incompatible activity, to
bring about a redistribution of the elements of organized
activity between those have been respectively
central and subsidiary. Thus the hand guided by the
eye moves toward a surface. Visual quality is the dominant
element. The hand comes in contact with an
object. The eye does not cease to operate but some
unexpected quality of touch, a voluptuous smoothness
or annoying heat, compels a readjustment in which the
touching, handling activity strives to dominate the action.
Now at these moments of a shifting in activity
conscious feeling and thought arise and are accentuated.
The disturbed adjustment of organism and environment
is reflected in a temporary strife which concludes
in a coming to terms of the old habit and the new
impulse.

\hncpage{180}
In this period of redistribution impulse determines
the direction of movement. It furnishes the focus about
which reorganization swirls. Our attention in short is
always directed forward to bring to notice something
which is imminent but which as yet escapes us. Impulse
defines the peering, the search, the inquiry. It is, in
logical language, the movement into the unknown, not
into the immense inane of the unknown at large, but into
that special unknown which when it is hit upon restores
an ordered, unified action. During this search, old
habit supplies content, filling, definite, recognizable,
subject-matter. It begins as vague presentiment of
what we are going towards. As organized habits are
definitely deployed and focused, the confused situation
takes on form, it is ``cleared up''---the essential function
of intelligence. Processes become objects. Without
habit there is only irritation and confused hesitation.
With habit alone there is a machine-like repetition,
a duplicating recurrence of old acts. With conflict
of habits and release of impulse there is conscious
search.


\subsection*{II}
\hncpage{181}

We are going far afield from any direct moral issue.
But the problem of the place of knowledge and judgment
in conduct depends upon getting the fundamental
psychology of thought straightened out. So the excursion
must be continued. We compare life to a traveler
faring forth. We may consider him first at a
moment where his activity is confident, straightforward,
organized. He marches on giving no direct attention to
his path, nor thinking of his destination. Abruptly he
is pulled up, arrested. Something is going wrong in
his activity. From the standpoint of an onlooker, he
has met an obstacle which must be overcome before his
behavior can be unified into a successful ongoing. From
his own standpoint, there is shock, confusion, perturbation,
uncertainty. For the moment he doesn't know
what hit him, as we say, nor where he is going. But
a new impulse is stirred which becomes the starting
point of an investigation, a looking into things, a trying
to see them, to find out what is going on. Habits which
were interfered with begin to get a new direction as they
cluster about the impulse to look and see. The blocked
habits of locomotion give him a sense of where he \emph{was}
going, of what he had set out to do, and of the ground
already traversed. As he looks, he sees definite things
which are not just things at large but which are related
\hncpage{182}
to his course of action. The momentum of the activity
entered upon persists as a sense of direction, of aim;
it is an anticipatory project. In short, he recollects,
observes and plans.

The trinity of these forecasts, perceptions and remembrances
form a subject-matter of discriminated
and identified objects. These objects represent habits
turned inside out. They exhibit both the onward tendency
of habit and the objective conditions which have
been incorporated within it. Sensations in immediate
consciousness are elements of action dislocated through
the shock of interruption. They never, however, completely
monopolize the scene; for there is a body of
residual undisturbed habits which is reflected in remembered
and perceived objects having a meaning. Thus
out of shock and puzzlement there gradually emerges a
figured framework of objects, past, present, future.
These shade off variously into a vast penumbra of
vague, unfigured things, a setting which is taken for
granted and not at all explicitly presented. The complexity
of the figured scene in its scope and refinement
of contents depends wholly upon prior habits and their
organization. The reason a baby can know little and
an experienced adult know much when confronting the
same things is not because the latter has a ``mind''
which the former has not, but because one has already
formed habits which the other has still to acquire. The
scientific man and the philosopher like the carpenter,
the physician and politician know with their habits not
with their ``consciousness.'' The latter is eventual, not
\hncpage{183}
a source. Its occurrence marks a peculiarly delicate
connection between highly organized habits and unorganized
impulses. Its contents or objects, observed,
recollected, projected and generalized into principles,
represent the incorporated material of habits coming
to the surface, because habits are disintegrating at the
touch of conflicting impulses. But they also gather
themselves together to comprehend impulse and make
it effective.

This account is more or less strange as psychology
but certain aspects of it are commonplaces in a static
logical formulation. It is, for example, almost a truism
that knowledge is both synthetic and analytic; a set of
discriminated elements connected by relations. This
combination of opposite factors of unity and difference,
elements and relations, has been a standing paradox and
mystery of the theory of knowledge. It will remain so
until we connect the theory of knowledge with an empirically
verifiable theory of behavior. The steps of
this connection have been sketched and we may enumerate
them. We know at such times as habits are
impeded, when a conflict is set up in which impulse is
released. So far as this impulse sets up a definite forward
tendency it constitutes the forward, prospective
character of knowledge. In this phase unity or synthesis
is found. We are striving to unify our responses,
to achieve a consistent environment which will restore
unity of conduct. Unity, relations, are prospective;
they mark out lines converging to a focus. They are
``ideal.'' But \emph{what} we know, the objects that present
\hncpage{184}
themselves with definiteness and assurance, are retrospective;
they are the conditions which have been mastered,
incorporated in the past. They are elements,
discriminated, analytic just because old habits so far
as they are checked are also broken into objects which
define the obstruction of ongoing activity. They are
``real,'' not ideal. Unity is something sought; split,
division is something given, at hand. Were we to carry
the same psychology into detail we should come upon
the explanation of perceived particulars and conceived
universals, of the relation of discovery and proof, induction
and deduction, the discrete and the continuous.
Anything approaching an adequate discussion is too
technical to be here in place. But the main point,
however technical and abstract it may be in statement,
is of far reaching importance for everything concerned
with moral beliefs, conscience and judgments of right
and wrong.

The most general, if vaguest issue, concerns the nature
of the organ of moral knowledge. As long as
knowledge in general is thought to be the work of a
special agent, whether soul, consciousness, intellect or
a knower in general, there is a logical propulsion towards
postulating a special agent for knowledge of
moral distinctions. Consciousness and conscience have
more than a verbal connection. If the former is something
in itself, a seat or power which antecedes intellectual
functions, why should not the latter be also a
unique faculty with its own separate jurisdiction? If
reason in general is independent of empirically verifiable
\hncpage{185}
realities of human nature, such as instincts and
organized habits, why should there not also exist a
moral or practical reason independent of natural operations?
On the other hand if it is recognized that
knowing is carried on through the medium of natural
factors, the assumption of special agencies for moral
knowing becomes outlawed and incredible. Now the
matter of the existence or non-existence of such special
agencies is no technically remote matter. The belief
in a separate organ involves belief in a separate and
independent subject-matter. The question fundamentally
at issue is nothing more or less than whether
moral values, regulations, principles and objects form
a separate and independent domain or whether they are
part and parcel of a normal development of a life
process.

These considerations explain why the denial of a
separate organ of knowledge, of a separate instinct or
impulse toward knowing, is not the wilful philistinism
it is sometimes alleged to be. There is of course a sense
in which there is a distinctive impulse, or rather habitual
disposition, to know. But in the same sense there
is an impulse to aviate, to run a typewriter or write
stories for magazines. Some activities result in knowledge,
as others result in these other things. The result
may be so important as to induce distinctive attention to
the activities in order to foster them. From an incident,
almost a by-product, attainment of truth, physical, social,
moral, may become the leading characteristic of
some activities. Under such circumstances, they become
\hncpage{186}
transformed. Knowing is then a distinctive activity,
with its own ends and its peculiarly adapted processes.
All this is a matter of course. Having hit
upon knowledge accidentally, as it were, and the product
being liked and its importance noted, knowledge-getting
becomes, upon occasion, a definite occupation.
And education confirms the disposition, as it may confirm
that of a musician or carpenter or tennis-player.
But there is no more an original separate impulse
or power in one case than in the other. Every
habit is impulsive, that is projective, urgent, and the
habit of knowing is no exception.

The reason for insisting on this fact is not failure
to appreciate the distinctive value of knowledge when
once it comes into existence. This value is so immense
it may be called unique. The aim of the discussion is
not to subordinate knowing to some hard, prosaic utilitarian
end. The reason for insistence upon the derivative
position of knowing in activity, roots in a sense for
fact, and in a realization that the doctrine of a separate
original power and impulse of knowledge cuts
knowledge off from other phases of human nature, and
results in its non-natural treatment. The isolation of
intellectual disposition from concrete empirical facts
of biological impulse and habit-formation entails a denial
of the continuity of mind with nature. Aristotle
asserted that the faculty of pure knowing enters a man
from without as through a door. Many since his day
have asserted that knowing and doing have no intrinsic
connection with each other. Reason is asserted to have
\hncpage{187}
no responsibility to experience; conscience is said to be
a sublime oracle independent of education and social influences.
All of these views follow naturally from a
failure to recognize that all knowing, judgment, belief
represent an acquired result of the workings of natural
impulses in connection with environment.

Upon the ethical side, as has been intimated, the matter
at issue concerns the nature of conscience. Conscience
has been asserted by orthodox moralists to be
unique in origin and subject-matter. The same view is
embodied by implication in all those popular methods
of moral training which attempt to fix rigid authoritative
notions of right and wrong by disconnecting moral
judgments from the aids and tests which are used in
other forms of knowledge. Thus it has been asserted
that conscience is an original faculty of illumination
which (if it has not been dimmed by indulgence in sin)
shines upon moral truths and objects and reveals them
without effort for precisely what they are. Those who
hold this view differ enormously among themselves as
to the nature of the objects of conscience. Some hold
them to be general principles, others individual acts,
others the order of worth among motives, others the
sense of duty in general, others the unqualified authority
of right. Still others carry the implied logic of
authority to conclusion, and identify knowledge of
moral truths with a divine supernatural revelation of a
code of commandments.

But among these diversities there is agreement about
one fundamental. There must be a separate non-natural
\hncpage{188}
faculty of moral knowledge because the things
to be known, the matters of right and wrong, good and
evil, obligation and responsibility, form a separate domain,
separate that is from that of ordinary action in
its usual human and social significance. The latter activities
may be prudential, political, scientific, economic.
But, from the standpoint of these theories, they have
no moral meaning until they are brought under the
purview of this separate unique department of our
nature. It thus turns out that the so-called intuitional
theories of moral knowledge concentrate in themselves
all the ideas which are subject to criticism in these
pages: Namely, the assertion that morality is distinct
in origin, working and destiny from the natural structure
and career of human nature. This fact is the excuse,
if excuse be desired, for a seemingly technical
excursion that links intellectual activity with the conjoint
operation of habit and impulse.

\bigskip
\centerline{\ldots}

\section*{Part Four: Conclusion}
\hncpage{278}

Conduct when distributed under heads like habit, impulse
and intelligence gets artificially shredded. In
discussing each of these topics we have run into the
others. We conclude, then, with an attempt to gather
together some outstanding considerations about conduct
as a whole.

\subsection*{I}

The foremost conclusion is that morals has to do
with all activity into which alternative possibilities
enter. For wherever they enter a difference between
better and worse arises. Reflection upon action means
uncertainty and consequent need of decision as to which
course is better. The better is the good; the best is
not better than the good but is simply the discovered
good. Comparative and superlative degrees are only
paths to the positive degree of action. The worse or
evil is a rejected good. In deliberation and before
choice no evil presents itself as evil. Until it is rejected,
it is a competing good. After rejection, it figures not
as a lesser good, but as the bad of that situation.

\hncpage{279}
Actually then only deliberate action, conduct into
which reflective choice enters, is distinctively moral, for
only then does there enter the question of better and
worse. Yet it is a perilous error to draw a hard and
fast line between action into which deliberation and
choice enter and activity due to impulse and matter-of-fact
habit. One of the consequences of action is to involve
us in predicaments where we have to reflect upon
things formerly done as matter of course. One of the
chief problems of our dealings with others is to induce
them to reflect upon affairs which they usually perform
from unreflective habit. On the other hand, every reflective
choice tends to relegate some conscious issue
into a deed or habit henceforth taken for granted and
not thought upon. Potentially therefore every and
any act is within the scope of morals, being a candidate
for possible judgment with respect to its better-or-worse
quality. It thus becomes one of the most perplexing
problems of reflection to discover just how far
to carry it, what to bring under examination and what
to leave to unscrutinized habit. Because there is no
final recipe by which to decide this question all moral
judgment is experimental and subject to revision by its
issue.

The recognition that conduct covers every act that
is judged with reference to better and worse and that
the need of this judgment is potentially coextensive
with all portions of conduct, saves us from the mistake
which makes morality a separate department of life.
Potentially conduct is one hundred per cent of our acts.
\hncpage{280}
Hence we must decline to admit theories which identify
morals with the purification of motives, edifying character,
pursuing remote and elusive perfection, obeying
supernatural command, acknowledging the authority of
duty. Such notions have a dual bad effect. First they
get in the way of observation of conditions and consequences.
They divert thought into side issues. Secondly,
while they confer a morbid exaggerated quality
upon things which are viewed under the aspect of morality,
they release the larger part of the acts of life
from serious, that is moral, survey. Anxious solicitude
for the few acts which are deemed moral is accompanied
by edicts of exemption and baths of immunity for most
acts. A moral moratorium prevails for everyday
affairs.

When we observe that morals is at home wherever
considerations of the worse and better are involved, we
are committed to noting that morality is a continuing
process not a fixed achievement. Morals means growth
of conduct in meaning; at least it means that kind of
expansion in meaning which is consequent upon observations
of the conditions and outcome of conduct. It
is all one with growing. Growing and growth are the
same fact expanded in actuality or telescoped in
thought. In the largest sense of the word, morals is
education. It is learning the meaning of what we are
about and employing that meaning in action. The
good, satisfaction, ``end,'' of growth of present action
in shades and scope of meaning is the only good within
our control, and the only one, accordingly, for which
\hncpage{281}
responsibility exists. The rest is luck, fortune. And
the tragedy of the moral notions most insisted upon by
the morally self-conscious is the relegation of the only
good which can fully engage thought, namely present
meaning of action, to the rank of an incident of a remote
good, whether that future good be defined as
pleasure, or perfection, or salvation, or attainment of
virtuous character.

``Present'' activity is not a sharp narrow knife-blade
in time. The present is complex, containing
within itself a multitude of habits and impulses. It is
enduring, a course of action, a process including memory,
observation and foresight, a pressure forward, a
glance backward and a look outward. It is of \emph{moral}
moment because it marks a transition in the direction
of breadth and clarity of action or in that of triviality
and confusion. Progress is present reconstruction adding
fullness and distinctness of meaning, and retrogression
is a present slipping away of significance, determinations,
grasp. Those who hold that progress can
be perceived and measured only by reference to a remote
goal, first confuse meaning with space, and then treat
spatial position as absolute, as limiting movement instead
of being bounded in and by movement. There are
plenty of negative elements, due to conflict, entanglement
and obscurity, in most of the situations of life,
and we do not require a revelation of some supreme
perfection to inform us whether or no we are making
headway in present rectification. We move on from
the worse and into, not just towards, the better, which
\hncpage{282}
is authenticated not by comparison with the foreign but
in what is indigenous. Unless progress is a present
reconstructing, it is nothing; if it cannot be told by
qualities belonging to the movement of transition it
can never be judged.

Men have constructed a strange dream-world when
they have supposed that without a fixed ideal of a remote
good to inspire them, they have no inducement to
get relief from present troubles, no desires for liberation
from what oppresses and for clearing-up what
confuses present action. The world in which we could
get enlightenment and instruction about the direction
in which we are moving only from a vague conception of
an unattainable perfection would be totally unlike our
present world. Sufficient unto the day is the evil
thereof. Sufficient it is to stimulate us to remedial
action, to endeavor in order to convert strife into harmony,
monotony into a variegated scene, and limitation
into expansion. The converting is progress, the only
progress conceivable or attainable by man. Hence
every situation has its own measure and quality of
progress, and the need for progress is recurrent, constant.
If it is better to travel than to arrive, it is because
traveling is a constant arriving, while arrival
that precludes further traveling is most easily attained
by going to sleep or dying. We find our clews to direction
in the projected recollections of definite experienced
goods not in vague anticipations, even when
we label the vagueness perfection, the Ideal, and proceed
to manipulate its definition with dry dialectic logic.
\hncpage{283}
Progress means increase of present meaning, which involves
multiplication of sensed distinctions as well as
harmony, unification. This statement may, perhaps, be
made generally, in application to the experience of
humanity. If history shows progress it can hardly be
found elsewhere than in this complication and extension
of the significance found within experience. It is clear
that such progress brings no surcease, no immunity
from perplexity and trouble. If we wished to transmute
this generalization into a categorical imperative
we should say: ``So act as to increase the meaning of
present experience.'' But even then in order to get instruction
about the concrete quality of such increased
meaning we should have to run away from the law and
study the needs and alternative possibilities lying within
a unique and localized situation. The imperative,
like everything absolute, is sterile. Till men give up
the search for a general formula of progress they will
not know where to look to find it.

A business man proceeds by comparing today's liabilities
and assets with yesterday's, and projects plans
for tomorrow by a study of the movement thus indicated
in conjunction with study of the conditions of
the environment now existing. It is not otherwise with
the business of living. The future is a projection of the
subject-matter of the present, a projection which is not
arbitrary in the extent in which it divines the movement
of the moving present. The physician is lost who would
guide his activities of healing by building up a picture
of perfect health, the same for all and in its nature
\hncpage{284}
complete and self-enclosed once for all. He employs
what he has discovered about actual cases of good
health and ill health and their causes to investigate the
present ailing individual, so as to further his recovering;
recovering, an intrinsic and living process rather
than recovery, which is comparative and static. Moral
theories, which however have not remained mere theories
but which have found their way into the opinions of
the common man, have reversed the situation and made
the present subservient to a rigid yet abstract future.

The ethical import of the doctrine of evolution is
enormous. But its import has been misconstrued because
the doctrine has been appropriated by the very
traditional notions which in truth it subverts. It has
been thought that the doctrine of evolution means the
complete subordination of present change to a future
goal. It has been constrained to teach a futile dogma
of approximation, instead of a gospel of present
growth. The usufruct of the new science has been
seized upon by the old tradition of fixed and external
ends. In fact evolution means continuity of change;
and the fact that change may take the form of present
growth of complexity and interaction. Significant
stages in change are found not in access of fixity of
attainment but in those crises in which a seeming fixity
of habits gives way to a release of capacities that have
not previously functioned: in times that is of readjustment
and redirection.

No matter what the present success in straightening
out difficulties and harmonizing conflicts, it is certain
\hncpage{285}
that problems will recur in the future in a new form
or on a different plane. Indeed every genuine accomplishment
instead of winding up an affair and enclosing
it as a jewel in a casket for future contemplation,
complicates the practical situation. It effects a new
distribution of energies which have henceforth to be
employed in ways for which past experience gives no
exact instruction. Every important satisfaction of an
old want creates a new one; and this new one has to
enter upon an experimental adventure to find its satisfaction.
From the side of what has gone before
achievement settles something. From the side of what
comes after, it complicates, introducing new problems,
unsettling factors. There is something pitifully juvenile
in the idea that ``evolution,'' progress, means a
definite sum of accomplishment which will forever stay
done, and which by an exact amount lessens the amount
still to be done, disposing once and for all of just so
many perplexities and advancing us just so far on our
road to a final stable and unperplexed goal. Yet the
typical nineteenth century, mid-victorian conception of
evolution was precisely a formulation of such a consummate
juvenilism.

If the true ideal is that of a stable condition free
from conflict and disturbance, then there are a number
of theories whose claims are superior to those of the
popular doctrine of evolution. Logic points rather in
the direction of Rousseau and Tolstoi who would recur
to some primitive simplicity, who would return from
complicated and troubled civilization to a state of nature.
\hncpage{286}
For certainly progress in civilization has not only
meant increase in the scope and intricacy of problems
to be dealt with, but it entails increasing instability.
For in multiplying wants, instruments and possibilities,
it increases the variety of forces which enter into relations
with one another and which have to be intelligently
directed. Or again, Stoic indifference or Buddhist
calm have greater claims. For, it may be argued,
since all objective achievement only complicates the situation,
the victory of a final stability can be secured
only by renunciation of desire. Since every satisfaction
of desire increases force, and this in turn creates
new desires, withdrawal into an inner passionless state,
indifference to action and attainment, is the sole road
to possession of the eternal, stable and final reality.

Again, from the standpoint of definite approximation
to an ultimate goal, the balance falls heavily on the side
of pessimism. The more striving the more attainments,
perhaps; but also assuredly the more needs and the
more disappointments. The more we do and the more
we accomplish, the more the end is vanity and vexation.
From the standpoint of attainment of good that
stays put, that constitutes a definite sum performed
which lessens the amount of effort required in order to
reach the ultimate goal of final good, progress \emph{is} an
illusion. But we are looking for it in the wrong place.
The world war is a bitter commentary on the nineteenth
century misconception of moral achievement---a misconception
however which it only inherited from the
traditional theory of fixed ends, attempting to bolster
\hncpage{287}
up that doctrine with aid from the ``scientific'' theory
of evolution. The doctrine of progress is not yet bankrupt.
The bankruptcy of the notion of fixed goods to
be attained and stably possessed may possibly be the
means of turning the mind of man to a tenable theory
of progress---to attention to present troubles and possibilities.

Adherents of the idea that betterment, growth in
goodness, consists in approximation to an exhaustive,
stable, immutable end or good, have been compelled to
recognize the truth that in fact we envisage the good
in specific terms that are relative to existing needs, and
that the attainment of every specific good merges insensibly
into a new condition of maladjustment with its
need of a new end and a renewed effort. But they
have elaborated an ingenious dialectical theory to account
for the facts while maintaining their theory intact.
The goal, the ideal, is infinite; man is finite, subject
to conditions imposed by space and time. The
specific character of the ends which man entertains
and of the satisfaction he achieves is due therefore
precisely to his empirical and finite nature in its contrast
with the infinite and complete character of the
true reality, the end. Consequently when man reaches
what he had taken to be the destination of his journey
he finds that he has only gone a piece on the road. Infinite
vistas still stretch before him. Again he sets his
mark a little way further ahead, and again when he
reaches the station set, he finds the road opening before
him in unexpected ways, and sees new distant objects
\hncpage{288}
beckoning him forward. Such is the popular doctrine.

By some strange perversion this theory passes for
moral idealism. An office of inspiration and guidance is
attributed to the thought of the goal of ultimate completeness
or perfection. As matter of fact, the idea
sincerely held brings discouragement and despair not
inspiration or hopefulness. There is something either
ludicrous or tragic in the notion that inspiration to
continued progress is had in telling man that no matter
what he does or what he achieves, the outcome is negligible
in comparison with what he set out to achieve, that
every endeavor he makes is bound to turn out a failure
compared with what should be done, that every attained
satisfaction is only forever bound to be only a
disappointment. The honest conclusion is pessimism.
All is vexation, and the greater the effort the greater
the vexation. But the fact is that it is not the negative
aspect of an outcome, its failure to reach infinity,
which renews courage and hope. Positive attainment,
actual enrichment of meaning and powers opens new
vistas and sets new tasks, creates new aims and stimulates
new efforts. The facts are not such as to yield
unthinking optimism and consolation; for they render
it impossible to rest upon attained goods. New struggles
and failures are inevitable. The total scene of
action remains as before, only for us more complex,
and more subtly unstable. But this very situation is a
consequence of expansion, not of failures of power, and
when grasped and admitted it is a challenge to intelligence.
Instruction in what to do next can never come
\hncpage{289}
from an infinite goal, which for us is bound to be empty.
It can be derived only from study of the deficiencies,
irregularities and possibilities of the actual situation.

In any case, however, arguments about pessimism and
optimism based upon considerations regarding fixed
attainment of good and evil are mainly literary in quality.
Man continues to live because he is a living creature
not because reason convinces him of the certainty
or probability of future satisfactions and achievements.
He is instinct with activities that carry him on. Individuals
here and there cave in, and most individuals
sag, withdraw and seek refuge at this and that point.
But man as man still has the dumb pluck of the animal.
He has endurance, hope, curiosity, eagerness, love of
action. These traits belong to him by structure, not by
taking thought. Memory of past and foresight of future
convert dumbness to some degree of articulateness.
They illumine curiosity and steady courage.
Then when the future arrives with its inevitable disappointments
as well as fulfilments, and with new
sources of trouble, failure loses something of its fatality,
and suffering yields fruit of instruction not of bitterness.
Humility is more demanded at our moments
of triumph than at those of failure. For humility is
not a caddish self-depreciation. It is the sense of our
slight inability even with our best intelligence and effort
to command events; a sense of our dependence
upon forces that go their way without our wish and
plan. Its purport is not to relax effort but to make
us prize every opportunity of present growth. In
\hncpage{290}
morals, the infinitive and the imperative develop from
the participle, present tense. Perfection means perfecting,
fulfilment, fulfilling, and the good is now or
never.

Idealistic philosophies, those of Plato, Aristotle, Spinoza,
like the hypothesis now offered, have found the
good in meanings belonging to a conscious life, a life
of reason, not in external achievement. Like it, they
have exalted the place of intelligence in securing fulfilment
of conscious life. These theories have at least
not subordinated conscious life to external obedience,
not thought of virtue as something different from excellence
of life. But they set up a transcendental meaning
and reason, remote from present experience and
opposed to it; or they insist upon a special form of
meaning and consciousness to be attained by peculiar
modes of knowledge inaccessible to the common man,
involving not continuous reconstruction of ordinary
experience, but its wholesale reversal. They have
treated regeneration, change of heart, as wholesale and
self-enclosed, not as continuous.

The utilitarians also made good and evil, right and
wrong, matters of conscious experience. In addition
they brought them down to earth, to everyday experience.
They strove to humanize other-worldly goods.
But they retained the notion that the good is future,
and hence outside the meaning of present activity. In
so far it is sporadic, exceptional, subject to accident,
passive, an enjoyment not a joy, something hit upon,
not a fulfilling. The future end is for them not \emph{so}
\hncpage{291}
remote from present action as the Platonic realm of
ideals, or as the Aristotelian rational thought, or the
Christian heaven, or Spinoza's conception of the universal
whole. But still it is separate in principle and
in fact from present activity. The next step is to identify
the sought for good with the meaning of our
impulses and our habits, and the specific \emph{moral} good
or virtue with \emph{learning} this meaning, a learning that
takes us back not into an isolated self but out into the
open-air world of objects and social ties, terminating
in an increment of present significance.

Doubtless there are those who will think that we
thus escape from remote and external ends only to fall
into an Epicureanism which teaches us to subordinate
everything else to present satisfactions. The hypothesis
preferred may seem to some to advise a subjective,
self-centered life of intensified consciousness, an esthetically
dilettante type of egoism. For is not its lesson
that we should concentrate attention, each upon the
consciousness accompanying his action so as to refine
and develop it? Is not this, like all subjective morals,
an anti-social doctrine, instructing us to subordinate
the objective consequences of our acts, those which promote
the welfare of others, to an enrichment of our
private conscious lives?

It can hardly be denied that as compared with the
dogmas against which it reacted there is an element of
truth in Epicureanism. It strove to center attention
upon what is actually within control and to find the
good in the present instead of in a contingent uncertain
\hncpage{292}
future. The trouble with it lies in its account of
present good. It failed to connect this good with the
full reach of activities. It contemplated good of withdrawal
rather than of active participation. That is
to say, the objection to Epicureanism lies in its conception
of what constitutes present good, not in its
emphasis upon satisfaction as at present. The same remark
may be made about every theory which recognizes
the individual self. If any such theory is objectionable,
the objection is against the character or quality
assigned to the self. Of course an individual is the
bearer or carrier of experience. What of that? Everything
depends upon the kind of experience that centers
in him. Not the residence of experience counts, but its
contents, what's in the house. The center is not in the
abstract amenable to our control, but what gathers
about it is our affair. We can't help being individual
selves, each one of us. If selfhood as such is a bad
thing, the blame lies not with the self but with the universe,
with providence. But in fact the distinction between
a selfishness with which we find fault and an
unselfishness which we esteem is found in the quality
of the activities which proceed from and enter into the
self, according as they are contractive, exclusive, or
expansive, outreaching. Meaning exists for some self,
but this truistic fact doesn't fix the quality of any particular
meaning. It may be such as to make the self
small, or such as to exalt and dignify the self. It is
as impertinent to decry the worth of experience because
it is connected with a self as it is fantastic to
\hncpage{293}
idealize personality just as personality aside from the
question what sort of a person one is.

Other persons are selves too. If one's own present
experience is to be depreciated in its meaning because
it centers in a self, why act for the welfare of others?
Selfishness for selfishness, one is as good as another;
our own is worth as much as another's. But the recognition
that good is always found in a present growth
of significance in activity protects us from thinking
that welfare can consist in a soup-kitchen happiness,
in pleasures we can confer upon others from without.
It shows that good is the same in quality wherever it is
found, whether in some other self or in one's own. An
activity has meaning in the degree in which it establishes
and acknowledges variety and intimacy of connections.
As long as any social impulse endures, so long an activity
that shuts itself off will bring inward dissatisfaction
and entail a struggle for compensatory goods, no matter
what pleasures or external successes acclaim its
course.

To say that the welfare of others, like our own,
consists in a widening and deepening of the perceptions
that give activity its meaning, in an educative growth,
is to set forth a proposition of political import. To
``make others happy'' except through liberating their
powers and engaging them in activities that enlarge
the meaning of life is to harm them and to indulge
ourselves under cover of exercising a special virtue.
Our moral measure for estimating any existing arrangement
or any proposed reform is its effect upon
\hncpage{294}
impulse and habits. Does it liberate or suppress, ossify
or render flexible, divide or unify interest? Is perception
quickened or dulled? Is memory made apt and
extensive or narrow and diffusely irrelevant? Is imagination
diverted to fantasy and compensatory dreams,
or does it add fertility to life? Is thought creative or
pushed one side into pedantic specialisms? There is a
sense in which to set up social welfare as an end of
action only promotes an offensive condescension, a
harsh interference, or an oleaginous display of complacent
kindliness. It always tends in this direction
when it is aimed at giving happiness to others
directly, that is, as we can hand a physical thing to
another. To foster conditions that widen the horizon
of others and give them command of their own powers,
so that they can find their own happiness in their own
fashion, is the way of ``social'' action. Otherwise the
prayer of a freeman would be to be left alone, and to be
delivered, above all, from ``reformers'' and ``kind''
people.

\bigskip
\centerline{\ldots}

\end{document}