\documentclass[12pt]{article}
\usepackage{hyperref,geometry,setspace}
\usepackage[]{accessibility}
\usepackage[symbol*,perpage]{footmisc}
\geometry{width=6in,height=9in,centering}
% changes font to TeX Gyre Schola (Century Schoolbook)
\usepackage{tgschola}\usepackage[T1]{fontenc}

\begin{document}
%\renewcommand*{\thefootnote}{\fnsymbol{footnote}}
\setcounter{footnote}{0}
\hypersetup{pdfinfo={Title={Philosophical conceptions and practical results}, Author={William James}}, pdfborder = {0 0 0 0}}

\centerline{\emph{University Chronicle}}

\textsc{Vol}. I \hfill SEPTEMBER, 1898 \hfill No. 4 

\centerline{PHILOSOPHICAL CONCEPTIONS AND 
PRACTICAL RESULTS\footnote{An address delivered before the Philosophical Union, at Berkeley, August 26, 1898, by William James, M.D., LL.D., Professor of Psychology in Harvard University.}\footnote{This text was prepared from a scan of the original by P.D. Magnus. The marginal numbers indicate pages in the original.}}

\centerline{By \textsc{William James}.}


\bigskip
\setstretch{1.2}
An occasion like the present would seem to call for an absolutely untechnical discourse. I ought to speak of something connected with life rather than with logic. I ought to give a message with a practical outcome and an emotional musical accompaniment, so to speak, fitted to interest men as men, and yet also not altogether to disappoint philosophers since philosophers, let them be as queer as they will, still are men in the secret recesses of their hearts, even here at Berkeley. I ought, I say, to produce something simple enough to catch and inspire the rest of you, and yet with just enough of ingenuity and oddity about it to keep the members of the Philosophical Union from yawning and letting their attention wander away. 

I confess that I have something of this kind in my mind, a perfectly ideal discourse for the present occasion. Were I to set it down on paper, I verily believe it would be regarded by everyone as the final word of philosophy. It would bring theory down to a single point, at which every human being's practical life would begin. It would solve \marginpar{288}all the antinomies and contradictions, it would let loose all the right impulses and emotions; and everyone, on hearing it, would say, ``Why, that \emph{is} the truth! --- \emph{that} is what I have been believing, that is what I have really been living on all this time, but I never could find the words for it before. All that eludes, all that flickers and twinkles, all that invites and vanishes even whilst inviting, is here made a solidity and a possession. Here is the end of unsatisfactoriness, here the beginning of unimpeded clearness, joy, and power.'' Yes, my friends, I have such a discourse within me! But, do not judge me harshly, I cannot produce it on the present occasion. I humbly apologize; I have come across the continent to this wondrous Pacific Coast--- to this Eden, not of the mythical antiquity, but of the solid future of mankind--- I ought to give you something worthy of your hospitality, and not altogether unworthy of your great destiny, to help cement our rugged East and your wondrous West together in a spiritual bond,--- and yet, and yet, and yet, I simply cannot. I have tried to articulate it, but it will not come. Philosophers are after all like poets. They are path-finders. What everyone can feel, what everyone can know in the bone and marrow of him, they sometimes can find words for and express. The words and thoughts of the philosophers are not exactly the words and thoughts of the poets--- worse luck. But both alike have the same function. They are, if I may use a simile, so many spots, or blazes,--- blazes made by the axe of the human intellect on the trees of the otherwise trackless forest of human experience. They give you somewhere to go from. They give you a direction and a place to reach. They do not give you the integral forest with all its sunlit glories and its moonlit witcheries and wonders. Ferny dells, and mossy waterfalls, and secret magic nooks escape you, owned only by the wild things to whom the region is a home. Happy they without the need of blazes! But to us the blazes give a sort of ownership. We can now use the forest, wend across it with companions, \marginpar{289}and enjoy its quality. It is no longer a place merely to get lost in and never return. The poet's words and the philosopher's phrases thus are helps of the most genuine sort, giving to all of us hereafter the freedom of the trails they made. Though they create nothing, yet for this marking and fixing function of theirs we bless their names and keep them on our lips, even whilst the thin and spotty and half-casual character of their operations is evident to our eyes. 

No one like the path-finder himself feels the immensity of the forest, or knows the accidentality of his own trails. Columbus, dreaming of the ancient East, is stopped by poor pristine simple America, and gets no farther on that day; and the poets and philosophers themselves know as no one else knows that what their formulas express leaves unexpressed almost everything that they organically divine and feel. So I feel that there is a center in truth's forest where I have never been: to track it out and get there is the secret spring of all my poor life's philosophic efforts; at moments I almost strike into the final valley, there is a gleam of the end, a sense of certainty, but always there comes still another ridge, so my blazes merely circle towards the true direction; and although now, if ever, would be the fit occasion, yet I cannot take you to the wondrous hidden spot to-day. To-morrow it must be, or to-morrow, or to-morrow, and pretty surely death will overtake me ere the promise is fulfilled. 

Of such postponed achievements do the lives of all philosophers consist. Truth's fullness is elusive; ever not quite, not quite! So we fall back on the preliminary blazes--- a few formulas, a few technical conceptions, a few verbal pointers--- which at least define the initial direction of the trail. And that to my sorrow, is all that I can do here at Berkeley to-day. Inconclusive I must be, and merely suggestive, though I will try to be as little technical as I can. 

I will seek to define with you merely what seems to be the most likely direction in which to start upon the trail of \marginpar{290}truth. Years ago this direction was given to me by an American philosopher whose home is in the East, and whose published works, few as they are and scattered in periodicals, are no fit expression of his powers. I refer to Mr. Charles S. Peirce, with whose very existence as a philosopher I dare say many of you are unacquainted. He is one of the most original of contemporary thinkers; and the principle of practicalism or pragmatism, as he called it, when I first heard him enunciate it at Cambridge in the early $'$70's is the clue or compass by following which I find myself more and more confirmed in believing we may keep our feet upon the proper trail. 

Peirce's principle, as we may call it, may be expressed in a variety of ways, all of them very simple. In the \emph{Popular Science Monthly} for January, 1878, he introduces it as follows: The soul and meaning of thought, he says, can never be made to direct itself towards anything but the production of belief, belief being the demicadence which closes a musical phrase in the symphony of our intellectual life. Thought in movement has thus for its only possible motive the attainment of thought at rest. But when our thought about an object has found its rest in belief, then our action on the subject can firmly and safely begin. Beliefs, in short, are really rules for action; and the whole function of thinking is but one step in the production of habits of action. If there were any part of a thought that made no difference in the thought's practical consequences, then that part would be no proper element of the thought's significance. Thus the same thought may be clad in different words; but if the different words suggest no different conduct, they are mere outer accretions, and have no part in the thought's meaning. If, however, they determine conduct differently, they are essential elements of the significance. ``Please open the door,'' and, ``\emph{Veuillez ouvrir la porte},'' in French, mean just the same thing; but ``D---n you, open the door,'' although in English, \emph{means} something very different. Thus to develop a thought's \marginpar{291}meaning we need only determine what conduct it is fitted to produce; that conduct is for us its sole significance. And the tangible fact at the root of all our thought-distinctions, however subtle, is that there is no one of them so fine as to consist in anything but a possible difference of practice. To attain perfect clearness in our thoughts of an object, then, we need only consider what effects of a conceivably practical kind the object may involve --- what sensations we are to expect from it, and what reactions we must prepare. Our conception of these effects, then, is for us the whole of our conception of the object, so far as that conception has positive significance at all. 

This is the principle of Peirce, the principle of pragmatism. I think myself that it should be expressed more broadly  than Mr. Peirce expresses it. The ultimate test for us of what a truth means is indeed the conduct it dictates or inspires. But it inspires that conduct because it first foretells some particular turn to our experience which shall call for just that conduct from us. And I should prefer for our purposes this evening to express Peirce's principle by saying that the effective meaning of any philosophic proposition can always be brought down to some particular consequence, in our future practical experience, whether active or passive; the point lying rather in the fact that the experience must be particular, than in the fact that it must be active. 

To take in the importance of this principle, one must get accustomed to applying it to concrete cases. Such use as I am able to make of it convinces me that to be mindful of it in philosophical disputations tends wonderfully to smooth out misunderstandings and to bring in peace. If it did nothing else, then, it would yield a sovereignly valuable rule of method for discussion. So I shall devote the rest of this precious hour with you to its elucidation, because I sincerely think that if you once grasp it, it will shut your steps out from many an old false opening, and head you in the true direction for the trail. \marginpar{292}One of its first consequences is this. Suppose there are two different philosophical definitions, or propositions, or maxims, or what not, which seem to contradict each other, and about which men dispute. If, by supposing the truth of the one, you can foresee no conceivable practical consequence to anybody at any time or place, which is different from what you would foresee if you supposed the truth of the other, why then the difference between the two propositions is no difference,--- it is only a specious and verbal difference, unworthy of further contention. Both formulas mean radically the same thing, although they may say it in such different words. It is astonishing to see how many philosophical disputes collapse into insignificance the moment you subject them to this simple test. There can be no difference which doesn't make a difference--- no difference in abstract truth which does not express itself in a difference of concrete fact, and of conduct consequent upon the fact, imposed on somebody, somehow, somewhere, and somewhen. It is true that a certain shrinkage of values often seems to occur in our general formulas when we measure their meaning in this prosaic and practical way. They diminish. But the vastness that is merely based on vagueness is a false appearance of importance, and not a vastness worth retaining. The $x$'s, $y$'s, and $z$'s always do shrivel, as I have heard a learned friend say, whenever at the end of your algebraic computation they change into so many plain $a$'s, $b$'s, and $c$'s;--- but the whole function of algebra is, after all, to get them into that more definite shape; and the whole function of philosophy ought to be to find out what definite difference it will make to you and me, at definite instants of our life, if this world-formula or that world-formula be the one which is true. 

If we start off with an impossible case, we shall perhaps all the more clearly see the use and scope of our principle. Let us, therefore, put ourselves, in imagination, in a position from which no forecasts of consequence, no dictates of conduct, can possibly be made, so that the principle of \marginpar{293}pragmatism finds no field of application. Let us, I mean, assume that the present moment is the absolutely last moment of the world, with bare nonentity beyond it, and no hereafter for either experience or conduct. 

Now I say that in that case there would be no sense whatever in some of our most urgent and envenomed philosophical and religious debates. The question, ``Is matter the producer of all things, or is a God there too?'' would, for example, offer a perfectly idle and insignificant alternative if the world were finished and no more of it to come. Many of us, most of us, I think, now feel as if a terrible coldness and deadness would come over the world were we forced to believe that no informing spirit or purpose had to do with it, but it merely accidentally had come. The actually experienced details of fact might be the same on either hypothesis, some sad, some joyous; some rational, some odd and grotesque; but without a God behind them, we think they would have something ghastly, they would tell no genuine story, there would be no speculation in those eyes that they do glare with. With the God, on the other hand, they would grow solid, warm, and altogether full of real significance. 

But I say that such an alternation of feelings, reasonable enough in a consciousness that is prospective, as ours now is, and whose world is partly yet to come, would be absolutely senseless and irrational in a purely retrospective consciousness summing up a world already past. For such a consciousness, no emotional interest could attach to the alternative. The problem would be purely intellectual; and if unaided matter could, with any scientific plausibility, be shown to cipher out the actual facts, then not the faintest shadow ought to cloud the mind, of regret for the God that by the same ciphering would prove needless and disappear from our belief. 

For just consider the case sincerely, and say what would be the \emph{worth} of such a God if he \emph{were} there, with his work accomplished and his world run down. He would be worth \marginpar{294}no more than just that world was worth. To that amount of result, with its mixed merits and defects, his creative power could attain, but go no farther. And since there is to be no future; since the whole value and meaning of the world has been already paid in and actualized in the feelings that went with it in the passing, and now go with it in the ending ; since it draws no supplemental significance (such as our real world draws) from its function of preparing some thing yet to come; why then, by it we take God's measure, as it were. He is the Being who could once for all do \emph{that}; and for that much we are thankful to him, but for nothing more. But now, on the contrary hypothesis, namely, that the bits of matter following their ``laws'' could make that world and do no less, should we not be just as thankful to them? Wherein should we suffer loss, then, if we dropped God as an hypothesis and made the matter alone responsible? Where would the special deadness, ``crassness,'' and ghastliness come in? And how, experience being what it is once for all, would God's presence in it make it any more ``living,'' any richer in our sight? 

Candidly, it is impossible to give any answer to this question. The actually experienced world is supposed to be the same in its details on either hypothesis, ``the same, for our praise or blame,'' as Browning says. It stands there indefeasibly; a gift which can't be taken back. Calling matter the cause of it retracts no single one of the items that have made it up, nor does calling God the cause augment them. They are the God or the atoms, respectively, of just that and no other world. The God, if there, has been doing just what atoms could do--- appearing in the character of atoms, so to speak--- and earning such gratitude as is due to atoms, and no more. If his presence lends no different turn or issue to the performance, it surely can lend it no increase of dignity. Nor would indignity come to it were he absent, and did the atoms remain the only actors on the stage. When a play is once over, and the curtain down, you really make it no better by claiming an illustrious \marginpar{295}genius for its author, just as you make it no worse by calling him a common hack. 

Thus if no future detail of experience or conduct is to be deduced from our hypothesis, the debate between materialism and theism becomes quite idle and insignificant. Matter and God in that event mean exactly the same thing--- the power, namely, neither more nor less, that can make just this mixed, imperfect, yet completed world--- and the wise man is he who in such a case would turn his back on such a supererogatory discussion. Accordingly most men instinctively--- and a large class of men, the so-called positivists or scientists, deliberately--- do turn their backs on philosophical disputes from which nothing in the line of definite future consequences can be seen to follow. The verbal and empty character of our studies is surely a reproach with which you of the Philosophical Union are but too sadly familiar. An escaped Berkeley student said to me at Harvard the other day--- he had never been in the philosophical department here--- ``Words, words, words, are all that you philosophers care for.'' We philosophers think it all unjust; and yet, if the principle of pragmatism be true, it is a perfectly sound reproach unless the metaphysical alternatives under investigation can be shown to have alternative practical outcomes, however delicate and distant these may be. The common man and the scientist can discover no such outcomes. And if the metaphysician can discern none either, the common man and scientist certainly are in the right of it, as against him. His science is then but pompous trifling; and the endowment of a professorship for such a being would be something really absurd. 

Accordingly, in every genuine metaphysical debate some practical issue, however remote, is really involved.To realize this, revert with me to the question of materialism or theism; and place yourselves this time in the real world we live in, the world that has a future, that is yet uncompleted whilst we speak. In this unfinished world the alternative of ``materialism or theism?'' is intensely \marginpar{296}practical; and it is worth while for us to spend some minutes of our hour in seeing how truly this is the case. 

How, indeed, does the programme differ for us, according as we consider that the facts of experience up to date are purposeless configurations of atoms moving according to eternal elementary laws, or that on the other hand they are due to the providence of God? As far as the past facts go, indeed there is no difference. These facts are in, are bagged, are captured; and the good that's in them is gained, be the atoms or be the God their cause. There are accordingly many materialists about us to-day who, ignoring altogether the future and practical aspects of the question, seek to eliminate the odium attaching to the word materialism, and even to eliminate the word itself, by showing that, if matter could give birth to all these gains, why then matter, functionally considered, is just as divine an entity as God, in fact coalesces with God, is what you mean by God. Cease, these persons advise us, to use either of these terms, with their outgrown opposition. Use terms free of the clerical connotations on the one hand; of the suggestion of grossness, coarseness, ignobility, on the other. Talk of the primal mystery, of the unknowable energy, of the one and only power, instead of saying either God or matter. This is the course to which Mr. Spencer urges us at the end of the first volume of his Psychology. In some well-written pages he there shows us that a ``matter'' so infinitely subtile, and performing motions as inconceivably quick and fine as modern science postulates in her explanations, has no trace of grossness left. He shows that the conception of spirit, as we mortals hitherto have framed it, is itself too gross to cover the exquisite complexity of Nature's facts. Both terms, he says, are but symbols, pointing to that one unknowable reality in which their oppositions cease. 

Throughout these remarks of Mr. Spencer, eloquent, and even noble in a certain sense, as they are, he seems to think that the dislike of the ordinary man to materialism \marginpar{297}comes from a purely {\ae}sthetic disdain of matter, as some thing gross in itself, and vile and despicable. Undoubtedly such an aesthetic disdain of matter has played a part in philosophic history. But it forms no part whatever of an intelligent modern man's dislikes. Give him a matter bound forever by its laws to lead our world nearer and nearer to perfection, and any rational man will worship that matter as readily as Mr. Spencer worships his own so-called unknowable power. It not only has made for righteousness up to date, but it will make for righteousness forever; and that is all we need. Doing practically all that a God can do, it is equivalent to God, its function is a God's function, and in a world in which a God would be superfluous; from such a world a God could never lawfully be missed. 

But is the matter by which Mr. Spencer's process of cosmic evolution is carried on any such principle of never-ending perfection as this? Indeed it is not, for the future end of every cosmically evolved thing or system of things is tragedy; and Mr. Spencer, in confining himself to the aesthetic and ignoring the practical side of the controversy, has really contributed nothing serious to its relief. But apply now our principle of practical results, and see what a vital significance the question of materialism or theism immediately acquires. 

Theism and materialism, so indifferent when taken retrospectively, point when we take them prospectively to wholly different practical consequences, to opposite outlooks of experience. For, according to the theory of mechanical evolution, the laws of redistribution of matter and motion, though they are certainly to thank for all the good hours which our organisms have ever yielded us and for all the ideals which our minds now frame, are yet fatally certain to undo their work, again, and to redissolve everything that they have once evolved. You all know the picture of the last foreseeable state of the dead universe, as evolutionary science gives it forth. I cannot \marginpar{298}state it better than in Mr. Balfour's words: ``The energies of our system will decay, the glory of the sun will be dimmed, and the earth, tideless and inert, will no longer tolerate the race which has for a moment disturbed its solitude. Man will go down into the pit, and all his thoughts will perish. The uneasy consciousness which in this obscure corner has for a brief space broken the contented silence of the universe, will be at rest. Matter will know itself no longer. `Imperishable monuments' and `immortal deeds,' death itself, and love stronger than death, will be as if they had not been. Nor will anything that is, be better or worse for all that the labor, genius, devotion, and suffering of man have striven through countless ages to effect.''\footnote{The Foundations of Belief, p. 30.}

That is the sting of it, that in the vast driftings of the cosmic weather, though many a jewelled shore appears, and many an enchanted cloud-bank floats away, long lingering ere it be dissolved--- even as our world now lingers, for our joy--- yet when these transient products are gone, nothing, absolutely \emph{nothing} remains, to represent those particular qualities, those elements of preciousness which they may have enshrined. Dead and gone are they, gone utterly from the very sphere and room of being. Without an echo; without a memory; without an influence on aught that may come after, to make it care for similar ideals. This utter final wreck and tragedy is of the essence of scientific materialism as at present understood. The lower and not the higher forces are the eternal forces, or the last surviving forces within the only cycle of evolution which we can definitely see. Mr. Spencer believes this as much as anyone; so why should he argue with us as if we were making silly esthetic objections to the ``grossness'' of ``matter and motion,''--- the principles of his philosophy,--- when what really dismays us in it is the disconsolateness of its ulterior practical results? 

No, the true objection to materialism is not positive but \marginpar{299}negative. It would be farcical at this day to make complaint of it for what it is, for ``grossness.'' Grossness is what grossness \emph{does}--- we now know \emph{that}. We make complaint of it, on the contrary, for what it is \emph{not}--- not a permanent warrant for our more ideal interests, not a fulfiller of our remotest hopes. 

The notion of God, on the other hand, however inferior it may be in clearness to those mathematical notions so current in mechanical philosophy, has at least this practical superiority over them, that it guarantees an ideal order that shall be permanently preserved. A world with a God in it to say the last word, may indeed burn up or freeze, but we then think of Him as still mindful of the old ideals and sure to bring them elsewhere to fruition; so that, where He is, tragedy is only provisional and partial, and shipwreck and dissolution not the absolutely final things. This need of an eternal moral order is one of the deepest needs of our breast. And those poets, like Dante and Wordsworth, who live on the conviction of such an order, owe to that fact the extraordinary tonic and consoling power of their verse. Here then, in these different emotional and practical appeals, in these adjustments of our concrete attitudes of hope and expectation, and all the delicate consequences which their differences entail, lie the real meanings of materialism and theism--- not in hair-splitting abstractions about matter's inner essence, or about the metaphysical attributes of God. Materialism means simply the denial that the moral order is eternal, and the cutting off of ultimate hopes; theism means the affirmation of an eternal moral order and the letting loose of hope. Surely here is an issue genuine enough, for any one who feels it; and, as long as men are men, it will yield matter for serious philosophic debate. Concerning this question, at any rate, the positivists and pooh-pooh-ers of metaphysics are in the wrong. 

But possibly some of you may still rally to their defense. Even whilst admitting that theism and materialism make \marginpar{300}different prophecies of the world's future, you may yourselves pooh-pooh the difference as something so infinitely remote as to mean nothing for a sane mind. The essence of a sane mind, you may say, is to take shorter views, and to feel no concern about such chim{\ae}ras as the latter end of the world. Well, I can only say that if you say this, you do injustice to human nature. Religious melancholy is not disposed of by a simple nourish of the word insanity. The absolute things, the last things, the overlapping things, are the truly philosophic concern; all superior minds feel seriously about them, and the mind with the shortest views is simply the mind of the more shallow man. 

However, I am willing to pass over these very distant outlooks on the ultimate, if any of you so insist. The theistic controversy can still serve to illustrate the principle of pragmatism for us well enough, without driving us so far afield. If there be a God, it is not likely that he is confined solely to making differences in the world's latter end; he improbably makes differences all along its course. Now the principle of practicalism says that the very meaning of the conception of God lies in those differences which must be made in our experience if the conception be true. God's famous inventory of perfections, as elaborated by dogmatic theology, either means nothing, says our principle, or it implies certain definite things that we can feel and do at particular moments of our lives, things which we could not feel and should not do were no God present and were the business of the universe carried on by material atoms instead. So far as our conceptions of the Deity involve no such experiences, so far they are meaningless and verbal,--- scholastic entities and abstractions, as the positivists say, and fit objects for their scorn. But so far as they do involve such definite experiences, God means something for us, and may be real. 

Now if we look at the definitions of God made by dogmatic theology, we see immediately that some stand and some fall when treated by this test. God, for example, \marginpar{301}as any orthodox text-book will tell us, is a being existing not only \emph{per se}, or by himself, as created beings exist, but \emph{a se}, or from himself; and out of this ``aseity'' flow most of his perfections. He is, for example, necessary; absolute; infinite in all respects; and single. He is simple, not compounded of essence and existence, substance and accident, actuality and potentiality, or subject and attributes, as are other things. He belongs to no genus; he is inwardly and outwardly unalterable; he knows and wills all things, and first of all his own infinite self, in one indivisible eternal act. And he is absolutely self -sufficing, and infinitely happy.--- Now in which one of us practical Americans here assembled does this conglomeration of attributes awaken any sense of reality? And if in no one, then why not? Surely because such attributes awaken no responsive active feelings and call for no particular conduct of our own. How does God's ``aseity'' come home to \emph{you}? What specific thing can I do to adapt myself to his ``simplicity''? Or how determine our behavior henceforward if his ``felicity'' is anyhow absolutely complete? In the $'$50's and $'$60's Captain Mayne Reid was the great writer of boys' books of out-of-door adventure. He was forever extolling the hunters and field-observers of living animals' habits, and keeping up a fire of invective against the ``closet-naturalists,'' as he called them, the collectors and classifiers, and handlers of skeletons and skins. When I was a boy I used to think that a closet-naturalist must be the vilest type of wretch under the sun. But surely the systematic theologians are the closet-naturalists of the Deity, even in Captain Mayne Reid's sense. Their orthodox deduction of God's attributes is nothing but a shuffling and matching of pedantic dictionary-adjectives, aloof from morals, aloof from human needs, something that might be worked out from the mere word ``God'' by a logical machine of wood and brass as well as by a man of flesh and blood. The attributes which I have quoted have absolutely nothing to do with religion, for religion is a living practical affair. Other \marginpar{302}parts, indeed, of God's traditional description do have practical connection with life, and have owed all their historic importance to that fact. His omniscience, for example, and his justice. With the one he sees us in the dark, with the other he rewards and punishes what he sees. So do his ubiquity and eternity and unalterability appeal to our confidence, and his goodness banish our fears. Even attributes of less meaning to this present audience have in past times so appealed. One of the chief attributes of God, according to the orthodox theology, is his infinite love of himself, proved by asking the question, ``By what but an infinite object can an infinite affection be appeased!'' An immediate consequence of this primary self-love of God is the orthodox dogma that the manifestation of his own glory is God's primal purpose in creation; and that dogma has certainly made very efficient practical connection with life. It is true that we ourselves are tending to outgrow this old monarchical conception of a Deity with his ``court'' and pomp ``his state is kingly, thousands at his bidding speed,'' etc. but there is no denying the enormous influence it has had over ecclesiastical history, nor, by repercussion, over the history of European states. And yet even these more real and significant attributes have the trail of the serpent over them as the books on theology have actually worked them out. One feels that, in the theologians hands, they are only a set of dictionary-adjectives, mechanically deduced; logic has stepped into the place of vision, professionalism into that of life. Instead of bread we get a stone; instead of a fish, a serpent. Did such a conglomeration of abstract general terms give really the gist of our knowledge of the Deity, divinity-schools might indeed continue to flourish, but religion, vital religion, would have taken its flight from this world. What keeps religion going is something else than abstract definitions and systems of logically concatenated adjectives, and something different from faculties of theology and their professors. All these things are after-effects, secondary accretions upon a mass of \marginpar{303}concrete religious experiences, connecting themselves with feeling and conduct that renew themselves \emph{in s{\ae}cula s{\ae}culorum}\footnote{[Literally `in a century of centuries'; that is, forever and ever.]} in the lives of humble private men. If you ask what these experiences are, they are conversations with the unseen, voices and visions, responses to prayer, changes of heart, deliverances from fear, inflowings of help, assurances of support, whenever certain persons set their own internal attitude in certain appropriate ways. The power comes and goes and is lost, and can be found only in a certain definite direction, just as if it were a concrete material thing. These direct experiences of a wider spiritual life, with which our superficial consciousness is continuous, and with which it keeps up an intense commerce, form the primary mass of direct religious experience on which all hearsay religion rests, and which furnishes that notion of an ever-present God, out of which systematic theology thereupon proceeds to make capital in its own unreal pedantic way. What the word ``God'' means is just those passive and active experiences of your life. Now, my friends, it is quite immaterial to my purpose whether yourselves enjoy and venerate these experiences, or whether you stand aloof and, viewing them in others, suspect them of being illusory and vain. Like all other human experiences, they too certainly share in the general liability to illusion and mistake. They need not be infallible. But they are certainly the originals of the God-idea, and theology is the translation; and you remember that I am now using the God-idea merely as an example, not to discuss as to its truth or error, but only to show how well the principle of pragmatism works. That the God of systematic theology should exist or not exist is a matter of small practical moment. At most it means that you may continue uttering certain abstract words and that you must stop using others. But if the God of these particular experiences be false, it is an awful thing for you, if you are one of those whose lives are stayed on such experiences. The theistic controversy, trivial enough if we take it merely academically and theologically, is of \marginpar{304}tremendous significance if we test it by its results for actual life. 

I can best continue to recommend the principle of practicalism to you by keeping in the neighborhood of this theological idea. I reminded you a few minutes ago that the old monarchical notion of the Deity as a sort of Louis the Fourteenth of the Heavens is losing nowadays much of its ancient prestige. Religious philosophy, like all philosophy, is growing more and more idealistic. And in the philosophy of the Absolute, so called, that post-Kantian form of idealism which is carrying so many of our higher minds before it, we have the triumph of what in old times was summarily disposed of as the pantheistic heresy,--- I mean the conception of God, not as the extraneous creator, but as the indwelling spirit and substance of the world. I know not where one can find a more candid, more clear, or, on the whole, more persuasive statement of this theology of Absolute Idealism than in the addresses made before this very Union three years ago by your own great Californian philosopher (whose colleague at Harvard I am proud to be), Josiah Royce. His contributions to the resulting volume, \emph{The Conception of God}, form a very masterpiece of popularization. Now you will remember, many of you, that in the discussion that followed Professor Royce's first address, the debate turned largely on the ideas of unity and plurality, and on the question whether, if God be One in All and All in All, ``One with the unity of a single instant,'' as Royce calls it, ``forming in His wholeness one luminously transparent moment,'' any room is left for real morality or freedom. Professor Howison, in particular, was earnest in urging that morality and freedom are relations between a manifold of selves, and that under the regime of Royce's monistic Absolute Thought ``no true manifold of selves is or can be provided for.'' I will not go into any of the details of that particular discussion, but just ask you to consider for a moment whether, in general, any discussion about monism or pluralism, any argument over the unity of the universe, \marginpar{305}would not necessarily be brought into a shape where it tends to straighten itself out, by bringing our principle of practical results to bear. 

The question whether the world is at bottom One or Many is a typical metaphysical question. Long has it raged! In its crudest form it is an exquisite example of the \emph{loggerheads} of metaphysics. ``I say it is one great  fact,'' Parmenides and Spinoza exclaim. ``I say it is many little facts,'' reply the atomists and associationists. ``I say it is both one and many, many in one,'' say the Hegelians; and in the ordinary popular discussions we rarely get beyond this barren reiteration by the disputants of their pet adjectives of number. But is it not first of all clear that when we take such an adjective as ``One'' absolutely and abstractly, its meaning is so vague and empty that it makes no difference whether we affirm or deny it? Certainly this universe is not the mere number One; and yet you can number it ``one,'' if you like, in talking about it as contrasted with other possible worlds numbered ``two'' and ``three'' for the occasion. What exact thing do you practically mean by ``One,'' when you call the universe One, is the first question you must ask. In what ways does | the oneness come home to your own personal life? By what difference does it express itself in your experience? How can you act differently towards a universe which is one? Inquired into in this way, the unity might grow clear and be affirmed in some ways and denied in others, and so cleared up, even though a certain vague and worshipful portentousness might disappear from the notionof it in the process. 

For instance, one practical result that follows when we have one thing to handle, is that we can pass from one part of it to another without letting go of the thing. In this sense oneness must be partly denied and partly affirmed of our universe. Physically we can pass continuously in various manners from one part of it to another part. But logically and psychically the passage seems less easy, for \marginpar{306}there is no obvious transition from one mind to another, or from minds to physical things. You have to step off and get on again; so that in these ways the world is not one, as measured by that practical test. 

Another practical meaning of oneness is susceptibility of collection. A collection is one, though the things that compose it be many. Now, can we practically ``collect'' the universe? Physically, of course we cannot. And mentally we cannot, if we take it concretely in its details. But if we take it summarily and abstractly, then we collect it mentally whenever we refer to it, even as I do now when I fling the term ``universe'' at it, and so seem to leave a mental ring around it. It is plain, however, that such abstract noetic unity (as one might call it) is practically an extremely insignificant thing. 

Again, oneness may mean generic sameness, so that you can treat all parts of the collection by one rule and get the same results. It is evident that in this sense the oneness of our world is incomplete, for in spite of much generic sameness in its elements and items, they still remain of many irreducible kinds. You can't pass by mere logic all over the field of it. 

Its elements have, however, an affinity or commensurability with each other, are not wholly irrelevant, but can be compared, and fit together after certain fashions. This again might practically mean that they were one \emph{in origin}, and that, tracing them backwards, we should find them arising in a single primal causal fact. Such unity of origin would have definite practical consequences, would have them for our scientific life at least. 

I can give only these hasty superficial indications of what I mean when I say that it tends to clear up the quarrel between monism and pluralism to subject the notion of unity to such practical tests. On the other hand, it does but perpetuate strife and misunderstanding to continue talking of it in an absolute and mystical way. I have little doubt myself that this old quarrel might be \marginpar{307}completely smoothed out to the satisfaction of all claimants, if only the maxim of Peirce were methodically followed here. The current monism on the whole still keeps talking in too abstract a way. It says the world must be either pure disconnectedness, no universe at all, or absolute unity. It insists that there is no stopping-place half way. Any connection whatever, says this monism, is only possible if there be still more connection, until at last we are driven to admit the absolutely total connection required. But this absolutely total connection either means nothing, is the mere word ``one'' spelt long; or else it means the sum of all the partial connections that can possibly be con ceived. I believe that when we thus attack the question, and set ourselves to search for these possible connections, and conceive each in a definite practical way, the dispute is already in a fair way to be settled beyond the chance of misunderstanding, by a compromise in which the Many and the One both get their lawful rights. 

But I am in danger of becoming technical; so I must stop right here, and let you go. 

\bigskip

I am happy to say that it is the English-speaking philosophers who first introduced the custom of interpreting the meaning of conceptions by asking what difference they make for life. Mr. Peirce has only expressed in the form of an explicit maxim what their sense for reality led them all instinctively to do. The great English way of investigating a conception is to ask yourself right off, ``What is it known as? In what facts does it result? What is its cash-value, in terms of particular experience! and what special difference would come into the world according as it were true or false?'' Thus does Locke treat the conception of personal identity. What you mean by it is just your chain of memories, says he. That is the only concretely verifiable part of its significance. All further ideas about it, such as the oneness or manyness of the spiritual substance on which it is based, are therefore \marginpar{308}void of intelligible meaning; and propositions touching such ideas may be indifferently affirmed or denied. So Berkeley with his ``matter.'' The cash-value of matter is our physical sensations. That is what it is known as, all that we concretely verify of its conception. That therefore is the whole meaning of the word ``matter''--- any other pretended meaning is mere wind of words. Hume does the same thing with causation. It is known as habitual antecedence, and tendency on our part to look for some thing definite to come. Apart from this practical meaning it has no significance whatever, and books about it may be committed to the flames, says Hume. Stewart and Brown, James Mill, John Mill, and Bain, have followed more or less consistently the same method; and Shadworth Hodgson has used it almost as explicitly as Mr. Peirce. These writers have many of them no doubt been too sweep ing in their negations; Hume, in particular, and James Mill, and Bain. But when all is said and done, it was they, not Kant, who introduced ``the critical method'' into philosophy, the one method fitted to make philosophy a study worthy of serious men. For what seriousness can I possibly remain in debating philosophic propositions that I will never make an appreciable difference to us in action? And what matters it, when all propositions are practically meaningless, which of them be called true or false?

The shortcomings and the negations and baldnesses of the English philosophers in question come, not from their eye to merely practical results, but solely from their failure to track the practical results completely enough to see how far they extend. Hume can be corrected and built out, and his beliefs enriched, by using Humian principles exclusively, and without making any use of the circuitous and ponderous artificialities of Kant. It is indeed a somewhat pathetic matter, as it seems to me, that this is not the course which the actual history of philosophy has followed. Hume had no English successors of adequate ability to complete him and correct his negations; so it \marginpar{309}happened, as a matter of fact, that the building out of critical philosophy has mainly been left to thinkers who were under the influence of Kant. Even in England and this country it is with Kantian catch-words and categories that the fuller view of life is pursued, and in our universities it is the courses in transcendentalism that kindle the enthusiasm of the more ardent students, whilst the courses in English philosophy are committed to a secondary place. I cannot think that this is exactly as it should be. And I say this not out of national jingoism, for jingoism has no place in philosophy; or out of excitement over the great Anglo-American alliance against the world, of which we nowadays hear so much though heaven knows that to that alliance I wish a God-speed. I say it because I sincerely believe that the English spirit in philosophy is intellectually, as well as practically and morally, on the saner, sounder, and truer path. Kant's mind is the rarest and most intricate of all possible antique bric-a-brac museums, and connoisseurs and dilettanti will always wish to visit it and see the wondrous and racy contents. The temper of the dear old man about his work is perfectly delectable. And yet he is really--- although I shrink with some terror from saying such a thing before some of you here present--- at bottom a mere curio, a ``specimen.'' I mean by this a perfectly definite thing: I believe that Kant bequeathes to us not one single conception which is both indispensable to philosophy and which philosophy either did not possess before him, or was not destined inevitably to acquire after him through the growth of men's reflection upon the hypothesis by which science interprets nature. The trueline of philosophic progress lies, in short, it seems to me, not so much \emph{through} Kant as \emph{round} him to the point where now we stand. Philosophy can perfectly well outflank him, and build herself up into adequate fulness by prolonging more directly the older English lines. 

May I hope, as I now conclude, and release your attention from the strain to which you have so kindly put it on \marginpar{310}my behalf, that on this wonderful Pacific Coast, of which our race is taking possession, the principle of practicalism, in which I have tried so hard to interest you, and with it the whole English tradition in philosophy, will come to its rights, and in your hands help the rest of us in our struggle towards the light. 

\end{document}