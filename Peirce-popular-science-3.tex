% This file does not have the appropriate header and tags to compile in latex. It is meant to be input into another file for compiling. 



\section*{The Doctrine of Chances}
\emph{This is part three in the series. It was originally published in \emph{Popular Science Monthly} 12 (March 1878), pgs. 604--615.}

\subsection*{I}

It is a common observation that a science first begins to be exact when it is quantitatively treated. What are called the exact sciences are no others than the mathematical ones. Chemists reasoned vaguely until Lavoisier showed them how to apply the balance to the verification of their theories, when chemistry leaped suddenly into the position of the most perfect of the classificatory sciences. It has thus become so precise and certain that we usually think of it along with optics, thermotics, and electrics. But these are studies of general laws, while chemistry considers merely the relations and classification of certain objects; and belongs, in reality, in the same category as systematic botany and zo\"ology. Compare it with these last, however, and the advantage that it derives from its quantitative treatment is very evident.

The rudest numerical scales, such as that by which the mineralogists distinguish the different degrees of hardness, are found useful. The mere counting of pistils and stamens sufficed to bring botany out of total chaos into some hind of form. It is not, however, so much from \emph{counting} as from \emph{measuring}, not so much from the conception of number as from that of continuous quantity, that the advantage of mathematical treatment comes. Number, after all, only serves to pin us down to a precision in our thoughts which, however beneficial, can seldom lead to lofty conceptions, and frequently descends to pettiness. Of those two faculties of which Bacon speaks, that which marks differences and that which notes resemblances, the employment of number can only aid the lesser one; and the excessive use of it must tend to narrow the powers of the mind. But the conception of continuous quantity has a great office to fulfill, independently of any attempt at precision. Far from tending to the exaggeration of differences, it is the direct instrument of the finest generalizations. When a naturalist wishes to study a species, he collects a considerable number of specimens more or less similar. In contemplating them, he observes certain ones which are more or less alike in some particular respect. They all have, for instance, a certain S-shaped marking. He observes that they are not \emph{precisely} alike, in this respect; the S has not precisely the same shape, but the differences are such as to lead him to believe that forms could be found intermediate between any two of those he possesses. He, now, finds other forms apparently quite dissimilar--- say a marking in the form of a C--- and the question is, whether he can find intermediate ones which will connect these latter with the others. This he often succeeds in doing in cases where it would at first be thought impossible; whereas, he sometimes finds those which differ, at first glance, much less, to be separated in Nature by the non-occurrence of intermediaries. In this way, he builds up from the study of Nature a new general conception of the character in question. He obtains, for example, an idea of a leaf which includes every part of the flower, and an idea of a vertebra which includes the skull. I surely need not say much to show what a logical engine there is here. It is the essence of the method of the naturalist. How he applies it first to one character, and then to another, and finally obtains a notion of a species of animals, the differences between whose members, however great, are confined within limits, is a matter which does not here concern us. The whole method of classification must be considered later; but, at present, I only desire to point out that it is by taking advantage of the idea of continuity, or the passage from one form to another by insensible degrees, that the naturalist builds his conceptions. Now, the naturalists are the great builders of conceptions; there is no other branch of science where so much of this work is done as in theirs; and we must, in great measure, take them for our teachers in this important part of logic. And it will be found everywhere that the idea of continuity is a powerful aid to the formation of true and fruitful conceptions. By means of it, the greatest differences are broken down and resolved into differences of degree, and the incessant application of it is of the greatest value in broadening our conceptions. I propose to make a great use of this idea in the present series of papers; and the particular series of important fallacies, which, arising from a neglect of it, have desolated philosophy, must further on be closely studied. At present, I simply call the reader's attention to the utility of this conception.

In studies of numbers, the idea of continuity is so indispensable, that it is perpetually introduced even where there is no continuity in fact, as where we say that there are in the United States 10.7 inhabitants per square mile, or that in New York 14.72 persons live in the average house.\footnote{This mode of thought is so familiarly associated with all exact numerical consideration, that the phrase appropriate to it is imitated by shallow writers in order to produce of the appearance of exactitude where none exists. Certain newspapers which affect a learned tone talk of ``the average man,'' when they simply mean \emph{most men}, and have no idea of striking an average.} Another example is that law of the distribution of errors which Quetelet, Galton, and others, have applied with so much success to the study of biological and social matters. This application of continuity to cases where it does not really exist illustrates, also, another point which will hereafter demand a separate study, namely, the great utility which fictions sometimes have in science.

\subsection*{II}

The theory of probabilities is simply the science of logic quantitatively treated. There are two conceivable certainties with reference to any hypothesis, the certainty of its truth and the certainty of its falsity. The numbers \emph{one} and \emph{zero} are appropriated, in this calculus, to marking these extremes of knowledge; while fractions having values intermediate between them indicate, as we may vaguely say, the degrees in which the evidence leans toward one or the other. The general problem of probabilities is, from a given state of facts, to determine the numerical probability of a possible fact. This is the same as to inquire how much the given facts are worth, considered as evidence to prove the possible fact. Thus the problem of probabilities is simply the general problem of logic.

Probability is a continuous quantity, so that great advantages may be expected from this mode of studying logic. Some writers have gone so far as to maintain that, by means of the calculus of chances, every solid inference may be represented by legitimate arithmetical operations upon the numbers given in the premises. If this be, indeed, true, the great problem of logic, how it is that the observation of one fact can give us knowledge of another independent fact, is reduced to a mere question of arithmetic. It seems proper to examine this pretension before undertaking any more recondite solution of the paradox.

But, unfortunately, writers on probabilities are not agreed in regard to this result. This branch of mathematics is the only one, I believe, in which good writers frequently get results entirely erroneous. In elementary geometry the reasoning is frequently fallacious, but erroneous conclusions are avoided; but it may be doubted if there is a single extensive treatise on probabilities in existence which does not contain solutions absolutely indefensible. This is partly owing to the want of any regular method of procedure; for the subject involves too many subtilties to make it easy to put its problems into equations without such an aid. But, beyond this, the fundamental principles of its calculus are more or less in dispute. In regard to that class of questions to which it is chiefly applied for practical purposes, there is comparatively little doubt; but in regard to others to which it has been sought to extend it, opinion is somewhat unsettled.

This last class of difficulties can only be entirely overcome by making the idea of probability perfectly clear in our minds in the way set forth in our last paper.

\subsection*{III}

To get a clear idea of what we mean by probability, we have to consider what real and sensible difference there is between one degree of probability and another.

The character of probability belongs primarily, without doubt, to certain inferences. Locke explains it as follows: After remarking that the mathematician positively knows that the sum of the three angles of a triangle is equal to two right angles because he apprehends the geometrical proof, he thus continues: \begin{quote}
But another man who never took the pains to observe the demonstration, hearing a mathematician, a man of credit, affirm the three angles of a triangle to be equal to two right ones,\emph{assents} to it; i.e., receives it for true. In which case the foundation of his assent is the probability of the thing, the proof being such as, for the most part, carries truth with it; the man on whose testimony he receives it not being wont to affirm anything contrary to, or besides his knowledge, especially in matters of this kind.
\end{quote}
The celebrated \{Essay concerning Humane Understanding} contains many passages which, like this one, make the first steps in profound analyses which are not further developed. It was shown in the first of these papers that the validity of an inference does not depend on any tendency of the mind to accept it, however strong such tendency may be; but consists in the real fact that, when premises like those of the argument in question are true, conclusions related to them like that of this argument are also true. It was remarked that in a logical mind an argument is always conceived as a member of a genus of arguments all constructed in the same way, and such that, when their premises are real facts, their conclusions are so also. If the argument is demonstrative, then this is always so; if it is only probable, then it is for the most part so. As Locke says, the probable argument is ``\emph{such as} for the most part carries truth with it.''

According to this, that real and sensible difference between one degree of probability and another, in which the meaning of the distinction lies, is that in the frequent employment of two different modes of inference, one will carry truth with it oftener than the other. It is evident that this is the only difference there is in the existing fact. Having certain premises, a man draws a certain conclusion, and as far as this inference alone is concerned the only possible practical question is whether that conclusion is true or not, and between existence and non-existence there is no middle term. ``Being only is and nothing is altogether not,'' said Parmenides; and this is in strict accordance with the analysis of the conception of reality given in the last paper. For we found that the distinction of reality and fiction depends on the supposition that sufficient investigation would cause one opinion to he universally received and all others to he rejected. That presupposition involved in the very conceptions of reality and figment involves a complete sundering of the two. It is the heaven-and-hell idea in the domain of thought. But, in the long run, there is a real fact which corresponds to the idea of probability, and it is that a given mode of inference sometimes proves successful and sometimes not, and that in a ratio ultimately fixed. As we go on drawing inference after inference of the given kind, during the first ten or hundred cases the ratio of successes may he expected to show considerable fluctuations; but when we come into the thousands and millions, these fluctuations become less and less; and if we continue long enough, the ratio will approximate toward a fixed limit. We may therefore define the probability of a mode of argument as the proportion of cases in which it carries truth with it.

The inference from the premise, A, to the conclusion, B, depends, as we have seen, on the guiding principle, that if a fact of the class A is true, a fact of the class B is true. The probability consists of the fraction whose numerator is the number of times in which both A and B are true, and whose denominator is the total number of times in which A is true, whether B is so or not. Instead of speaking of this as the probability of the inference, there is not the slighest objection to calling it the probability that, if A happens, B happens. But to speak of the probability of the event B, without naming the condition, really has no meaning at all. It is true that when it is perfectly obvious what condition is meant, the ellipsis may be permitted. But we should avoid contracting the habit of using language in this way (universal as the habit is), because it gives rise to a vague way of thinking, as if the action of causation might either determine an event to happen or determine it not to happen, or leave it more or less free to happen or not, so as to give rise to an \emph{inherent} chance in regard to its occurrence. It is quite clear to me that some of the worst and most persistent errors in the use of the doctrine of chances have arisen from this vicious mode of expression.\footnote{The conception of probability here set forth is substantially that first developed by Mr. Venn, in his ``Logic of Chance.'' Of course, a vague apprehension of the idea had always existed, but the problem was to make it perfectly clear, and to him belongs the credit of first doing this.}

\subsection*{IV}


But there remains an important point to he cleared up. According to what has been said, the idea of probability essentially belongs to a kind of inference which is repeated indefinitely. An individual inference must be either true or false, and can show no effect of probability; and, therefore, in reference to a single case considered in itself, probability can have no meaning. Yet if a man bad to choose between drawing a card from a pack containing twenty-five red cards and a black one, or from a pack containing twenty-five black cards and a red one, and if the drawing of a red card were destined to transport him to eternal felicity, and that of a black one to consign him to everlasting woe, it would be folly to deny that he ought to prefer the pack containing the larger proportion of red cards, although, from the nature of the risk, it could not be repeated. It is not easy to reconcile this with our analysis of the conception of chance. But suppose he should choose the red pack, and should draw the wrong card, what consolation would he have? He might say that he had acted in accordance with reason, but that would only show that his reason was absolutely worthless. And if he should choose the right card, how could he regard it as anything but a happy accident? He could not say that if he had drawn from the other pack, he might have drawn the wrong one, because an hypothetical proposition such as, ``if A, then B,'' means nothing with reference to a single case. Truth consists in the existence of a real fact corresponding to the true proposition. Corresponding to the proposition, ``if A, then B,'' there may be the fact that \emph{whenever} such an event as A happens such an event as B happens. But in the case supposed, which has no parallel as far as this man is concerned, there would be no real fact whose existence could give any truth to the statement that, if he had drawn from the other pack, he might have drawn a black card. Indeed, since the validity of an inference consists in the truth of the hypothetical proposition that \emph{if} the premises be true the conclusion will also be true, and since the only real fact which can correspond to such a proposition is that whenever the antecedent is true the consequent is so also, it follows that there can be no sense in reasoning in an isolated case, at all.

These considerations appear, at first sight, to dispose of the difficulty mentioned. Yet the case of the other side is not yet exhausted. Although probability will probably manifest its effect in, say, a thousand risks, by a certain proportion between the numbers of successes and failures, yet this, as we have seen, is only to say that it certainly will, at length, do so. Now the number of risks, the number of probable inferences, which a man draws in his whole life, is a finite one, and he cannot be absolutely \emph{certain} that the mean result will accord with the probabilities at all. Taking all his risks collectively, then, it cannot be certain that they will not fail, and his case does not differ, except in degree, from the one last supposed. It is an indubitable result of the theory of probabilities that every gambler, if he continues long enough, must ultimately be ruined. Suppose he tries the martingale, which some believe infallible, and which is, as I am informed, disallowed in the gambling-houses. In this method of playing, he first bets say \$1; if he loses it he bets \$2; if he loses that he bets \$4; if he loses that he bets \$8; if he then gains he has lost $1+2+4=7$, and he has gained \$1 more; and no matter how many bets he loses, the first one he gains will make him \$1 richer than he was in the beginning. In that way, he will probably gain at first; but, at last, the time will come when the run of luck is so against him that he will not have money enough to double, and must therefore let his bet go. This will \emph{probably} happen before he has won as much as he had in the first place, so that this run against him will leave him poorer than he began; some time or other it will be sure to happen. It is true that there is always a possibility of his winning any sum the bank can pay, and we thus come upon a celebrated paradox that, though he is certain to be ruined, the value of his expectation calculated according to the usual rules (which omit this consideration) is large. But, whether a gambler plays in this way or any other, the same thing is true, namely, that if plays long enough he will be sure some time to have such a run against him as to exhaust his entire fortune. The same thing is true of an insurance company. Let the directors take the utmost pains to be independent of great conflagrations and pestilences, their actuaries can tell them that, according to the doctrine of chances, the time must come, at last, when their losses will bring them to a stop. They may tide over such a crisis by extraordinary means, but then they will start again in a weakened state, and the same thing will happen again all the sooner. An actuary might be inclined to deny this, because he knows that the expectation of his company is large, or perhaps (neglecting the interest upon money) is infinite. But calculations of expectations leave out of account the circumstance now under consideration, which reverses the whole thing. However, I must not be understood as saying that insurance is on this account unsound, more than other kinds of business. All human affairs rest upon probabilities, and the same thing is true everywhere. If man were immortal he could be perfectly sure of seeing the day when everything in which he had trusted should betray his trust, and, in short, of coming eventually to hopeless misery. He would break down, at last, as every great fortune, as every dynasty, as every civilization does. In place of this we have death.

But what, without death, would happen to every man, with death must happen to some man. At the same time, death makes the number of our risks, of our inferences, finite, and so makes their mean result uncertain. The very idea of probability and of reasoning rests on the assumption that this number is indefinitely great. We are thus landed in the same difficulty as before, and I can see but one solution of it. It seems to me that we are driven to this, that logicality inexorably requires that our interests shall \emph{not} be limited. They must not stop at our own fate, but must embrace the whole community. This community, again, must not be limited, but must extend to all races of beings with whom we can come into immediate or mediate intellectual relation. It must reach, however vaguely, beyond this geological epoch, beyond all bounds. He who would not sacrifice his own soul to save the whole world, is, as it seems to me, illogical in all his inferences, collectively. Logic is rooted in the social principle.

To be logical men should not be selfish; and, in point of fact, they are not so selfish as they are thought. The willful prosecution of one's desires is a different thing from selfishness. The miser is not selfish; his money does him no good, and he cares for what shall become of it after his death. We are constantly speaking of \emph{our} possessions on the Pacific, and of \emph{our} destiny as a republic, where no personal interests are involved, in a way which shows that we have wider ones. We discuss with anxiety the possible exhaustion of coal in some hundreds of years, or the cooling-off of the sun in some millions, and show in the most popular of all religious tenets that we can conceive the possibility of a man's descending into hell for the salvation of his fellows.

Now, it is not necessary for logicality that a man should himself be capable of the heroism of self-sacrifice. It is sufficient that he should recognize the possibility of it, should perceive that only that man's inferences who has it are really logical, and should consequently regard his own as being only so far valid as they would be accepted by the hero. So far as he thus refers his inferences to that standard, he becomes identified with such a mind.

This makes logicality attainable enough. Sometimes we can personally attain to heroism. The soldier who runs to scale a wall knows that he will probably be shot, but that is not all he cares for. He also knows that if all the regiment, with whom in feeling he identifies himself, rush forward at once, the fort will be taken. In other cases we can only imitate the virtue. The man whom we have supposed as having to draw from the two packs, who if he is not a logician will draw from the red pack from mere habit, will see, if he is logician enough, that he cannot be logical so long as he is concerned only with his own fate, but that that man who should care equally for what was to happen in all possible cases of the sort could act logically, and would draw from the pack with the most red cards, and thus, though incapable himself of such sublimity, our logician would imitate the effect of that man's courage in order to share his logicality.

But all this requires a conceived identification of one's interests with those of an unlimited community. Now, there exist no reasons, and a later discussion will show that there can be no reasons, for thinking that the human race, or any intellectual race, will exist forever. On the other hand, there can be no reason against it;\footnote{I do not here admit an absolutely unknowable. Evidence could show us what would probably be the case after any given lapse of time; and though a subsequent time might be assigned which that evidence might not cover, yet further evidence would cover it.} and, fortunately, as the whole requirement is that we should have certain sentiments, there is nothing in the facts to forbid our having a hope, or calm and cheerful wish, that the community may last beyond any assignable date.

It may seem strange that I should put forward three sentiments, namely, interest in an indefinite community, recognition of the possibility of this interest being made supreme, and hope in the unlimited continuance of intellectual activity, as indispensable requirements of logic. Yet, when we consider that logic depends on a mere struggle to escape doubt, which, as it terminates in action, must begin in emotion, and that, furthermore, the only cause of our planting ourselves on reason is that other methods of escaping doubt fail on account of the social impulse, why should we wonder to find social sentiment presupposed in reasoning? As for the other two sentiments which I find necessary, they are so only as supports and accessories of that. It interests me to notice that these three sentiments seem to be pretty much the same as that famous trio of Charity, Faith, and Hope, which, in the estimation of St. Paul, are the finest and greatest of spiritual gifts. Neither Old nor New Testament is a text-book of the logic of science, but the latter is certainly the highest existing authority in regard to the dispositions of heart which a man ought to have.

\bigskip
\centerline{\emph{[ You can skip the rest for our purposes. ---P.D.M. ]}}

\subsection*{V}

Such average statistical numbers as the number of inhabitants per square mile, the average number of deaths per week, the number of convictions per indictment, or, generally speaking, the number of x's per y, where the x's are a class of things some or all of which are connected with another class of things, their y's, I term \emph{relative} numbers. Of the two classes of things to which a relative number refers, that one of which it is a number may be called its \emph{relate}, and that one per which the numeration is made may be called its \emph{correlate}.

Probability is a kind of relative number; namely, it is the ratio of the number of arguments of a certain genus which carry truth with them to the total number of arguments of that genus, and the rules for the calculation of probabilities are very easily derived from this consideration. They may all be given here, since they are extremely simple, and it is sometimes convenient to know something of the elementary rules of calculation of chances.

\paragraph{Rule I. Direct Calculation.} --- To calculate, directly, any relative number, say for instance the number of passengers in the average trip of a street-car, we must proceed as follows:

Count the number of passengers for each trip; add all these numbers, and divide by the number of trips. There are cases in which this rule may be simplified. Suppose we wish to know the number of inhabitants to a dwelling in New York. The same person cannot inhabit two dwellings. If he divide his time between two dwellings he ought to be counted a half-inhabitant of each. In this case we have only to divide the total number of the inhabitants of New York by the number of their dwellings, without the necessity of counting separately those which inhabit each one. A similar proceeding will apply wherever each individual relate belongs to one individual correlate exclusively. If we want the number of x's per y, and no x belongs to more than one y, we have only to divide the whole number of x's of y's by the number of y's. Such a method would, of course, fail if applied to finding the average number of street-car passengers per trip. We could not divide the total number of travelers by the number of trips, since many of them would have made many passages. To find the probability that from a given class of premises, A, a given class of conclusions, B, follow, it is simply necessary to ascertain what proportion of the times in which premises of that class are true, the appropriate conclusions are also true. In other words, it is the number of cases of the occurrence of both the events A and B, divided by the total number of cases of the occurrence of the event A.

\paragraph{Rule II. Addition of Relative Numbers.} --- Given two relative numbers having the same correlate, say the number of x's per y, and the number of z's per y; it is required to find the number of x's and z's together per y. If there is nothing which is at once an x and a z to the same y, the sum of the two given numbers would give the required number. Suppose, for example, that we had given the average number of friends that men have, and the average number of enemies, the sum of these two is the average number of persons interested in a man. On the other hand, it plainly would not do to add the average number of persons having constitutional diseases to the average number over military age, and to the average number exempted by each special cause from military service, in order to get the average number exempt in any way, since many are exempt in two or more ways at once.

This rule applies directly to probabilities. Given the probability that two different and mutually exclusive events will happen under the same supposed set of circumstances. Given, for instance, the probability that if A then B, and also the probability that if A then C, then the sum of these two probabilities is the probability that if A then either B or C, so long as there is no event which belongs at once to the two classes B and C.

\paragraph{Rule III. Multiplication of Relative Numbers.} --- Suppose that we have given the relative number of x's per y also the relative number of z's per x of y; or, to take a concrete example, suppose that we have given, first, the average number of children in families living in New York; and, second, the average number of teeth in the head of a New York child--- then the product of these two numbers would give the average number of children's teeth in a New York family. But this mode of reckoning will only apply in general under two restrictions. In the first place, it would not be true if the same child could belong to different families, for in that case those children who belonged to several different families might have an exceptionally large or small number of teeth, which would affect the average number of children's teeth in a family more than it would affect the average number of teeth in a child's head. In the second place, the rule would not be true if different children could share the same teeth, the average number of children's teeth being in that case evidently something different from the average number of teeth belonging to a child.

In order to apply this rule to probabilities, we must proceed as follows: Suppose that we have given the probability that the conclusion B follows from the premise A, B and A representing as usual certain classes of propositions. Suppose that we also knew the probability of an inference in which B should be the premise, and a proposition of a third kind, C, the conclusion. Here, then, we have the materials for the application of this rule. We have, first, the relative number of B's per A. We next should have the relative number of C's per B following from A. But the classes of propositions being so selected that the probability of C following from any B in general is just the same as the probability of C's following from one of those B's which is deducible from an A, the two probabilities may be multiplied together, in order to give the probability of C following from A. The same restrictions exist as before. It might happen that the probability that B follows from A was affected by certain propositions of the class B following from several different propositions of the class A. But, practically speaking, all these restrictions are of very little consequence, and it is usually recognized as a principle universally true that the probability that, if A is true, B is, multiplied by the probability that, if B is true, C is, gives the probability that, if A is true, C is.

There is a rule supplementary to this, of which great use is made. It is not universally valid, and the greatest caution has to be exercised in making use of it--- a double care, first, never to use it when it will involve serious error; and, second, never to fail to take advantage of it in cases in which it can be employed. This rule depends upon the fact that in very many cases the probability that C is true if B is, is substantially the same as the probability that C is true if A is. Suppose, for example, we have the average number of males among the children born in New York; suppose that we also have the average number of children born in the winter months among those born in New York. Now, we may assume without doubt, at least as a closely approximate proposition (and no very nice calculation would be in place in regard to probabilities), that the proportion of males among all the children born in New York is the same as the proportion of males born in summer in New York, and, therefore, if the names of all the children born during a year were put into an urn, we might multiply the probability that any name drawn would be the name of a male child by the probability that it would be the name of a child born in summer, in order to obtain the probability that it would be the name of a male child born in summer. The questions of probability, in the treatises upon the subject, have usually been such as relate to balls drawn from urns, and games of cards, and so on, in which the question of the \emph{independence} of events, as it is called--- that is to say, the question of whether the probability of C, under the hypothesis B, is the same as its probability under the hypothesis A, has been very simple; but, in the application of probabilities to the ordinary questions of life, it is often an exceedingly nice question whether two events may be considered as independent with sufficient accuracy or not. In all calculations about cards it is assumed that the cards are thoroughly shuffled, which makes one deal quite independent of another. In point of fact the cards seldom are, in practice, shuffled sufficiently to make this true; thus, in a game of whist, in which the cards have fallen in suits of four of the same suit, and are so gathered up, they will lie more or less in sets of four of the same suit, and this will be true even after they are shuffled. At least some traces of this arrangement will remain, in consequence of which the number of ``short suits,'' as they are called--- that is to say, the number of hands in which the cards are very unequally divided in regard to suits--- is smaller than the calculation would make it to be; so that, when there is a misdeal, where the cards, being thrown about the table, get very thoroughly shuffled, it is a common saying that in the hands next dealt out there are generally short suits. A few years ago a friend of mine, who plays whist a great deal, was so good as to count the number of spades dealt to him in 165 hands, in which the cards had been, if anything, shuffled better than usual. According to calculation, there should have been 85 of these hands in which my friend held either three or four spades, but in point of fact there were 94, showing the influence of imperfect shuffling.

According to the view here taken, these are the only fundamental rules for the calculation of chances. An additional one, derived from a different conception of probability, is given in some treatises, which if it be sound might be made the basis of a theory of reasoning. Being, as I believe it is, absolutely absurd, the consideration of it serves to bring us to the true theory; and it is for the sake of this discussion, which must be postponed to the next number, that I have brought the doctrine of chances to the reader's attention at this early stage of our studies of the logic of science.

