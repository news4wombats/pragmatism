\documentclass[12pt]{article}
\usepackage{hyperref,geometry,setspace}
\usepackage[]{accessibility}
\usepackage[symbol*,perpage]{footmisc}
\geometry{width=6in,height=9in,centering}
% changes font to TeX Gyre Schola (Century Schoolbook)
%\usepackage{tgschola}\usepackage[T1]{fontenc}
\usepackage{baskervald}

\begin{document}
\hypersetup{pdfinfo={Title={Transcendentalism}, Author={Theodore Parker}}, pdfborder = {0 0 0 0}}

\section*{Theodore Parker's ``Transcendentalism''}



\paragraph{A note about the author:}
Parker was born near Boston in 1810. As a young man, he was admitted to Harvard but could not afford to attend. He read the curriculum on his own and later attended the Harvard Divinity school.

From 1837 to 1845, he was minister of the West Roxbury Unitarian church, which had only 60 adult members. In the 1840s, his theological views caused some outcry--- but his parishioners stood by him.

In 1845 he began to preach in rented theatre space around Boston. His sermons were first attended by several hundred people. By 1852, attendance was 2000 and required the space of the Boston Music Hall. He continued preaching in this way until tuberculosis made him unable to preach in 1859.

He was an outspoken social activist. He vehemently opposed slavery, wrote in support of slave revolts, and was indicted for violating the Fugitive Slave Act in 1854. He also supported women's suffrage, temperance, and prison reform.

%``I do not pretend to understand the moral universe. The arc is a long one. My eye reaches but little ways. I cannot calculate the curve and complete the figure by experience of sight. I can divine it by conscience. And from what I see I am sure it bends toward justice.''

He died in 1860.




\paragraph{A note about the text:}
This was prepared by P.D. Magnus from a scan of a 1907 collection of Parker's writings:
\begin{quote}
Theodore Parker, \emph{The World of Matter and the Spirit of Man: Latest Discourses of Religion}. Edited with notes by George Willis Cooke. Boston: American Unitarian Association, 1907.
\end{quote}
The marginal numbers reflect the pagination in the Cooke collection.

Parker himself says at one place that this is a \emph{lecture}, but it is unclear exactly when it was written or who the intended audience was. Cooke's note regarding the text, which is given below as an introduction, only points back to an earlier note on the text which itself provides only approximate answers.

\section*{Introduction}

\marginpar{399}
It is not known when this lecture was written, nor is it known where or how often it was delivered. It was published in pamphlet form (24mo, 39 pages) about February 1, 1876, by the Free Religious Association, Boston. It was No. 4 of the ``Free Religious Tracts'' issued at that time. The ``preliminary note'' was probably written by William Channing Gannett, editor of the series, and was as follows:---

``Who were the `New England Transcendentalists,' and what was the new wine that filled them full of its enthusiasm, a generation ago? Ralph Waldo Emerson, George Ripley, Mr. Alcott, Margaret Fuller, Theodore Parker and the rest,--- many of our fathers can name their names, but it might be harder to answer the other question and tell what they believed. No one can tell that better than Theodore Parker himself, who was the Paul of the movement,--- its most doctrinal interpreter, its systematic theologian, its most ardent missionary. Much of him that would be welcomed by the world lies buried in the hieroglyphics of his manuscript, only readable by eyes that learned to love him in his life-time. This essay on transcendentalism in contrast with sensationalism, now for the first time printed, has been rescued from the burial, because it is so clear-cut an account of the two great rival philosophies as viewed by the band of which he was the champion; a champion not blind to the dangers of his own nor to the good achieved by the opposite system, but so thorough-going in his loyalty that even Bishop Berkeley and Jonathan Edwards stand to him for arch-sensationalists. 

\marginpar{400}
``Save a few slight amendments or omissions where some sentence is imperfect in the manuscript or past finding out, the lecture is printed as written in 1850 or thereabouts, with all the time-marks left in; the reference, for instance, to the politics then reigning in the North and South. As plain a time-mark, too, is the necessary estimate of the sensational philosophy by its earlier and barer, not its more recent and deepening statement. Probably a tract presenting the beliefs of this new school of sensationalism will also be issued by the Free Religious Association.''

The reference here is to the philosophy growing out of Darwinism and the theory of evolution.
% Parker's relations to this later sensationalism will be mentioned in these notes in considering his sermons on the revelations of matter and mind.
\ldots
The promised tract on  the newer sensationalism was not published by the Free Religious Association. 

\section*{Transcendentalism}

\marginpar{1}
The will is father to the deed, but the thought and sentiment are father and mother of the will. Nothing seems more impotent than a thought, it has neither hands nor feet,--- but nothing proves so powerful. The thought turns out a thing; its vice or virtue becomes manners, habits, laws, institutions; the abstraction becomes concrete; the most universal proposition is the most particular; and in the end it is the abstract thinker who is the most practical man and sets mills a-running and ships to sail. 

A change of ideas made all the difference between Catholic and Protestant, monarchical and democratic. You see that all things are first an idea in the mind, then a fact out of the mind. The architect, the farmer, the railroad-calculator, the founder of empires, has his temple, his farm, his railroad, or his empire, in his head as an idea before it is a fact in the world. As the thought is the thing becomes. Every idea bears fruit after its kind,--- the good, good; the bad, bad. Some few hundred years ago John Huss, Luther, Lord Bacon, Descartes said, We will not be ruled by authority in the church or the school, but by common sense and reason. That was nothing but an idea; but out of it has come the Protestant Reformation, the English Revolution, the American Revolution, the French Revolution, the cycle of Revolutions that fill up the year 1848. Yes, all the learned societies of Europe, all the Protestant churches, all the liberal governments,--- of\marginpar{2} Holland, England, France, Germany, America,--- have come of that idea. The old fellows in Galileo's time would not look through his telescope lest it should destroy the authorized theory of vision; they knew what they were about. So have all the old fellows known ever since who refuse to look through a new telescope, or even at it, but only talk against it. Once the Egyptian sculptors copied men into stone with their feet joined and their hands fixed to their sides. The copy indicated the immutableness of things in Egypt, where a mummy was the type of a man. A Greek sculptor separated the feet, as in life, illegally taking a live man for his type. The sculptor lost his head, for the government saw a revolution of the empire in this departure from the authorized type of man. Such is the power of ideas. The first question to ask of a civilized nation is. How do they think.'' what is their philosophy? 

Now it is the design of philosophy to explain the phenomena of the universe by showing their order, connection, cause, law, use and meaning. These phenomena are of two kinds or forms, as they belong to the material world--- facts of observation; and as they belong to the spiritual world--- facts of consciousness: facts without, and facts within. From these two forms of phenomena or facts there come two grand divisions of philosophy: the philosophy of outward things,--- physics; the philosophy of inward things,--- metaphysics. 

In the material world, to us, there are only facts. Man carries something thither, to wit, ideas. Thus the world has quite a different look; for he finds the facts without have a certain relation to the ideas within. The world is one thing to Newton's dog Dia\marginpar{3}mond, quite another to Newton himself. The dog saw only the facts and some of their uses; the philosopher saw therein the reflected image of his own ideas,--- saw order, connection, cause, law and meaning, as well as use. 

Now in the pursuit of philosophy there are two methods which may be followed, namely, the deductive and the inductive. 

\paragraph{I.} By the deductive the philosopher takes a certain maxim or principle, assumes it as a fact and therefrom deduces certain other maxims or principles as conclusions, as facts. But in the conclusions there must be nothing which is not in the primary fact else the conclusion does not conclude. All pure science is of this character--- geometry, algebra, arithmetic. $1+1 = 2$ is a maxim, let us suppose: $1000 + 1000 = 2000$ is one deduction from it; $25 {\times} 25 = 625$, another deduction. Thus the philosopher must be certain of the fact he starts from, of the method he goes by, and the conclusion he stops at is made sure of beforehand. 

The difficulty is that the philosopher often assumes his first fact, takes a fancy for a fact; then, though the method be right, the conclusion is wrong. For instance, Aristotle assumed this proposition,--- the matter of the sun is incorruptible; thence he deduced this fact, that the sun does not change, that its light and heat are constant quantities. The conclusion did not agree with observation, the theory with the facts. His first fact was not proved, could not be, was disproved. But when Galileo looked at the sun with a telescope he saw spots on the sun, movable spots. Aristotle's first fact turned out a fancy, so all conclusion from it. The Koran is written by the infallible inspiration of God, the Pope is infallible, the King can do no wrong, the\marginpar{4} People are always right,--- these are assumptions. If taken as truths, you see the conclusions which may be deduced therefrom,--- which have been. There is in God somewhat not wholly good, is an assumption which lies at the bottom of a good deal of theology, whence conclusions quite obvious are logically deduced,--- 1, Manicheism, God and the devil; 2, God and an evil never to be overcome. God is absolute good is another assumption from which the opposite deductions are to be made. The method of deduction is of the greatest value and cannot be dispensed with. 

\paragraph{II.} By the inductive method the philosopher takes facts, puts them together after a certain order, seen in nature or devised in his own mind, and tries to find a more comprehensive fact common to many facts, i.e., what is called a \emph{law}, which applies to many facts and so Is a general law, or to all facts and so Is a universal law. In the deductive method you pass from a universal fact to a particular fact; In the inductive, from the particular to the general. In the deductive process there is nothing in the conclusion which was not first In the premises; by the Inductive something new is added at every step. The philosopher is sifting in his own conjecture or thought in order to get at a general idea which takes in all the particular facts in the case and explains them. When this general idea and the facts correspond the Induction is correct. But it is as easy to arrive at a false conclusion by the inductive process as to assume a false maxim from which to make deductions. A physician's apprentice once visited his master's patient and found him dead, and reported the case accordingly. ``What killed him.'' said the old doctor. ``He died of eating a horse.'' ``Eating a horse!'' expostulated the man of experi\marginpar{5}ence; ``impossible! how do you know that?'' `` He did,'' said the inductive son of {\AE}sculapius, ``for I saw the saddle and bridle under the bed.'' Another, but a grown-up doctor, once gave a sick blacksmith a certain medicine; he recovered. ``\emph{Post hoc, ergo propter hoc},'' said the doctor, and tried the same drug on the next sick man, who was a shoemaker. The shoemaker died, and the doctor wrote down his induction: ``This drug will cure all sick blacksmiths, but kill all sick shoemakers. (Rule for phosphorus.).'' 

The inductive method is also indispensable in all the sciences which depend on observation or experiment. The process of induction is as follows: After a number of facts is collected, the philosopher looks for some one fact common to all and explanatory thereof. To obtain this he assumes a fact as a law, and applies it to the facts before him. This is an hypothesis. If it correspond to the facts, the hypothesis is true. Two great forms of error are noticeable in the history of philosophy: 1, the assumption of false maxims, whence deductions are to be made,--- the assumption of no-fact for a fact; 2, the making of false inductions from actual facts. In the first, a falsehood is assumed, and then falsehood deduced from it; in the second, from a truth falsehood is induced, and this new falsehood is taken as the basis whence other falsehoods are deduced. 

Pythagoras declared the sun was the centre of the planets which revolved about it; that was an hypothesis,--- guess-work, and no more. He could not compare the hypothesis with facts, so his hypothesis could not be proved or disproved. But long afterwards others made the comparison and confirmed the hypothesis. Kepler wished to find out what ratio the time of a planet's revolution bears to its distance from\marginpar{6} the sun. He formed an hypothesis,--- ``The time is proportionable to the distance,'' No, that did not agree with the facts. ``To the square of the distance?'' No. ``To the cube of the distance?'' No. 11The square of the time to the cube of the distance?'' This he found to be the case, and so he estabhshed his celebrated law,--- Kepler's third law. But he examined only a few planets: how should he know the law was universal? He could not learn that by induction. That would only follow from this postulate, ``The action of nature is always uniform,'' which is not an induction, nor a deduction, but an assumption. The inductive method alone never establishes a universal law, for it cannot transcend the particular facts in the hands of the philosopher. The axioms of mathematics are not learned by inductions, but assumed outright as self-evident. ``Kepler's third law is universal of all bodies moving about a centre,''--- now there are three processes by which that conclusion is arrived at: 1. The process of induction, by which the law is proved general and to apply to all the cases investigated. 2. A process of deduction from the doctrine or axiom, that the action of nature is always uniform. 3. That maxim is obtained by a previous process of assumption from some source or another. 

Such is the problem of philosophy, to explain the facts of the universe; such the two departments of philosophy, physics and metaphysics; such the two methods of inquiry, deductive and inductive; such are the two forms of error,--- the assumption of a false fact as the starting-point of deduction, the induction of a false fact by the inductive process. Now these methods are of use in each department of philosophy, indispensable in each, in physics and in metaphysics. 

\marginpar{7} 

This is the problem of metaphysics,--- to explain the facts of human consciousness. In metaphysics there are and have long been two schools of philosophers. The first is the sensational school. Its most important metaphysical doctrine is this: There is nothing in the intellect which was not first in the senses. Here ``intellect'' means the whole intellectual, moral, affectional and religious consciousness of man. The philosophers of this school claim to have reached this conclusion legitimately by the inductive method. It was at first an hypothesis; but after analyzing the facts of consciousness, interrogating all the ideas and sentiments and sensations of man, they say the hypothesis is proved by the most careful induction. They appeal to it as a principle, as a maxim, from which other things are deduced. They say that experience by one or more of the senses is the ultimate appeal in philosophy: all that I know is of sensational origin; the senses are the windows which let in all the light I have; the senses afford a sensation. I reflect upon this, and by reflection transform a sensation into an idea. An idea, therefore, is a transformed sensation. 

A school in metaphysics soon becomes a school in physics, in politics, ethics, religion. The sensational school has been long enough in existence to assert itself in each of the four great forms of human action. Let us see what it amounts to. 

\paragraph{I.} In physics. \textbf{1.} It does not afford us a certainty of the existence of the outward world. The sensationalist believes it, not on account of his sensational philosophy, but in spite of it; not by his philosophy, but by his common sense: he does not philosophically know it. While I am awake the senses give me various sensations, and I refer the sensations to an object out\marginpar{8} of me, and so perceive its existence. But while I am asleep the senses give me various sensations, and for the time I refer the sensations to an object out of me, and so perceive its existence,--- but when I awake it seems a dream. Now, if the senses deceive me in sleep, why not when awake? How can I \emph{know} philosophically the existence of the material world.'' With only the sensational philosophy I cannot! I can only \emph{know} the facts of consciousness. I cannot pass from ideas to things, from psychology to ontology. Indeed there is no ontology, and I am certain only of my own consciousness. Bishop Berkeley, a thorough sensationalist, comes up with the inductive method in his hand, and annihilates the outward material world, annihilates mankind, leaves me nothing but my own consciousness, and no consciousness of any certainty there. Dr. Priestley, a thorough sensationalist, comes up with the same inductive method in his hand, and annihilates the spiritual world, annihilates the soul. Berkeley, with illogical charity, left me the soul as an existence, but stripped me of matter; I was certain I had a soul, not at all sure of my body. Priestley, as illoglcally, left me the body as an existence, but stripped me of the soul. Both of these gentlemen I see were entirely in the right, if their general maxim be granted; and so, between the two, I am left pretty much without soul or sense! Soul and body are philosophically hurled out of existence! 

\paragraph{2.} From its hypothetical world sensationalism proceeds to the laws of matter; but it cannot logically get beyond its facts. Newton says, ``Gravitation prevails,--- its power diminishing as the square of the distance Increases between two bodies, so far as I have seen.'' ``Is it so where you have not seen?'' Newton\marginpar{9} don't know; he cannot pass from a general law to a universal law. As the existence of the world is hypothetical, so the universality of laws of the world is only hypothetical universality. The Jesuits who edited the Prinicipia were wise men when they published them as an hypothesis.

The sensational philosophy has prevailed chiefly in England; that is the home of its ablest representatives,--- Bacon, Locke. See the effect. England turns her attention to sciences that depend chiefly on observation, on experiment,--- botany, chemistry, the descriptive part of astronomy, zoology, geology. England makes observations on the tides, on variations of the magnetic needle, on the stars; fits out exploring expeditions; learns the facts; looks after the sources of the Nile, the Niger; hunts up the North Pole; tests the strength of iron, wood, gunpowder; makes improvements in all the arts. In mechanics. But in metaphysics she does nothing; in the higher departments of physics--- making comprehensive generalizations--- she does little. Even in mathematics, after Newton, for a hundred years England fell behind the rest of Europe. She is great at experiment, little at pure thinking. 

The sensational philosophy has no Idea of cause, except that of empirical connection in time and place; no idea of substance, only of body, or form of substance; no ontology, but phenomenology. It refers all questions--- say of the planets about the sun--- to an outward force: when they were made, God, standing outside, gave them a push and set them a-going; or else their motion is the result of a fortuitous concourse of atoms, a blind fate. Neither conclusion is a philosophical conclusion, each an hypothesis. Its physics\marginpar{10} are mere materialism; hence it delights in the atomistic theory of nature and repels the dynamic theory of matter. The sensationalist's physics appear well in a celebrated book, ``The Vestiges of the Natural History of Creation.''\footnote{An 1844 book published anonymously by the amateur geologist Robert Chambers.} The book has many valuable things in it, but the philosophy of its physics is an unavoidable result of sensationalism. There is nothing but materialism in his world. All is material, effects material, causes material, his God material,--- not surpassing the physical universe, but co-extensive therewith. In zoology life is the result of organization, but is an immanent life. In anthropology the mind is the result of organization, but is an immanent mind; in theology God is the result of organization, but is an immanent God. Life does not \emph{transcend} organization, nor does mind, nor God. All is matter. 

\paragraph{II.} In politics. Sensationalism knows nothing of absolute right, absolute justice; only of historical right, historical justice. ``There is nothing in the intellect which was not first in the senses.'' The senses by which we learn of justice and right are hearing and seeing. Do I reflect, and so get a righter right and juster justice than I have seen or heard of, it does me no good, for ``nothing is in the intellect which was not in the senses.'' Thus absolute justice is only a whim, a no-thing, a dream. Men that talk of absolute justice, absolute right, are visionary men.

In politics, sensationalism knows nothing of ideas, only of facts; ``the only lamp by which its feet are guided is the lamp of experience.'' All its facts are truths of observation, not of necessity. ``There is no right but might,'' is the political philosophy of sensationalism. It may be the might of a king, of an aristocracy, of a democracy, the might of passions, the\marginpar{11} might of intellect, the might of muscle,--- it has a right to what it will. It appeals always to human history, not human nature. Now human history shows what has been, not what should be or will be. To reason about war it looks not to the natural justice, only to the cost and present consequences. To reason about free trade or protection, it looks not to the natural justice or right of mankind, but only to the present expediency of the thing. Political expediency is the only right or justice it knows in its politics. So it always looks back, and says ``it worked well at Barcelona or Venice,'' or ``did not work well.'' It loves to cite precedents out of history, not laws out of nature. It claims a thing not as a human right, but as an historical privilege received by Magna Charta or the Constitution; as if a right were more of a right because time-honored and written on parchment; or less, because just claimed and for the first time and by a single man. The sensationalist has no confidence in ideas, so asks for facts to hold on to and to guide him in his blindness. Said a governor in America, ``The right of suffrage is universal.'' `` How can that be,'' said a sensationalist, ``when the Constitution of the state declares that certain persons shall not vote.'' He knew no rights before they became constitutional, no rights but vested rights,--- perhaps none but ``\emph{in}vested.''

The sensationalists in politics divide into two parties, each with the doctrine that in politics ``might makes right.'' One party favors the despotism of the few,--- is an oligarchy; or of the one,--- is a monarchy. Hence the doctrine is, ``The king can do no wrong.'' All power is his; he may delegate it to the people as a privilege; it is not theirs by right, by nature, and\marginpar{12} his as a trust. He has a right to make any laws he will, not merely any just laws. The people must pay passive obedience to the king, he has eminent domain over them. The celebrated Thomas Hobbes is the best representative of this party, and has one great merit,--- of telling what he thought. 

The other party favors the despotism of the many,--- is a democracy. The doctrine is, ``The people can do no wrong.'' The majority of the people have the right to make any laws they will, not merely any just laws; and the minority must obey, right or wrong. You must not censure the measures of the majority, you afford ``aid and comfort to the enemy.'' The state has absolute domain over the citizen, the majority over the minority; this holds good of the voters, and of any political party in the nation. For the majority has power of its own right, for its own behoof; not in trust, and for the good of all and each! The aim of sensational politics is the greatest good of the greatest number; this may be obtained by sacrificing the greatest good of the lesser number,--- by sacrificing any individual,--- or sacrificing absolute good. In No-man's-land this party prevails: the dark -haired men, over forty million,--- the red-haired, only three million five hundred thousand,--- the dark-haired enslave the red-haired for the greatest good of the greatest number. But in a hundred years the red-haired men are most numerous, and turn round and enslave the black-haired. 

Thomas Paine is a good representative of this party; so is Marat, Robespierre, the author of the ``Systeme de la Nature.'' In the old French Revolution you see the legitimate consequence of this doctrine, that might makes right, that there is no absolute justice,\marginpar{13} in the violence, the murder, the wholesale assassination. The nation did to masses, and in the name of democracy, what all kings had done to the nation and in the name of monarchy,--- sought the greatest good of the controlling power at the sacrifice of an opponent. It is the same maxim which in cold blood hangs a single culprit, enslaves three million negroes, and butchers thousands of men as in the September massacres. The sensational philosophy established the theory that might makes right,--- and the mad passions of a solitary despot, or a million-headed mob, made it a fact. Commonly the two parties unite by a compromise, and then it consults not the greatest good of its king alone, as in a brutal, pure monarchy; not of the greatest number, as in a pure and brutal democracy; but the greatest good of a class,--- the nobility and gentry in England, the landed proprietors and rich burghers in Switzerland, the slaveholders in South Carolina. Voltaire is a good representative of this type of sensational politics, not to come nearer home. In peaceful times England shares the defect of the sensational school in politics. Her legislation is empirical; great ideas do not run through her laws; she loves a precedent better than a principle; appeals to an accidental fact of human history, not an essential fact of human nature which is prophetic. Hence legislative politics is not a great science which puts the facts of human consciousness into a state, making natural justice common law; nothing but a poor dealing with precedents, a sort of national housekeeping and not very thrifty housekeeping. In our own nation you see another example of the same,--- result of the same sensational philosophy. There is no right, says Mr. Calhoun, but might; the white man has that, so the black man\marginpar{14} is his political prey. And Mr. Polk tells us that Vermont, under the Constitution, has the same right to establish slavery as Georgia to abolish it. 

\paragraph{III.} In ethics. Ethics are the morals of the individual; politics of the mass. The sensationalist knows no first truths in morals; the source of maxims in morals is experience; in experience there is no absolute right. Absolute justice, absolute right, were never in the senses, so not in the intellect; only whimsies, words in the mouth. The will is not free, but wholly conditioned, in bondage; character made always for you, not by you. The intellect is a smooth table; the moral power a smooth table; and experience writes there what she will, and what she writes is law of morality. Morality is expediency, nothing more; nothing is good of itself, right of itself, just of itself,--- but only because it produces agreeable consequences, which are agreeable sensations. Dr. Paley is a good example of the sensational moralist. I ask him ``What is right, just?'' He says, ``There are no such things; they are the names to stand for what works well in the long run.'' ``How shall I know what to do in a matter of morals? by referring to a moral sense?'' ``Not at all: only by common sense, by observation, by experience, by learning what works well in the long run; by human history, not human nature. To make a complete code of morals by sensationalism you must take the history of mankind, and find what has worked well, and follow that because it worked well.'' ``But human history only tells what has been and worked well, not what is right. I want what is right!'' He answers, ``It is pretty much the same thing.'' ``But suppose the first men endowed with faculties perfectly developed, would they know what to do?'' ``Not at all. Instinct\marginpar{15} would tell the beast antecedent to experience, but man has no moral instinct, must learn only by actual trial.'' ``Well,'' say I, ``let alone that matter, let us come to details. What is honesty?'' ``It is the best policy.'' ``Why must I tell the truth, keep my word, be chaste, temperate?'' ``For the sake of the reward, the respect of your fellows, the happiness of a long life and heaven at last. On the whole God pays well for virtue; though slow pay, he is sure.'' ``But suppose the devil paid the better pay?'' ``Then serve him, for the end is not the service, but the pay. Virtue, and by virtue I mean all moral excellence, is not a good in itself, but good as producing some other good.'' ``Why should I be virtuous?'' ``For the sake of the reward.'' ``But vice has its rewards, they are present and not future, immediate and certain, not merely contingent and mediate. I should think them greater than the reward of virtue.'' Then vice to you is virtue, for it pays best. The sensational philosophy knows no conscience to sound in the man's ears the stern word. Thou oughtest so to do, come what will come!

In politics might makes right, so in morals. Success is the touchstone; the might of obtaining the reward the right of doing the deed. Bentham represents the sensational morals of politics; Paley of ethics. Both are Epicureans. The sensationalist and the Epicurean agree in this,--- enjoyment is the touchstone of virtue and determines what is good, what bad, what indifferent: this is the generic agreement. Heathen Epicurus spoke only of enjoyment in this life; Christian Archdeacon Paley--- and a very \emph{arch}deacon--- spoke of enjoyment also in the next: this is the specific difference. In either case virtue ceases to be virtue, for it is only a bargain. 

\marginpar{16}

There is a school of sensationalists who turn off and say, ``Oh, you cannot answer the moral questions and tell what is right, just, fair, good. We must settle that by revelation.'' That, of course, only adjourns the question and puts the decision on men who received the revelation or God who made it. They do not meet the philosopher's question; they assume that the difference between right and wrong is not knowable by human faculties, and, if there be any difference between right and wrong, there is no faculty in man which naturally loves right and abhors wrong, still less any faculty which can find out what \emph{is} right, what wrong. So all moral questions are to be decided by authority, because somebody said so; not by reference to facts of consciousness, but to phenomena of history. Of course the moral law is not a law which is of me, rules in me and by me; only one put on me, which rules over me! Can any lofty virtue grow out of this theory? any heroism? Verily not. Regulus did not ask a reward for his virtue; if so, he made but a bargain, and who would honor him more than a desperate trader who made a good speculation.'' There is something in man which scoffs at expediency; which will do right, justice, truth, though hell itself should gape and bid him hold his peace; the morality which anticipates history, loves the right for itself. Of this Epicurus knew nothing, Paley nothing, Bentham nothing, sensationalism nothing. Sensationalism takes its standard of political virtue from the House of Commons; of right from the Constitution and common law; of commercial virtue from the board of brokers at their best, and the old bankrupt law; or virtue in general from the most comfortable classes of society, from human history, not human nature; and knows nothing more. The virtue\marginpar{17} of a Regulus, of a Socrates, of a Christ, it knows not. 

See the practical effect of this. A young man goes into trade. Experience meets him with the sensationalist morals in its hand, and says, {''}'\emph{Caveat emptor}, Let the buyer look to it, not you;' you must be righteous, young man, but not righteous overmuch; you must tell the truth to all who have the right to ask you, and when and where they have a right to ask you,--- otherwise you may lie. The mistake is not in lying, or deceit; but in lying and deceiving to your own disadvantage. You must not set up a private conscience of your own in your trade, you will lose the confidence of respectable people. You must have a code of morals which works well and produces agreeable sensations in the long run. To learn the true morals of business you must not ask conscience, that is a whim and very unphilosophical. You must ask, How did Mr. Smith make his money.'' He cheated, and so did Mr. Brown and Mr. Jones, and they cheat all round. Then you must do the same, only be careful not to cheat so as to `hurt your usefulness' and `injure your reputation.'{''} 

Shall I show the practical effects of this, not on very young men, in politics? It would hurt men's feeling and I have no time for that. 

\paragraph{IV.} In religion. Sensationalism must have a philosophy of religion, a theology; let us see what theology. There are two parties; one goes by philosophy, the other mistrusts philosophy. 

\paragraph{1.} The first thing in theology is to know God. The idea of God is the touchstone of a theologian. Now to know the existence of God is to be certain thereof as of my own existence. ``Nothing in the intellect which was not first in the senses,'' says sensationalism; ``all comes by sensational experience and reflection thereon.''\marginpar{18} Sensationalism--- does that give us the idea of God? I ask the sensationalist, ``Does the sensational eye see God?'' ``No.'' ``The ear hear him?'' ``No.'' ``Do the organs of sense touch or taste him?'' ``No.'' ``How then do you get the idea of God?'' ``By induction from facts of observation \emph{a posteriori}. The senses deal with finite things; I reflect on them, put them all together I assume that they have \emph{cause}; then by the inductive method I find out the character of that cause: that is God.'' Then I say, ``But the senses deal with only finite things, so you must infer only a finite maker, else the induction is imperfect. So you have but a finite God. Then these finite things, measured only by my experience, are Imperfect things. Look at disorders in the frame of nature; the sufferings of animals, the miseries of men; here are seeming imperfections which the sensational philosopher staggers at. But to go on with this induction: from an imperfect work you must infer an imperfect author. So the God of sensationalism is not only finite, but imperfect even at that. But am I certain of the existence of the finite and imperfect God? The existence of the outward world is only an hypothesis, its laws hypothetical; all that depends on that or them is but an hypothesis,--- the truth of your faculties, the forms of matter only an hypothesis: so the existence of God is not a certainty; he is but our hypothetical God. But a hypothetical God is no God at all, not the living God: an imperfect God is no God at all, not the true God: a finite God is no God at all, not the absolute God. But this hypothetical, finite, imperfect God, where is he? In matter? No. In spirit? No. Does he act in matter or spirit? No, only now and then he did act by miracle; he is outside of the world of matter and\marginpar{19} spirit. Then he is a non-residcnt, an absentee. A non-resident God is no God at all, not the all-present God.'' 

The above is the theory on which Mr. Hume constructs his notion of God with the sensational philosophy, the inductive method; and he arrives at the hypothesis of a God, of a finite God, of an imperfect God, of a non-resident God. Beyond that the sensational philosophy as philosophy cannot go. 

But another party comes out of the same school to treat of religious matters; they give their philosophy a vacation, and to prove the existence of God they go back to tradition, and say, ``Once God revealed himself to the senses of men; they heard him, they saw him, they felt him; so to them the existence of God was not an induction, but a fact of observation; they told it to others, through whom it comes to us; we can say it is not a fact of observation but a fact of testimony.'' 

``Well,'' I ask, ``are you certain then?'' ``Yes.'' ``Quite sure.? Let me look. The man to whom God revealed himself may have been mistaken; it may have been a dream, or a whim of his own, perhaps a fib; at any rate, he was not philosophically certain of the existence of the outward world in general; how could he be of anything that took place in it? Next, the evidence which relates the transaction is not wholly reliable: how do I know the books which tell of it tell the truth, that they were not fabricated to deceive me? All that rests on testimony is a little uncertain if it took place one or two thousand years ago; especially if I know nothing about the persons who testify or of that whereof they testify; still more so if it be a thing, as you say, unphilosophical and even supernatural.'' 

So, then, the men who give a vacation to their\marginpar{20} philosophy have slurred the philosophical argument for a historical, the theological for the mythological, and have gained nothing except the tradition of God. By this process we are as far from the infinite God as before, and have only arrived at the same point where the philosophy left us. 

The English Deists and the Socinians and others have approached religion with the sensational philosophy in their hands; we are to learn of God philosophically only by induction. And such is their God. They tell us that God is not knowable; the existence of God is not a certainty to us; it is a probability, a credibility, a possibility,--- a certainty to none. You ask of sensationalism, the greatest question, ``Is there a God?'' Answer: ``Probably.'' ``What is his character?'' ``Finite, imperfect.'' ``Can I trust him?'' ``If we consult tradition it is creditable; if philosophy, possible.'' 

\paragraph{2.} The next great question in theology is that of the immortality of the soul. That is a universal hope of mankind; what does it rest on? Can I know my immortality? Here are two wings of the sensational school. The first says, ``No, you cannot know it; it is not true. Mind, soul, are two words to designate the result of organization. Man is not a mind, not a soul, not a free will. Man is a body, with blood, brains, nerves--- nothing more; the organization gone, all is gone.'' Now that is sound, logical, consistent; that was the conclusion of Hume, of many of the English Deists, and of many French philosophers in the last century; they looked the fact in the face. But mortality, annihilation, is rather an ugly fact to look fairly in the face; but Mr. Hume and others have done it, and died brave with the sensational philosophy. 

\marginpar{21} 

The other wing of the sensational school gives its philosophy another vacation, rests the matter not on philosophy but history; not on the theological but the mythological argument; on authority of tradition asserting a phenomenon of human history, they try to establish the immortality of man by a single precedent, a universal law by the tradition of a single, empirical, contingent phenomenon. 

But I ask the sensational philosopher, ``Is immortality certain?'' ``No.'' ``Probable?'' ``No.'' ``Credible?'' ``No.'' ``Possible?'' ``Barely.'' I ask the traditional division, ``Is immortality certain?'' ``No, it is left uncertain to try your faith.'' ``Is it probable?'' ``Yes, there is one witness in six thousand years, one out of ten million times ten million.'' ``Well, suppose it is probable; is immortality, if it be sure to be a good thing, for me, for mankind?'' ``Not at all! There is nothing in the nature of man, nothing in the nature of the world, nothing in the nature of God to make you sure immortality will prove a blessing to mankind in general, to yourself in special!'' 

\paragraph{3.} That is not quite all. Sensationalism does not allow freedom of the will; I say not, absolute freedom--- that belongs only to God,--- but it allows no freedom of the will. See the result: all will is God's, all willing therefore is equally divine, and the worst vice of Pantheism follows. ``But what is the will of God, is that free?'' ``Not at all; man is limited by the organization of his body, God by the organization of the universe.'' So God is not absolute God, not absolutely free; and as man's will is necessitated by God's, so God's will by the universe of matter; and only a boundless fate and pitiless encircles man and God. 

\marginpar{22}

This is the philosophy of sensationaHsm; such its doctrine in physics, politics, ethics, religion. It leads to boundless uncertainty. Berkeley resolves the universe into subjective ideas; no sensationalist knows a law in physics to be universal. Hobbes and Bentham and Condillac in politics know of no right but might; Priestley denies the spirituality of man, Collins and Edwards his liberty; Dodwell affirms the materiality of the soul, and the mortality of all men not baptized; Mandeville directly, and others indirectly, deny all natural distinction between virtue and vice; Archdeacon Paley knows no motive but expediency. 

The materialist is puzzled with the existence of matter; finds its laws general, not universal. The sensational philosophy meets the politician and tells him through Rousseau and others, ``Society has no divine original, only the social compact; there is no natural justice, natural right; no right, but might; no greater good than the greatest good of the greatest number, and for that you may sacrifice all you will; to defend a constitution is better than to defend justice.'' In morals the sensational philosophy meets the young man and tells him all is uncertain; he had better be content with things as they are, himself as he is; to protest against a popular wrong is foolish, to make money by it, or ease, or power, is a part of wisdom; only the fool is wise above what is written. It meets the young minister with its proposition that the existence of God is not a certainty, nor the immortality of the soul; that religion is only traditions of the elders and the keeping of a form. It says to him, ``Look there, Dr. Humdrum has got the tallest pulpit and the quietest pews, the fattest living and the cosiest nook in all the land; how do you think he won it? Why, by letting well\marginpar{23}-enough alone; he never meddles with sin; it would break his heart to hurt a sinner's feelings,--- he might lose a parishioner; he never dreams to make the world better, or better off. Go thou and do likewise.'' 

I come now to the other school. This is distinguished by its chief metaphysical doctrine, that there is in the intellect (or consciousness), something that never was in the senses, to wit, the intellect (or consciousness) itself; that man has faculties which transcend the senses; faculties which give him ideas and intuitions that transcend sensational experience; ideas whose origin is not from sensation, nor their proof from sensation. This is the transcendental school. They maintain that the mind (meaning thereby all which is not sense) is not a smooth tablet on which sensation writes its experience, but is a living principle which of itself originates ideas when the senses present the occasion; that, as there is a body with certain senses, so there is a soul or mind with certain powers which give the man sentiments and ideas. This school maintains that it is a fact of consciousness itself that there is in the intellect somewhat that was not first in the senses; and also that they have analyzed consciousness, and by the inductive method established the conclusion that there is a consciousness that never was sensation, never could be; that our knowledge is in part \emph{a priori}; that we know, 1, certain truths of necessity; 2, certain truths of intuition, or spontaneous consciousness; 3, certain truths of demonstration, a voluntary consciousness; all of these truths not dependent on sensation for cause, origin, or proof. Facts of observation, sensational experience, it has in common with the other school.

\marginpar{24} 

Transcendentalism, also, reports itself in the four great departments of human activity--- in physics, politics, ethics, religion. 

\paragraph{I.} In physics it starts with the maxim that the senses acquaint us actually with body, and therefrom the mind gives us the idea of substance, answering to an objective reality. Thus is the certainty of the material world made sure of. Then \emph{a priori} it admits the uniformity of the action of nature; and its laws are \emph{a priori} known to be universal, and not general alone. These two doctrines it finds as maxims resulting from the nature of man, facts given. Then it sets out with other maxims, first truths, which are facts of necessity, known to be such without experience. All the first truths of mathematics are of this character, \emph{e.g.}, that the whole is greater than a part. From these, by the deductive method, it comes at other facts,--- facts of demonstration; these also are transcendental, that is, transcend the senses, transcend the facts of observation. For example, the three angles of a triangle are equal to two right angles,--- that is universally true; it is a fact of demonstration, and is a deduction from a first truth which is self-evident, a fact of necessity. But here the fact of demonstration transcends the fact of experience, philosophy is truer than sensation. The whole matter of geometry is transcendental. 

Transcendentalism does not take a few facts out of human history and say they are above nature; all that appears in nature it looks on as natural, not supernatural, not subternatural; so the distinction between natural and supernatural does not appear. By this means philosophy is often in advance of observation; \emph{e.g.}, Newton's law of gravitation, Kepler's third law, the theory that a diamond might be burned, and\marginpar{25} Berkeley's theory of vision,--- these are interpretations of nature, but also anticipations of nature, as all true philosophy must be. Those men, however, did not philosophically know it to be so. So by an actual law of nature, not only are known facts explained, but the unknown anticipated. 

Evils have come from the transcendental method in physics; men have scorned observation, have taken but a few facts from which to learn universal laws, and so failed of getting what is universal, even general. They have tried to divine the constitution of the world, to do without sensational experience in matters where knowledge depends on that and that is the \emph{sine qu\^{a} non}. The generalizations of the transcendental naturalists have been often hasty; they attempt to determine what nature shall be, not to learn what nature is. Thus a famous philosopher said there are only seven primary planets in the solar system, and from the nature of things, \emph{a priori} known, it is impossible there should be more. He had intelligence in advance of the mail; but the mail did not confirm, for six months afterwards Dr. Piazzi discovered one of the asteroids; and in a few years three more were found, and now several more have been discovered, not to mention the new planet Neptune. Many of the statements of Schelling in physics are of this same character. 

\paragraph{II.} In politics, transcendentalism starts not from experience alone, but from consciousness; not merely from human history, but also from human nature. It does not so much quote precedents, contingent facts of experience, as ideas, necessary facts of consciousness. It only quotes the precedent to obtain or illustrate the idea. It appeals to a natural justice, natural right; absolute justice, absolute right. Now the source and\marginpar{26} original of this justice and right it finds in God--- the conscience of God; the channel through which we receive this justice and right is our own moral sense, our conscience, which is our consciousness of the conscience of God. This conscience in politics and in ethics transcends experience, and \emph{a priori} tells us of the just, the right, the good, the fair; not the relatively right alone, but absolute right also. As it transcends experience, so it anticipates history; and the ideal justice of conscience is juster than the empirical and contingent justice actually exercised at Washington or at Athens, as the ideal circle is rounder than one the stone-cutter scratches on his rough seal. In transcendental politics the question of expediency is always subordinate to the question of natural right; it asks not merely about the cost of a war, but its natural justice. It aims to organize the ideals of man's moral and social nature into political institutions; to have a government which shall completely represent the facts of man's social consciousness so far as his nature is now developed. But as this development is progressive, so must government be; yet not progressive by revolution, by violence; but by harmonious development, progressive by growth. The transcendental politician does not merely interpret history, and look back to Magna Charta and the Constitution; but into human nature, through to divine nature; and so anticipates history, and in man and God finds the origin and primary source of all just policy, all right legislation. So looking he transcends history. 

For example, the great political idea of America, the idea of the Declaration of Independence, is a composite idea made up of three simple ones: 1. Each man is endowed with certain unalienable rights. 2. In respect\marginpar{27} of these rights all men are equal. 3. A government is to protect each man in the entire and actual enjoyment of all the unalienable rights. Now the first two ideas represent ontological facts, facts of human consciousness; they are facts of necessity. The third is an idea derived from the two others, is a synthetic judgment \emph{a priori}; it was not learned from sensational experience; there never was a government which did this, nor is there now. Each of the other ideas transcended history: every unalienable right has been alienated, still is; no two men have been actually equal in actual rights. Yet the idea is true, capable of proof by human nature, not of verification by experience; as true as the proposition that three angles of a triangle are equal to two right angles; but no more capable of a sensational proof than that. The American Revolution, with American history since, is an attempt to prove by experience this transcendental proposition, to organize the transcendental idea of politics. The idea demands for its organization a democracy--- a government of all, for all, and by all; a government by natural justice, by legislation that is divine as much as a true astronomy is divine, legislation which enacts law representing a fact of the universe, a resolution of God. 

All human history said, ``That cannot be.'' Human nature said, ``It can, must, shall.'' The authors of the American Revolution, as well as the fathers of New England, were transcendentalists to that extent. America had such faith in the idea that she made the experiment in part. She will not quite give up yet. But there is so much of the sensational philosophy in her politics that in half the land the attempt is not made at all, the composite idea is denied, each of the simple\marginpar{28} ideas is also denied; and in the otlier half it is but poorly made. 

In France men have an idea yet more transcendental; to the intellectual idea of liberty, and the moral idea of equality, they add the religious idea of fraternity, and so put politics and all legislation on a basis divine and incontestable as the truths of mathematics. They say that rights and duties are before all human laws and above all human laws. America says, ``The Constitution of the United States is above the President, the Supreme Court above Congress.'' France says, ``The Constitution of the Universe is above the Constitution of France.'' Forty million people say that. It transcends experience. The grandest thing a nation ever said in history. 

The transcendental politician does not say that might makes right, but that there is an immutable morality for nations as for men. Legislation must represent that, or the law is not binding on any man. By birth man is a citizen of the universe, subject to God; no oath of allegiance, no king, no parliament, no congress, no people, can absolve him from his natural fealty thereto, and alienate a man born to the rights, bom to the duties, of a citizen of God's universe. Society, government, politics come not from a social compact which men made and may unmake, but from a social nature of God's making; a nation is to be self-ruled by justice. In a monarchy, the king holds power as a trust, not a right: in a democracy, the people have it as a right, the majority as a trust; but the minority have lost no right, can alienate none, delegate none beyond power of ultimate recall. A nation has a right to make just laws, binding because just. Justice is the point common to one man and the world of men, the balance-\marginpar{29}point. A nation is to seek the greatest good of all, not of the greatest number; not to violate the constitution of the universe, not sacrifice the minority to the majority, nor one single man to the whole. But over all human law God alone has eminent domain. 

Here too is a danger: the transcendental politician may seek to ignore the past, and scorn its lessons; may take his own personal whims for oracles of human nature; and so he may take counsel from the selfishness of lazy men against the selfishness of active men, counsel from the selfishness of poor men against the selfishness of rich men, and think he hears the voice of justice, or the reverse, as himself is rich or poor, active or idle; there is danger that he be rash and question as hastily in politics as in physics, and reckon without his host, to find that the scot is not free when the day of reckoning comes. 

\paragraph{III.} In ethics. Transcendentalism affirms that man has moral faculties which lead him to justice and right, and by his own nature can find out what is right and j ust, and can know it and be certain of it. Right is to be done come what will come. I am not answerable for the consequences of doing right, only of not doing it, only of doing wrong. The conscience of each man is to him the moral standard; so to mankind is the conscience of the race. In morals conscience is complete and reliable as the eye for colors, the ear for sounds, the touch and taste for their purposes. While experience shows what has been or is, conscience shows what should be and shall. 

Transcendental ethics look not to the consequences of virtue, in this life or the next, as motive, therefore, to lead men to virtue. That is itself a good, an absolute good, to be loved not for what it brings, but is. It\marginpar{30} represents the even poise or balance-point between individual and social development. To know what is right, I need not ask what is the current practice, what say the Revised Statutes, what said holy men of old, but what says conscience? what, God? The common practice, the Revised Statutes, the holy men of old are helps, not masters. I am to be co-ordinate with justice. 

Conscience transcends experience, and not only explains but anticipates that, and the transcendental system of morals is to be founded on human nature and absolute justice. 

I am to respect my own nature and be an individual man,--- your nature and be a social man. Truth is to be told and asked, justice done and demanded, right claimed and allowed, affection felt and received. The will of man is free; not absolutely free as God's, but partially free, and capable of progress to yet higher degrees of freedom. 

Do you ask an example of a transcendental moralist? A scheme of morals was once taught to mankind wholly transcendental, the only such scheme that I know. In that was no alloy of expediency, no deference to experience, no crouching behind a fact of human history to hide from ideas of human nature; a scheme of morals which demands that you be you--- I, I; balances individualism and socialism on the central point of justice; which puts natural right, natural duty, before all institutions, all laws, all traditions. You will pardon me for mentioning the name of Jesus of Nazareth in a lecture. But the whole of human history did not justify his ethics; only human nature did that. Hebrew ethics, faulty in detail, were worse in method and principle, referring all to an outward command, not\marginpar{31} an inward law. Heathen ethics less faulty in detail, not less in principles, referred all to experience and expediency, knew only what was, and what worked well here or there; not what ought to be, and worked well anywhere and forever. He transcended that, taught what should be, must, shall, and forever.

The danger is that the transcendental moralist shall too much abhor the actual rules of morality; where much is bad and ill-founded, shall deem all worthless. Danger, too, that he take a transient impulse, personal and fugitive, for a universal law; follow a passion for a principle, and come to naught; surrender his manhood, his free will to his unreflecting instinct, become subordinate thereto. Men that are transcendental-mad we have all seen in morals; to be transcendental-wise, sober, is another thing. The notion that every impulse is to be followed, every instinct totally obeyed, will put man among the beasts, not angels. 

\paragraph{IV.} In religion. Transcendentalism admits a religious faculty, element, or nature in man, as it admits a moral, intellectual and sensational faculty,--- that man by nature is a religious being as well as moral, intellectual, sensational; that this religious faculty is adequate to its purposes and wants, as much so as the others, as the eye acquainting us with light; and that this faculty is the source of religious emotions, of the sentiments of adoration, worship. Through this we have consciousness of God as through the senses consciousness of matter. In connection with reason it gives us the primary ideas of religion, ideas which transcend experience. 

Now the transcendental philosophy legitimates the ideas of religion by reference to human nature. Some of them it finds truths of necessity, which cannot be\marginpar{32} conceived of as false or unreal without violence to reason; some it finds are truths of consciousness,--- of spontaneous consciousness, or intuition; some, truths of voluntary consciousness, or demonstration, inductive or deductive. Such ideas, capable of this legitimation, transcend experience, require and admit no further proof; as true before experience as after; true before time, after time, eternally; absolutely true. On that rock transcendentalism founds religion, sees its foundation, and doubts no more of religious truths than of the truths of mathematics. All the truths of religion it finds can be verified in consciousness to-day, what cannot is not religion. But it does not neglect experience. In human history it finds confirmations, illustrations, of the ideas of human nature, for history represents the attempt of mankind to develop human nature. So then as transcendentalism in philosophy legitimates religion by a reference to truths of necessity, to truths of consciousness, it illustrates religion by facts of observation, facts of testimony. 

By sensationalism religious faith is a belief, more or less strange, in a probability, a credibility, a possibility. By transcendentalism religious faith is the normal action of the whole spiritual nature of man, which gives him certain knowledge of a certainty not yet attainable by experience; where understanding ends, faith begins, and out-travels the understanding. Religion is natural to man, is justice, piety--- free justice, free piety, free thought. The form thereof should fit the individual; hence there will be a unity of substance, diversity of form. So a transcendental religion demands a transcendental theology. 

\paragraph{1.} The transcendental philosophy appears in its doctrine of God. The idea of God is a fact given in the\marginpar{33} consciousness of man; consciousness of the infinite is the condition of a consciousness of the finite. I learn of a finite thing by sensation, I get an idea thereof; at the same time the idea of the infinite unfolds in me. I am not conscious of my own existence except as a finite existence, that is, as a dependent existence; and the idea of the infinite, of God on whom I depend, comes at the same time as the logical correlative of a knowledge of myself. So the existence of God is a certainty; I am as certain of that as of my own existence. Indeed without that knowledge I know nothing. Of this I am certain,--- I am; but of this as certain,--- God is; for if I am, and am finite and dependent, then this presupposes the infinite and independent. So the idea of God is \emph{a priori}; rests on facts of necessity, on facts of consciousness. 

Then transcendentalism uses the other mode, the \emph{a posteriori}. Starting with the infinite, it finds signs and proofs of him everywhere, and gains evidence of God's existence in the limits of sensational observation; the thing refers to its maker, the thought to the mind, the eff*ect to the cause, the created to the creator, the finite to the infinite; at the end of my arms are two major prophets, ten minor prophets, each of them pointing the transcendental philosopher to the infinite God, of which he has consciousness without the logical process of induction. 

Then the character of God as given in the idea of him, given in consciousness,--- that represents God as a being, not with the limitations of impersonality (that is to confound God with matter); not with the limitations of personality (that confounds him with man); but God with no limitations, infinite, absolute; looked at from sensation, infinite power; from thought, in\marginpar{34}finite intellect; from the moral sense, infinite conscience; from the emotional, infinite affection; from the religious, infinite soul; from all truth, the whole human nature names him Infinite Father! 

God is immanent in matter, so it is; immanent in spirit, so it is. He acts also as God in matter and spirit, acts perfectly; laws of matter or of spirit are modes of God's acting, being; as God is perfect, so the mode of his action is perfect and unchangeable. Therefore, as God is ever in matter and spirit, and where God is is wholly God active, so no intervention is possible. God cannot come where he already is, so no miracle is possible. A miracle \emph{a parte human\^{a}} is a violation of what is a law to man; a miracle to God--- \emph{a parte divin\^{a}}--- is a violation of what is law to God: the most extraordinary things that have been seem miracles \emph{a parte human\^{a}},--- laws, \emph{a parte divin\^{a}}. But though God is immanent in matter and in spirit, he yet transcends both matter and spirit, has no limitations. Indeed all perfection of immanence and transcendence belong to him,--- the perfection of existence, infinite being; the perfection of space, immensity; the perfection of time, eternity; of power, all-mightiness; of mind, all-knowingness; of affection, all-lovingness; of will, absolute freedom, absolute justice, absolute right. His providence is not merely general, but universal, so special in each thing. Hence the universe partakes of his perfection, is a perfect universe for the end he made it for. 

\paragraph{2.} The doctrine of the soul. This teaches that man by nature is immortal. This doctrine it legitimates: 1. By reference to facts of consciousness that men feel In general; in the heart it finds the longing after immortality, in the mind the idea of immortality, in religious\marginpar{35} consciousness the faith in immortality, in human nature altogether the strong confidence and continued trust therein. 2. It refers also to the nature of God, and reasons thus: God is all-powerful and can do the best; all-wise, and can know it; all-good, and must will it; immortality is the best, therefore it is. All this anticipates experience \emph{a priori}. 3. It refers to the general arrangements of the world, where everything gets ripe, matures, but man. In the history of mankind it finds confirmation of this doctrine, for every rude race and all civilized tribes have been certain of immortality; but here and there are men, sad and unfortunate, who have not by the mind legitimated the facts of spontaneous consciousness, whose nature the sensational philosophy has made blind, and they doubt or deny what nature spontaneously affirms. 

The nature of God being such, he immanent and active in matter and spirit; the nature of man such, so provided with faculties to love the true, the just, the fair, the good,--- it follows that man is capable of inspiration from God, communion with God; not in raptures, not by miracle, but by the sober use of all his faculties, moral, intellectual, affectional, religious. The condition thereof is this, the faithful use of human nature, the coincidence of man's will with God's. Inspiration is proportionate to the man's quantity of being, made up of a constant and a variable, his quantity of gifts, his quantity of faithful use. In this way transcendentalism can legitimate the highest inspiration, and explain the genius of God's noblest son, not as monstrous, but natural. In religion as in all things else there has been a progressive development of mankind. The world is a school, prophets, saints, saviours, men more eminently gifted and faithful, and so most\marginpar{36} eminently inspired,--- they are the school-masters to lead men up to God. 

There is danger in this matter also lest the transcendental religionist should despise the past and its sober teachings, should take a fancy personal and fugitive for a fact of universal consciousness, embrace a cloud for an angel, and miserably perish. It is not for man to transcend his faculties, to be above himself, above reason, conscience, affection, religious trust. It is easy to turn off from these and be out of reason, conscience, affection, religion--- beside himself. Madmen in religion are not rare, enthusiasts, fanatics. 

The sensational philosophy, with all its evils, has done the world great service. It has stood up for the body, for common sense, protested against spiritual tyranny, against the spiritualism of the middle ages which thought the senses wicked and the material world profane. To sensationalism we are indebted for the great advance of mankind in physical science, in discovery, arts, mechanics, and for many improvements in government. Some of its men are great names,--- Bacon, Locke, Newton. Let us do them no dishonor; they saw what they could, told it; they saw not all things that are, saw some which are not. In our day no one of them would be content with the philosophy they all agreed in then. Hobbes and Hume have done us service; the Socinians, Priestley, Collins, Berkeley, Dodwell, Mandeville, Edwards. To take the good and leave the ill is our part; but the doubts which this philosophy raises, the doubt of Hume, the doubt of Hobbes, of the English Deists in general, do not get answered by this philosophy. For this we have weapons forged by other hands, tempered in another spring. 

\marginpar{37}

Transcendentalism has a work to do, to show that physics, politics, ethics, religion rest on facts of necessity, facts of intuition, facts of demonstration, and have their witness and confirmation in facts of observation. It is the work of transcendentalism to give us politics which represent God's thought of a state,--- the whole world, each man free; to give us morals which leave the man a complete individual, no chord rent from the human harp,--- yet complete in his social character, no string discordant in the social choir; to give us religion worthy of God and man,--- free goodness, free piety, free thought. That is not to be done by talking at random, not by idleness, not by railing at authority, calumniating the past or the present; not by idle brains with open mouth, who outrage common sense; but by diligent toil, brave discipline, patience to wait, patience to work. Nothing comes of nothing, foolishness of fools; but something from something, wise thought from thinking men; and of the wise thought comes a lovely deed, life, laws, institutions for mankind. 

The problem of transcendental philosophy is no less than this, to revise the experience of mankind and try its teachings by the nature of mankind; to test ethics by conscience, science by reason; to try the creeds of the churches, the constitutions of the states by the constitution of the universe; to reverse what is wrong, supply what is wanting, and command the just. To do this in a nation like ours, blinded still by the sensational philosophy, devoted chiefly to material interests, its politics guided by the madness of party more than sober reason; to do this in a race like the Anglo-Saxon, which has an obstinate leaning to a sensational philosophy, which loves facts of experience, not ideas of consciousness, and believes not in the First-Fair, First-\marginpar{38}Perfect, First-Good, is no light work; not to be taken in hand by such as cannot bear the strife of tongues, the toil, the heat, the war of thought; not to be accomplished by a single man, however well-born and wellbred; not by a single age and race. It has little of history behind, for this philosophy is young. It looks to a future, a future to be made; a church whose creed is truth, whose worship love; a society full of industry and abundance, full of wisdom, virtue, and the poetry of life; a state with unity among all, with freedom for each; a church without tyranny, a society without ignorance, want, or crime, a state without oppression; yes, a world with no war among the nations to consume the work of their hands, and no restrictive policy to hinder the welfare of mankind. That is the human dream of the transcendental philosophy. Shall it ever become a fact? History says, No; human nature says, Yes. 


\end{document}