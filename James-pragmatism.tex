\documentclass[]{article}
\usepackage{hyperref,geometry,setspace}
\usepackage[symbol*,perpage]{footmisc}
\geometry{width=6in,height=9in,centering}
\begin{document}
%\renewcommand*{\thefootnote}{\fnsymbol{footnote}}
%\setcounter{footnote}{0}
\hypersetup{pdfinfo={Title={Pragmatism}, Author={William James}}, pdfborder = {0 0 0 0}}

\title{Pragmatism:\\ A New Name for Some Old Ways of Thinking}
\author{William James}
\date{}
\maketitle


\begin{center}
To the Memory of John Stuart Mill

from whom I first learned the pragmatic openness of mind and whom my
fancy likes to picture as our leader were he alive to-day.
\end{center}


\setstretch{1.2}

\section*{Preface}

The lectures that follow were delivered at the Lowell Institute in
Boston in November and December, 1906, and in January, 1907, at
Columbia University, in New York. They are printed as delivered, without
developments or notes.\footnote{This version of lectures 1--2 was prepared by P.D. Magnus from \href{http://www.gutenberg.org/ebooks/5116}{the Project Gutenberg edition}.} The pragmatic movement, so-called--- I do not like
the name, but apparently it is too late to change it--- seems to have
rather suddenly precipitated itself out of the air. A number of
tendencies that have always existed in philosophy have all at once
become conscious of themselves collectively, and of their combined
mission; and this has occurred in so many countries, and from so many
different points of view, that much unconcerted statement has resulted.
I have sought to unify the picture as it presents itself to my own eyes,
dealing in broad strokes, and avoiding minute controversy. Much futile
controversy might have been avoided, I believe, if our critics had been
willing to wait until we got our message fairly out.

If my lectures interest any reader in the general subject, he will
doubtless wish to read farther. I therefore give him a few references.

\medskip

\ldots [James cites works by John Dewey, F.C.S. Schiller, and others. ---P.D.M.]

\medskip
%In America, John Dewey's `Studies in Logical Theory' are the foundation. Read also by Dewey the articles in the Philosophical Review, vol. xv, pp. 113 and 465, in Mind, vol. xv, p. 293, and in the \emph{Journal of Philosophy}, vol. iv, p. 197.

%Probably the best statements to begin with however, are F. C. S. Schiller's in his `Studies in Humanism,' especially the essays numbered i, v, vi, vii, xviii and xix. His previous essays and in general the polemic literature of the subject are fully referred to in his footnotes.

%Furthermore, see G. Milhaud: le Rationnel, 1898, and the fine articles by Le Roy in the \emph{Revue de Metaphysique}, vols. 7, 8 and 9. Also articles by Blondel and de Sailly in the \emph{Annales de Philosophie Chretienne}, 4me Serie, vols. 2 and 3. Papini announces a book on \emph{Pragmatism}, in the French language, to be published very soon.

To avoid one misunderstanding at least, let me say that there is no logical connexion between pragmatism, as I understand it, and a doctrine
which I have recently set forth as `radical empiricism.' The latter
stands on its own feet. One may entirely reject it and still be a
pragmatist.

~\hfill Harvard University, April, 1907.


\section*{Lecture I: The Present Dilemma in Philosophy}

In the preface to that admirable collection of essays of his called
\emph{Heretics}, Mr. Chesterton writes these words: ``There are some
people--- and I am one of them--- who think that the most practical and
important thing about a man is still his view of the universe. We think
that for a landlady considering a lodger, it is important to know his
income, but still more important to know his philosophy. We think that
for a general about to fight an enemy, it is important to know
the enemy's numbers, but still more important to know the enemy's
philosophy. We think the question is not whether the theory of the
cosmos affects matters, but whether, in the long run, anything else
affects them.''

I think with Mr. Chesterton in this matter. I know that you, ladies and
gentlemen, have a philosophy, each and all of you, and that the most
interesting and important thing about you is the way in which it
determines the perspective in your several worlds. You know the same
of me. And yet I confess to a certain tremor at the audacity of the
enterprise which I am about to begin. For the philosophy which is so
important in each of us is not a technical matter; it is our more or
less dumb sense of what life honestly and deeply means. It is only
partly got from books; it is our individual way of just seeing and
feeling the total push and pressure of the cosmos. I have no right to
assume that many of you are students of the cosmos in the class-room
sense, yet here I stand desirous of interesting you in a philosophy
which to no small extent has to be technically treated. I wish to fill
you with sympathy with a contemporaneous tendency in which I profoundly
believe, and yet I have to talk like a professor to you who are not
students. Whatever universe a professor believes in must at any rate be
a universe that lends itself to lengthy discourse. A universe definable
in two sentences is something for which the professorial intellect has
no use. No faith in anything of that cheap kind! I have heard friends
and colleagues try to popularize philosophy in this very hall, but they
soon grew dry, and then technical, and the results were only partially
encouraging. So my enterprise is a bold one. The founder of pragmatism
himself recently gave a course of lectures at the Lowell Institute with
that very word in its title-flashes of brilliant light relieved
against Cimmerian darkness! None of us, I fancy, understood \emph{all} that he
said--- yet here I stand, making a very similar venture.

I risk it because the very lectures I speak of \emph{drew}--- they brought good
audiences. There is, it must be confessed, a curious fascination in
hearing deep things talked about, even tho neither we nor the disputants
understand them. We get the problematic thrill, we feel the presence of
the vastness. Let a controversy begin in a smoking-room anywhere, about
free-will or God's omniscience, or good and evil, and see how everyone
in the place pricks up his ears. Philosophy's results concern us all
most vitally, and philosophy's queerest arguments tickle agreeably our
sense of subtlety and ingenuity.

Believing in philosophy myself devoutly, and believing also that a kind
of new dawn is breaking upon us philosophers, I feel impelled, \emph{per fas
aut nefas},\footnote{[Latin for `by good or ill'; that is, `by any means'. ---P.D.M.]} to try to impart to you some news of the situation.

Philosophy is at once the most sublime and the most trivial of human
pursuits. It works in the minutest crannies and it opens out the widest
vistas. It `bakes no bread,' as has been said, but it can inspire our
souls with courage; and repugnant as its manners, its doubting and
challenging, its quibbling and dialectics, often are to common people,
no one of us can get along without the far-flashing beams of light it
sends over the world's perspectives. These illuminations at least, and
the contrast-effects of darkness and mystery that accompany them, give
to what it says an interest that is much more than professional.

The history of philosophy is to a great extent that of a certain clash
of human temperaments. Undignified as such a treatment may seem to some
of my colleagues, I shall have to take account of this clash and explain
a good many of the divergencies of philosophers by it. Of whatever
temperament a professional philosopher is, he tries when philosophizing
to sink the fact of his temperament. Temperament is no conventionally
recognized reason, so he urges impersonal reasons only for his
conclusions. Yet his temperament really gives him a stronger bias than
any of his more strictly objective premises. It loads the evidence
for him one way or the other, making for a more sentimental or a more
hard-hearted view of the universe, just as this fact or that principle
would. He trusts his temperament. Wanting a universe that suits it, he
believes in any representation of the universe that does suit it.
He feels men of opposite temper to be out of key with the world's
character, and in his heart considers them incompetent and `not in
it,' in the philosophic business, even tho they may far excel him in
dialectical ability.

Yet in the forum he can make no claim, on the bare ground of his
temperament, to superior discernment or authority. There arises thus a
certain insincerity in our philosophic discussions: the potentest of
all our premises is never mentioned. I am sure it would contribute to
clearness if in these lectures we should break this rule and mention it,
and I accordingly feel free to do so.

Of course I am talking here of very positively marked men, men
of radical idiosyncracy, who have set their stamp and likeness on
philosophy and figure in its history. Plato, Locke, Hegel, Spencer,
are such temperamental thinkers. Most of us have, of course, no
very definite intellectual temperament, we are a mixture of opposite
ingredients, each one present very moderately. We hardly know our own
preferences in abstract matters; some of us are easily talked out of
them, and end by following the fashion or taking up with the beliefs of
the most impressive philosopher in our neighborhood, whoever he may be.
But the one thing that has \emph{counted} so far in philosophy is that a man
should see things, see them straight in his own peculiar way, and be
dissatisfied with any opposite way of seeing them. There is no reason
to suppose that this strong temperamental vision is from now onward to
count no longer in the history of man's beliefs.

Now the particular difference of temperament that I have in mind
in making these remarks is one that has counted in literature, art,
government and manners as well as in philosophy. In manners we find
formalists and free-and-easy persons. In government, authoritarians and
anarchists. In literature, purists or academicals, and realists. In art,
classics and romantics. You recognize these contrasts as familiar; well,
in philosophy we have a very similar contrast expressed in the pair of
terms `rationalist' and `empiricist,' `empiricist' meaning your lover of
facts in all their crude variety, `rationalist' meaning your devotee to
abstract and eternal principles. No one can live an hour without both
facts and principles, so it is a difference rather of emphasis; yet it
breeds antipathies of the most pungent character between those who
lay the emphasis differently; and we shall find it extraordinarily
convenient to express a certain contrast in men's ways of taking their
universe, by talking of the `empiricist' and of the `rationalist'
temper. These terms make the contrast simple and massive.

More simple and massive than are usually the men of whom the terms are
predicated. For every sort of permutation and combination is possible in
human nature; and if I now proceed to define more fully what I have in
mind when I speak of rationalists and empiricists, by adding to each
of those titles some secondary qualifying characteristics, I beg you to
regard my conduct as to a certain extent arbitrary. I select types
of combination that nature offers very frequently, but by no means
uniformly, and I select them solely for their convenience in helping
me to my ulterior purpose of characterizing pragmatism. Historically we
find the terms `intellectualism' and `sensationalism' used as synonyms
of `rationalism' and `empiricism.' Well, nature seems to combine most
frequently with intellectualism an idealistic and optimistic tendency.
Empiricists on the other hand are not uncommonly materialistic, and
their optimism is apt to be decidedly conditional and tremulous.
Rationalism is always monistic. It starts from wholes and universals,
and makes much of the unity of things. Empiricism starts from the parts,
and makes of the whole a collection-is not averse therefore to calling
itself pluralistic. Rationalism usually considers itself more religious
than empiricism, but there is much to say about this claim, so I merely
mention it. It is a true claim when the individual rationalist is what
is called a man of feeling, and when the individual empiricist prides
himself on being hard-headed. In that case the rationalist will usually
also be in favor of what is called free-will, and the empiricist will
be a fatalist--- I use the terms most popularly current. The rationalist
finally will be of dogmatic temper in his affirmations, while the
empiricist may be more sceptical and open to discussion.

I will write these traits down in two columns. I think you will
practically recognize the two types of mental make-up that I mean if
I head the columns by the titles `tender-minded' and `tough-minded'
respectively.

\medskip

\begin{minipage}{.45\linewidth}
\textsc{The Tender-Minded}\par
Rationalistic (going by `principles')\\
Intellectualistic\\
Idealistic\\
Optimistic\\
Religious\\
Free-willist\\
Monistic\\
Dogmatical
\end{minipage}\hfill
\begin{minipage}{.45\linewidth}
\textsc{The Tough-Minded}\par
Empiricist (going by `facts')\\ Sensationalistic\\ Materialistic\\
Pessimistic\\ Irreligious\\ Fatalistic\\ Pluralistic\\ Sceptical
\end{minipage}

\medskip

Pray postpone for a moment the question whether the two contrasted
mixtures which I have written down are each inwardly coherent and
self-consistent or not--- I shall very soon have a good deal to say on
that point. It suffices for our immediate purpose that tender-minded and
tough-minded people, characterized as I have written them down, do both
exist. Each of you probably knows some well-marked example of each type,
and you know what each example thinks of the example on the other side
of the line. They have a low opinion of each other. Their antagonism,
whenever as individuals their temperaments have been intense, has formed
in all ages a part of the philosophic atmosphere of the time. It forms a
part of the philosophic atmosphere to-day. The tough think of the tender
as sentimentalists and soft-heads. The tender feel the tough to be
unrefined, callous, or brutal. Their mutual reaction is very much like
that that takes place when Bostonian tourists mingle with a population
like that of Cripple Creek. Each type believes the other to be inferior
to itself; but disdain in the one case is mingled with amusement, in the
other it has a dash of fear.

Now, as I have already insisted, few of us are tender-foot Bostonians
pure and simple, and few are typical Rocky Mountain toughs, in
philosophy. Most of us have a hankering for the good things on both
sides of the line. Facts are good, of course--- give us lots of facts.
Principles are good--- give us plenty of principles. The world is
indubitably one if you look at it in one way, but as indubitably is
it many, if you look at it in another. It is both one and many--- let us
adopt a sort of pluralistic monism. Everything of course is necessarily
determined, and yet of course our wills are free: a sort of free-will
determinism is the true philosophy. The evil of the parts is undeniable;
but the whole can't be evil: so practical pessimism may be combined with
metaphysical optimism. And so forth--- your ordinary philosophic layman
never being a radical, never straightening out his system, but living
vaguely in one plausible compartment of it or another to suit the
temptations of successive hours.

But some of us are more than mere laymen in philosophy. We are worthy
of the name of amateur athletes, and are vexed by too much inconsistency
and vacillation in our creed. We cannot preserve a good intellectual
conscience so long as we keep mixing incompatibles from opposite sides
of the line.

And now I come to the first positively important point which I wish to
make. Never were as many men of a decidedly empiricist proclivity in
existence as there are at the present day. Our children, one may say,
are almost born scientific. But our esteem for facts has not neutralized
in us all religiousness. It is itself almost religious. Our scientific
temper is devout. Now take a man of this type, and let him be also a
philosophic amateur, unwilling to mix a hodge-podge system after the
fashion of a common layman, and what does he find his situation to be,
in this blessed year of our Lord 1906? He wants facts; he wants
science; but he also wants a religion. And being an amateur and not an
independent originator in philosophy he naturally looks for guidance to
the experts and professionals whom he finds already in the field. A
very large number of you here present, possibly a majority of you, are
amateurs of just this sort.

Now what kinds of philosophy do you find actually offered to meet your
need? You find an empirical philosophy that is not religious enough, and
a religious philosophy that is not empirical enough for your purpose.
If you look to the quarter where facts are most considered you find
the whole tough-minded program in operation, and the `conflict between
science and religion' in full blast. Either it is that Rocky Mountain
tough of a Haeckel with his materialistic monism, his ether-god and his
jest at your God as a `gaseous vertebrate'; or it is Spencer treating
the world's history as a redistribution of matter and motion solely, and
bowing religion politely out at the front door:--- she may indeed continue
to exist, but she must never show her face inside the temple. For a
hundred and fifty years past the progress of science has seemed to mean
the enlargement of the material universe and the diminution of man's
importance. The result is what one may call the growth of naturalistic
or positivistic feeling. Man is no law-giver to nature, he is an
absorber. She it is who stands firm; he it is who must accommodate
himself. Let him record truth, inhuman tho it be, and submit to it! The
romantic spontaneity and courage are gone, the vision is materialistic
and depressing. Ideals appear as inert by-products of physiology; what
is higher is explained by what is lower and treated forever as a case of
`nothing but'--- nothing but something else of a quite inferior sort. You
get, in short, a materialistic universe, in which only the tough-minded
find themselves congenially at home.

If now, on the other hand, you turn to the religious quarter for
consolation, and take counsel of the tender-minded philosophies, what do
you find?

Religious philosophy in our day and generation is, among us
English-reading people, of two main types. One of these is more radical
and aggressive, the other has more the air of fighting a slow retreat.
By the more radical wing of religious philosophy I mean the so-called
transcendental idealism of the Anglo-Hegelian school, the philosophy of
such men as Green, the Cairds, Bosanquet, and Royce. This philosophy has
greatly influenced the more studious members of our protestant ministry.
It is pantheistic, and undoubtedly it has already blunted the edge of
the traditional theism in protestantism at large.

That theism remains, however. It is the lineal descendant, through one
stage of concession after another, of the dogmatic scholastic theism
still taught rigorously in the seminaries of the catholic church. For a
long time it used to be called among us the philosophy of the Scottish
school. It is what I meant by the philosophy that has the air of
fighting a slow retreat. Between the encroachments of the hegelians and
other philosophers of the `Absolute,' on the one hand, and those of the
scientific evolutionists and agnostics, on the other, the men that
give us this kind of a philosophy, James Martineau, Professor Bowne,
Professor Ladd and others, must feel themselves rather tightly squeezed.
Fair-minded and candid as you like, this philosophy is not radical
in temper. It is eclectic, a thing of compromises, that seeks a \emph{modus
vivendi} above all things. It accepts the facts of darwinism, the facts
of cerebral physiology, but it does nothing active or enthusiastic with
them. It lacks the victorious and aggressive note. It lacks \emph{prestige} in
consequence; whereas absolutism has a certain \emph{prestige} due to the more
radical style of it.

These two systems are what you have to choose between if you turn to the
tender-minded school. And if you are the lovers of facts I have
supposed you to be, you find the trail of the serpent of rationalism, of
intellectualism, over everything that lies on that side of the line. You
escape indeed the materialism that goes with the reigning empiricism;
but you pay for your escape by losing contact with the concrete parts
of life. The more absolutistic philosophers dwell on so high a level
of abstraction that they never even try to come down. The absolute mind
which they offer us, the mind that makes our universe by thinking it,
might, for aught they show us to the contrary, have made any one of a
million other universes just as well as this. You can deduce no single
actual particular from the notion of it. It is compatible with any state
of things whatever being true here below. And the theistic God is almost
as sterile a principle. You have to go to the world which he has created
to get any inkling of his actual character: he is the kind of god that
has once for all made that kind of a world. The God of the theistic
writers lives on as purely abstract heights as does the Absolute.
Absolutism has a certain sweep and dash about it, while the usual theism
is more insipid, but both are equally remote and vacuous. What you want
is a philosophy that will not only exercise your powers of intellectual
abstraction, but that will make some positive connexion with this actual
world of finite human lives.

You want a system that will combine both things, the scientific
loyalty to facts and willingness to take account of them, the spirit of
adaptation and accommodation, in short, but also the old confidence in
human values and the resultant spontaneity, whether of the religious or
of the romantic type. And this is then your dilemma: you find the two
parts of your quaesitum hopelessly separated. You find empiricism with
inhumanism and irreligion; or else you find a rationalistic philosophy
that indeed may call itself religious, but that keeps out of all
definite touch with concrete facts and joys and sorrows.

I am not sure how many of you live close enough to philosophy to realize
fully what I mean by this last reproach, so I will dwell a little longer
on that unreality in all rationalistic systems by which your serious
believer in facts is so apt to feel repelled.

I wish that I had saved the first couple of pages of a thesis which
a student handed me a year or two ago. They illustrated my point so
clearly that I am sorry I cannot read them to you now. This young man,
who was a graduate of some Western college, began by saying that he had
always taken for granted that when you entered a philosophic class-room
you had to open relations with a universe entirely distinct from the one
you left behind you in the street. The two were supposed, he said, to
have so little to do with each other, that you could not possibly occupy
your mind with them at the same time. The world of concrete personal
experiences to which the street belongs is multitudinous beyond
imagination, tangled, muddy, painful and perplexed. The world to which
your philosophy-professor introduces you is simple, clean and noble.
The contradictions of real life are absent from it. Its architecture is
classic. Principles of reason trace its outlines, logical necessities
cement its parts. Purity and dignity are what it most expresses. It is a
kind of marble temple shining on a hill.

In point of fact it is far less an account of this actual world than
a clear addition built upon it, a classic sanctuary in which the
rationalist fancy may take refuge from the intolerably confused and
gothic character which mere facts present. It is no \emph{explanation} of our
concrete universe, it is another thing altogether, a substitute for it,
a remedy, a way of escape.

Its temperament, if I may use the word temperament here, is utterly
alien to the temperament of existence in the concrete. \emph{refinement} is
what characterizes our intellectualist philosophies. They exquisitely
satisfy that craving for a refined object of contemplation which is so
powerful an appetite of the mind. But I ask you in all seriousness to
look abroad on this colossal universe of concrete facts, on their awful
bewilderments, their surprises and cruelties, on the wildness which
they show, and then to tell me whether `refined' is the one inevitable
descriptive adjective that springs to your lips.

Refinement has its place in things, true enough. But a philosophy that
breathes out nothing but refinement will never satisfy the empiricist
temper of mind. It will seem rather a monument of artificiality. So we
find men of science preferring to turn their backs on metaphysics as on
something altogether cloistered and spectral, and practical men shaking
philosophy's dust off their feet and following the call of the wild.

Truly there is something a little ghastly in the satisfaction with which
a pure but unreal system will fill a rationalist mind. Leibnitz was
a rationalist mind, with infinitely more interest in facts than
most rationalist minds can show. Yet if you wish for superficiality
incarnate, you have only to read that charmingly written `Theodicee' of
his, in which he sought to justify the ways of God to man, and to prove
that the world we live in is the best of possible worlds. Let me quote a
specimen of what I mean.

Among other obstacles to his optimistic philosophy, it falls to Leibnitz
to consider the number of the eternally damned. That it is infinitely
greater, in our human case, than that of those saved he assumes as a
premise from the theologians, and then proceeds to argue in this way.
Even then, he says:

``The evil will appear as almost nothing in comparison with the good, if
we once consider the real magnitude of the City of God. Coelius Secundus
Curio has written a little book, \emph{De Amplitudine Regni Coelestis}, which
was reprinted not long ago. But he failed to compass the extent of the
kingdom of the heavens. The ancients had small ideas of the works of
God. ... It seemed to them that only our earth had inhabitants, and even
the notion of our antipodes gave them pause. The rest of the world for
them consisted of some shining globes and a few crystalline spheres.
But to-day, whatever be the limits that we may grant or refuse to the
Universe we must recognize in it a countless number of globes, as big
as ours or bigger, which have just as much right as it has to support
rational inhabitants, tho it does not follow that these need all be men.
Our earth is only one among the six principal satellites of our sun. As
all the fixed stars are suns, one sees how small a place among visible
things our earth takes up, since it is only a satellite of one among
them. Now all these suns \emph{may} be inhabited by none but happy creatures;
and nothing obliges us to believe that the number of damned persons is
very great; for a \emph{very few instances and samples suffice for the utility
which good draws from evil}. Moreover, since there is no reason to
suppose that there are stars everywhere, may there not be a great space
beyond the region of the stars? And this immense space, surrounding all
this region, ... may be replete with happiness and glory. ... What now
becomes of the consideration of our Earth and of its denizens? Does it
not dwindle to something incomparably less than a physical point, since
our Earth is but a point compared with the distance of the fixed stars.
Thus the part of the Universe which we know, being almost lost in
nothingness compared with that which is unknown to us, but which we
are yet obliged to admit; and all the evils that we know lying in this
almost-nothing; it follows that the evils may be almost-nothing in
comparison with the goods that the Universe contains.''

Leibnitz continues elsewhere: ``There is a kind of justice which aims
neither at the amendment of the criminal, nor at furnishing an example
to others, nor at the reparation of the injury. This justice is founded
in pure fitness, which finds a certain satisfaction in the expiation
of a wicked deed. The Socinians and Hobbes objected to this punitive
justice, which is properly vindictive justice and which God has reserved
for himself at many junctures. ... It is always founded in the fitness
of things, and satisfies not only the offended party, but all wise
lookers-on, even as beautiful music or a fine piece of architecture
satisfies a well-constituted mind. It is thus that the torments of the
damned continue, even tho they serve no longer to turn anyone away from
sin, and that the rewards of the blest continue, even tho they confirm
no one in good ways. The damned draw to themselves ever new penalties
by their continuing sins, and the blest attract ever fresh joys by their
unceasing progress in good. Both facts are founded on the principle of
fitness, ... for God has made all things harmonious in perfection as I
have already said.''

Leibnitz's feeble grasp of reality is too obvious to need comment from
me. It is evident that no realistic image of the experience of a damned
soul had ever approached the portals of his mind. Nor had it occurred to
him that the smaller is the number of `samples' of the genus `lost-soul'
whom God throws as a sop to the eternal fitness, the more unequitably
grounded is the glory of the blest. What he gives us is a cold literary
exercise, whose cheerful substance even hell-fire does not warm.

And do not tell me that to show the shallowness of rationalist
philosophizing I have had to go back to a shallow wigpated age. The
optimism of present-day rationalism sounds just as shallow to the
fact-loving mind. The actual universe is a thing wide open, but
rationalism makes systems, and systems must be closed. For men in
practical life perfection is something far off and still in process of
achievement. This for rationalism is but the illusion of the finite
and relative: the absolute ground of things is a perfection eternally
complete.

I find a fine example of revolt against the airy and shallow optimism
of current religious philosophy in a publication of that valiant
anarchistic writer Morrison I. Swift. Mr. Swift's anarchism goes a
little farther than mine does, but I confess that I sympathize a
good deal, and some of you, I know, will sympathize heartily with his
dissatisfaction with the idealistic optimisms now in vogue. He begins
his pamphlet on \emph{Human Submission} with a series of city reporter's
items from newspapers (suicides, deaths from starvation and the like) as
specimens of our civilized regime. For instance:

`` `After trudging through the snow from one end of the city to the other
in the vain hope of securing employment, and with his wife and six
children without food and ordered to leave their home in an upper east
side tenement house because of non-payment of rent, John Corcoran, a
clerk, to-day ended his life by drinking carbolic acid. Corcoran lost
his position three weeks ago through illness, and during the period of
idleness his scanty savings disappeared. Yesterday he obtained work with
a gang of city snow shovelers, but he was too weak from illness and was
forced to quit after an hour's trial with the shovel. Then the
weary task of looking for employment was again resumed. Thoroughly
discouraged, Corcoran returned to his home late last night to find his
wife and children without food and the notice of dispossession on the
door.' On the following morning he drank the poison.

``The records of many more such cases lie before me [Mr. Swift goes on];
an encyclopedia might easily be filled with their kind. These few I cite
as an interpretation of the universe. `We are aware of the presence of
God in His world,' says a writer in a recent English Review. [The very
presence of ill in the temporal order is the condition of the perfection
of the eternal order, writes Professor Royce (\emph{The World and the
Individual}, II, 385).] `The Absolute is the richer for every discord,
and for all diversity which it embraces,' says F. H. Bradley (Appearance
and Reality, 204). He means that these slain men make the universe
richer, and that is Philosophy. But while Professors Royce and Bradley
and a whole host of guileless thoroughfed thinkers are unveiling
Reality and the Absolute and explaining away evil and pain, this is the
condition of the only beings known to us anywhere in the universe with
a developed consciousness of what the universe is. What these people
experience \emph{is} Reality. It gives us an absolute phase of the universe. It
is the personal experience of those most qualified in all our circle
of knowledge to \emph{have} experience, to tell us \emph{what} is. Now, what does
\emph{thinking about} the experience of these persons come to compared with
directly, personally feeling it, as they feel it? The philosophers are
dealing in shades, while those who live and feel know truth. And the
mind of mankind-not yet the mind of philosophers and of the proprietary
class-but of the great mass of the silently thinking and feeling men,
is coming to this view. They are judging the universe as they have
heretofore permitted the hierophants of religion and learning to judge
\emph{them}. ...

``This Cleveland workingman, killing his children and himself [another
of the cited cases], is one of the elemental, stupendous facts of this
modern world and of this universe. It cannot be glozed over or minimized
away by all the treatises on God, and Love, and Being, helplessly
existing in their haughty monumental vacuity. This is one of the simple
irreducible elements of this world's life after millions of years of
divine opportunity and twenty centuries of Christ. It is in the moral
world like atoms or sub-atoms in the physical, primary, indestructible.
And what it blazons to man is the ... imposture of all philosophy
which does not see in such events the consummate factor of conscious
experience. These facts invincibly prove religion a nullity. Man will
not give religion two thousand centuries or twenty centuries more to try
itself and waste human time; its time is up, its probation is ended.
Its own record ends it. Mankind has not sons and eternities to spare for
trying out discredited systems....'' [Footnote: Morrison I. Swift, Human
Submission, Part Second, Philadelphia, Liberty Press, 1905, pp. 4-10.]

Such is the reaction of an empiricist mind upon the rationalist bill of
fare. It is an absolute `No, I thank you.' ``Religion,'' says Mr. Swift,
``is like a sleep-walker to whom actual things are blank.'' And such,
tho possibly less tensely charged with feeling, is the verdict of
every seriously inquiring amateur in philosophy to-day who turns to the
philosophy-professors for the wherewithal to satisfy the fulness of his
nature's needs. Empiricist writers give him a materialism, rationalists
give him something religious, but to that religion ``actual things are
blank.'' He becomes thus the judge of us philosophers. Tender or tough,
he finds us wanting. None of us may treat his verdicts disdainfully, for
after all, his is the typically perfect mind, the mind the sum of whose
demands is greatest, the mind whose criticisms and dissatisfactions are
fatal in the long run.

It is at this point that my own solution begins to appear. I offer the
oddly-named thing pragmatism as a philosophy that can satisfy both kinds
of demand. It can remain religious like the rationalisms, but at the
same time, like the empiricisms, it can preserve the richest intimacy
with facts. I hope I may be able to leave many of you with as favorable
an opinion of it as I preserve myself. Yet, as I am near the end of my
hour, I will not introduce pragmatism bodily now. I will begin with it
on the stroke of the clock next time. I prefer at the present moment to
return a little on what I have said.

If any of you here are professional philosophers, and some of you I know
to be such, you will doubtless have felt my discourse so far to have
been crude in an unpardonable, nay, in an almost incredible degree.
Tender-minded and tough-minded, what a barbaric disjunction! And, in
general, when philosophy is all compacted of delicate intellectualities
and subtleties and scrupulosities, and when every possible sort of
combination and transition obtains within its bounds, what a brutal
caricature and reduction of highest things to the lowest possible
expression is it to represent its field of conflict as a sort of
rough-and-tumble fight between two hostile temperaments! What a
childishly external view! And again, how stupid it is to treat the
abstractness of rationalist systems as a crime, and to damn them because
they offer themselves as sanctuaries and places of escape, rather than
as prolongations of the world of facts. Are not all our theories just
remedies and places of escape? And, if philosophy is to be religious,
how can she be anything else than a place of escape from the crassness
of reality's surface? What better thing can she do than raise us out of
our animal senses and show us another and a nobler home for our minds in
that great framework of ideal principles subtending all reality, which
the intellect divines? How can principles and general views ever be
anything but abstract outlines? Was Cologne cathedral built without an
architect's plan on paper? Is refinement in itself an abomination? Is
concrete rudeness the only thing that's true?

Believe me, I feel the full force of the indictment. The picture I
have given is indeed monstrously over-simplified and rude. But like all
abstractions, it will prove to have its use. If philosophers can treat
the life of the universe abstractly, they must not complain of an
abstract treatment of the life of philosophy itself. In point of fact
the picture I have given is, however coarse and sketchy, literally true.
Temperaments with their cravings and refusals do determine men in their
philosophies, and always will. The details of systems may be reasoned
out piecemeal, and when the student is working at a system, he may
often forget the forest for the single tree. But when the labor is
accomplished, the mind always performs its big summarizing act, and the
system forthwith stands over against one like a living thing, with that
strange simple note of individuality which haunts our memory, like the
wraith of the man, when a friend or enemy of ours is dead.

Not only Walt Whitman could write ``who touches this book touches a man.''
The books of all the great philosophers are like so many men. Our
sense of an essential personal flavor in each one of them, typical but
indescribable, is the finest fruit of our own accomplished philosophic
education. What the system pretends to be is a picture of the great
universe of God. What it is--- and oh so flagrantly!--- is the revelation of
how intensely odd the personal flavor of some fellow creature is. Once
reduced to these terms (and all our philosophies get reduced to them in
minds made critical by learning) our commerce with the systems reverts
to the informal, to the instinctive human reaction of satisfaction or
dislike. We grow as peremptory in our rejection or admission, as when a
person presents himself as a candidate for our favor; our verdicts are
couched in as simple adjectives of praise or dispraise. We measure the
total character of the universe as we feel it, against the flavor of the
philosophy proffered us, and one word is enough.

``Statt der lebendigen Natur,'' we say, ``da Gott die Menschen schuf
hinein''--- that nebulous concoction, that wooden, that straight-laced
thing, that crabbed artificiality, that musty schoolroom product, that
sick man's dream! Away with it. Away with all of them! Impossible!
Impossible!

Our work over the details of his system is indeed what gives us our
resultant impression of the philosopher, but it is on the resultant
impression itself that we react. Expertness in philosophy is measured
by the definiteness of our summarizing reactions, by the immediate
perceptive epithet with which the expert hits such complex objects
off. But great expertness is not necessary for the epithet to come. Few
people have definitely articulated philosophies of their own. But almost
everyone has his own peculiar sense of a certain total character in
the universe, and of the inadequacy fully to match it of the peculiar
systems that he knows. They don't just cover \emph{his} world. One will be too
dapper, another too pedantic, a third too much of a job-lot of opinions,
a fourth too morbid, and a fifth too artificial, or what not. At any
rate he and we know offhand that such philosophies are out of plumb and
out of key and out of `whack,' and have no business to speak up in the
universe's name. Plato, Locke, Spinoza, Mill, Caird, Hegel--- I prudently
avoid names nearer home!--- I am sure that to many of you, my hearers,
these names are little more than reminders of as many curious personal
ways of falling short. It would be an obvious absurdity if such ways of
taking the universe were actually true. We philosophers have to reckon
with such feelings on your part. In the last resort, I repeat, it will
be by them that all our philosophies shall ultimately be judged. The
finally victorious way of looking at things will be the most completely
\emph{impressive} way to the normal run of minds.

One word more--- namely about philosophies necessarily being abstract
outlines. There are outlines and outlines, outlines of buildings
that are \emph{fat}, conceived in the cube by their planner, and outlines of
buildings invented flat on paper, with the aid of ruler and compass.
These remain skinny and emaciated even when set up in stone and mortar,
and the outline already suggests that result. An outline in itself is
meagre, truly, but it does not necessarily suggest a meagre thing. It is
the essential meagreness of \emph{what is suggested} by the usual rationalistic
philosophies that moves empiricists to their gesture of rejection. The
case of Herbert Spencer's system is much to the point here. Rationalists
feel his fearful array of insufficiencies. His dry schoolmaster
temperament, the hurdy-gurdy monotony of him, his preference for
cheap makeshifts in argument, his lack of education even in mechanical
principles, and in general the vagueness of all his fundamental ideas,
his whole system wooden, as if knocked together out of cracked hemlock
boards--- and yet the half of England wants to bury him in Westminster
Abbey.

Why? Why does Spencer call out so much reverence in spite of his
weakness in rationalistic eyes? Why should so many educated men who
feel that weakness, you and I perhaps, wish to see him in the Abbey
notwithstanding?

Simply because we feel his heart to be \emph{in the right place}
philosophically. His principles may be all skin and bone, but at any
rate his books try to mould themselves upon the particular shape of
this, particular world's carcase. The noise of facts resounds through
all his chapters, the citations of fact never cease, he emphasizes
facts, turns his face towards their quarter; and that is enough. It
means the right kind of thing for the empiricist mind.

The pragmatistic philosophy of which I hope to begin talking in my
next lecture preserves as cordial a relation with facts, and, unlike
Spencer's philosophy, it neither begins nor ends by turning positive
religious constructions out of doors--- it treats them cordially as well.

I hope I may lead you to find it just the mediating way of thinking that
you require.


\section{Lecture II: What Pragmatism Means}

Some years ago, being with a camping party in the mountains, I
returned from a solitary ramble to find everyone engaged in a ferocious
metaphysical dispute. The corpus of the dispute was a squirrel--- a live
squirrel supposed to be clinging to one side of a tree-trunk; while over
against the tree's opposite side a human being was imagined to stand.
This human witness tries to get sight of the squirrel by moving rapidly
round the tree, but no matter how fast he goes, the squirrel moves
as fast in the opposite direction, and always keeps the tree between
himself and the man, so that never a glimpse of him is caught. The
resultant metaphysical problem now is this: \emph{Does the man go round the
squirrel or not?} He goes round the tree, sure enough, and the squirrel
is on the tree; but does he go round the squirrel? In the unlimited
leisure of the wilderness, discussion had been worn threadbare. Everyone
had taken sides, and was obstinate; and the numbers on both sides were
even. Each side, when I appeared, therefore appealed to me to make it
a majority. Mindful of the scholastic adage that whenever you meet a
contradiction you must make a distinction, I immediately sought and
found one, as follows: ``Which party is right,'' I said, ``depends on what
you \emph{practically mean} by `going round' the squirrel. If you mean passing
from the north of him to the east, then to the south, then to the west,
and then to the north of him again, obviously the man does go round him,
for he occupies these successive positions. But if on the contrary you
mean being first in front of him, then on the right of him, then behind
him, then on his left, and finally in front again, it is quite as
obvious that the man fails to go round him, for by the compensating
movements the squirrel makes, he keeps his belly turned towards the man
all the time, and his back turned away. Make the distinction, and there
is no occasion for any farther dispute. You are both right and both
wrong according as you conceive the verb `to go round' in one practical
fashion or the other.''

Altho one or two of the hotter disputants called my speech a shuffling
evasion, saying they wanted no quibbling or scholastic hair-splitting,
but meant just plain honest English `round,' the majority seemed to
think that the distinction had assuaged the dispute.

I tell this trivial anecdote because it is a peculiarly simple example
of what I wish now to speak of as \emph{the pragmatic method}. The pragmatic
method is primarily a method of settling metaphysical disputes that
otherwise might be interminable. Is the world one or many?--- fated or
free?--- material or spiritual?--- here are notions either of which may
or may not hold good of the world; and disputes over such notions are
unending. The pragmatic method in such cases is to try to interpret each
notion by tracing its respective practical consequences. What difference
would it practically make to anyone if this notion rather than that
notion were true? If no practical difference whatever can be traced,
then the alternatives mean practically the same thing, and all dispute
is idle. Whenever a dispute is serious, we ought to be able to show some
practical difference that must follow from one side or the other's being
right.

A glance at the history of the idea will show you still better what
pragmatism means. The term is derived from the same Greek word [pi rho
alpha gamma mu alpha], meaning action, from which our words `practice'
and `practical' come. It was first introduced into philosophy by Mr.
Charles Peirce in 1878. In an article entitled `How to Make Our Ideas
Clear,' in the `Popular Science Monthly' for January of that year
[Footnote: Translated in the Revue Philosophique for January, 1879 (vol.
vii).] Mr. Peirce, after pointing out that our beliefs are really rules
for action, said that to develope a thought's meaning, we need only
determine what conduct it is fitted to produce: that conduct is for
us its sole significance. And the tangible fact at the root of all our
thought-distinctions, however subtle, is that there is no one of them so
fine as to consist in anything but a possible difference of practice.
To attain perfect clearness in our thoughts of an object, then, we need
only consider what conceivable effects of a practical kind the object
may involve--- what sensations we are to expect from it, and what
reactions we must prepare. Our conception of these effects, whether
immediate or remote, is then for us the whole of our conception of the
object, so far as that conception has positive significance at all.

This is the principle of Peirce, the principle of pragmatism. It lay
entirely unnoticed by anyone for twenty years, until I, in an address
before Professor Howison's philosophical union at the university of
California, brought it forward again and made a special application
of it to religion. By that date (1898) the times seemed ripe for its
reception. The word `pragmatism' spread, and at present it fairly
spots the pages of the philosophic journals. On all hands we find the
`pragmatic movement' spoken of, sometimes with respect, sometimes with
contumely, seldom with clear understanding. It is evident that the term
applies itself conveniently to a number of tendencies that hitherto have
lacked a collective name, and that it has `come to stay.'

To take in the importance of Peirce's principle, one must get accustomed
to applying it to concrete cases. I found a few years ago that Ostwald,
the illustrious Leipzig chemist, had been making perfectly distinct
use of the principle of pragmatism in his lectures on the philosophy of
science, tho he had not called it by that name.

``All realities influence our practice,'' he wrote me, ``and that influence
is their meaning for us. I am accustomed to put questions to my classes
in this way: In what respects would the world be different if this
alternative or that were true? If I can find nothing that would become
different, then the alternative has no sense.''

That is, the rival views mean practically the same thing, and meaning,
other than practical, there is for us none. Ostwald in a published
lecture gives this example of what he means. Chemists have long wrangled
over the inner constitution of certain bodies called `tautomerous.'
Their properties seemed equally consistent with the notion that an
instable hydrogen atom oscillates inside of them, or that they are
instable mixtures of two bodies. Controversy raged; but never was
decided. ``It would never have begun,'' says Ostwald, ``if the combatants
had asked themselves what particular experimental fact could have been
made different by one or the other view being correct. For it would then
have appeared that no difference of fact could possibly ensue; and the
quarrel was as unreal as if, theorizing in primitive times about the
raising of dough by yeast, one party should have invoked a `brownie,'
while another insisted on an `elf' as the true cause of the phenomenon.''\footnote{`Theorie und Praxis,' Zeitsch. des Oesterreichischen
Ingenieur u. Architecten-Vereines, 1905, Nr. 4 u. 6. I find a still
more radical pragmatism than Ostwald's in an address by Professor W.
S. Franklin: ``I think that the sickliest notion of physics, even if a
student gets it, is that it is `the science of masses, molecules and the
ether.' And I think that the healthiest notion, even if a student does
not wholly get it, is that physics is the science of the ways of taking
hold of bodies and pushing them!'' (Science, January 2, 1903.)}

It is astonishing to see how many philosophical disputes collapse
into insignificance the moment you subject them to this simple test of
tracing a concrete consequence. There can \emph{be} no difference any-where
that doesn't \emph{make} a difference elsewhere--- no difference in abstract
truth that doesn't express itself in a difference in concrete fact and
in conduct consequent upon that fact, imposed on somebody, somehow,
somewhere and somewhen. The whole function of philosophy ought to be
to find out what definite difference it will make to you and me,
at definite instants of our life, if this world-formula or that
world-formula be the true one.

There is absolutely nothing new in the pragmatic method. Socrates was
an adept at it. Aristotle used it methodically. Locke, Berkeley and Hume
made momentous contributions to truth by its means. Shadworth Hodgson
keeps insisting that realities are only what they are `known-as.'
But these forerunners of pragmatism used it in fragments: they were
preluders only. Not until in our time has it generalized itself, become
conscious of a universal mission, pretended to a conquering destiny. I
believe in that destiny, and I hope I may end by inspiring you with my
belief.

Pragmatism represents a perfectly familiar attitude in philosophy, the
empiricist attitude, but it represents it, as it seems to me, both in
a more radical and in a less objectionable form than it has ever yet
assumed. A pragmatist turns his back resolutely and once for all upon
a lot of inveterate habits dear to professional philosophers. He turns
away from abstraction and insufficiency, from verbal solutions, from bad
a priori reasons, from fixed principles, closed systems, and pretended
absolutes and origins. He turns towards concreteness and adequacy,
towards facts, towards action, and towards power. That means the
empiricist temper regnant, and the rationalist temper sincerely given
up. It means the open air and possibilities of nature, as against dogma,
artificiality and the pretence of finality in truth.

At the same time it does not stand for any special results. It is
a method only. But the general triumph of that method would mean an
enormous change in what I called in my last lecture the `temperament'
of philosophy. Teachers of the ultra-rationalistic type would be frozen
out, much as the courtier type is frozen out in republics, as the
ultramontane type of priest is frozen out in protestant lands. Science
and metaphysics would come much nearer together, would in fact work
absolutely hand in hand.

Metaphysics has usually followed a very primitive kind of quest. You
know how men have always hankered after unlawful magic, and you know
what a great part, in magic, \emph{words} have always played. If you have his
name, or the formula of incantation that binds him, you can control the
spirit, genie, afrite, or whatever the power may be. Solomon knew the
names of all the spirits, and having their names, he held them subject
to his will. So the universe has always appeared to the natural mind as
a kind of enigma, of which the key must be sought in the shape of
some illuminating or power-bringing word or name. That word names the
universe's \emph{principle}, and to possess it is, after a fashion, to
possess the universe itself. `God,' `Matter,' `Reason,' `the Absolute,'
`Energy,' are so many solving names. You can rest when you have them.
You are at the end of your metaphysical quest.

But if you follow the pragmatic method, you cannot look on any such word
as closing your quest. You must bring out of each word its practical
cash-value, set it at work within the stream of your experience. It
appears less as a solution, then, than as a program for more work,
and more particularly as an indication of the ways in which existing
realities may be \emph{changed}.

\emph{Theories thus become instruments, not answers to enigmas, in which
we can rest.} We don't lie back upon them, we move forward, and, on
occasion, make nature over again by their aid. Pragmatism unstiffens all
our theories, limbers them up and sets each one at work. Being nothing
essentially new, it harmonizes with many ancient philosophic tendencies.
It agrees with nominalism for instance, in always appealing to
particulars; with utilitarianism in emphasizing practical aspects; with
positivism in its disdain for verbal solutions, useless questions, and
metaphysical abstractions.

All these, you see, are \emph{anti-intellectualist} tendencies. Against
rationalism as a pretension and a method, pragmatism is fully armed
and militant. But, at the outset, at least, it stands for no particular
results. It has no dogmas, and no doctrines save its method. As the
young Italian pragmatist Papini has well said, it lies in the midst of
our theories, like a corridor in a hotel. Innumerable chambers open out
of it. In one you may find a man writing an atheistic volume; in the
next someone on his knees praying for faith and strength; in a third
a chemist investigating a body's properties. In a fourth a system
of idealistic metaphysics is being excogitated; in a fifth the
impossibility of metaphysics is being shown. But they all own the
corridor, and all must pass through it if they want a practicable way of
getting into or out of their respective rooms.

No particular results then, so far, but only an attitude of orientation,
is what the pragmatic method means. \emph{The attitude of looking away from
first things, principles, `categories,' supposed necessities; and of
looking towards last things, fruits, consequences, facts.}

So much for the pragmatic method! You may say that I have been praising
it rather than explaining it to you, but I shall presently explain it
abundantly enough by showing how it works on some familiar problems.
Meanwhile the word pragmatism has come to be used in a still wider
sense, as meaning also a certain theory of \emph{truth}. I mean to give a whole
lecture to the statement of that theory, after first paving the way,
so I can be very brief now. But brevity is hard to follow, so I ask
for your redoubled attention for a quarter of an hour. If much remains
obscure, I hope to make it clearer in the later lectures.

One of the most successfully cultivated branches of philosophy in our
time is what is called inductive logic, the study of the conditions
under which our sciences have evolved. Writers on this subject have
begun to show a singular unanimity as to what the laws of nature and
elements of fact mean, when formulated by mathematicians, physicists and
chemists. When the first mathematical, logical and natural uniformities,
the first \emph{laws}, were discovered, men were so carried away by the
clearness, beauty and simplification that resulted, that they believed
themselves to have deciphered authentically the eternal thoughts of the
Almighty. His mind also thundered and reverberated in syllogisms.
He also thought in conic sections, squares and roots and ratios, and
geometrized like Euclid. He made Kepler's laws for the planets to
follow; he made velocity increase proportionally to the time in falling
bodies; he made the law of the sines for light to obey when refracted;
he established the classes, orders, families and genera of plants and
animals, and fixed the distances between them. He thought the archetypes
of all things, and devised their variations; and when we rediscover any
one of these his wondrous institutions, we seize his mind in its very
literal intention.

But as the sciences have developed farther, the notion has gained ground
that most, perhaps all, of our laws are only approximations. The laws
themselves, moreover, have grown so numerous that there is no counting
them; and so many rival formulations are proposed in all the branches of
science that investigators have become accustomed to the notion that no
theory is absolutely a transcript of reality, but that any one of them
may from some point of view be useful. Their great use is to summarize
old facts and to lead to new ones. They are only a man-made language,
a conceptual shorthand, as someone calls them, in which we write our
reports of nature; and languages, as is well known, tolerate much choice
of expression and many dialects.

Thus human arbitrariness has driven divine necessity from scientific
logic. If I mention the names of Sigwart, Mach, Ostwald, Pearson,
Milhaud, Poincare, Duhem, Ruyssen, those of you who are students will
easily identify the tendency I speak of, and will think of additional
names.

Riding now on the front of this wave of scientific logic Messrs.
Schiller and Dewey appear with their pragmatistic account of what truth
everywhere signifies. Everywhere, these teachers say, `truth' in our
ideas and beliefs means the same thing that it means in science. It
means, they say, nothing but this, \emph{that ideas (which themselves are but
parts of our experience) become true just in so far as they help us to
get into satisfactory relation with other parts of our experience}, to
summarize them and get about among them by conceptual short-cuts instead
of following the interminable succession of particular phenomena. Any
idea upon which we can ride, so to speak; any idea that will carry us
prosperously from any one part of our experience to any other part,
linking things satisfactorily, working securely, simplifying,
saving labor; is true for just so much, true in so far forth, true
\emph{instrumentally}. This is the `instrumental' view of truth taught so
successfully at Chicago, the view that truth in our ideas means their
power to `work,' promulgated so brilliantly at Oxford.

Messrs. Dewey, Schiller and their allies, in reaching this general
conception of all truth, have only followed the example of geologists,
biologists and philologists. In the establishment of these other
sciences, the successful stroke was always to take some simple process
actually observable in operation--- as denudation by weather, say, or
variation from parental type, or change of dialect by incorporation of
new words and pronunciations--- and then to generalize it, making it apply
to all times, and produce great results by summating its effects through
the ages.

The observable process which Schiller and Dewey particularly singled out
for generalization is the familiar one by which any individual settles
into \emph{new opinions}. The process here is always the same. The individual
has a stock of old opinions already, but he meets a new experience that
puts them to a strain. Somebody contradicts them; or in a reflective
moment he discovers that they contradict each other; or he hears of
facts with which they are incompatible; or desires arise in him which
they cease to satisfy. The result is an inward trouble to which his
mind till then had been a stranger, and from which he seeks to escape
by modifying his previous mass of opinions. He saves as much of it as he
can, for in this matter of belief we are all extreme conservatives. So
he tries to change first this opinion, and then that (for they resist
change very variously), until at last some new idea comes up which he
can graft upon the ancient stock with a minimum of disturbance of the
latter, some idea that mediates between the stock and the new experience
and runs them into one another most felicitously and expediently.

This new idea is then adopted as the true one. It preserves the older
stock of truths with a minimum of modification, stretching them just
enough to make them admit the novelty, but conceiving that in ways as
familiar as the case leaves possible. An outree explanation, violating
all our preconceptions, would never pass for a true account of a
novelty. We should scratch round industriously till we found something
less excentric. The most violent revolutions in an individual's beliefs
leave most of his old order standing. Time and space, cause and effect,
nature and history, and one's own biography remain untouched. New truth
is always a go-between, a smoother-over of transitions. It marries old
opinion to new fact so as ever to show a minimum of jolt, a maximum of
continuity. We hold a theory true just in proportion to its success in
solving this `problem of maxima and minima.' But success in solving
this problem is eminently a matter of approximation. We say this theory
solves it on the whole more satisfactorily than that theory; but that
means more satisfactorily to ourselves, and individuals will emphasize
their points of satisfaction differently. To a certain degree,
therefore, everything here is plastic.

The point I now urge you to observe particularly is the part played by
the older truths. Failure to take account of it is the source of much
of the unjust criticism leveled against pragmatism. Their influence is
absolutely controlling. Loyalty to them is the first principle--- in
most cases it is the only principle; for by far the most usual way
of handling phenomena so novel that they would make for a serious
rearrangement of our preconceptions is to ignore them altogether, or to
abuse those who bear witness for them.

You doubtless wish examples of this process of truth's growth, and the
only trouble is their superabundance. The simplest case of new truth is
of course the mere numerical addition of new kinds of facts, or of new
single facts of old kinds, to our experience--- an addition that involves
no alteration in the old beliefs. Day follows day, and its contents are
simply added. The new contents themselves are not true, they simply \emph{come}
and \emph{are}. Truth is what we say about them, and when we say that they have
come, truth is satisfied by the plain additive formula.

But often the day's contents oblige a rearrangement. If I should now
utter piercing shrieks and act like a maniac on this platform, it
would make many of you revise your ideas as to the probable worth of my
philosophy. `Radium' came the other day as part of the day's content,
and seemed for a moment to contradict our ideas of the whole order of
nature, that order having come to be identified with what is called
the conservation of energy. The mere sight of radium paying heat away
indefinitely out of its own pocket seemed to violate that conservation.
What to think? If the radiations from it were nothing but an escape of
unsuspected `potential' energy, pre-existent inside of the atoms, the
principle of conservation would be saved. The discovery of `helium' as
the radiation's outcome, opened a way to this belief. So Ramsay's view
is generally held to be true, because, altho it extends our old ideas of
energy, it causes a minimum of alteration in their nature.

I need not multiply instances. A new opinion counts as `true' just in
proportion as it gratifies the individual's desire to assimilate the
novel in his experience to his beliefs in stock. It must both lean on
old truth and grasp new fact; and its success (as I said a moment ago)
in doing this, is a matter for the individual's appreciation. When
old truth grows, then, by new truth's addition, it is for subjective
reasons. We are in the process and obey the reasons. That new idea is
truest which performs most felicitously its function of satisfying our
double urgency. It makes itself true, gets itself classed as true, by
the way it works; grafting itself then upon the ancient body of truth,
which thus grows much as a tree grows by the activity of a new layer of
cambium.

Now Dewey and Schiller proceed to generalize this observation and
to apply it to the most ancient parts of truth. They also once were
plastic. They also were called true for human reasons. They also
mediated between still earlier truths and what in those days were novel
observations. Purely objective truth, truth in whose establishment the
function of giving human satisfaction in marrying previous parts of
experience with newer parts played no role whatever, is nowhere to be
found. The reasons why we call things true is the reason why they \emph{are}
true, for `to be true' \emph{means} only to perform this marriage-function.

The trail of the human serpent is thus over everything. Truth
independent; truth that we \emph{find} merely; truth no longer malleable to
human need; truth incorrigible, in a word; such truth exists indeed
superabundantly--- or is supposed to exist by rationalistically minded
thinkers; but then it means only the dead heart of the living tree, and
its being there means only that truth also has its paleontology and its
`prescription,' and may grow stiff with years of veteran service and
petrified in men's regard by sheer antiquity. But how plastic even the
oldest truths nevertheless really are has been vividly shown in our
day by the transformation of logical and mathematical ideas, a
transformation which seems even to be invading physics. The ancient
formulas are reinterpreted as special expressions of much wider
principles, principles that our ancestors never got a glimpse of in
their present shape and formulation.

Mr. Schiller still gives to all this view of truth the name of
`Humanism,' but, for this doctrine too, the name of pragmatism seems
fairly to be in the ascendant, so I will treat it under the name of
pragmatism in these lectures.

Such then would be the scope of pragmatism--- first, a method; and second,
a genetic theory of what is meant by truth. And these two things must be
our future topics.

What I have said of the theory of truth will, I am sure, have appeared
obscure and unsatisfactory to most of you by reason of us brevity. I
shall make amends for that hereafter. In a lecture on `common sense' I
shall try to show what I mean by truths grown petrified by antiquity. In
another lecture I shall expatiate on the idea that our thoughts become
true in proportion as they successfully exert their go-between function.
In a third I shall show how hard it is to discriminate subjective from
objective factors in Truth's development. You may not follow me wholly
in these lectures; and if you do, you may not wholly agree with me. But
you will, I know, regard me at least as serious, and treat my effort
with respectful consideration.

You will probably be surprised to learn, then, that Messrs. Schiller's
and Dewey's theories have suffered a hailstorm of contempt and ridicule.
All rationalism has risen against them. In influential quarters Mr.
Schiller, in particular, has been treated like an impudent schoolboy who
deserves a spanking. I should not mention this, but for the fact that it
throws so much sidelight upon that rationalistic temper to which I have
opposed the temper of pragmatism. Pragmatism is uncomfortable away from
facts. Rationalism is comfortable only in the presence of abstractions.
This pragmatist talk about truths in the plural, about their utility
and satisfactoriness, about the success with which they `work,' etc.,
suggests to the typical intellectualist mind a sort of coarse lame
second-rate makeshift article of truth. Such truths are not real truth.
Such tests are merely subjective. As against this, objective truth must
be something non-utilitarian, haughty, refined, remote, august, exalted.
It must be an absolute correspondence of our thoughts with an equally
absolute reality. It must be what we \emph{ought} to think, unconditionally.
The conditioned ways in which we \emph{do} think are so much irrelevance and
matter for psychology. Down with psychology, up with logic, in all this
question!

See the exquisite contrast of the types of mind! The pragmatist clings
to facts and concreteness, observes truth at its work in particular
cases, and generalizes. Truth, for him, becomes a class-name for all
sorts of definite working-values in experience. For the rationalist it
remains a pure abstraction, to the bare name of which we must defer.
When the pragmatist undertakes to show in detail just \emph{why} we must defer,
the rationalist is unable to recognize the concretes from which his own
abstraction is taken. He accuses us of \emph{denying} truth; whereas we have
only sought to trace exactly why people follow it and always ought
to follow it. Your typical ultra-abstractionist fairly shudders at
concreteness: other things equal, he positively prefers the pale and
spectral. If the two universes were offered, he would always choose the
skinny outline rather than the rich thicket of reality. It is so much
purer, clearer, nobler.

I hope that as these lectures go on, the concreteness and closeness to
facts of the pragmatism which they advocate may be what approves itself
to you as its most satisfactory peculiarity. It only follows here the
example of the sister-sciences, interpreting the unobserved by the
observed. It brings old and new harmoniously together. It converts the
absolutely empty notion of a static relation of `correspondence' (what
that may mean we must ask later) between our minds and reality, into
that of a rich and active commerce (that anyone may follow in detail and
understand) between particular thoughts of ours, and the great universe
of other experiences in which they play their parts and have their uses.

But enough of this at present? The justification of what I say must be
postponed. I wish now to add a word in further explanation of the claim
I made at our last meeting, that pragmatism may be a happy harmonizer
of empiricist ways of thinking, with the more religious demands of human
beings.

Men who are strongly of the fact-loving temperament, you may remember me
to have said, are liable to be kept at a distance by the small sympathy
with facts which that philosophy from the present-day fashion of
idealism offers them. It is far too intellectualistic. Old fashioned
theism was bad enough, with its notion of God as an exalted monarch,
made up of a lot of unintelligible or preposterous `attributes'; but, so
long as it held strongly by the argument from design, it kept some touch
with concrete realities. Since, however, darwinism has once for all
displaced design from the minds of the `scientific,' theism has lost
that foothold; and some kind of an immanent or pantheistic deity working
\emph{in} things rather than above them is, if any, the kind recommended to our
contemporary imagination. Aspirants to a philosophic religion turn, as a
rule, more hopefully nowadays towards idealistic pantheism than towards
the older dualistic theism, in spite of the fact that the latter still
counts able defenders.

But, as I said in my first lecture, the brand of pantheism offered is
hard for them to assimilate if they are lovers of facts, or empirically
minded. It is the absolutistic brand, spurning the dust and reared upon
pure logic. It keeps no connexion whatever with concreteness. Affirming
the Absolute Mind, which is its substitute for God, to be the rational
presupposition of all particulars of fact, whatever they may be, it
remains supremely indifferent to what the particular facts in our world
actually are. Be they what they may, the Absolute will father them. Like
the sick lion in Esop's fable, all footprints lead into his den,
but nulla vestigia retrorsum. You cannot redescend into the world of
particulars by the Absolute's aid, or deduce any necessary consequences
of detail important for your life from your idea of his nature. He gives
you indeed the assurance that all is well with Him, and for his eternal
way of thinking; but thereupon he leaves you to be finitely saved by
your own temporal devices.

Far be it from me to deny the majesty of this conception, or its
capacity to yield religious comfort to a most respectable class of
minds. But from the human point of view, no one can pretend that it
doesn't suffer from the faults of remoteness and abstractness. It is
eminently a product of what I have ventured to call the rationalistic
temper. It disdains empiricism's needs. It substitutes a pallid outline
for the real world's richness. It is dapper; it is noble in the bad
sense, in the sense in which to be noble is to be inapt for humble
service. In this real world of sweat and dirt, it seems to me that
when a view of things is `noble,' that ought to count as a presumption
against its truth, and as a philosophic disqualification. The prince of
darkness may be a gentleman, as we are told he is, but whatever the
God of earth and heaven is, he can surely be no gentleman. His menial
services are needed in the dust of our human trials, even more than his
dignity is needed in the empyrean.

Now pragmatism, devoted tho she be to facts, has no such materialistic
bias as ordinary empiricism labors under. Moreover, she has no objection
whatever to the realizing of abstractions, so long as you get about
among particulars with their aid and they actually carry you somewhere.
Interested in no conclusions but those which our minds and our
experiences work out together, she has no a priori prejudices against
theology. \emph{If theological ideas prove to have a value for concrete life,
they will be true, for pragmatism, in the sense of being good for so
much. For how much more they are true, will depend entirely on their
relations to the other truths that also have to be acknowledged.}

What I said just now about the Absolute of transcendental idealism is a
case in point. First, I called it majestic and said it yielded religious
comfort to a class of minds, and then I accused it of remoteness and
sterility. But so far as it affords such comfort, it surely is not
sterile; it has that amount of value; it performs a concrete function.
As a good pragmatist, I myself ought to call the Absolute true `in so
far forth,' then; and I unhesitatingly now do so.

But what does \emph{true in so far forth} mean in this case? To answer, we need
only apply the pragmatic method. What do believers in the Absolute mean
by saying that their belief affords them comfort? They mean that since
in the Absolute finite evil is `overruled' already, we may, therefore,
whenever we wish, treat the temporal as if it were potentially the
eternal, be sure that we can trust its outcome, and, without sin,
dismiss our fear and drop the worry of our finite responsibility. In
short, they mean that we have a right ever and anon to take a moral
holiday, to let the world wag in its own way, feeling that its issues
are in better hands than ours and are none of our business.

The universe is a system of which the individual members may relax their
anxieties occasionally, in which the don't-care mood is also right for
men, and moral holidays in order--- that, if I mistake not, is part, at
least, of what the Absolute is `known-as,' that is the great difference
in our particular experiences which his being true makes for us, that
is part of his cash-value when he is pragmatically interpreted. Farther
than that the ordinary lay-reader in philosophy who thinks favorably of
absolute idealism does not venture to sharpen his conceptions. He can
use the Absolute for so much, and so much is very precious. He is pained
at hearing you speak incredulously of the Absolute, therefore, and
disregards your criticisms because they deal with aspects of the
conception that he fails to follow.

If the Absolute means this, and means no more than this, who can
possibly deny the truth of it? To deny it would be to insist that men
should never relax, and that holidays are never in order. I am well
aware how odd it must seem to some of you to hear me say that an idea is
`true' so long as to believe it is profitable to our lives. That it is
\emph{good}, for as much as it profits, you will gladly admit. If what we do
by its aid is good, you will allow the idea itself to be good in so far
forth, for we are the better for possessing it. But is it not a strange
misuse of the word `truth,' you will say, to call ideas also `true' for
this reason?

To answer this difficulty fully is impossible at this stage of
my account. You touch here upon the very central point of Messrs.
Schiller's, Dewey's and my own doctrine of truth, which I cannot discuss
with detail until my sixth lecture. Let me now say only this, that truth
is \emph{one species of good}, and not, as is usually supposed, a category
distinct from good, and co-ordinate with it. \emph{The true is the name of
whatever proves itself to be good in the way of belief, and good, too,
for definite, assignable reasons.} Surely you must admit this, that if
there were \emph{no} good for life in true ideas, or if the knowledge of them
were positively disadvantageous and false ideas the only useful ones,
then the current notion that truth is divine and precious, and its
pursuit a duty, could never have grown up or become a dogma. In a world
like that, our duty would be to \emph{shun} truth, rather. But in this world,
just as certain foods are not only agreeable to our taste, but good for
our teeth, our stomach and our tissues; so certain ideas are not only
agreeable to think about, or agreeable as supporting other ideas that we
are fond of, but they are also helpful in life's practical struggles. If
there be any life that it is really better we should lead, and if there
be any idea which, if believed in, would help us to lead that life,
then it would be really \emph{better for us} to believe in that idea, \emph{unless,
indeed, belief in it incidentally clashed with other greater vital
benefits.}

`What would be better for us to believe'! This sounds very like a
definition of truth. It comes very near to saying `what we \emph{ought} to
believe': and in \emph{that} definition none of you would find any oddity.
Ought we ever not to believe what it is \emph{better for us} to believe? And
can we then keep the notion of what is better for us, and what is true
for us, permanently apart?

Pragmatism says no, and I fully agree with her. Probably you also agree,
so far as the abstract statement goes, but with a suspicion that if
we practically did believe everything that made for good in our own
personal lives, we should be found indulging all kinds of fancies about
this world's affairs, and all kinds of sentimental superstitions about a
world hereafter. Your suspicion here is undoubtedly well founded, and it
is evident that something happens when you pass from the abstract to the
concrete, that complicates the situation.

I said just now that what is better for us to believe is true \emph{unless the
belief incidentally clashes with some other vital benefit}. Now in real
life what vital benefits is any particular belief of ours most liable
to clash with? What indeed except the vital benefits yielded by \emph{other
beliefs} when these prove incompatible with the first ones? In other
words, the greatest enemy of any one of our truths may be the rest
of our truths. Truths have once for all this desperate instinct of
self-preservation and of desire to extinguish whatever contradicts them.
My belief in the Absolute, based on the good it does me, must run the
gauntlet of all my other beliefs. Grant that it may be true in giving me
a moral holiday. Nevertheless, as I conceive it,--- and let me speak now
confidentially, as it were, and merely in my own private person,--- it
clashes with other truths of mine whose benefits I hate to give up on
its account. It happens to be associated with a kind of logic of which I
am the enemy, I find that it entangles me in metaphysical paradoxes
that are inacceptable, etc., etc.. But as I have enough trouble in
life already without adding the trouble of carrying these intellectual
inconsistencies, I personally just give up the Absolute. I just \emph{take} my
moral holidays; or else as a professional philosopher, I try to justify
them by some other principle.

If I could restrict my notion of the Absolute to its bare holiday-giving
value, it wouldn't clash with my other truths. But we cannot easily thus
restrict our hypotheses. They carry supernumerary features, and these it
is that clash so. My disbelief in the Absolute means then disbelief
in those other supernumerary features, for I fully believe in the
legitimacy of taking moral holidays.

You see by this what I meant when I called pragmatism a mediator and
reconciler and said, borrowing the word from Papini, that he unstiffens
our theories. She has in fact no prejudices whatever, no obstructive
dogmas, no rigid canons of what shall count as proof. She is completely
genial. She will entertain any hypothesis, she will consider any
evidence. It follows that in the religious field she is at a great
advantage both over positivistic empiricism, with its anti-theological
bias, and over religious rationalism, with its exclusive interest in
the remote, the noble, the simple, and the abstract in the way of
conception.

In short, she widens the field of search for God. Rationalism sticks
to logic and the empyrean. Empiricism sticks to the external senses.
Pragmatism is willing to take anything, to follow either logic or the
senses, and to count the humblest and most personal experiences. She
will count mystical experiences if they have practical consequences.
She will take a God who lives in the very dirt of private fact-if that
should seem a likely place to find him.

Her only test of probable truth is what works best in the way of leading
us, what fits every part of life best and combines with the collectivity
of experience's demands, nothing being omitted. If theological ideas
should do this, if the notion of God, in particular, should prove to do
it, how could pragmatism possibly deny God's existence? She could see
no meaning in treating as `not true' a notion that was pragmatically so
successful. What other kind of truth could there be, for her, than all
this agreement with concrete reality?

In my last lecture I shall return again to the relations of pragmatism
with religion. But you see already how democratic she is. Her manners
are as various and flexible, her resources as rich and endless, and her
conclusions as friendly as those of mother nature.
\end{document}