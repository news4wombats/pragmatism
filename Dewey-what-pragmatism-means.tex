\documentclass[12pt]{article}
\usepackage{hyperref,setspace,graphicx}
\usepackage[]{accessibility}
%\usepackage[symbol*,perpage]{footmisc}
\usepackage[width=6in,height=9in,top=.5in,centering]{geometry}
\usepackage[T1]{fontenc}
% changes font to TeX Gyre Schola (Century Schoolbook)
\usepackage{tgschola}


\begin{document}
\hypersetup{pdfinfo={Title={What Pragmatism Means by Practical}, Author={John Dewey}}, pdfborder = {0 0 0 0}}

\setlength{\parskip}{4pt}
\setlength{\parindent}{0pt}


From John Dewey, \emph{Essays in Experimental Logic}, The University of Chicago Press, 1916.


\paragraph{About this text:}
Based on the \href{https://www.gutenberg.org/ebooks/40794}{Project Gutenberg edition}, eBook 40794, prepared by Barbara Tozier, Bill Tozier, JoAnn Greenwood, and the Online Distributed Proofreading Team. Formatted by P.D. Magnus in Fall 2025. Marginal page numbers, following the Project Gutenberg edition, correspond to the 1916 original.





\providecommand*{\marginpage}[1]{\marginpar{\tiny #1}}

\section*{What Pragmatism Means by Practical}

\marginpage{303}
Pragmatism, according to Mr. James, is a temper of mind, an attitude;
it is also a theory of the nature of ideas and truth; and, finally, it
is a theory about reality. It is pragmatism as method which is
emphasized, I take it, in the subtitle, ``a new name for some old ways
of thinking.''\footnote{William James, \emph{Pragmatism. A New Name for Some Old Ways
of Thinking.} (Popular Lectures on Philosophy.) New York: Longmans,
Green, {\&} Co., 1907. Pp. xiii+309.} It is this aspect which I suppose to be uppermost in
Mr. James's own mind; one frequently gets the impression that he
conceives the discussion of the other two points to be illustrative
material, more or less hypothetical, of the method. The briefest and
at the same time the most comprehensive formula for the method is:
``The attitude of looking away from first things, principles,
`categories,' supposed necessities; and of looking towards last
things, fruits, consequences, facts'' (pp. 54-55). And as the attitude
looked ``away from'' is the rationalistic, perhaps the chief aim of the
lectures is to exemplify some typical differences resulting from
taking one outlook or the other.


But pragmatism is ``used in a still wider sense, as meaning also a
certain theory of truth'' (p. 55);
\marginpage{304}it is ``a genetic theory of what is meant by truth'' (p. 65). Truth
means, as a matter of course, agreement, correspondence, of idea and
fact (p. 198), but what do agreement, correspondence, mean? With
rationalism they mean ``a static, inert relation,'' which is so ultimate
that of it nothing more can be said. With pragmatism they signify the
guiding or leading power of ideas by which we ``dip into the
particulars of experience again,'' and if by its aid we set up the
arrangements and connections among experienced objects which the idea
intends, the idea is verified; it corresponds with the things it means
to square with (pp. 205-6). The idea is true which works in leading us
to what it purports (p. 80).\footnote{Certain aspects of the doctrine are here purposely
omitted, and will meet us later.} Or, ``any idea that will carry us
prosperously from any one part of experience to any other part,
linking things satisfactorily, working securely, simplifying, saving
labor, is true for just so much, true in so far forth'' (p. 58). This
notion presupposes that ideas are essentially intentions (plans and
methods), and that what they, as ideas, ultimately intend is
\emph{prospective}---certain changes in prior existing things. This
contrasts again with rationalism, with its copy theory, where ideas,
\emph{as} ideas, are ineffective and impotent, since they mean only to
mirror a reality (p. 69) complete without them. Thus we are led to the
third aspect of pragmatism. The alternative between rationalism and
pragmatism ``concerns the structure
\marginpage{305}of the universe itself'' (p. 258). ``The essential contrast is that
reality ... for pragmatism is still in the making'' (p. 257). And in a
recent number of the \emph{Journal of Philosophy, Psychology, and
Scientific Methods},\footnote{Vol. IV, p. 547.} he says: ``I was primarily concerned in my
lectures with contrasting the belief that the world is still in the
process of making with the belief that there is an eternal edition of
it ready-made and complete.''


\subsection*{I}
It will be following Mr. James's example, I think, if we here regard
pragmatism as primarily a method, and treat the account of ideas and
their truth and of reality somewhat incidentally so far as the
discussion of them serves to exemplify or enforce the method.
Regarding the attitude of orientation which looks to outcomes and
consequences, one readily sees that it has, as Mr. James points out,
points of contact with historic empiricism, nominalism, and
utilitarianism. It insists that general notions shall ``cash in'' as
particular objects and qualities in experience; that ``principles'' are
ultimately subsumed under facts, rather than the reverse; that the
empirical consequence rather than the a priori basis is the
sanctioning and warranting factor. But all of these ideas are colored
and transformed by the dominant influence of experimental science: the
method of treating conceptions, theories, etc., as working hypotheses,
as
\marginpage{306}directors for certain experiments and experimental observations.
Pragmatism as attitude represents what Mr. Peirce has happily termed
the ``laboratory habit of mind'' extended into every area where inquiry
may fruitfully be carried on. A scientist would, I think, wonder not
so much at the method as at the lateness of philosophy's conversion to
what has made science what it is. Nevertheless it is impossible to
forecast the intellectual change that would proceed from carrying the
method sincerely and unreservedly into all fields of inquiry. Leaving
philosophy out of account, what a change would be wrought in the
historical and social sciences---in the conceptions of politics and law
and political economy! Mr. James does not claim too much when he says:
``The center of gravity of philosophy must alter its place. The earth
of things, long thrown into shadow by the glories of the upper ether,
must resume its rights.... It will be an alteration in the 'seat of
authority' that reminds one almost of the Protestant Reformation'' (p.
123).


I can imagine that many would not accept this method in philosophy for
very diverse reasons, perhaps among the most potent of which is lack
of faith in the power of the elements and processes of experience and
life to guarantee their own security and prosperity; because, that is,
of the feeling that the world of experience is so unstable, mistaken,
and fragmentary that it must have an absolutely permanent, true,\marginpage{307} and
complete ground. I cannot imagine, however, that so much uncertainty
and controversy as actually exists should arise about the content and
import of the doctrine on the basis of the general formula. It is when
the method is applied to special points that questions arise. Mr.
James reminds us in his preface that the pragmatic movement has found
expression ``from so many points of view, that much unconcerted
statement has resulted.'' And speaking of his lectures he goes on to
say: ``I have sought to unify the picture as it presents itself to my
own eyes, dealing in broad strokes.'' The ``different points of view''
here spoken of have concerned themselves with viewing pragmatically a
number of different things. And it is, I think, Mr. James's effort to
combine them, as they stand, which occasions misunderstanding among
Mr. James's readers. Mr. James himself applied it, for example, in
1898 to philosophic controversies to indicate what they mean in terms
of practical issues at stake. Before that, Mr. Peirce himself (in
1878) had applied the method to the proper way of \emph{conceiving} and
defining objects. Then it has been applied to \emph{ideas} in order to find
out what they mean in terms of what they intend, and what and how they
must intend in order to be true. Again, it has been applied to
\emph{beliefs}, to what men actually accept, hold to, and affirm. Indeed,
it lies in the nature of pragmatism that it should be applied as
widely as possible; and to things as diverse as controversies,
beliefs, truths,\marginpage{308} ideas, and objects. But yet the situations and
problems \emph{are} diverse; so much so that, while the meaning of each may
be told on the basis of ``last things,'' ``fruits,'' ``consequences,''
``facts,'' \emph{it is quite certain that the specific last things and facts
will be very different in the diverse cases, and that very different
types of meaning will stand out}. ``Meaning'' will itself \emph{mean}
something quite different in the case of ``objects'' from what it will
mean in the case of ``ideas,'' and for ``ideas'' something different from
``truths.'' Now the explanation to which I have been led of the
unsatisfactory condition of contemporary pragmatic discussion is that
in composing these ``different points of view'' into a single pictorial
whole, the distinct type of consequence and hence of meaning of
``practical'' appropriate to each has not been sufficiently emphasized.


1. When we consider separately the subjects to which the pragmatic
method has been applied, we find that Mr. James has provided the
necessary formula for each---with his never-failing instinct for the
concrete. We take first the question of the significance of an object:
the meaning which should properly be contained in its conception or
definition. ``To attain perfect clearness in our thoughts of an object,
then, we need only consider what conceivable effects of a practical
kind the object may involve---what sensations we are to expect from it
and what reactions we must prepare'' (pp. 46-47). Or, more\marginpage{309} shortly, as
it is quoted from Ostwald, ``All realities influence our practice, and
that influence is their meaning for us'' (p. 48). Here it will be noted
that the start is from objects already empirically given or presented,
existentially vouched for, and the question is as to their proper
conception---What is the proper meaning, or idea, of an object? And the
meaning is the effects \emph{these given objects produce}. One might doubt
the correctness of this theory, but I do not see how one could doubt
its import, or could accuse it of subjectivism or idealism, since the
object with its power to produce effects is assumed. Meaning is
expressly distinguished from objects, not confused with them (as in
idealism), and is said to consist in the practical reactions objects
exact of us or impose upon us. When, then, it is a question of an
object, ``meaning'' signifies its \emph{conceptual content or connotation,
and ``practical'' means the future responses which an object requires of
us or commits us to}.


2. But we may also start from a given idea, and ask what the \emph{idea}
means. Pragmatism will, of course, look to future consequences, but
they will clearly be of a different sort when we start from an idea as
idea, than when we start from an object. For what an idea as idea
means, is precisely that an object is \emph{not} given. The pragmatic
procedure here is to set the idea ``at work within the stream of
experience. It appears less as a solution than as a program for more
work, and particularly as an indication of the ways in which\marginpage{310} existing
realities may be changed. Theories, thus, become instruments.... We
don't lie back on them, we move forward, and, on occasion, make nature
over again by their aid'' (p. 53). In other words, an idea is a draft
drawn upon existing things, and intention to act so as to arrange them
in a certain way. From which it follows that if the draft is honored,
if existences, following upon the actions, rearrange or readjust
themselves in the way the idea intends, the idea is true. When, then,
it is a question of an idea, it is the idea itself which is practical
(being an intent) and its \emph{meaning} resides in the existences which,
as changed, it intends. While the meaning of an object is the changes
it requires in our attitude,\footnote{Only those who have already lost in the idealistic
confusion of existence and meaning will take this to mean that the
object is those changes in our reactions.} the meaning of an idea is the changes
it, as our attitude, effects in objects.


3. Then we have another formula, applicable not to objects nor ideas
as objects and ideas, but to \emph{truths}---to things, that is, where the
meaning of the object and of the idea is assumed to be already
ascertained. It reads: ``What difference would it practically make to
anyone if this notion rather than that notion were true? If no
practical difference whatever can be traced, then the alternatives
mean practically the same thing, and all dispute is idle'' (p. 45).
There can be ``no difference in abstract truth that
\marginpage{311}doesn't express itself in a difference in concrete fact, and in
conduct consequent upon the fact, imposed on somebody'' (p. 50).\footnote{I assume that the reader is sufficiently familiar with
Mr. James's book not to be misled by the text into thinking that Mr.
James himself discriminates as I have done these three types of
problems from one another. He does not; but, none the less, the three
formulae for the three situations are there.} Now
when we start with something which is already a truth (or taken to be
truth), and ask for its meaning in terms of its consequences, it is
implied that the conception, or conceptual significance, is already
clear, and that the existences it refers to are already in hand.
Meaning here, then, can be neither the connotative nor denotative
reference of a term; they are covered by the two prior formulae.
Meaning here means \emph{value}, importance. The practical factor is, then,
the worth character of these consequences: they are good or bad;
desirable or undesirable; or merely \emph{nil}, indifferent, in which
latter case belief is idle, the controversy a vain and conventional,
or verbal, one.


The term ``meaning'' and the term ``practical'' taken in isolation, and
without explicit definition from their specific context and problem,
are triply ambiguous. The meaning may be the conception or definition
of an \emph{object}; it may be the denotative existential reference of an
\emph{idea}; it may be actual value or \emph{importance}. So practical in the
corresponding cases may mean the attitudes and conduct exacted of us
by objects; or the capacity and tendency of an idea to
\marginpage{312}effect changes in prior existences; or the desirable and undesirable
quality of certain ends. The general pragmatic attitude, none the
less, is applied in all cases.


If the differing problems and the correlative diverse significations
of the terms ``meaning'' and ``practical'' are borne in mind, not all will
be converted to pragmatism, but the present uncertainty as to what
pragmatism is, anyway, and the present constant complaints on both
sides of misunderstanding will, I think, be minimized. At all events,
I have reached the conclusion that what the pragmatic movement just
now wants is a clear and consistent bearing in mind of these different
problems and of what is meant by practical in each. Accordingly the
rest of this paper is an endeavor to elucidate from the standpoint of
pragmatic method the importance of enforcing these distinctions.


\subsection*{II}
First, as to the problems of philosophy when pragmatically approached,
Mr. James says: ``The whole function of philosophy ought to be to find
out what definite difference it will make to you and me, at definite
instants of our life, if this world-formula or that world-formula be
true'' (p. 50). Here the world-formula is assumed as already given; it
is there, defined and constituted, and the question is as to its
import if believed. But from the second standpoint, that of idea as
working hypothesis, the chief function of philosophy is not to find
out what difference\marginpage{313} ready-made formulae make, \emph{if true}, but to
arrive at and to clarify their \emph{meaning as programs of behavior for
modifying the existent world}. From this standpoint, the meaning of a
world-formula is practical and moral, not merely in the consequences
which flow from accepting a certain conceptual content as true, but as
regards that content itself. And thus at the very outset we are
compelled to face this question: Does Mr. James employ the pragmatic
method to discover the value in terms of consequences in life of some
formula which has its logical content already fixed; or does he employ
it to criticize and revise and, ultimately, to constitute the meaning
of that formula? If it is the first, there is danger that the
pragmatic method will be employed only to vivify, if not validate,
doctrines which in themselves are pieces of rationalistic metaphysics,
not inherently pragmatic. If the last, there is danger that some
readers will think old notions are being confirmed, when in truth they
are being translated into new and inconsistent notions.


Consider the case of design. Mr. James begins with accepting a
ready-made notion, to which he then applies the pragmatic criterion.
The traditional notion is that of a ``seeing force that runs things.''
This is rationalistically and retrospectively empty; its being there
makes no difference. (This seems to overlook the fact that the past
world may be just what it is in virtue of the difference which a blind
force or a seeing force has already made in it. A pragmatist\marginpage{314} as well
as a rationalist may reply that it makes no difference retrospectively
only because we leave out the most important retrospective
difference). But ``returning with it into experience, we gain a more
confiding outlook on the future. If not a blind force, but a seeing
force, runs things, we may reasonably expect better issues. \emph{This
vague confidence in the future is the sole pragmatic meaning at
present discernible in the terms design and designer}'' (p. 115,
italics mine). Now is this meaning intended to \emph{replace} the meaning
of a ``seeing force which runs things''? Or is it intended to superadd a
pragmatic value and validation to that concept of a seeing force? Or
does it mean that, irrespective of the existence of any such object, a
belief in it has that value? Strict pragmatism would seem to require
the first interpretation.


The same difficulties arise in the discussion of spiritualistic theism
\emph{versus} materialism. Compare the two following statements: ``The
notion of God ... guarantees an ideal order that shall be permanently
preserved'' (p. 106). ``Here, then, in these different emotional and
practical appeals, in these adjustments of our attitudes of hope and
expectation, and all the delicate consequences which their differences
entail, \emph{lie the real meanings of materialism and spiritualism}'' (p.
107, italics mine). Does the latter method of determining the meaning
of, say, a spiritual God afford the substitute for the conception of
him as a ``superhuman power'' effecting the eternal preservation\marginpage{315} of
something; does it, that is, define God, supply the content for our
notion of God? Or does it merely superadd a value to a meaning already
fixed? And, if the latter, does the object, God as defined, or the
notion, or the belief (the acceptance of the notion) effect these
consequent values? In either of the latter alternatives, the good or
valuable consequences cannot clarify the meaning or conception of God;
for, by the argument, they proceed from a prior definition of God.
They cannot prove, or render more probable, the existence of such a
being, for, by the argument, these desirable consequences depend upon
accepting such an existence; and not even pragmatism can prove an
existence from desirable consequences which themselves exist only when
and if that other existence is there. On the other hand, if the
pragmatic method is not applied simply to tell the value of a belief
or controversy, but to fix the meaning of the terms involved in the
belief, resulting consequences would serve to constitute the entire
meaning, intellectual as well as practical, of the terms; and hence
the pragmatic method would simply abolish the meaning of an antecedent
power which will perpetuate eternally some existence. For that
consequence flows not from the belief or idea, but from the existence,
the power. It is not pragmatic at all.


Accordingly, when Mr. James says: ``Other than this \emph{practical}
significance, the words God, free will, design, \emph{have none}. Yet dark
though they be in\marginpage{316} themselves, or intellectualistically taken, when we
bear them on to life's thicket with us, the darkness then grows light
about us'' (p. 121, italics mine), what is meant? Is it meant that when
we take the intellectualistic notion and employ it, it gets value in
the way of results, and hence then has some value of its own; or is it
meant that the intellectual content itself must be determined in terms
of the changes effected in the ordering of life's thicket? An explicit
declaration on this point would settle, I think, not merely a point
interesting in itself, but one essential to the determination of what
is pragmatic method. For myself, I have no hesitation in saying that
it seems unpragmatic for pragmatism to content itself with finding out
the value of a conception whose own inherent significance pragmatism
has not first determined; a fact which entails that it be taken not as
a truth but simply as a working hypothesis. In the particular case in
question, moreover, it is difficult to see how the pragmatic method
could possibly be applied to a notion of ``eternal perpetuation,''
which, by its nature, can never be empirically verified, or cashed in
any particular case.


This brings us to the question of truth. The problem here is also
ambiguous in advance of definition. Does the problem of what is truth
refer to discovering the ``true meaning'' of something; or to
discovering what an idea has to effect, and how, in order to be true;
or to discovering what the value of\marginpage{317} truth is when it is an existent
and accomplished fact? (1) We may, of course, find the ``true meaning''
of a thing, as distinct from its incorrect interpretation, without
thereby establishing the truth of the ``true meaning''---as we may
dispute about the ``true meaning'' of a passage in the classics
concerning Centaurs, without the determination of its true sense
establishing the truth of the notion that there are Centaurs.
Occasionally this ``true meaning'' seems to be what Mr. James has in
mind, as when, after the passage upon design already quoted, he goes
on: ``But if cosmic confidence is right, not wrong, better, not worse,
that [vague confidence in the future] is a most important meaning.
That much at least of possible 'truth' the terms will then have in
them'' (p. 115). ``Truth'' here seems to mean that design has a genuine,
not merely conventional or verbal, meaning: that something is at
stake. And there are frequently points where ``truth'' seems to mean
just meaning that is genuine as distinct from empty or verbal. (2) But
the problem of the meaning of truth may also refer to the meaning or
value of truths that already exist as truths. We have them; they
exist; now what do they mean? The answer is: ``True ideas lead us into
useful verbal and conceptual quarters as well as directly up to useful
sensible termini. They lead to consistency, stability, and flowing
human intercourse'' (p. 215). This, referring to things already true, I
do not suppose the most case-hardened rationalist\marginpage{318} would question; and
even if he questions the pragmatic contention that these consequences
define the meaning of truth, he should see that here is not given an
account of what it means for an idea to \emph{become true}, but only of
what it means \emph{after} it has become true, truth as \emph{fait accompli}. It
is the meaning of truth as \emph{fait accompli} which is here defined.


Bearing this in mind, I do not know why a mild-tempered rationalist
should object to the doctrine that truth is valuable not \emph{per se}, but
because, when given, it leads to desirable consequences. ``The true
thought is useful here because the home which is its object is useful.
The practical value of true ideas is thus primarily derived from the
practical importance of their objects to us'' (p. 203). And many
besides confirmed pragmatists, any utilitarian, for example, would be
willing to say that our duty to pursue ``truth'' is conditioned upon its
leading to objects which upon the whole are valuable. ``The concrete
benefits we gain are what we mean by calling the pursuit a duty'' (p.
231, compare p. 76). (3) Difficulties have arisen chiefly because Mr.
James is charged with converting simply the foregoing proposition, and
arguing that since true ideas are good, any idea if good in any way is
true. Certainly transition from one of these conceptions to the other
is facilitated by the fact that ideas are tested as to their validity
by a certain goodness, viz., whether they are good for accomplishing
what they intend, for what they claim to be good\marginpage{319} for, that is,
certain modifications in prior given existences. In this case, it is
the idea which is practical, since it is essentially an intent and
plan of altering prior existences in a specific situation, which is
indicated to be unsatisfactory by the very fact that it needs or
suggests a specific modification. Then arises the theory that ideas as
ideas are always working hypotheses concerning the attaining of
particular empirical results, and are tentative programs (or sketches
of method) for attaining them. If we stick consistently to this notion
of ideas, only \emph{consequences which are actually produced by the
working of the idea in co-operation with, or application to, prior
existences are good consequences in the specific sense of good which
is relevant to establishing the truth of an idea}. This is, at times,
unequivocally recognized by Mr. James. (See, for example, the
reference to veri-\emph{fication}, on p. 201; the acceptance of the idea
that verification means the advent of the object intended, on p. 205.)


But at other times any good which flows from acceptance of a belief is
treated as if it were an evidence, \emph{in so far}, of the truth of the
idea. This holds particularly when theological notions are under
consideration. Light would be thrown upon how Mr. James conceives this
matter by statements on such points as these: If ideas terminate in
good consequences, but yet the goodness of the consequences was no
part of the intention of an idea, does the goodness have any verifying
force? If the goodness of\marginpage{320} consequences arises from the context of the
idea in belief rather than from the idea itself, does it have any
verifying force?\footnote{The idea of immortality, or the traditional theistic idea
of God, for example, may produce its good consequences, not in virtue
of the idea as idea, but from the character of the person who
entertains the belief; or it may be the idea of the supreme value of
ideal considerations, rather than that of their temporal duration,
which works.} If an idea leads to consequences which are good
in the \emph{one} respect only of fulfilling the intent of the idea (as
when one drinks a liquid to test the idea that it is a poison), does
the badness of the consequences in every other respect detract from
the verifying force of consequences?


Since Mr. James has referred to me as saying ``truth is what gives
satisfaction'' (p. 234), I may remark (apart from the fact that I do
not think I ever said that truth is what \emph{gives} satisfaction) that I
have never identified any satisfaction with the truth of an idea, save
\emph{that} satisfaction which arises when the idea as working hypothesis
or tentative method is applied to prior existences in such a way as to
fulfil what it intends.


My final impression (which I cannot adequately prove) is that upon the
whole Mr. James is most concerned to enforce, as against rationalism,
two conclusions about the character of truths as \emph{faits accomplis}:
namely, that they are made, not a priori, or eternally in
existence,\footnote{``Eternal truth'' is one of the most ambiguous phrases that
philosophers trip over. It may mean eternally in existence; or that a
statement which is ever true is always true (if it is true a fly is
buzzing, it is eternally true that just now a fly buzzed); or it may
mean that some truths, \emph{in so far as wholly conceptual}, are
irrelevant to any particular time determination, since they are
non-existential in import—e.g., the truth of geometry dialectically
taken—that is, without asking whether any particular existence
exemplifies them.} and that their value or
\marginpage{321}importance is not static, but dynamic and practical. The special
question of \emph{how} truths are made is not particularly relevant to this
anti-rationalistic crusade, while it is the chief question of interest
to many. Because of this conflict of problems, what Mr. James says
about the value of truth when accomplished is likely to be interpreted
by some as a criterion of the truth of ideas; while, on the other
hand, Mr. James himself is likely to pass lightly from the
consequences that determine the worth of a belief to those which
decide the worth of an idea. When Mr. James says the function of
giving ``satisfaction in marrying previous parts of experience with
newer parts'' is necessary in order to establish truth, the doctrine is
unambiguous. The satisfactory character of consequences is itself
measured and defined by the conditions which led up to it; the
inherently satisfactory quality of results is not taken as validating
the antecedent intellectual operations. But when he says (not of his
own position, but of an opponent's\footnote{Such statements, it ought in fairness to be said,
generally come when Mr. James is speaking of a doctrine which he does
not himself believe, and arise, I think, in that fairness and
frankness of Mr. James, so unusual in philosophers, which cause him to
lean over backward—unpragmatically, it seems to me. As to the claim
of his own doctrine, he consistently sticks to his statement: ``Pent
in, as the pragmatist, more than any one, sees himself to be, between
the whole body of funded truths squeezed from the past and the
coercions of the world of sense about him, who, so well as he, feels
the immense pressure of objective control under which our minds
perform their operations? If anyone imagines that this law is lax, let
him keep its commandments one day, says Emerson'' (p. 233).}) of the idea of an absolute, ``so
far as it affords such comfort it surely
\marginpage{322}is not sterile, it has that amount of value; it performs a concrete
function. As a good pragmatist I myself ought to call the absolute
true \emph{in so far forth} then; and I unhesitatingly now do so'' (p. 73),
the doctrine seems to be as unambiguous in the other direction: that
any good, consequent upon acceptance of a belief is, in so far
forth,\footnote{Of course, Mr. James holds that this ``in so far'' goes a
very small way. See pp. 77-79. But even the slightest concession is, I
think, non-pragmatic unless the satisfaction is relevant to the idea
as intent. Now the satisfaction in question comes not from the idea as
\emph{idea}, but from its acceptance as \emph{true}. Can a satisfaction
dependent on an assumption that an idea is already true be relevant to
testing the truth of an idea? And can an idea, like that of the
absolute, which, if true, ``absolutely'' precludes any appeal to
consequences as test of truth, be confirmed by use of the pragmatic
test without sheer self-contradiction? In other words, we have a
confusion of the test of an idea as idea, with that of the value of a
belief as belief. On the other hand, it is quite possible that all Mr.
James intends by truth here is true (i.e., genuine) meaning at stake
in the issue—true not as distinct from false, but from meaningless or
verbal.} a warrant of truth. In such passages as the following (which
are of the common type) the two notions seem blended together: ``Ideas
become true just in so far as they help us to get
\marginpage{323}into satisfactory relations with other parts of our experience'' (p.
58); and, again, on the same page: ``Any idea that will carry us
\emph{prosperously} from any one part of our experience to any other part,
linking things \emph{satisfactorily}, working securely, simplifying, saving
labor, is true for just so much'' (italics mine). An explicit statement
as to whether the carrying function, the linking of things, is
satisfactory and prosperous and hence true in so far as it executes
the intent of an idea; or whether the satisfaction and prosperity
reside in the material consequences on their own account and in that
aspect make the idea true, would, I am sure, locate the point at issue
and economize and fructify future discussion. At present pragmatism is
accepted by those whose own notions are thoroughly rationalistic in
make-up as a means of refurbishing, galvanizing, and justifying those
very notions. It is rejected by non-rationalists (empiricists and
naturalistic idealists) because it seems to them identified with the
notion that pragmatism holds that the desirability of certain beliefs
overrides the question of the meaning of the ideas involved in them
and the existence of objects denoted by them. Others (like myself),
who believe thoroughly in pragmatism as a method of orientation, as
defined by Mr. James, and who would apply the method to the
determination of the meaning of objects, the intent and worth of ideas
as ideas, and to the human and moral value of beliefs, when these
various problems\marginpage{324} are carefully distinguished from one another, do not
know whether they are pragmatists in some other sense, because they
are not sure whether the practical, in the sense of desirable facts
which define the worth of a belief, is confused with the practical as
an attitude imposed by objects, and with the practical as a power and
function of ideas to effect changes in prior existences. Hence the
importance of knowing which one of the three senses of practical is
conveyed in any given passage.


It would do Mr. James an injustice, however, to stop here. His real
doctrine is that a belief is true when it satisfies both personal
needs and the requirements of objective things. Speaking of
pragmatism, he says, ``Her only test of probable truth is what works
best in the way of \emph{leading us}, what fits every part of life best and
\emph{combines with the collectivity of experience's demands}, nothing
being omitted'' (p. 80, italics mine). And again, ``That new idea is
truest which performs most felicitously its function of satisfying
\emph{our double urgency}'' (p. 64). It does not appear certain from the
context that this ``double urgency'' is that of the personal and the
objective demands, respectively, but it is probable (see, also, p.
217, where ``consistency with previous truth and novel fact'' is said to
be ``always the most imperious claimant''). On this basis, the ``in so
far forth'' of the truth of the absolute because of the comfort it
supplies, means that one of the two conditions which need to be
satisfied has\marginpage{325} been met, so that if the idea of the absolute met the
other one also, it would be quite true. I have no doubt this is Mr.
James's meaning, and it sufficiently safeguards him from the charge
that pragmatism means that anything which is agreeable is true. At the
same time, I do not think, in logical strictness, that satisfying one
of two tests, when satisfaction of both is required, can be said to
constitute a belief true even ``in so far forth.''


\subsection*{III}
At all events this raises a question not touched so far: the place of
the personal in the determination of truth. Mr. James, for example,
emphasizes the doctrine suggested in the following words: ``We say this
theory solves it [the problem] more satisfactorily than that theory;
but that means more satisfactorily \emph{to ourselves}, and individuals
will emphasize their points of satisfaction differently'' (p. 61,
italics mine). This opens out into a question which, in its larger
aspects---the place of the personal factor in the constitution of
knowledge systems and of reality---I cannot here enter upon, save to
say that a synthetic pragmatism such as Mr. James has ventured upon
will take a very different form according as the point of view of what
he calls the ``Chicago School'' or that of humanism is taken as a basis
for interpreting the nature of the personal. According to the latter
view, the personal appears to be ultimate and unanalyzable,\marginpage{326} the
metaphysically real. Associations with idealism, moreover, give it an
idealistic turn, a translation, in effect, of monistic
intellectualistic idealism into pluralistic, voluntaristic idealism.
But, according to the former, the personal is not ultimate, but is to
be analyzed and defined, biologically on its genetic side, ethically
on its prospective and functioning side.


There is, however, one phase of the teaching illustrated by the
quotation which is directly relevant here. Because Mr. James
recognizes that the personal element enters into judgments passed upon
whether a problem has or has not been satisfactorily solved, he is
charged with extreme subjectivism, with encouraging the element of
personal preference to run rough-shod over all objective controls. Now
the question raised in the quotation is primarily one of fact, not of
doctrine. Is or is not a personal factor found in truth evaluations?
If it is, pragmatism is not responsible for introducing it. If it is
not, it ought to be possible to refute pragmatism by appeal to
empirical fact, rather than by reviling it for subjectivism. Now it is
an old story that philosophers, in common with theologians and social
theorists, are as sure that personal habits and interests shape their
opponents' doctrines as they are that their own beliefs are
``absolutely'' universal and objective in quality. Hence arises that
dishonesty, that insincerity characteristic of philosophic discussion.
As Mr. James says (p. 8), ``The most potential of all our premises is
never mentioned.''\marginpage{327} Now the moment the complicity of the personal
factor in our philosophic valuations is recognized, is recognized
fully, frankly, and generally, that moment a new era in philosophy
will begin. We shall have to discover the personal factors that now
influence us unconsciously, and begin to accept a new and moral
responsibility for them, a responsibility for judging and testing them
by their consequences. So long as we ignore this factor, its deeds
will be largely evil, not because \emph{it} is evil, but because,
flourishing in the dark, it is without responsibility and without
check. The only way to control it is by recognizing it. And while I
would not prophesy of pragmatism's future, I would say that this
element which is now so generally condemned as intellectual dishonesty
(perhaps because of an uneasy, instinctive recognition of the
searching of hearts its acceptance would involve) will in the future
be accounted unto philosophy for righteousness' sake.


So much in general. In particular cases, it is possible that Mr.
James's language occasionally leaves the impression that the fact of
the inevitable involution of the personal factor in every belief gives
some special sanction to some special belief. Mr. James says that his
essay on the \emph{right} to believe was unluckily entitled the ``\emph{Will} to
believe'' (p. 258). Well, even the term ``right'' is unfortunate, if the
personal or belief factor is inevitable---unfortunate because it seems
to indicate a privilege which might\marginpage{328} be exercised in special cases, in
religion, for example, though not in science; or, because it suggests
to some minds that the fact of the personal complicity involved in
belief is a warrant for this or that special personal attitude,
instead of being a warning to locate and define it so as to accept
responsibility for it. If we mean by ``will'' not something deliberate
and consciously intentional (much less, something insincere), but an
active personal participation, then belief \emph{as} will, rather than
either the right or the will to believe seems to phrase the matter
correctly.


I have attempted to review not so much Mr. James's book as the present
status of the pragmatic movement which is expressed in the book; and I
have selected only those points which seem to bear directly upon
matters of contemporary controversy. Even as an account of this
limited field, the foregoing pages do an injustice to Mr. James, save
as it is recognized that his lectures were ``popular lectures,'' as the
title-page advises us. We cannot expect in such lectures the kind of
explicitness which would satisfy the professional and technical
interests that have inspired this review. Moreover, it is inevitable
that the attempt to compose different points of view, hitherto
unco-ordinated, into a single whole should give rise to problems
foreign to any one factor of the synthesis, left to itself. The need
and possibility of the discrimination of various elements in the
pragmatic meaning of ``practical,'' attempted in this\marginpage{329} review, would
hardly have been recognized by me were it not for by-products of
perplexity and confusion which Mr. James's combination has effected.
Mr. James has given so many evidences of the sincerity of his
intellectual aims, that I trust to his pardon for the injustice which
the character of my review may have done \emph{him}, in view of whatever
service it may render in clarifying the problem to which he is
devoted.


As for the book itself, it is in any case beyond a critic's praise or
blame. It is more likely to take place as a philosophical classic than
any other writing of our day. A critic who should attempt to appraise
it would probably give one more illustration of the sterility of
criticism compared with the productiveness of creative genius. Even
those who dislike pragmatism can hardly fail to find much of profit in
the exhibition of Mr. James's instinct for concrete facts, the breadth
of his sympathies, and his illuminating insights. Unreserved
frankness, lucid imagination, varied contacts with life digested into
summary and trenchant conclusions, keen perceptions of human nature in
the concrete, a constant sense of the subordination of philosophy to
life, capacity to put things into an English which projects ideas as
if bodily into space till they are solid things to walk around and
survey from different sides---these things are not so common in
philosophy that they may not smell sweet even by the name of
pragmatism.


\section*{An Added Note as to the ``Practical''}
\marginpage{330}

It is easier to start a legend than to prevent its
continued circulation. No misconception of the
instrumental logic has been more persistent than the
belief that it makes knowledge merely a means to a
practical end, or to the satisfaction of practical needs---practical
being taking to signify some quite definite
utilities of a material or bread-and-butter type.
Habitual associations aroused by the word ``pragmatic''
have been stronger than the most explicit
and emphatic statements which any pragmatist has
been able to make. But I again affirm that the
term ``pragmatic'' means only the rule of referring all
thinking, all reflective considerations, to \emph{consequences}
for final meaning and test. Nothing is said about
the nature of the consequences; they may be aesthetic,
or moral, or political, or religious in quality---anything
you please. All that the theory requires is that they
be in some way consequences of thinking; not, indeed,
of it alone, but of it acted upon in connection with
other things. This is no after-thought inserted to
lessen the force of objections. Mr. Peirce explained
that he took the term ``pragmatic'' from Kant, in
order to denote empirical consequences. When he
refers to their practical character it is only to indicate\marginpage{331}
a criterion by which to avoid purely verbal disputes.
Different consequences are alleged to constitute rival
meanings of a term. Is a difference more than
merely one of formulation? The way to get an answer
is to ask whether, if realized, these consequences would
exact of us different modes of behavior. If they do
not make such a difference in conduct the difference
between them is conventional. It is not that consequences
are themselves practical, but that practical
consequences from them may at times be appealed to
in order to decide the specific question of whether
two proposed meanings differ save in words. Mr.
James says expressly that what is important is that
the consequences should be specific, not that they
should be active. When he said that general notions
must ``cash in,'' he meant of course that they must
be translatable into verifiable specific things. But
the words ``cash in'' were enough for some of his
critics, who pride themselves upon a logical rigor
unattainable by mere pragmatists.


In the logical version of pragmatism termed instrumentalism,
action or practice does indeed play a
fundamental r\^{o}le. But it concerns not the nature
of consequences but the nature of knowing. To use
a term which is now more fashionable (and surely to
some extent in consequence of pragmatism) than it
was earlier, instrumentalism means a behaviorist
theory of thinking and knowing. It means that
knowing is literally something which we do; that\marginpage{332}
analysis is ultimately physical and active; that meanings
in their logical quality are standpoints, attitudes,
and methods of behaving toward facts, and that
active experimentation is essential to verification.
Put in another way it holds that thinking does not
mean any transcendent states or acts suddenly
introduced into a previously natural scene, but that
the operations of knowing are (or are artfully derived
from) natural responses of the organism, which constitute
knowing in virtue of the situation of doubt
in which they arise and in virtue of the uses of
inquiry, reconstruction, and control to which they are
put. There is no warrant in the doctrine for carrying
over \emph{this} practical quality into the consequences
in which action culminates, and by which it is tested
and corrected. A knowing as an act is instrumental
to the resultant controlled and more significant situation;
this does not imply anything about the intrinsic
or the instrumental character of the consequent
situation. That is whatever it may be in a given case.


There is nothing novel nor heterodox in the notion
that thinking is instrumental. The very word is
redolent of an \emph{Organum}---whether \emph{novum} or \emph{veterum}.
The term ``instrumentality,'' applied to thinking,
raises at once, however, the question of whether
thinking as a tool falls within or without the subject-matter
which it shapes into knowledge. The answer of
formal logic (adopted moreover by Kant and followed
in some way by all neo-Kantian logics) is unambiguous.\marginpage{333}
To call logic ``formal'' means precisely that
mind or thought supplies forms foreign to the original
subject-matter, but yet required in order that it
should have the appropriate form of knowledge. In
this regard it deviates from the Aristotelian \emph{Organon}
which it professes to follow. For according to
Aristotle, the processes of knowing---of teaching and
learning---which lead up to knowledge are but the
actualization through the potentialities of the human
body of the \emph{same} forms or natures which are previously
actualized in Nature through the potentialities
of extra-organic bodies. Thinking which is not
instrumental to truth, which is merely formal in the
modern sense, would have been a monstrosity inconceivable
to him. But the discarding of the metaphysics
of form and matter, of cyclic actualizations and
eternal species, deprived the Aristotelian ``thought''
of any place within the scheme of things, and left it
an activity with forms alien to subject-matter. To
conceive of thinking as instrumental to truth or
knowledge, and as a tool shaped out of the same
subject-matter as that to which it is applied, is but
to return to the Aristotelian tradition about logic.
That the practice of science has in the meantime
substituted a logic of experimental discovery (of
which definition and classification are themselves but
auxiliary tools) for a logic of arrangement and exposition
of what is already known, necessitates, however,
a very different sort of \emph{Organon}. It makes\marginpage{334}
necessary the conception that the object of knowledge
is not something with which thinking sets out,
but something with which it ends: something which
the processes of inquiry and testing, that constitute
thinking, themselves produce. Thus the object of
knowledge is practical in the sense that it depends
upon a specific kind of practice for its existence---for its
existence as an object of knowledge. How practical
it may be in any other sense than this is quite another
story. The \emph{object of knowledge} marks an achieved
triumph, a secured control---that holds by the very
nature of knowledge. What other uses it may have
depends upon its own inherent character, not upon
anything in the nature of knowledge. We do not
know the origin and nature and the cure of malaria
till we can both produce and eliminate
malaria; the \emph{value} of either the production or the
removal depends upon the character of malaria in
relation to other things. And so it is with mathematical
knowledge, or with knowledge of politics or
art. Their respective objects are not known till they
are made in course of the process of experimental
thinking. Their usefulness when made is whatever,
from infinity to zero, experience may subsequently
determine it to be.




\end{document}